\chapter{Getting an HPC Account}
\label{ch:getting-a-hpc-account}

\section{Getting ready to request an account}
\label{sec:getting-ready-to-request-an-account}

\ifbrussel
 If you are affiliated with the \university, you can access \hpc using your NetID
 (the same username and password that you use for accessing your university email).
 Your \hpc account is not activated by default, you can do it via the Personal
 Account Manager (PAM) homepage (\url{https://idsapp.vub.ac.be/pam/}).
 Once you have activated your \hpcname account you will be able to use only the HPC
 infrastructure of the \university. If you would also like to use the HPC infrastructure
 located at other sites within the Flemish Supercomputing Centre (VSC), then you have
 to apply for a VSC account.
\fi

All users of \association can request
\ifbrussel
 a VSC
 \else
 an
\fi
account on
the \hpc, which is part of the Flemish Supercomputing Centre (VSC).

See \autoref{ch:hpc-policies} for more information on who is entitled to an account.

The VSC, abbreviation of Flemish Supercomputer Centre, is a virtual
supercomputer centre. It is a partnership between the five Flemish
associations: the Association KU~Leuven,  Ghent University Association, Brussels
University Association, Antwerp University Association and the University
Colleges-Limburg. The VSC is funded by the Flemish Government.


The \hpcInfra clusters use public/private key pairs for user authentication
(rather than passwords). Technically, the private key is stored on your local
computer and always stays there; the public key is stored on the \hpc.
Access to the \hpc is granted to anyone who can prove to have access to the
corresponding private key on his local computer.

\subsection{How do SSH keys work?}
\label{subsec:how-do-ssh-keys-work}

\begin{itemize}
    \item an SSH public/private key pair can be seen as a lock and a key
    \item the SSH public key is equivalent with a \emph{lock}: you give it to
        the VSC and they put it on the \emph{door} that gives access to your account.
    \item the SSH private key is like a \emph{physical key}: you don't hand it out
        to other people.
    \item anyone who has the key (and the optional password) can unlock the \emph{door}
        and log in to the account.
    \item the \emph{door} to your VSC account is special: it can have multiple \emph{locks}
        (SSH public keys) attached to it, and you only need to open one \emph{lock}
        with the corresponding \emph{key} (SSH private key) to open the \emph{door}
        (log in to the account).
\end{itemize}

\begin{center}
\ifantwerpen
\includegraphics*[width=5.75in, height=3.01in, keepaspectratio=false]{ch2-authentication-general}
\fi
\ifbrussel
\includegraphics*[width=5.75in, height=3.01in, keepaspectratio=false]{ch2-authentication-general}
\fi
\ifgent
\includegraphics*[width=5.75in, height=3.01in, keepaspectratio=false]{ch2-authentication-general}
\fi
\ifleuven
\includegraphics*[width=5.75in, height=3.01in, keepaspectratio=false]{ch2-authentication-general-leuven}
\fi
\end{center}

Since all VSC clusters use Linux as their main operating system, you will
need to get acquainted with using the command-line interface and using the
terminal.

% IFDEF WINDOWS
\ifwindows
  A typical Windows environment does not come with pre-installed
  software to connect and run command-line executables on a \hpc. Some tools
  need to be installed on your Windows machine first, before we can start the
  actual work.

  \subsection{Install PuTTY: A free telnet/SSH client}
  \label{sec:install-putty}

  We recommend to use the PuTTY tools package, which is freely available. You
  do not need to install PuTTY, you can download the PuTTY and PuTTYgen executable
  and run it. This can be useful in situations where you do not have the
  required permissions to install software on the computer you are using.
  Alternatively, an installation package is also available.

  You can download PuTTY from the official address:
  \url{http://www.chiark.greenend.org.uk/\~sgtatham/putty/}

  The PuTTY package consists of several components:

  \begin{enumerate}
    \item  \strong{PuTTY}: the Telnet and SSH client itself
    \item  \strong{PSCP}: an SCP client, i.e., command-line secure file copy
    \item  \strong{PSFTP}: an SFTP client, i.e., general file transfer sessions much like FTP
    \item  \strong{Plink}: a command-line interface to the PuTTY back ends
    \item  \strong{Pageant}: an SSH authentication agent for PuTTY, PSCP and Plink
    \item  \strong{PuTTYgen}: an RSA and DSA key generation utility
    \item  \strong{pterm}: a standalone terminal emulator
  \end{enumerate}

  \subsection{Generating a public/private key pair}
  \label{sec:generate-key-pair}

  Before requesting a VSC account, you need to generate a pair of \emph{ssh}
  keys. You need 2 keys, a public and a private key. You can visualise the public
  key as a lock to which only you have the key (your private key). You can send
  a copy of your lock to anyone without any problems, because only you can
  open it, as long as you keep your private key secure.
  To generate a public/private key pair, you can use the PuTTYgen key generator.

  Start \strong{\emph{PuTTYgen.exe}} it and follow these steps:

  \begin{enumerate}
    \item  In \keys{Parameters} (at the bottom of the window), choose
      ``RSA'' and keep the number of bits in the key to 2048.

    \begin{center}
    \includegraphics*[width=3.49in, height=3.38in, keepaspectratio=false]{ch2-puttygen-bits}
    \end{center}

    \item  Click on \keys{Generate}. To generate the key, you must move
      the mouse cursor over the PuTTYgen window (this generates some random
      data that PuTTYgen uses to generate the key pair). Once the key pair is
      generated, your public key is shown in the field \\
      \keys{Public key for pasting into OpenSSH authorized\_keys file}.
    \item  Next, it is advised to fill in the \keys{Key comment} field
      to make it easier identifiable afterwards.
    \item  Next, you should specify a passphrase in the \keys{Key passphrase}
      field and retype it in the \keys{Confirm passphrase} field.
      Remember, the passphrase protects the private key against unauthorised
      use, so it is best to choose one that is not too easy to guess but that
      you can still remember.
      Using a passphrase is not required, but we recommend you to use a good
      passphrase unless you are certain that your computer's hard disk is
      encrypted with a decent password. (If you are not sure your disk is
      encrypted, it probably isn't.)

    \begin{center}
    \includegraphics*[width=3.39in, height=3.28in, keepaspectratio=false]{ch2-puttygen-password}
    \end{center}

    \item  Finally, save both the public and private keys in a folder on your
      personal computer (We recommend to create and put them in the folder
      ``C:\textbackslash Users\textbackslash \%USERNAME\%\textbackslash
      AppData\textbackslash Local\textbackslash PuTTY\textbackslash .ssh'')
      with the buttons \keys{Save public key} and \keys{Save private key}.
      We recommend using the name ``\strong{id\_rsa.pub}'' for the public key,
      and ``\strong{id\_rsa.ppk}'' for the private key.
  \end{enumerate}

  If you use another program to generate a key pair, please remember that they
  need to be in the OpenSSH format to access the \hpc clusters.
\fi %WINDOWS

\ifmac
  To open a Terminal window in macOS, open the Finder and choose

  \emph{$>$$>$ Applications $>$ Utilities $>$ Terminal}
\fi

\iflinux
  Launch a terminal from your desktop's application menu and you will see the
  bash shell.
  There are other shells, but most Linux distributions use bash by default.
\fi

\ifmacORlinux
  Before requesting an account, you need to generate a pair of ssh keys. One
  popular way to do this on \OS is using the OpenSSH client included with \OS
  , which you can then also use to log on to the clusters.

  \subsection{Test OpenSSH}
  \label{sec:test-openssh}

  Secure Shell (ssh) is a cryptographic network protocol for secure data
  communication, remote command-line login, remote command execution, and other
  secure network services between two networked computers. In short, ssh
  provides a secure connection between 2 computers via insecure channels
  (Network, Internet, telephone lines, \ldots).

  ``Secure'' means that:
  \begin{enumerate}
    \item  the User is authenticated to the System; and
    \item  the System is authenticated to the User; and
    \item  all data is encrypted during transfer.
  \end{enumerate}

  OpenSSH is a FREE implementation of the SSH connectivity protocol. \OS comes
  with its own implementation of OpenSSH, so you don't need to install any
  third-party software to use it. Just open a terminal window and jump in!

  On all popular Linux distributions, the OpenSSH software is readily
  available, and most often installed by default. You can check whether the
  OpenSSH software is installed by opening a terminal and typing:

\begin{prompt}
%\shellcmd{ssh -V}%
OpenSSH_7.4p1, OpenSSL 1.0.2k-fips  26 Jan 2017
\end{prompt}

  To access the clusters and transfer your files, you will use the following commands:

  \begin{enumerate}
    \item  ssh-keygen: to generate the ssh keys
    \item  ssh: to open a shell on a remote machine;
    \item  sftp: a secure equivalent of ftp;
    \item  scp: a secure equivalent of the remote copy command rcp.
  \end{enumerate}

  \subsection{Generate a public/private key pair with OpenSSH}
  \label{sec:generate-key-pair-with-openssh}

  A key pair might already be present in the default location inside your home
  directory. Therefore, we first check if a key is available with the ``list
  short'' (``ls'')  command:

\begin{prompt}
%\shellcmd{ls ~/.ssh}%
\end{prompt}


  If a key-pair is already available, you would normally get:

\begin{prompt}
authorized_keys    id_rsa            id_rsa.pub         known_hosts
\end{prompt}

  Otherwise, the command will show:

\begin{prompt}
ls: .ssh: No such file or directory
\end{prompt}

  You can recognise a public/private key pair when a pair of files has the same
  name except for the extension ``.pub'' added to one of them. In this particular
  case, the private key is ``id\_rsa'' and public key is ``id\_rsa.pub''. You may
  have multiple keys (not necessarily in the directory ``\tilde/.ssh'') if you or
  your operating system requires this.

  You will need to generate a new key pair, when:
  \begin{enumerate}
    \item  you don't have a key pair yet,
    \item  you forgot the passphrase protecting your private key,
    \item  or your private key was compromised.
  \end{enumerate}

  For extra security, the private key itself can be encrypted using a ``passphrase'',
  to prevent anyone from using your private key even when they manage to copy
  it. You have to ``unlock'' the private key by typing the passphrase.
  Be sure to never give away your private key, it is private and should stay private.
  You should not even copy it to one of your other machines, instead, you should create a
  new public/private key pair for each machine.

\begin{prompt}
%\shellcmd{ssh-keygen}%
Generating public/private rsa key pair.
Enter file in which to save the key (/home/user/.ssh/id_rsa):
Enter passphrase (empty for no passphrase):
Enter same passphrase again:
Your identification has been saved in /home/user/.ssh/id_rsa.
Your public key has been saved in /home/user/.ssh/id_rsa.pub.
\end{prompt}

  This will ask you for a file name to store the private and public key, and a
  passphrase to protect your private key. It needs to be emphasised that you
  really should choose the passphrase wisely! The system will ask you for it
  every time you want to use the private key that is every time you want to
  access the cluster or transfer your files.

  \strong{Without your key pair, you won't be able to apply for a personal VSC account.}
\fi %macORlinux

\ifmacORlinux

\subsection{Using an SSH agent (optional)}
\label{sec:using-ssh-agent-with-openssh}
  Most recent Unix derivatives include by default an SSH agent
  \iflinux(``gnome-keyring-daemon'' in most cases) \fi to keep and manage the user SSH keys.
  If you use one of these derivatives you \strong{must} include the new keys into
  the SSH manager keyring to be able to connect to the HPC cluster. If not,
  SSH client will display an error message (see \autoref{ch:preparing-the-environment})
  similar to this:

  \begin{flattext}
   Agent admitted failure to sign using the key.
   Permission denied (publickey,gssapi-keyex,gssapi-with-mic).
  \end{flattext}

  This could be fixed using the \verb|ssh-add| command.
  You can include the new private keys' identities in your keyring with:

\begin{prompt}
%\shellcmd{ssh-add}%
\end{prompt}

\begin{tip}
  Without extra options \verb|ssh-add| adds any key located at \verb|$HOME/.ssh|
  directory, but you can specify the private key location path as argument,
  as example: \verb|ssh-add /path/to/my/id_rsa|.
\end{tip}

Check that your key is available from the keyring with:
\begin{prompt}
%\shellcmd{ssh-add -l}%
\end{prompt}

  After these changes the key agent will keep your SSH key to connect
  to the clusters as usual.
\begin{tip}
  You should execute \verb|ssh-add| command again if you generate a new SSH key.
\end{tip}

\iflinux
  Visit \url{https://wiki.gnome.org/Projects/GnomeKeyring/Ssh} for
  more information.
\fi
\fi %macORlinux

\section{Applying for the account}
\label{sec:applying-for-the-account}

Visit \url{https://account.vscentrum.be/}

You will be redirected to our WAYF (Where Are You From) service where you have to select your ``Home Organisation''.

\begin{center}
\includegraphics*[width=5.73in, height=4.00in, keepaspectratio=false]{ch2-browser-authenticate}
\end{center}

Select '\wayf' in the dropdown box and optionaly select 'Save my preference' and 'permanently.'

Click \keys{Confirm}

You will now be taken to the authentication page of your institute.

\ifantwerpen
\begin{center}
\includegraphics*[width=5.73in, height=4.00in, keepaspectratio=false]{ch2-browser-authenticate-antwerpen}
\end{center}

\strong{The site is only accessible from within the \university
domain}, so the page won't load from, e.g., home. However, you can also get
external access to the \university domain using VPN. We refer to the Pintra
pages of the ICT Department
% (\url{https://pintra.uantwerpen.be/webapps/ua-pintrasite-BBLEARN/module/index.jsp?course_id=_8_1&tid=_525_1&lid=_11434_1&l=nl_PINTRA})
for more information.

\subsection{Users of the \association}
\label{sec:users-of-association}

All users (researchers, academic staff, etc.) from the higher education
institutions associated with \university can get a VSC account via the \university. There is not
yet an automated form to request your personal VSC account.

Please e-mail the \hpc staff to get an account (see Contacts information).
You will have to provide a public \emph{ssh} key generated as described
above. Please attach your public key (i.e., the file named ``id\_rsa.pub''),
which you will normally find in your .ssh subdirectory within your HOME
Directory. (i.e., /Users/$<$username$>$/.ssh/id\_rsa.pub).


\fi % END IF ANTWERPEN

\ifgent
  You will now have to log in with CAS using your UGent account.

  You either have a
  login name of maximum 8 characters, or a (non-UGent) email address if you are an external
  user. In case of problems with your UGent password, please visit:
  \url{https://password.ugent.be/}. After logging in, you may be requested
  to share your information. Click ``Yes, continue''.
  \begin{center}
  \includegraphics*[width=\textwidth]{ch2-browser-authenticate-gent}
  \end{center}
\fi

\ifbrussel
  \begin{center}
  \includegraphics*[width=5.73in, height=4.00in, keepaspectratio=false]{ch2-browser-authenticate-brussel}
  \end{center}

  You will now have to log in using your NetID. (The same username and password that you use for accessing your e-mail)

\fi

\ifleuven
\hpc users should login using their staff or student id
(The same username and password that is used for accessing e-mail).

 \begin{center}
  \includegraphics*[width=5.73in, height=4.00in, keepaspectratio=false]{ch2-browser-authenticate-leuven}
  \includegraphics*[width=5.73in, height=4.00in, keepaspectratio=false]{ch2-browser-authenticate-hasselt}
  \end{center}


\fi

%TODO: other institute login information explained here?

After you log in using your \university login and password,
you will be asked to upload the file that contains
your public key, i.e., the file ``id\_rsa.pub'' which you have
generated earlier.

\ifwindows
  This file should have been stored in the directory
  ``C:\textbackslash Users\textbackslash \%USERNAME\%\textbackslash
  AppData\textbackslash Local\textbackslash PuTTY\textbackslash .ssh''
\fi
\ifmacORlinux
  This file has been stored in the directory ``\emph{\tilde/.ssh/}''.
\fi

\ifmac
\begin{tip}
  As ``.ssh'' is an invisible directory, the Finder will
  not show it by default. The easiest way to access the folder, is by pressing
  \keys{cmd} + \keys{shift} + \keys{g}, which will allow you to enter the name of a directory, which
  you would like to open in Finder. Here, type ``\tilde/.ssh'' and press enter.
\end{tip}
\fi

After you have uploaded your public key you will receive an e-mail with a link to
confirm your e-mail address. After confirming your e-mail address the VSC staff will
review and if applicable approve your account.

\subsection{Welcome e-mail}
\label{sec:welcome-email}

Within one day, you should receive a Welcome e-mail with your VSC account details.

\begin{flattext}
Dear (Username),
Your VSC-account has been approved by an administrator.
Your vsc-username is %\userid%

Your account should be fully active within one hour.

To check or update your account information please visit
https://account.vscentrum.be/

For further info please visit https://www.vscentrum.be/en/user-portal

Kind regards,
-- The VSC administrators
\end{flattext}

Now, you can start using the \hpc. You can always look up your vsc id later by visiting\\*
\url{https://account.vscentrum.be}.

\subsection{Adding multiple SSH public keys (optional)}
\label{sec:adding-multiple-keys}
In case you are connecting from different computers to the login nodes, it is advised to use separate SSH public keys to do so.
You should follow these steps.

\begin{enumerate}
\ifmacORlinux
 \item Create a new public/private SSH key pair from the new computer. Repeat the process described in section~\ref{sec:generate-key-pair-with-openssh}.
\fi %macORlinux
\ifwindows
 \item Create a new public/private SSH key pair from Putty. Repeat the process described in section~\ref{sec:generate-key-pair}.
\fi %ifwindows
 \item Go to \url{https://account.vscentrum.be/django/vo/edit}
 \item Upload the new SSH public key using the \strong{Add public key} section.
 \item (optional) If you lost your key, you can delete the old key on the same page. You should keep at least one valid public SSH key in your account.
 \item Take into account that it will take some time before the new SSH public key is active in your account on the system; waiting for 15-30 minutes should be sufficient.
\end{enumerate}

\section{Computation Workflow on the \hpc}
\label{sec:compuation-workflow-on-the-hpc}

A typical Computation workflow will be:

\begin{enumerate}
  \item  Connect to the \hpc
  \item  Transfer your files to the \hpc
  \item  Compile your code and test it
  \item  Create a job script
  \item  Submit your job
  \item  Wait while
  \begin{enumerate}
    \item  your job gets into the queue
    \item  your job gets executed
    \item  your job finishes
  \end{enumerate}
  \item  Move your results
\end{enumerate}

We'll take you through the different tasks one by one in the following
chapters.
