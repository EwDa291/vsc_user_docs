\chapter{Mympirun}
\label{ch:mympirun}

\lstinline|mympirun| is a tool to make it easier for users of HPC clusters to run
MPI programs with good performance.

\section{Basic usage}
\label{sec:myrun-basic-usage}

The most basic form of using \lstinline|mympirun| is \lstinline|mympirun [options] -- your_program [your_program options]|.

\section{Controlling number of processes}

There are four options from which you can choose to control the number of processes
\lstinline|mympirun| will start. In the following examples, the program \lstinline|mpi_hello|
prints a single line: \lstinline|Hello world from processor <node> ...|.
\lstinline|mympirun| normally starts one process per \emph{core}.

\subsection{\texttt{--hybrid}/\texttt{-h}}

The \lstinline|--hybrid| option requires a positive number. This number specifies
the number of processes started on each available physical \emph{node}.

\begin{prompt}
%\shellcmd{echo \$PBS\_NUM\_NODES}%
2
%\shellcmd{mympirun --hybrid 2 ./mpi_hello}%
Hello world from processor node2157.delcatty.os, rank 1 out of 4 processors
Hello world from processor node2158.delcatty.os, rank 3 out of 4 processors
Hello world from processor node2158.delcatty.os, rank 2 out of 4 processors
Hello world from processor node2157.delcatty.os, rank 0 out of 4 processors
\end{prompt}

\subsection{\texttt{--universe}}

The \lstinline|--universe| option requires a positive number. This number specifies
the exact number of processes \lstinline|mympirun| will start
(so the number of nodes doesn't affect the number of processes).

\begin{prompt}
%\shellcmd{echo \$PBS\_NUM\_NODES}%
16
%\shellcmd{mympirun --universe 2 ./mpi_hello}%
Hello world from processor node2157.delcatty.os, rank 1 out of 2 processors
Hello world from processor node2157.delcatty.os, rank 0 out of 2 processors
\end{prompt}

\subsection{\texttt{--double} and \texttt{--multi}}

As the name suggests, when using the \lstinline|--double| option, \lstinline|mympirun| will start double
the amount of processes as it normally would. The \lstinline|--multi| option works the same
but it requires an integer, indicating the multiplier, for example, \lstinline|--multi 3| will
start triple the amount of processes. This means \lstinline|--double| and \lstinline|--multi 2| will have
the exact same effect.

\begin{prompt}
%\shellcmd{echo \$PBS\_NUM\_NODES}%
2

%\shellcmd{echo \$PBS\_NUM\_PPN}%
2

%\shellcmd{mympirun --double ./mpi\_hello}%
Hello world from processor node2157.delcatty.os, rank 1 out of 8 processors
Hello world from processor node2158.delcatty.os, rank 3 out of 8 processors
Hello world from processor node2157.delcatty.os, rank 0 out of 8 processors
Hello world from processor node2158.delcatty.os, rank 2 out of 8 processors
...
\end{prompt}

\section{Redirecting output with \texttt{--output}}

You can specify a file to output stdout and stderr.

\section{FAQ}

\subsection{\texttt{mympirun} seems to ignore its arguments}

For example, we have a simple script (\lstinline|./hello.sh|):

\begin{code}{bash}
#!/bin/bash
echo "hello world"
\end{code}

And we run it like \lstinline|mympirun ./hello.sh --output output.txt|.

To our suprise, this doesn't output to the file \lstinline|output.txt|, but to standard out!
This is because \lstinline|mympirun| expects the program name and the arguments of the program to
be its last arguments. Here, the \lstinline|--output output.txt| arguments are passed to
\lstinline|./hello.sh| instead of to \lstinline|mympirun|.
