\usepackage[english]{babel}
\usepackage{amssymb}
\usepackage{amsmath}
\usepackage{hyperref}
% allow conditional compilation
\usepackage{etoolbox}
\usepackage[toc,nonumberlist]{glossaries}

\usepackage{color}
\usepackage{listings}
\usepackage{tikz}
\usepackage[T1]{fontenc}
\usepackage{graphics}
\graphicspath{{img//}}

\lstset{basicstyle=\ttfamily,breaklines=true}
\lstdefinestyle{prompt}{language=bash,frame=tb,columns=fullflexible,
  escapechar=\%}
\lstdefinestyle{code}{numbers=left}

\lstnewenvironment{prompt} {\lstset{style=prompt}} {}
\lstnewenvironment{code}[1]{\lstset{style=code, language=#1}} {}
\lstnewenvironment{prog}{\lstset{style=code}} {}

% Taken from: http://tex.stackexchange.com/questions/9363/how-does-one-insert-a-backslash-or-a-tilde-into-latex
\renewcommand{\tilde}{\raise.17ex\hbox{$\scriptstyle\mathtt{\sim}$}}
\newcommand{\shellcmd}[1]{\textbf{\texttt{#1}\\}}
\parindent=0pt
\parskip=7pt
\newcommand{\ignore}[1]{}

\newif\ifremark
\long\def\remark#1{
  \ifremark%
  \begingroup%
  \dimen0=\columnwidth
  \advance\dimen0 by -1in%
  \setbox0=\hbox{\parbox[b]{\dimen0}{\protect\em #1}}
  \dimen1=\ht0\advance\dimen1 by 2pt%
  \dimen2=\dp0\advance\dimen2 by 2pt%
  \vskip 0.25pt%
  \hbox to \columnwidth{%
    \vrule height\dimen1 width 3pt depth\dimen2%
    \hss\copy0\hss%
    \vrule height\dimen1 width 3pt depth\dimen2%
  }%
  \endgroup%
\fi}

\remarktrue

\newcommand{\includesite}[1]{
  \ifleuven
    \include{#1-leuven}
  \fi
  \ifantwerpen
    \include{#1-antwerpen}
  \fi
  \ifhasselt
    \include{#1-hasselt}
  \fi
  \ifbrussel
    \include{#1-brussel}
  \fi
  \ifgent
    \include{#1-gent}
  \fi
}

\newif\ifwindows
\newif\ifmac
\newif\ifleuven
\newif\ifhasselt
\newif\ifbrussel
\newif\ifgent
\newif\ifantwerpen

\windowsfalse
\macfalse
\leuvenfalse
\hasseltfalse
\brusselfalse
\gentfalse
\antwerpenfalse

\ifdefined\iswindows
  \windowstrue
\fi
\ifdefined\ismac
  \mactrue
\fi
\ifdefined\isleuven
  \leuventrue
\fi
\ifdefined\ishasselt
  \hasselttrue
\fi
\ifdefined\isbrussel
  \brusseltrue
\fi
\ifdefined\isgent
  \genttrue
\fi
\ifdefined\isantwerpen
  \antwerpentrue
\fi
