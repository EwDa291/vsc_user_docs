\chapter{Compiling and testing your software on the HPC}

All nodes in the \hpc cluster are running the ``\operatingsystem'' Operating
system, which is a specific version of RedHat-Linux. This means that all the
software programs (executable) that the end-user wants to run on the \hpc first
must be compiled for \operatingsystem.  It also means that you first have to
install all the required external software packages on the \hpc.

Most commonly used compilers are already pre-installed on the \hpc and can
be used straight away. Also many popular external software packages, which are
regularly used in the scientific community, are also pre-installed.

\section{Check the pre-installed software on the \hpc}

In order to check all the available modules and their version numbers, which
are pre-installed on the \hpc enter:

\inputsite{available-modules}

When your required application is not available on the \hpc please contact any
\hpc member. Be aware of potential ``License Costs''.  ``Open Source'' software
is often preferred.

\section{Porting your code}

To \strong{port} a software-program is to translate it from the operating
system in which it was developed (e.g., Windows 7) to another operating system
(e.g., Scientific Linux on our \hpc so that it can be used there. Porting
implies some degree of effort, but not nearly as much as redeveloping the
program in the new environment.  It all depends on how ``portable'' you wrote
your code.

In the simplest case the file or files may simply be copied from one machine to
the other. However, in many cases the software is installed on a computer in a
way, which depends upon its detailed hardware, software, and setup, with device
drivers for particular devices, using installed operating system and supporting
software components, and using different directories.

In some cases software, usually described as ``portable software'' is
specifically designed to run on different computers with compatible operating
systems and processors without any machine-dependent installation; it is
sufficient to transfer specified directories and their contents. Hardware- and
software-specific information is often stored in configuration files in
specified locations (e.g., the registry on machines running MS Windows).

Software, which is not portable in this sense, will have to be transferred with
modifications to support the environment on the destination machine.

Whilst programming, it would be wise to stick to certain standards (e.g.,
ISO/ANSI/POSIX).  This will ease the porting of your code to other platforms.

Porting your code to the \operatingsystem platform is the responsibility of the
end-user.

\section{Compiling and building on the \hpc}

Compiling refers to the process of translating code written in some programming
language, e.g. Fortran, C, or C++, to machine code. Building is similar, but
includes gluing together the machine code resulting from different source files
into an executable (or library). The text below guides you through some basic
problems typical for small software projects. For larger projects it is more
appropriate to use makefiles or even an advanced build system like CMake.

All the \hpc nodes run the same version of the Operating System, i.e.
\operatingsystem. So, it is sufficient to compile your program on any compute
node.  Once you have generated an executable with your compiler, this
executable should be able to run on any other compute-node.

A typical process looks like:

\begin{enumerate}
\item  Copy your software to the login-node of the \hpc
\item  Start an interactive session on a compute node;
\item  Compile it;
\item  Test it locally;
\item  Generate your job-scripts;
\item  Test it on the \hpc
\item  Run it (in parallel);
\end{enumerate}

We assume you've copied your software to the \hpc The next step is to request your private compute node.
\begin{prompt}
%\shellcmd{qsub -I}%
qsub: waiting for job %\jobid% to start
\end{prompt}

\subsection{Compiling a sequential program in C}

Go to the examples for Chapter 9 and load the GCC modules:
\begin{prompt}
%\shellcmd{cd \tilde/\exampledir}%
%\shellcmd{module load GCC}%
\end{prompt}

We now list the directory and explore the contents of the ``\emph{hello.c}''
program:

\begin{prompt}
%\shellcmd{ls -l}%
total 512
-rw-r--r-- 1 %\userid% 214 Sep 16 09:42 hello.c
-rw-r--r-- 1 %\userid% 130 Sep 16 11:39 hello.pbs*
-rw-r--r-- 1 %\userid% 359 Sep 16 13:55 mpihello.c
-rw-r--r-- 1 %\userid% 304 Sep 16 13:55 mpihello.pbs
\end{prompt}

\examplecode{C}{hello.c}

The ``hello.c'' program is a simple source file, written in C. It'll print 500
times ``Hello \#$<$num$>$'', and waits one second between 2 printouts.

We first need to compile this C-file into an executable with the gcc-compiler.

First, check the command line options for \emph{``gcc'' (GNU C-Compiler)},
then we compile and list the contents of the directory again:

\begin{prompt}
%\shellcmd{gcc}% --help
%\shellcmd{gcc -o hello hello.c}%
%\shellcmd{ls -l}%
total 512
-rwxrwxr-x 1 %\userid% 7116 Sep 16 11:43 hello*
-rw-r--r-- 1 %\userid%  214 Sep 16 09:42 hello.c
-rwxr-xr-x 1 %\userid%  130 Sep 16 11:39 hello.pbs*
\end{prompt}

A new file ``hello'' has been created. Note that this file has ``execute''
rights, i.e. it is an executable. More often than not, calling gcc -- or any
other compiler for that matter -- will provide you with a list of errors and
warnings referring to mistakes the programmer made, such as typos, syntax
errors. You will have to correct them first in order to make the code compile.
Warnings pinpoint less crucial issues that may relate to performance problems,
using unsafe or obsolete language features, etc. It is good practice to remove
all warnings from a compilation process, even if they seem unimportant so that
a code change that produces a warning does not go unnoticed.

Let's test this program on the local compute node, which is at your disposal
after the ``qsub --I'' command:

\begin{prompt}
%\shellcmd{./hello}%
Hello #0
Hello #1
Hello #2
Hello #3
Hello #4
%\dots%
\end{prompt}


It seems to work, now run it on the \hpc
\begin{prompt}
%\shellcmd{qsub hello.pbs}%
\end{prompt}

\subsection{Compiling a parallel program in C/MPI}

\begin{prompt}
%\shellcmd{cd \tilde/\exampledir}%
\end{prompt}

List the directory and explore the contents of the ``\textit{mpihello.c}'' program:
\begin{prompt}
%\shellcmd{ls -l}%
total 512
total 512
-rw-r--r-- 1 %\userid% 214 Sep 16 09:42 hello.c
-rw-r--r-- 1 %\userid% 130 Sep 16 11:39 hello.pbs*
-rw-r--r-- 1 %\userid% 359 Sep 16 13:55 mpihello.c
-rw-r--r-- 1 %\userid% 304 Sep 16 13:55 mpihello.pbs
\end{prompt}

\examplecode{C}{mpihello.c}

The ``mpi\_hello.c'' program is a simple source file, written in C with MPI library calls.

We first need to compile this file into an executable with the
\emph{mpicc}-compiler. The mpicc compiler is simply a MPI-aware wrapper
around the gcc compiler. It is available from the impi modules.

First, search the relevant MPI modules, and load the most recent.
\begin{prompt}
%\shellcmd{module}% av 2>&1 | grep -i mpi
impi/3.2.1.009
impi/3.2.2.006
impi/4.0.0.025
impi/4.0.0.027
impi/4.0.0.028
impi/4.0.1.007
impi/4.1.0.027
impi/4.1.1.036
%\dots%
%\shellcmd{module load impi}%
\end{prompt}

Then, check the command line options for \emph{``mpicc'' (GNU C-Compiler with
MPI extensions)}, then we compile and list the contents of the directory again:

\begin{prompt}
%\shellcmd{mpicc --help}%
%\shellcmd{mpicc -o mpihello mpihello.c}%
%\shellcmd{ls -l}%
\end{prompt}

A new file ``hello'' has been created. Note that this program has ``execute''
rights.

Let's test this program on the ``login''-node first:

\begin{prompt}
%\shellcmd{./mpihello}%
Hello World from Node 0.
\end{prompt}

It seems to work, now run it on the \hpc.

\begin{prompt}
%\shellcmd{qsub mpihello.pbs}%
\end{prompt}

\subsection{Compiling a parallel program in INTEL Cluster Studio}

We will now compile the same program, but using the INTEL Cluster Studio
compilers. We stay in the examples directory for this chapter:

\begin{prompt}
%\textbf{cd \tilde/\exampledir}%
\end{prompt}

We will compile this C/MPI -file into an executable with the INTEL Cluster
Studio compiler. First, clear the modules (purge) and then load the latest
``ictce'' (Intel Cluster Toolkit Compiler Edition) module:

\begin{prompt}
%\shellcmd{module purge}%
%\shellcmd{module load ictce}%
\end{prompt}

Then, compile and list the contents of the directory again. The ictce
equivalent of mpicc is mpiicc.

\begin{prompt}
%\shellcmd{mpiicc -o mpihello mpihello.c}%
%\shellcmd{ls -l}%
\end{prompt}

Note that the old ``mpihello'' file has been overwritten.
Let's test this program on the ``login''-node first:

\begin{prompt}
%\shellcmd{./mpihello}%
Hello World from Node 0.
\end{prompt}

It seems to work, now run it on the \hpc.

\begin{prompt}
%\shellcmd{qsub mpihello.pbs}%
\end{prompt}

Note: The \association only has a license for the Intel Cluster Studio for a
fixed number of users. As such, it might happen that you have to wait a few
minutes before a floating license becomes available for your use.

Note: The Intel Cluster Studio contains equivalent compilers for all GNU
compilers. Hereafter the overview for C, C++ and Fortran compilers.

\begin{tabular}{|p{0.15\textwidth}|p{0.15\textwidth}|p{0.15\textwidth}|p{0.15\textwidth}|p{0.15\textwidth}|} \hline
& \multicolumn{2}{|p{0.3\textwidth}|}{\strong{Sequential Program}} & \multicolumn{2}{|p{0.3\textwidth}|}{\strong{Parallel Program (with MPI)}} \\ \hline
                 & {\strong{GNU}} & \strong{Intel} & \strong{GNU} & \strong{Intel} \\ \hline
\strong{C}       & gcc            & icc            & mpicc        & mpiicc \\ \hline
\strong{C++}     & g++            & icpc           & mpicxx       & mpiicpc \\ \hline
\strong{Fortran} & gfortran       & ifort          & mpif90       & mpiifort \\ \hline
\end{tabular}
