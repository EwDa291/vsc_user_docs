\chapter{Frequently Asked Questions}
\label{ch:faq}

\section{When will my job start?}

\ifgent
See the explanation about how jobs get prioritized in \autoref{subsec:priority}.
\else
\ifbrussel
See the explanation about how jobs get prioritized in \autoref{subsec:priority}.
\else
You can use the \verb|showstart| command. For more information, see \autoref{sec:monitoring-and-managing-your-jobs}.
\fi % BRUSSEL
\fi % GENT

\section{Can I share my account with someone else?}

\strong{NO.} You are not allowed to share your VSC account with anyone else, it is strictly personal.
\ifgent
See \url{https://helpdesk.ugent.be/account/en/regels.php}.
\fi
\ifleuven
See \url{https://admin.kuleuven.be/personeel/english_hrdepartment/ICT-codeofconduct-staff#section-5}.
\fi
\ifbrussel
See \url{http://www.vub.ac.be/sites/vub/files/reglement-gebruik-ict-infrastructuur.pdf}.
\fi
\ifantwerpen
See \url{https://pintra.uantwerpen.be/bbcswebdav/xid-23610_1}
\fi
\ifgent
If you want to share data, there are alternatives (like a shared
directories in VO space, see \autoref{sec:virtual-organisations}).
\fi

\section{Can I share my data with other \hpc users?}

Yes, you can use the \verb|chmod| or \verb|setfacl| commands to change permissions
of files so other users can access the data. For example, the following command
will enable a user named ``otheruser'' to read the file named \verb|dataset.txt|.
See

\begin{prompt}
%\shellcmd{setfacl -m u:otheruser:r dataset.txt}%
%\shellcmd{ls -l dataset.txt}%
-rwxr-x---+ 2 %\userid% mygroup      40 Apr 12 15:00 dataset.txt
\end{prompt}

For more information about \verb|chmod| or \verb|setfacl|, see \href{\LinuxManualURL#sec:chmod}
{the section on chmod in chapter 3 of the Linux intro manual}.
% \section{I no longer work for \university, can I transfer my data to another researcher working at \university}
% See https://github.com/hpcugent/vsc_user_docs/issues/230

\section{Can I use multple different SSH key pairs to connect ot my VSC account?}

Yes, and this is recommendend when working from different computers. Please see
\autoref{sec:adding-multiple-keys} on how to do this.
