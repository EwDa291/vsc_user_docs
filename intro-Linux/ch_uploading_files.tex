\chapter{Uploading/downloading/editing files}

\section{Uploading/downloading files}
\label{sec:uploading-files}

To transfer files from and to the HPC, see
\href{\HPCManualURL#sec:filetranfer}{the section about tranferring files in chapter 3 of the HPC manual}.

\ifwindows

\subsection{\texttt{dos2unix}}
\label{subsec:dos2unix}
\hypertarget{sec:dos2unix}{}
After copying files it is advised to run

\begin{prompt}
%\shellcmd{dos2unix filename}
\end{prompt}

as this will fix any problems with Windows/Unix conversion.
\fi
\section{Symlinks for data/scratch}

As we end up in the home directory when connecting, it would be convenient if we
could access our data and VO storage. To facilitate this we will create
symlinks to them in our home directory. This will create 4 symbolic links
% You're probably wondering about the duplication? I tried to fix it by putting
% the ifwindows inline. This resulted in either an ugly lone ) or the newly-invented word
% "desktopadd". total_wasted_minutes=10
\ifwindows
(they're like ``shortcuts'' on your desktop and they look like directories in WinSCP)
\else
(they're like ``shortcuts'' on your desktop)
\fi
pointing to the respective storages:

\begin{prompt}
%\shellcmd{cd \$HOME}%
%\shellcmd{ln -s \$VSC\_SCRATCH scratch}%
%\shellcmd{ln -s \$VSC\_DATA data}%
%\shellcmd{ls -l scratch data}%
lrwxrwxrwx 1 %\userid% %\userid% 31 Mar 27  2009 data -> %\datadir%
lrwxrwxrwx 1 %\userid% %\userid% 34 Jun  5  2012 scratch -> %\scratchdir%
\end{prompt}

\section{Editing with \texttt{nano}}

Nano is the simplest editor available on Linux. To open Nano, just type
\verb|nano|. To edit a file, you use \verb|nano the_file_to_edit.txt|. You will
be presented with the contents of the file and a menu at the bottom with
commands like \verb|^O Write Out| The \verb|^| is the Control key. So \verb|^O| means
\verb|Ctrl-O|. The main commands are:

\begin{enumerate}
\item Open ("Read"): \verb|^R|
\item Save ("Write Out"): \verb|^O|
\item Exit: \verb|^X|
\end{enumerate}

More advanced editors (beyond the scope of this page) are \verb|vim| and \verb|emacs|.
A simple tutorial on how to get started with \verb|vim| can be found at \url{https://www.openvim.com/}.
