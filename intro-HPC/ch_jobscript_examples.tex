\chapter{Job script examples}
\label{ch:jobscript-examples}

\section{Single-core job}

Here's an example of a single-core job script:

\examplecode{bash}{single_core.sh}

\begin{enumerate}
    \item Using \lstinline|#PBS| header lines, we specify the resource requirements for the job,
        see \autoref{ch:torque-options} for a list of these options
    \item A module for \lstinline|Python 3.6| is loaded, see also \autoref{sec:modules}
    \item We stage the data in: the file \lstinline|input.txt| is copied into the ``working'' directory, see \autoref{ch:running-jobs-with-input-output-data}
    \item The main part of the script runs a small Python program that counts the
        number of characters in the provided input file \lstinline|input.txt|
    \item We stage the results out: the output file \lstinline|output.txt| is copied from
        the ``working directory'' (\lstinline|$TMPDIR||) to a unique directory in \lstinline|$VSC_DATA|.
        For a list of possible storage locations, see \autoref{subsec:predefined-user-directories}.
\end{enumerate}

\section{Multi-core job}

Here's an example of a multi-core job script that uses \lstinline|mympirun|:

\examplecode{bash}{multi_core.sh}

An example MPI hello world program can be downloaded from \url{https://github.com/hpcugent/vsc-mympirun/blob/master/testscripts/mpi_helloworld.c}.


\section{Running a command with a maximum time limit}
\label{sec:maximum-timelimit-timeout-jobscript}

If you want to run a job, but you are not sure it will finish before the job runs
out of walltime and you want to copy data back before, you have to stop the main
command before the walltime runs out and copy the data back.

This can be done with the \lstinline|timeout| command. This command sets a limit
of time a program can run for, and when this limit is exceeded, it kills the program.
Here's an example job script using \lstinline|timeout|:

\examplecode{bash}{timeout.sh}

The example program used in this script is a dummy script that simply sleeps a specified amount of minutes:

\examplecode{bash}{example_program.sh}
