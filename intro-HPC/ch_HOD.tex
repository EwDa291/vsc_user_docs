\chapter{Hanythingondemand (HOD)}
\label{ch:hod}

Hanythingondemand (or HOD for short) is a tool to run a Hadoop (Yarn) cluster on
a traditional HPC system.

\section{Documentation}

The official documentation for HOD version 3.0.0 and newer is available at \url{https://hod.readthedocs.org/en/latest/}.
The slides of the 2016 HOD training session are available at \url{http://users.ugent.be/~kehoste/hod_20161024.pdf}.

\strong{This chapter only covers how HOD is installed on the \hpcInfra infrastructure.}

\section{Using HOD}

Before using HOD, you first need to load the \lstinline|hod| module. We don't specify
a version here (this is an exception, for most other modules you should, see \autoref{subsec:explicit-version-numbers}) because newer versions
might include important bug fixes.

\begin{prompt}
%\shellcmd{module load hod}%
\end{prompt}

\subsection{Compatibility with login nodes}

The \lstinline|hod| modules are constructed such that they can be used on the \hpcInfra login nodes,
regardless of which \lstinline|cluster| module is loaded (this is not the case for software installed
via modules in general, see \autoref{sec:running-software-incompatible-with-host}).

As such, you should experience no problems if you swap to a different cluster module
before loading the \lstinline|hod| module and subsequently running |hod|.

For example, this will work as expected:

\begin{prompt}
%\shellcmd{module swap cluster/\othercluster}%
%\shellcmd{module load hod}%
%\shellcmd{hod}%
hanythingondemand - Run services within an HPC cluster
usage: hod <subcommand> [subcommand options]
Available subcommands (one of these must be specified!):
    batch           Submit a job to spawn a cluster on a PBS job controller, run a job script, and tear down the cluster when it's done
    clean           Remove stale cluster info.
...
\end{prompt}

Note that also modules named \lstinline|hanythingondemand/*| are available. These should
however not be used directly, since they may not be compatible with the login nodes
(depending on which cluster they were installed for).

\subsection{Standard HOD configuration}

The hod module will also put a basic configuration in place for HOD, by defining a couple of
\lstinline|$HOD_*| environment variables:

\begin{prompt}
%\shellcmd{module load hod}%
%\shellcmd{env | grep HOD | sort}%
HOD_BATCH_HOD_MODULE=hanythingondemand/3.2.2-intel-2016b-Python-2.7.12
HOD_BATCH_WORKDIR=$VSC_SCRATCH/hod
HOD_CREATE_HOD_MODULE=hanythingondemand/3.2.2-intel-2016b-Python-2.7.12
HOD_CREATE_WORKDIR=$VSC_SCRATCH/hod
\end{prompt}

By defining these environment variables, we avoid that you have to specify \lstinline|--hod-module|
and \lstinline|--workdir| when using \lstinline|hod batch| or \lstinline|hod create|, since they are strictly required.

If you which to use a different parent working directory for HOD, it suffices to
either redefine \lstinline|$HOD_BATCH_WORKDIR| and \lstinline|$HOD_CREATE_WORKDIR|,
or to specify \lstinline|--workdir| (which will override the corresponding environment variable).

Changing the HOD module that is used by the HOD backend (i.e. using \lstinline|--hod-module|
or redefining \lstinline|$HOD_*_HOD_MODULE|) is strongly discouraged.

\subsection{Cleaning up}

After HOD clusters terminate, their local working directory and cluster information
is typically not cleaned up automatically (for example, because the job hosting an
interactive HOD cluster submitted via \lstinline|hod create| runs out of walltime).

These HOD clusters will still show up in the output of \lstinline|hod list|, and will be marked as \lstinline|<job-not-found>|.

You should occasionally clean this up using \lstinline|hod clean|:

\begin{prompt}
%\shellcmd{module list}%
Currently Loaded Modulefiles:
  1) cluster/%\defaultcluster{}%(default)   2) pbs_python/4.6.0            3) vsc-base/2.4.2              4) hod/3.0.0-cli

%\shellcmd{hod list}%
Cluster label	Job ID				                State          	Hosts
example1		%\jobid{}%	<job-not-found>	<none>

%\shellcmd{hod clean}%
Removed cluster localworkdir directory /user/scratch/gent/vsc400/vsc40000/hod/hod/%\jobid{}% for cluster labeled example1
Removed cluster info directory /user/home/gent/vsc400/vsc40000/.config/hod.d/wordcount for cluster labeled example1

%\shellcmd{module swap cluster/\othercluster{}}%
$ hod list
Cluster label	Job ID				State          	Hosts
example2		98765.master19.%\othercluster{}%.gent.vsc	<job-not-found>	<none>

%\shellcmd{hod clean}%
Removed cluster localworkdir directory /user/scratch/gent/vsc400/vsc40000/hod/hod/98765.master19.%\othercluster{}%.gent.vsc for cluster labeled example2
Removed cluster info directory /user/home/gent/vsc400/vsc40000/.config/hod.d/wordcount for cluster labeled example2
\end{prompt}

Note that \strong{only HOD clusters that were submitted to the currently loaded \lstinline|cluster|
module will be cleaned up}.

\section{Getting help}

If you have any questions, or are experiencing problems using HOD, you have a couple of options:

\begin{itemize}
    \item Subscribe to the HOD mailing list via \url{https://lists.ugent.be/wws/info/hod},
        and contact the HOD users and developers at hod@lists.ugent.be.
    \item Contact the \hpcTeam via \hpcinfo
    \item Open an issue in the \lstinline|hanythingondemand| GitHub repository, via \url{https://github.com/hpcugent/hanythingondemand/issues/190}.
\end{itemize}
