\chapter{Filesystems}

In this section, we briefly explain which different filesystems are available on
the \gls{HPC} infrastructure.

See \href{\HPCManualURL#predefined-user-directories}{Chapter 6, section titled "Where to store your data on the \hpc{}" of the HPC manual}
for a list of available locations.

\ifgent

\subsubsection{VO storage}

If you're member of a (non-default) virtual organisation (VO), see
\href{\HPCManualURL#sec:virtual-organisation}{Chapter 6, section titled "Virtual Organisations"}, you have access to
additional directories (with more quota) on the data and \gls{scratch} filesystems,
which you can share with other members in the VO.

\fi

\subsection{Quota}

Space is limited on the cluster's storage. To check your quota, see
\href{\HPCManualURL#predefined-quotas}{Chapter 6, section titled "Pre-defined quota" of the HPC manual}.

To figure out where you're quota is being spent, the \lstinline|du| (\strong{d}isk \strong{u}sage)
command can come in useful:

\begin{prompt}
%\shellcmd{du -sh test}%
59M   test
\end{prompt}

Do \emph{not} (frequently) run \lstinline|du| on directories where large amounts
of data are stored, since that will:
\begin{enumerate}
    \item take a long time
    \item result in increased
        load on the shared storage since (the metadata of) every file in those directories
        will have to be inspected.
\end{enumerate}

\section{Exercises}

\begin{itemize}
    \item What's the full path to your personal home/data/scratch directories?
    \item Determine how large your personal directories are.
    \item What's the difference between the size reported by \lstinline|du -sh $HOME|
        and by \lstinline|ls -ld $HOME|?
\end{itemize}
