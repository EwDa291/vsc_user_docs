\chapter{Getting Started}

\section{Logging in}

To get started with the HPC-UGent infrastructure, you need to obtain a VSC
account, see http://hpc.ugent.be/userwiki/index.php/User:VscRequests .

\strong{Keep in mind that you must keep your private key to yourself!}

You can look at your public/private key pair as a lock and a key: you give us
the lock (your public key), we put it on the door, and then you can use your
key to open the door and get access to the HPC infrastructure. \strong{Anyone
who has your key can use your VSC account!}

Details on connecting to the HPC infrastructure are available at
http://hpc.ugent.be/userwiki/index.php/User:VscConnect .

\section{Getting help}

To get help:

\begin{enumerate}
  \item use the documentation available on the system, through the ``help'',
    ``info'' and ``man'' commands (use ``q'' to exit).

\begin{prompt}
  help cd
  info ls
  man cp
\end{prompt}

  \item use Google
  \item contact hpc@ugent.be in case of problems or questions (even for basic
    things!)
\end{enumerate}

\subsection{Errors}

Sometimes when executing a command, an error occurs. Most likely there will be
error output or a message explaining you this. Read this carefully and try to
act on it. Try googling the error first to find any possible solution, but if
you can't come up with something in 15 minutes, don't hesitate to mail
hpc@ugent.be.


\section{Basic terminal usage}

The basic interface is the so-called shell prompt, typically ending with ``$''
(for ``bash'' shells).

You use the shell by executing commands, and hitting ``<enter>''. For example:

\begin{prompt}
  %shellcmd{echo hello}%
  hello
\end{prompt}

You can go to the start or end of the command line using ``Ctrl-A'' or ``Ctrl-E''.

To go through previous commands, use ``<up>'' and ``<down>'', rather than retyping them.

\subsection{Command history}

A powerful feature is that you can ''search'' through your command history, either using the ``history'' command, or using Ctrl-R:

\begin{prompt}
  %\shellcmd{history}%
      1  echo hello

  # hit Ctrl-R, type 'echo'
  (reverse-i-search)`echo': echo hello
\end{prompt}

\subsection{Stopping commands}
If for any reason you want to stop a command from executing, press Ctrl-C. For
example, if a command is taking too long and you want to rerun it with a
different command.

\section{Variables}

At the prompt we also have access to shell variables, which have both a \emph{name} and a \emph{value}.

They can be thought of as placeholders for things we need to remember.

For example, to print the path to your home directory, we can use the shell variable named ``HOME'':

\begin{prompt}
  %\shellcmd{echo \$HOME}%
  /user/home/gent/vsc400/vsc40000
\end{prompt}

This prints the value of this variable.

\subsection{Defining variables}

There are several variables already defined for you when you start your
session, such as ``$HOME'' which contains the path to your home directory. 

But we can also define our own. this is done with the ``export'' command (note:
variables are always all-caps as a convention):

\begin{prompt} %shellcmd{export MYVARIABLE="value"}% \end{prompt}

It is important you don't include spaces around the \'``=''\' sign. Also note
the lack of \'``$''\' sign in front of the variable name.

If we then do

\begin{prompt}
  echo $MYVARIABLE
\end{prompt}

this will output ``value''. Note that the quotes are not included, they were only used when defining the variable to escape potential spaces in the value.

\subsubsection{Changing your prompt using ``$PS1''}

You can change what your prompt looks like by redefining the special-purpose variable ``$PS1''.

For example: to include the current location in your prompt:

\begin{prompt}
  %\shellcmd{export PS1='\w \$'}%
  ~ $ cd test
  ~/test $
\end{prompt}

Note that ``\tilde'' is short representation of your home directory.

To make this persistent across session, you can define this custom value for
``$PS1'' in your ``.profile'' startup script:

\begin{prompt}
%\shellcmd{echo 'export PS1="\w \$ " ' >> ~/.profile}%
\end{prompt}

\subsection{Using non-defined variables}

One common pitfall is the (accidental) use of non-defined variables.  Contrary
to what you may expect, this does \emph{not} result in error messages, but the
variable is considered to be \emph{empty} instead.

This may lead to surprising results, for example:

\begin{prompt}
%\shellcmd{export WORKDIR=/tmp/test}%
%\shellcmd{cd \$WROKDIR}%
%\shellcmd{pwd}%
  /user/home/gent/vsc400/vsc40000
%\shellcmd{echo \$HOME}%
  /user/home/gent/vsc400/vsc40000
\end{prompt}

To understand what's going on here, see the section on <code>cd</code> below.

The moral here is: \strong{be very careful to not use empty variables unintentionally}.

More information can be found at \url{http://www.tldp.org/LDP/abs/html/variables.html} .

\subsection{Restoring your default environment}

If you've made a mess of your environment, you shouldn't waste too much time
trying to fix it. Just log out and log in again and you will be given a
pristine environment.

\section{Basic system information}

Basic information about the system you are logged into can be obtained in a variety of ways.

We limit ourselves to determining the hostname:

\begin{prompt}
  %\shellcmd{hostname}%
  gligar01.gligar.os

  %\shellcmd{echo \$HOSTNAME}%
  gligar01.gligar.os
\end{prompt}

And quering some basic information about the Linux kernel:

\begin{prompt}
  %\shellcmd{uname -a}%
  Linux gligar01.gligar.os 2.6.32-573.8.1.el6.ug.x86_64 #1 SMP Mon Nov 16 15:12:09 CET 2015 x86_64 x86_64 x86_64 GNU/Linux
\end{prompt}

