\chapter{Filesystems}

In this section, we briefly explain which different filesystems are available on
the HPC infrastructure.

See \url{http://hpc.ugent.be/userwiki/index.php/Tips:Filesystem_Information} and
\url{http://hpc.ugent.be/userwiki/index.php/User:StorageDetails} for more
information.

\subsubsection{Home}

Your own personal home directory: ``\$VSC\_HOME''

You end up here when you log in (and when jobs start). This is meant for small
configuration files, limited space available.

\subsubsection{Data}

Long-term personal storage (for large files, volumes): ``\$VSC\_DATA''

\subsubsection{Scratch}

Personal scratch (fast, but to be considered volatile) storage:
``\$VSC\_SCRATCH''. ``\$VSC\_SCRATCH'' is just a short\-hand for the default
scratch storage (``\$VSC\_SCRATCH\_DELCATTY'').

\subsubsection{Local}

When running jobs, you also have access to the local storage of a node (which
you cannot access when logging in, as it is node-dependent).

You access it through ``\$VSC\_SCRATCH\_NODE''.

Inside jobs, a unique directory located in ``\$VSC\_SCRATCH\_NODE'' is made
available via ``\$TMPDIR''.

\subsubsection{VO storage}

If you're member of a (non-default) virtual organisation (VO), see
\url{http://hpc.ugent.be/userwiki/index.php/User:VSCVos}, you have access to
additional directories (with more quota) on the data and scratch filesystems,
which you can share with other members in the VO.

See \url{http://hpc.ugent.be/userwiki/index.php/User:StorageDetails} for more
information.

\subsection{Quota}

Space is limited on the cluster's storage, you can check your quota with the
``show\_quota'' command:

\begin{prompt}
%\shellcmd{show\_quota}%
VSC_HOME: used 1.89 GiB (66%\%%) quota 2.85 GiB (3 GiB hard limit)
VSC_DATA: used 0 B (0%\%%) quota 23.8 GiB (25 GiB hard limit)
VSC_SCRATCH_DELCATTY: used 0 B (0%\%%) quota 23.8 GiB (25 GiB hard limit)
\end{prompt}

Quota for personal directories is limited to 3GiB (home) and 25GiB (data \&
scratch); for VO directories, it depends on the particular VO.

To figure out where you're quota is being spent, the ``du'' command can come in
useful:

\begin{prompt}
%\shellcmd{du -sh test}%
59M   test
\end{prompt}

Note: running ``du'' can take a while in case you have a lot of files and
directories; use it with caution!
