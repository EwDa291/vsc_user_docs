\chapter{Troubleshooting}
\label{ch:troubleshooting}



\section{Walltime issues}
If you get from your job output an error message similar to this:

\begin{prompt}
 =>> PBS: job killed: walltime %\emph{<value in seconds>}% exceeded limit %\emph{<value in seconds>}%
\end{prompt}

This occurs when your job did not complete within the requested walltime.
See section~\ref{sec:specifying-walltime-requirements} for more information about how to request the walltime.
It is recommended to use \emph{checkpointing} if the job requires \strong{72 hours} of walltime or more to be executed.
% FIXME: Refer to Checkpointing section.



\section{Out of quota issues}

Sometimes a job hangs at some point or it stops writing in the disk. These errors are usually
related to the quota usage. You may have reached your quota limit at some storage endpoint.
You should move (or remove) the data to a different storage endpoint (or request more quota) to be able to write to the disk and then resubmit the jobs.
\ifgent
Another option is to request extra quota for your VO to the VO moderator/s.
See section~\ref{subsec:predefined-user-directories} and section~\ref{subsec:predfined-quotas} for more information about
quotas and how to use the storage endpoints in an efficient way.
\fi
% FIXME: Add how to request quota section

\section{DOS/Windows text format}

If you get errors like:

\begin{prompt}
%\shellcmd{qsub fibo.pbs}%
qsub:  script is written in DOS/Windows text format
\end{prompt}

It's probably because you transferred the files from a Windows computer.
Please go to the section about \verb|dos2unix| in \href{\LinuxManualURL#sec:dos2unix}{chapter 5 of the intro to Linux}
to fix this error.

\section{Warning message when first connecting to new host}
\label{sec:warning-message-new-host}

\ifmacORlinux
\begin{prompt}
%\shellcmd{ssh \userid{}@\loginnode{}}%
The authenticity of host %\loginnode% (<IP-adress>)
can't be established.
%\underline{<algorithm> key fingerprint is <hash>}%
Are you sure you want to continue connecting (yes/no)?
\end{prompt}

Now you can check the authenticity by checking if the line that is at the place
of the underlined piece of text matches one of the following lines:
\begin{prompt}
%\opensshFirstConnect%
\end{prompt}

If it does, type  \strong{\emph{yes}}. If it doesn't, please contact support: \hpcinfo.
\else
\firsttimeconnection
\fi


\section{Memory limits}

To avoid jobs allocating too much memory, there are memory limits in place by default.
It is possible to specify higher memory limits if your jobs require this.

\subsection{How will I know if memory limits are the cause of my problem?}

If your program fails with a memory-related issue, there is a good chance it failed
because of the memory limits and you should increase the memory limits for your job.

Examples of these error messages are: \verb|malloc failed|, \verb|Out of memory|,
\verb|Could not allocate memory|
or in Java: \verb|Could not reserve enough space for object heap|. Your program can
also run into a \verb|Segmentation fault| (or \verb|segfault|) or crash due to bus errors.

You can check the amount of virtual memory (in Kb) that is available to you via the
\verb|ulimit -v| command \emph{in your jobscript}.

\subsection{How do I specify the amount of memory I need?}

See \autoref{subsec:generic-resource-requirements} to set memory and other requirements,
see \autoref{sec:specifying-memory-requirements} to finetune the amount of memory you request.
% See issue #248 to fix Java. This is software specific.
% See issue #196 to fix MATLAB. This is software specific.
