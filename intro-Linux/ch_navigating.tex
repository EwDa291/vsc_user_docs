\chapter{Navigating}

\section{Current directory: ``pwd'' and ``\$PWD''}

To print the current directory, use \lstinline|pwd| or \lstinline|$PWD|:

\begin{prompt}
%\shellcmd{cd \$HOME}%
%\shellcmd{pwd}%
%\homedir{}%
%\shellcmd{echo "The current directory is: \$PWD"}%
The current directory is: %\homedir{}%
\end{prompt}

\section{Listing files and directories: ``ls''}

A very basic and commonly used command is \lstinline|ls|, which can be used to list files and directories.

In it's basic usage, it just prints the names of files and directories in the current directory. For example:

\begin{prompt}
%\shellcmd{ls}%
afile.txt   some_directory
\end{prompt}

When provided an argument, it can be used to list the contents of a directory:

\begin{prompt}
%\shellcmd{ls some\_directory}%
one.txt  two.txt
\end{prompt}

A couple of commonly used options include:

\begin{itemize}
\item detailed listing using \lstinline|ls -l|:

\begin{prompt}
%\shellcmd{ls -l}%
total 4224
-rw-rw-r-- 1 %\userid{}% %\userid{}% 2157404 Apr 12 13:17 afile.txt
drwxrwxr-x 2 %\userid{}% %\userid{}%     512 Apr 12 12:51 some_directory
\end{prompt}

\item To print the size information in human-readable form, use the \lstinline|-h| flag:

\begin{prompt}
%\shellcmd{ls -lh}%
total 4.1M
-rw-rw-r-- 1 %\userid{}% %\userid{}% 2.1M Apr 12 13:16 afile.txt
drwxrwxr-x 2 %\userid{}% %\userid{}%  512 Apr 12 12:51 some_directory
\end{prompt}

\item also listing hidden files using the \lstinline|-a| flag:

\begin{prompt}
%\shellcmd{ls -lah}%
total 3.9M
drwxrwxr-x   3 %\userid{}% %\userid{}%  512 Apr 12 13:11 .
drwx------ 188 %\userid{}% %\userid{}% 128K Apr 12 12:41 ..
-rw-rw-r--   1 %\userid{}% %\userid{}% 1.8M Apr 12 13:12 afile.txt
-rw-rw-r--   1 %\userid{}% %\userid{}%    0 Apr 12 13:11 .hidden_file.txt
drwxrwxr-x   2 %\userid{}% %\userid{}%  512 Apr 12 12:51 some_directory
\end{prompt}

\item ordering files by the most recent change using \lstinline|-rt|:

\begin{prompt}
%\shellcmd{ls -lrth}%
total 4.0M
drwxrwxr-x 2 %\userid{}% %\userid{}%  512 Apr 12 12:51 some_directory
-rw-rw-r-- 1 %\userid{}% %\userid{}% 2.0M Apr 12 13:15 afile.txt
\end{prompt}

\end{itemize}

If you try to use \lstinline|ls| on a file that doesn't exist, you will get a clear error message:

\begin{prompt}
%\shellcmd{ls nosuchfile}%
ls: cannot access nosuchfile: No such file or directory
\end{prompt}

\section{Changing directory: ``cd''}

To change to a different directory, you can use the \lstinline|cd| command:

\begin{prompt}
%\shellcmd{cd some\_directory}%
\end{prompt}

To change back to the previous directory you were in, there's a shortcut: \lstinline|cd -|

Using \lstinline|cd| without an argument results in returning back to your home directory:

\begin{prompt}
%\shellcmd{cd}%
%\shellcmd{pwd}%
%\homedir{}%
\end{prompt}

\section{Inspecting file type: ``file''}

The \lstinline|file| command can be used to inspect what type of file you're dealing with:

\begin{prompt}
%\shellcmd{file afile.txt}%
afile.txt: ASCII text

%\shellcmd{file some\_directory}%
some_directory: directory
\end{prompt}

\section{Absolute vs relative file paths}

An \emph{absolute} filepath starts with \lstinline|/| (or a variable which value starts
with  \lstinline|/|), which is also called the \emph{root} of the filesystem.

Example: absolute path to your home directory:
\texttt{\homedir}.

A \emph{relative} path starts from the current directory, and points to another
location up or down the filesystem hierarchy.

Example: \lstinline|some_directory/one.txt| points to the file \lstinline|one.txt| that is
located in the subdirectory named \lstinline|some_directory| of the current directory.

There are two special relative paths worth mentioning:

\begin{itemize}
    \item \lstinline|.| is a shorthand for the current directory
    \item \lstinline|..| is a shorthand for the parent of the current directory
\end{itemize}

You can also use \lstinline|..| when constructing relative paths, for example:

\begin{prompt}
%\shellcmd{cd \$HOME/some\_directory}%
%\shellcmd{ls ../afile.txt}%
../afile.txt
\end{prompt}

\section{Permissions}
\label{sec:permissions}

Each file and directory has particular \emph{permissions} set on it, which can be queried using \lstinline|ls -l|.

For example:

\begin{prompt}
%\shellcmd{ls -l afile.txt}%
-rw-rw-r-- 1 %\userid{}% agroup 2929176 Apr 12 13:29 afile.txt
\end{prompt}

The \lstinline|-rwxrw-r--| specifies both the type of file (\lstinline|-| for files, \lstinline|d|
for directories (see first character)), and the permissions for user/group/others:

\begin{enumerate}
\item each triple of characters indicates whether the read (\lstinline|r|), write (\lstinline|w|), execute (\lstinline|x|) permission bits are set or not
\item the 1st part \lstinline|rwx| indicates that the \emph{owner} ``\userid'' of the file has all the rights
\item the 2nd part \lstinline|rw-| indicates the members of the \emph{group} ``agroup'' only have read/write permissions (not execute)
\item the 3rd part \lstinline|r--| indicates that \emph{other} users only have read permissions
\end{enumerate}

The default permission settings for new files/directories are determined by the so-called \emph{umask} setting, and are by default:

\begin{enumerate}
\item read-write permission on files for user/group (no execute), read-only for others (no write/execute)
\item read-write-execute permission for directories on user/group, read/execute-only for others (no write)
\end{enumerate}

See also \hyperref[sec:chmod]{the chmod command} later in this manual.

\section{Finding files/directories: ``find''}

\lstinline|find| will crawl a series of directories and lists files matching given criteria.

For example, to look for the file named \lstinline|one.txt|:

\begin{prompt}
%\shellcmd{cd \$HOME}%
%\shellcmd{find . -name one.txt}%
./some_directory/one.txt
\end{prompt}

To look for files using incomplete names, you can use a wildcard \lstinline|*|; note
that you need to escape the \lstinline|*| to avoid that Bash \emph{expands} it into
\lstinline|afile.txt| by adding double quotes:

\begin{prompt}
%\shellcmd{find . -name "*.txt"}%
./.hidden_file.txt
./afile.txt
./some_directory/one.txt
./some_directory/two.txt
\end{prompt}

A more advanced use of the \lstinline|find| command is to use the \lstinline|-exec| flag
to perform actions on the found file(s), rather than just printing their paths (see \lstinline|man find|).

\section{Exercises}

\begin{itemize}
    \item Go to \lstinline|/tmp|, then back to your home directory.
        How many different ways to do this can you come up with?
    \item When was your home directory created or last changed?
    \item Determine the name of the last changed file in \lstinline|/tmp|.
    \item See how home directories are organised.
        Can you access the home directory of other users?
\end{itemize}
