\chapter{Manipulating files and directories}

Being able to manage your data is an important part of using the HPC
infrastructure.  The bread and butter commands for doing this are mentioned
here.  It might seem annoyingly terse at first, but with practice you will
realise that it's very practical to have such common commands short to type.

\section{File contents: ``cat'', ``head'', ``tail'', ``less'', ``more''}

To print the contents of an entire file, you can use ``cat''; to only see the
first or last N lines, you can use ``head'' or ``tail'':

\begin{prompt}
%\shellcmd{cat one.txt}%
1
2
3
4
5

%\shellcmd{head -n 2 one.txt}%
1
2

%\shellcmd{tail -2 one.txt}%
4
5
\end{prompt}

To check the contents of long text files, you can use the ``less'' or ``more''
commands which support scrolling with ``<up>'', ``<down>'', ``<space>'', etc.

\section{Copying files: ``cp''}

\begin{prompt}
%\shellcmd{cp source target}%
\end{prompt}

This is the ``cp'' command, which copies a file from source to target. To copy a directory, we use the ``-r'' option:

\begin{prompt}
%\shellcmd{cp -r sourceDirectory target}%
\end{prompt}

A last more complicated example:
\begin{prompt}
%\shellcmd{cp -a sourceDirectory target}%
\end{prompt}

Here we used the same cp command, but instead we gave it the ``-a'' option
which tells cp to copy all the files and keep timestamps and permissions.

\section{Creating directories: ``mkdir''}

\begin{prompt}
%\shellcmd{mkdir directory}%
\end{prompt}

which will create a directory with the given name inside the current directory.

\section{Renaming/moving files: ``mv''}

\begin{prompt}
%\shellcmd{mv source target}%
\end{prompt}

``mv'' will move the source path to the destination path. Works for both directories as files.

\section{Removing files: ``rm''}

\strong{Note: there are NO backups, there is no 'trash bin'. If you
remove files/directories, they are gone.}

\begin{prompt}
%\shellcmd{ rm filename}%
\end{prompt}

``rm'' will remove a file or directory. (``rm -rf'' directory will remove every
file inside a given directory). WARNING: files removed will be lost forever,
there are no backups, so beware when using this command!

\subsubsection{Removing a directory: ``rmdir''}

You can remove directories using ``rm -r directory'', however, this is error
prone and can ruin your day if you make a mistake in typing. To prevent this
type of error, you can remove the contents of a directory using ``rm''
and then finally removing the directory with:

\begin{prompt}
%\shellcmd{rmdir directory}%
\end{prompt}

\section{Changing permissions: ``chmod''}

Every file, directory, and link has a set of permissions. These permissions
consist of permission groups and permission types. The permission groups are:

\begin{enumerate}
\item User - a particular user (account)
\item Group - a particular group of users (may be user-specific group with only one member)
\item Other - other users in the system
\end{enumerate}

The permission types are:

\begin{enumerate}
\item Read - For files, this gives permission to read the contents of a file
\item Write - For files, this gives permission to write data to the file. For directories it allows users to add or remove files to a directory.
\item Execute - For files this gives permission to execute a file as through it were a script. For directories, it allows users to open the directory and look at the contents.
\end{enumerate}

Any time you run ``ls -l'' you'll see a familiar line of ``rwx------'' or similar combination of the letters ``r'', ``w'', ``x'' and ``-'' (dashes). These are the permissions for the file or directory.

\begin{prompt}
%\shellcmd{ls -l}%
total 1
-rw-r--r--. 1 %\userid% mygroup 4283648 Apr 12 15:13 articleTable.csv
drwxr-x---. 2 %\userid% mygroup      40 Apr 12 15:00 Project_GoldenDragon
\end{prompt}

Here, we see that ``articleTable.csv'' is a file (beginning the line with ``-'')
has read and write permission for the user ``%\userid%''(``rw-''), and read
permission for the group ``mygroup'' as well as all other users (``r--'' and
``r--'').

The next entry is ``Project\_GoldenDragon''. We see it is a directory because the
line begins with a ``d''. It also has read, write, and execute permission for
the ``%\userid%'' user (``rwx''). So that user can look into the directory and add
or remove files. Users in the ``mygroup'' can also look into the directory and
read the files. But they can't add or remove files (``r-x''). Finally, other
users can read files in the directory, but other users have no permissions to
look in the directory at all (``---'').

Maybe we have a colleague who wants to be able to add files to the directory. We
use ``chmod'' to change the modifiers to the directory to let people in the
group write to the directory:

\begin{prompt}
%\shellcmd{chmod g+w Project\_GoldenDragon}%
%\shellcmd{ls -l}%
total 1
-rw-r--r--. 1 %\userid% mygroup 4283648 Apr 12 15:13 articleTable.csv
drwxrwx---. 2 %\userid% mygroup      40 Apr 12 15:00 Project_GoldenDragon
\end{prompt}

The syntax used here is ``g+x'' which means \strong{g}roup was given \strong{w}rite
permission. To revoke it again, we use ``g-w''. The other roles are ``u'' for
user and ``o'' for other.

You can put multiple changes on the same line: ``chmod o-rwx,g-rxw,u+rx,u-w
somefile'' will take everyone's permission away except the user's ability to
read or execute the file.

You can also use the ``-R'' to affect all the files within a directory but this
is dangerous. It's best to refine your search using ``find'' and then pass the
resulting list to ``chmod'' since it's not usual for all files in a directory
structure to have the same permissions.

\subsection{Access control lists (ACLs)}

However, this means that all users in ``mygroup'' can add or remove files. This
could be problematic if you only wanted one person to be allowed to help you
administer the files in the project. We need a new group. To do this in the HPC
environment, we need to use access control lists (ACLs):

\begin{prompt}
%\shellcmd{setfacl -m u:otheruser:w Project\_GoldenDragon}%
%\shellcmd{ls -l Project\_GoldenDragon}%
drwxr-x---+ 2 %\userid% mygroup      40 Apr 12 15:00 Project_GoldenDragon
\end{prompt}

Now there is a ``+'' at the end of the line. This means there is an ACL attached
to the directory. ``getfacl Project\_GoldenDragon'' will print the ACLs for the
directory.

Note: most people don't use ACLs, but it's sometimes the right thing and you
should be aware it exists.

See \url{http://linuxcommand.org/lc3_lts0090.php} for more information.

\section{Zipping: ``gzip''/``gunzip'', ``zip''/``unzip''}

Files should usually be stored in a compressed file if they're not being used
frequently. This means they will use less space and thus you get more out of
your quota. Some types of files (e.g. CSV files with a lot of numbers) compress
as much as 9:1. The most commonly used compression format on Linux is gzip. To
compress a file using gzip, we use:

\begin{prompt}
    %\shellcmd{ls -lh myfile}%
    -rw-r--r--. 1 %\userid% %\userid% 4.1M Dec  2 11:14 myfile
    %\shellcmd{gzip myfile}%
    %\shellcmd{ls -lh myfile.gz}%
    -rw-r--r--. 1 %\userid% %\userid% 1.1M Dec  2 11:14 myfile.gz
\end{prompt}


To compress the file again, use gzip:

\begin{prompt}
    %\shellcmd{gzip myfile}%
\end{prompt}

Note: if you zip a file, the original file will be removed. If you unzip a file,
the compressed file will be removed. To keep both, we send the data to
``stdout'' and redirect it to the target file:

\begin{prompt}
    %\shellcmd{gzip -c myfile $>$ myfile.gz}%
    %\shellcmd{gunzip -c myfile.gz $>$ myfile}%
\end{prompt}

\subsection{``zip'' and ``unzip''}

Windows and OS X seem to favour the zip file format, so it's also important to
know how to unpack those. We do this using unzip:

\begin{prompt}
    %\shellcmd{unzip myfile.zip}%
\end{prompt}

If we would like to make our own zip archive, we use zip:

\begin{prompt}
    %\shellcmd{zip myfiles.zip myfile1 myfile2 myfile3}%
\end{prompt}

\section{Working with tarballs: ``tar''}

Tar stands for "tape archive" and is a way to bundle files together in a bigger
file.

You will normally want to unpack these files more often than you make them. To
unpack a ``tar'' file you use:

\begin{prompt}
    %\shellcmd{tar -xf tarfile.tar}%
\end{prompt}

Often, you will find ``gzip'' compressed tar files on the web. These are called
tarballs. You can uncompress these using gunzip and then unpacking them using
tar. But tar knows how to open them using the ``-z'' option:

\begin{prompt}
    %\shellcmd{tar -zxf tarfile.tar.gz}%
    %\shellcmd{tar -zxf tarfile.tgz}%
\end{prompt}

\subsection{Order of arguments}

Note: Archive programs like ``zip'', ``tar'', and ``jar'' use arguments in the
''opposite direction'' of copy commands.

\begin{prompt}
    # cp, ln: <source(s)> <target>
    %\shellcmd{cp source1 source2 source3 target}%
    %\shellcmd{ln -s source target}%

    # zip, tar: <target> <source(s)>
    %\shellcmd{zip zipfile.zip source1 source2 source3}%
    %\shellcmd{tar -cf tarfile.tar source1 source2 source3}%
\end{prompt}

If you use tar with the source files first then the first file will be
overwritten. You can control the order of arguments of ``tar'' if it helps you
remember:

\begin{prompt}
    %\shellcmd{tar -c source1 source2 source3 -f tarfile.tar}%
\end{prompt}
