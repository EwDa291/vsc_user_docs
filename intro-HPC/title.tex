 \begin{center}

\includegraphics*[width=8truecm]{logo_vsc}

\vspace*{1.5\baselineskip}

\Huge \strong{HPC Tutorial} \\
<<<<<<< HEAD
\LARGE Version 20180904.05
=======
\LARGE Version 20180904.05
>>>>>>> 9049c6e64cab6f3721d09e72b456cf3f12a1262c

\ifwindows
\LARGE For Windows Users
\fi
\ifmac
\LARGE For macOS Users
\fi
\iflinux
\LARGE For Linux Users
\fi

\vspace*{.75\baselineskip}
\ifantwerpen
\includegraphics*[width=7truecm]{logo_ua}
\fi

\vspace*{0.75\baselineskip}


\normalsize\strong{Author:}

Geert Borstlap (UAntwerpen)

\vspace*{.5\baselineskip}

\strong{Co-authors:}

Kenneth Hoste, Toon Willems, Jens Timmerman (UGent)

Geert-Jan Bex (UHasselt and KU~Leuven)

Mag Selwa (KU~Leuven)

Stefan Becuwe, Franky Backeljauw, Bert Tijskens, Kurt Lust (UAntwerpen)

Bal\'azs Hajgat\'o (VUB)

\vspace*{.5\baselineskip}

Acknowledgement: VSCentrum.be

\vspace*{\baselineskip}

\begin{tabular}{ >{\centering\arraybackslash}m{0.25\textwidth}  >{\centering\arraybackslash}m{0.05\textwidth}  >{\centering\arraybackslash}m{0.20\textwidth}  >{\centering\arraybackslash}m{0.2\textwidth}} \\
\includegraphics*[width=0.2\textwidth, height=0.7in, keepaspectratio=true]{logo_auha} & \multicolumn{2}{ >{\centering\arraybackslash}m{.2\textwidth} }{\includegraphics*[width=0.2\textwidth, height=0.7in,, keepaspectratio=true]{logo_akuleuven}} & \includegraphics*[width=0.2\textwidth, height=0.7in,, keepaspectratio=true]{logo_auhl} \\
\multicolumn{2}{ >{\centering\arraybackslash}m{.32\textwidth} }{\includegraphics*[width=0.3\textwidth, height=0.7in, keepaspectratio=true]{logo_augent}} & \multicolumn{2}{ >{\centering\arraybackslash}m{.38\textwidth} }{\includegraphics*[width=0.3\textwidth, height=0.7in, keepaspectratio=false]{logo_uab}} \\
\end{tabular}
\end{center}


\cleardoublepage
\pagestyle{plain}
\strong{Audience:}

This HPC Tutorial is designed for \strong{researchers} at the
\strong{\university} and affiliated institutes who are in need of
computational power (computer resources) and wish to explore and use the High
Performance Computing (HPC) core facilities of the Flemish Supercomputing Centre (VSC)
to execute their computationally intensive tasks.


The audience may be completely unaware of the \hpc concepts but must have some
basic understanding of computers and computer programming.


\strong{\underbar{Contents:}}

This \strong{Beginners Part} of this tutorial gives answers to the typical
questions that a new \hpc user has. The aim is to learn how to make use of the
HPC.

\begin{tabular}{|p{\dimexpr 0.44\textwidth-2\tabcolsep}|>{\centering\arraybackslash}p{\dimexpr 0.12\textwidth-2\tabcolsep}|p{\dimexpr 0.44\textwidth-2\tabcolsep}|} \hline
\multicolumn{3}{|c|}{\strong{Beginners Part}} \\ \hline
\strong{Questions}                                                      & \strong{chapter} & \strong{title} \\ \hline
What is a \hpc exactly?\newline Can it solve my computational needs?    & \strong{\ref{ch:introduction-to-hpc}} & \nameref{ch:introduction-to-hpc} \\ \hline
How to get an account?                                                  & \strong{\ref{ch:getting-a-hpc-account}} & \nameref{ch:getting-a-hpc-account} \\ \hline
How do I connect to the \hpc and transfer my files and programs?        & \strong{\ref{ch:preparing-the-environment}} & \nameref{ch:preparing-the-environment} \\ \hline
How to start background jobs?                                           & \strong{\ref{ch:running-batch-jobs}} & \nameref{ch:running-batch-jobs} \\ \hline
How to start jobs with user interaction?                                & \strong{\ref{ch:running-interactive-jobs}} & \nameref{ch:running-interactive-jobs} \\ \hline
Where do the input and output go? Where to collect my results?          & \strong{\ref{ch:running-jobs-with-input-output-data}} & \nameref{ch:running-jobs-with-input-output-data} \\ \hline
Can I speed up my program by exploring parallel programming techniques? & \strong{\ref{ch:multi-core-jobs-parallel-computing}} & \nameref{ch:multi-core-jobs-parallel-computing} \\ \hline
Can I start many jobs at once?                                          & \strong{\ref{ch:multi-job-submission}} & \nameref{ch:multi-job-submission} \\ \hline
What are the rules and priorities of jobs?                              & \strong{\ref{ch:hpc-policies}} & \nameref{ch:hpc-policies} \\ \hline
\end{tabular}

The \strong{Advanced Part} focuses on in-depth issues. The aim is to assist the
end-users in running their own software on the \hpc.

\begin{tabular}{|p{\dimexpr 0.44\textwidth-2\tabcolsep}|>{\centering\arraybackslash}p{\dimexpr 0.12\textwidth-2\tabcolsep}|p{\dimexpr 0.44\textwidth-2\tabcolsep}|} \hline
\multicolumn{3}{|c|}{\strong{Advanced Part}} \\ \hline
\strong{Questions}                           & \strong{chapter} & \strong{title} \\ \hline
Can I compile my software on the \hpc?       & \strong{\ref{ch:compiling-and-testing-your-software-on-the-hpc}}      & \nameref{ch:compiling-and-testing-your-software-on-the-hpc} \\ \hline
What are the optimal Job Specifications?     & \strong{\ref{ch:fine-tuning-job-specifications}}      & \nameref{ch:fine-tuning-job-specifications} \\ \hline
%How do I check my programs at runtime?       & \strong{\ref{ch:monitoring-cluster-utilization}}      & \nameref{ch:monitoring-cluster-utilization} \\ \hline
%Can I stop my program and continue later on? & \strong{\ref{ch:checkpointing}}      & Checkpointing \\ \hline
Do you have more examples for me?            & \strong{\ref{ch:program-examples}}      & \nameref{ch:program-examples} \\ \hline
Any more advice?                             & \strong{\ref{ch:best-practices}}      & \nameref{ch:best-practices} \\ \hline
\end{tabular}

The \strong{Annexes} contains some useful reference guides.

\begin{tabular}{|l|c|} \hline
\multicolumn{2}{|c|}{\strong{Annex}} \\ \hline
\strong{Title}             & \strong{chapter} \\ \hline
\nameref{ch:quick-reference-guide} & \strong{\ref{ch:quick-reference-guide}} \\ \hline
\nameref{ch:torque-options}& \strong{\ref{ch:torque-options}} \\ \hline
\nameref{ch:useful-linux-commands}& \strong{\ref{ch:useful-linux-commands}} \\ \hline
\end{tabular}


\strong{\underbar{Notification:}}

In this tutorial specific commands are separated from the accompanying text:

\begin{prompt}
%\shellcmd{commands}%
\end{prompt}

These should be entered by the reader at a command line in a Terminal on the \hpcInfra. They appear in all exercises preceded by a \$ and printed in \textbf{bold}. You'll find those actions in a grey frame.

\keys{Button} are menus, buttons or drop down boxes to be pressed or selected.

``Directory'' is the notation for directories (called ``folders'' in
Windows terminology) or specific files. (e.g., ``\homedir'')

``Text'' Is the notation for text to be entered.

\begin{tip}
A ``Tip'' paragraph is used for remarks or tips.
\end{tip}

\strong{\underbar{More support:}}

Before starting the course, the example programs and configuration files used in this Tutorial must be copied to your home directory, so that you can work with your personal copy. If you have received a new VSC-account, all examples are present in an ``\examplesdir'' directory.

\begin{prompt}
%\shellcmd{cp --r \examplesdir \tilde/}%
%\shellcmd{cd}%
%\shellcmd{ls}%
\end{prompt}

They can also be downloaded from the VSC website at
\url{https://www.vscentrum.be}.
Apart from this \hpc Tutorial, the documentation on the VSC website
will serve as a reference for all the
operations.


\begin{tip}
The users are advised to get self-organised. There are
only limited resources available at the \hpc, which are best effort based.
The \hpc cannot give support for code fixing, the user applications and own
developed software remain solely the responsibility of the end-user.
\end{tip}

More documentation can be found at:

\begin{enumerate}
  \item  VSC documentation: \url{https://www.vscentrum.be/en/user-portal}
  \ifantwerpen
    \item CalcUA Core Facility web pages: \url{https://www.uantwerpen.be/hpc}
  \fi
  \ifbrussel
    \item \hpcname documentation: \url{http://cc.ulb.ac.be/hpc}
  \fi
  \ifgent
    \item \hpcname documentation: \url{http://hpc.ugent.be/userwiki}
  \fi
  \item  External documentation (TORQUE, Moab): \url{http://docs.adaptivecomputing.com}
\end{enumerate}

This tutorial is intended for users who want to connect and work on the HPC of the \strong{\university}.

This tutorial is available in a Windows, Mac or Linux version.

This tutorial is available for UAntwerpen, UGent, KU~Leuven, UHasselt and VUB users.

Request your appropriate version at \hpcinfo.

\strong{\underbar{Contact Information:}}

We welcome your feedback, comments and suggestions for improving the \hpc
Tutorial  (contact: \hpcinfo).

For all technical questions, please contact the \hpc staff:

\inputsite{contact-information}
