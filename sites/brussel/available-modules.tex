\begin{prompt}
%\shellcmd{module av 2>\&1 | more}%
------------------- /apps/brussel/shanghai/modules/2014b/all -------------------
annotate/1.42.1-intel-2014b-R-3.1.1
ant/1.9.3-Java-1.7.0_60
Autoconf/2.69-intel-2014b
biomaRt/2.20.0-intel-2014b-R-3.1.1
BioPerl/1.6.924-intel-2014b-Perl-5.20.0
Bio-SamTools/1.39-intel-2014b-Perl-5.20.0
BISICLES/2558-intel-2014b-Chombo-3.2-21908
%\dots{}%
\end{prompt}

``module av'' is an abbreviation for ``module available''.

Or when you want to check whether some specific software, some compiler or some
application (e.g., netCDF) is installed on the \hpc.

\begin{prompt}
%\shellcmd{module av 2>\&1 | grep -i -e "netcdf"}%
netCDF/4.3.2-intel-2014b
netCDF-C++4/4.2.1-intel-2014b
netCDF-Fortran/4.4.0-intel-2014b
\end{prompt}

As you are not aware of the capitals letters in the module name, we looked for
a case-insensitive name with the ``-i'' option.
