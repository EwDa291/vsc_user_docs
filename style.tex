\usepackage[english]{babel}
\usepackage{amssymb}
\usepackage{amsmath}
\usepackage{hyperref}
% allow conditional compilation
\usepackage{etoolbox}
\usepackage[toc,nonumberlist]{glossaries}
\usepackage{makeidx}

\usepackage{color}
\usepackage{listings}
\usepackage{tikz}
\usepackage[T1]{fontenc}
\usepackage{graphicx}
\usepackage{array}
\usepackage{multirow}
\usepackage{substr}
\usepackage{xstring}
\usepackage{verbatim}
\usepackage{courier}
\usepackage{xparse}
\usepackage{menukeys}
\renewcommand{\ttdefault}{pcr}

\setlength{\hoffset}{-1in}
\setlength{\voffset}{-1in}
\setlength{\topmargin}{2cm}
\setlength{\headheight}{0.5cm}
\setlength{\headsep}{1cm}
\setlength{\oddsidemargin}{2cm}
\setlength{\textwidth}{16cm}
\setlength{\textheight}{23.3cm}
\setlength{\footskip}{1.5cm}
\usepackage{fancyhdr}

\pagestyle{fancy}

\newcommand{\headerfmt}[1]{\textsl{\textsf{#1}}}
\newcommand{\headerfmtpage}[1]{\textsf{#1}}

\fancyhf{}
\fancyhead[L,RO]{\headerfmtpage{\thepage}}
\fancyhead[LO]{\headerfmt{\rightmark}}
\fancyhead[R]{\headerfmt{\leftmark}}
\fancyfoot[C]{\thepage}
\renewcommand{\headrulewidth}{0.5pt}
\renewcommand{\footrulewidth}{0pt}

\fancypagestyle{plain}{
\fancyhf{}
\fancyhead[L,RO]{\headerfmtpage{\thepage}}
\fancyhead[LO]{\headerfmt{\rightmark}}
\fancyhead[R]{\headerfmt{\leftmark}}
\fancyfoot[C]{\thepage}
\renewcommand{\headrulewidth}{0.5pt}
\renewcommand{\footrulewidth}{0pt}
}

\lstset{basicstyle=\ttfamily,breaklines=true, keepspaces=true}
\lstdefinestyle{prompt}{
% language=bash,
  frame=tblr,
  columns=fullflexible,
  escapechar=\%,
  fillcolor=\color{lightgray},
  backgroundcolor=\color{lightgray},
  }
\lstdefinestyle{code}{
  numbers=left,
  belowcaptionskip=1\baselineskip,
  breaklines=true,
  frame=tlbr,
  xleftmargin=\parindent,
  showstringspaces=false,
  basicstyle=\footnotesize\ttfamily,
  keywordstyle=\bfseries\color{green!40!black},
  commentstyle=\ttfamily\color{purple!40!black},
  identifierstyle=\color{blue},
  stringstyle=\color{orange},
}
\lstdefinestyle{flattext}{
  frame=tblr,
  breaklines=true,
  columns=fullflexible,
  escapechar=\%,
  fillcolor=\color{lightgray},
  backgroundcolor=\color{white},
  }

\lstnewenvironment{prompt} {
    \\[\baselineskip]
    \minipage{\linewidth}
    \lstset{style=prompt}
} {
   \endminipage
}
\lstnewenvironment{code}[1]{\lstset{style=code, language=#1}} {}
\lstnewenvironment{prog}{\lstset{style=code}} {}

\lstnewenvironment{flattext} {
    \\[\baselineskip]
    \minipage{\linewidth}
    \lstset{style=flattext}
} {
   \endminipage
}
% Taken from: http://tex.stackexchange.com/questions/9363/how-does-one-insert-a-backslash-or-a-tilde-into-latex
\protected\def\tilde{\raise.17ex\hbox{$\scriptstyle\mathtt{\sim}$}}
\newcommand{\shellcmd}[1]{\noexpandarg\StrSubstitute{#1}{~}{\tilde}[\temp]\$ \textbf{\texttt{\temp}}}

\newwrite\shellcmds
\immediate\openout\shellcmds=shellcmds.sh

\newcommand{\captureshellcmd}[1]{\noexpandarg\StrSubstitute{#1}{~}{\string~}[\wr]\immediate\write\shellcmds{\wr > \Chaptername-\the\inputlineno.out}\shellcmd{#1}}

% LaTeX hacks to allow underscores in filenames. Adapted from
% http://tex.stackexchange.com/questions/69142/include-figure-from-macro-with-underscore-in-filename
\newcommand{\examplecodeStub}[2]{
  \\[\baselineskip]
  \minipage{\linewidth}
  \lstinputlisting[
    language=#1,
    numbers=left,
    title=\IfSubStringInString{/}{#2}{\BehindSubString{/}{#2}} --- {#2} --- ,
    frame=tlbr,
    style=code,]
    {\exampledir/#2}
  \endminipage
    \endgroup

}
\def\examplecode{
  \begingroup
  \catcode`\_=12
  \examplecodeStub
}

\newcommand{\exampleoutputStub}[1]{
  \lstinputlisting[
    language=,
    style=prompt,
    escapechar=,
    numbers=none,
    ]
    {\exampledir/#1}
    \endgroup
}
\def\exampleoutput{\begingroup
  \catcode`\_=12
\exampleoutputStub}

\newcommand{\exampledir}{examples/\Chaptername}

% http://tex.stackexchange.com/questions/62241/how-to-get-the-current-chapter-name-section-name-subsection-name-etc
% getting the chapter name automatically
\renewcommand{\sectionmark}[1]{\markboth{}{\thesection~#1}}
\let\Chaptermark\chaptermark
\def\chaptermark#1{
  \markboth{#1}{}
  \StrSubstitute{#1}{ }{-}[\x]
  \StrSubstitute{\x}{/}{-}[\Chaptername]
  \Chaptermark{#1}}


\ExplSyntaxOn
    \NewDocumentEnvironment{tip} { o }
     {
      \par\noindent
      \IfNoValueTF{#1} { \textbf{Tip:~} }{ \textbf{Tip:~#1:~} }
       \ignorespaces
     }
     {}
\ExplSyntaxOff


\parindent=0pt
\parskip=7pt
\newcommand{\ignore}[1]{}

\newif\ifremark
\long\def\remark#1{
  \ifremark%
  \begingroup%
  \dimen0=\columnwidth
  \advance\dimen0 by -1in%
  \setbox0=\hbox{\parbox[b]{\dimen0}{\protect\em #1}}
  \dimen1=\ht0\advance\dimen1 by 2pt%
  \dimen2=\dp0\advance\dimen2 by 2pt%
  \vskip 0.25pt%
  \hbox to \columnwidth{%
    \vrule height\dimen1 width 3pt depth\dimen2%
    \hss\copy0\hss%
    \vrule height\dimen1 width 3pt depth\dimen2%
  }%
  \endgroup%
\fi}

% custom \strong declaration
% http://tex.stackexchange.com/questions/14667/does-latex-define-a-semantic-equivalent-of-textbf
\makeatletter
\newcommand{\strong}[1]{\@strong{#1}}
\newcommand{\@@strong}[1]{\textbf{\let\@strong\@@@strong#1}}
\newcommand{\@@@strong}[1]{\textnormal{\let\@strong\@@strong#1}}
\let\@strong\@@strong
\makeatother

\remarktrue

\newcommand{\includesite}[1]{
  \ifleuven
    \include{leuven/#1}
  \fi
  \ifantwerpen
    \include{antwerpen/#1}
  \fi
  \ifhasselt
    \include{hasselt/#1}
  \fi
  \ifbrussel
    \include{brussel/#1}
  \fi
  \ifgent
    \include{gent/#1}
  \fi
  \iftacc
    \include{tacc/#1}
  \fi
}
\newcommand{\inputsite}[1]{
  \ifleuven
    \input{leuven/#1}
  \fi
  \ifantwerpen
    \input{antwerpen/#1}
  \fi
  \ifhasselt
    \input{hasselt/#1}
  \fi
  \ifbrussel
    \input{brussel/#1}
  \fi
  \ifgent
    \input{gent/#1}
  \fi
  \iftacc
    \input{tacc/#1}
  \fi
}


\newif\ifwindows
\newif\iflinux
\newif\ifmac
\newif\ifmacORlinux
\newif\ifleuven
\newif\ifhasselt
\newif\ifleuvenORhasselt
\newif\ifbrussel
\newif\ifgent
\newif\ifantwerpen
\newif\iftacc
\newif\ifvsc

\windowsfalse
\linuxfalse
\macfalse
\leuvenfalse
\hasseltfalse
\brusselfalse
\gentfalse
\antwerpenfalse
\taccfalse
\vscfalse

\ifdefined\iswindows
  \windowstrue
\fi
\ifdefined\islinux
  \linuxtrue
  \macORlinuxtrue
\fi
\ifdefined\ismac
  \mactrue
  \macORlinuxtrue
\fi
\ifdefined\isleuven
  \leuventrue
  \leuvenORhasselttrue
  \vsctrue
\fi
\ifdefined\ishasselt
  \hasselttrue
  \leuvenORhasselttrue
  \vsctrue
\fi
\ifdefined\isbrussel
  \brusseltrue
  \vsctrue
\fi
\ifdefined\isgent
  \genttrue
  \vsctrue
\fi
\ifdefined\isantwerpen
  \antwerpentrue
  \vsctrue
\fi
\ifdefined\istacc
  \tacctrue
\fi

