\chapter{More PerfExpert Functionality}
\label{ch:ch05_more_perfexpert_functionality}

\section{Special Options}
\label{sec:Special_Options}

\subsection{Arguments}
\label{subsec:Arguments}

If your program takes any argument that starts with a ``-'' signal, PerfExpert will interpret this as a command line option of its own. To help PerfExpert handle arguments and options correctly, use quotes and add a space before the program's arguments (e.g., ``-s 50'').

If your program would normally executed by the command:
\begin{prompt}
%\shellcmd{my\_prog -n 1000 --s 50}%
\end{prompt}

the PerfExpert command line would become:
\begin{prompt}
%\shellcmd{perfexpert 0.1 ./my\_prog "-n 1000 -s 50"}%
\end{prompt}

\subsection{Input/Output pipes}
\label{subsec:IO_Pipes}

When your program reads from standard input via an input pipe (i.e., ``$<$''), use the ``-i'' option. The Output pipe ('$>$') is not available because PerfExpert already sets a default output file.

If your program would normally executed by the command:
\begin{prompt}
%\shellcmd{my\_prog $<$ input\_file}%
\end{prompt}

the PerfExpert command line would become:
\begin{prompt}
%\shellcmd{perfexpert -i input\_file 0.1 ./my\_prog}%
\end{prompt}

\subsection{Prologue and Epilogue}
\label{subsec:Prologue_Epilogue}

PerfExpert will run your application multiple times to collect different performance metrics. If you want to execute a program or script \strong{before} (or \strong{after}) each run, you may use the ``\strong{-b}'' (or ``\strong{-a}'') options. This is sometimes needed to reset the proper start situation for your program, to temporarily change environment variables, or to clean up after the run.

The target program should be the filename of the application you want to analyse, not a shell script. Otherwise, PerfExpert will analyse the performance of the shell script instead of the performance of you application.

If your program requires some scripts to run and after the real program, such as:

\begin{prompt}
%\shellcmd{before\_script}%
%\shellcmd{my\_prog}%
%\shellcmd{after\_script}%
\end{prompt}

the PerfExpert command line would become:

\begin{prompt}
%\shellcmd{perfexpert -b before\_script -a after\_script 0.1 ./my\_prog}%
\end{prompt}

\subsection{Debugging}
\label{subsec:Debugging}

In the event a bug (\textit{or call it a special feature}) might occur, the user can do some first aid troubleshooting.

\emph{Verbose:}

You could run perfexpert in verbose mode with the --v option. The verbose level runs from 0 to 10, and it will list the different programs that PerfExpert will start and also a lot of internal operations.

\emph{Garbage Directory:}

PerfExpert creates a hidden directory for each run, wherein it stores all its internal information. The directory name has a unique format such as ``\textit{.perfexpert-temp.kPUtGV}''. This directory structure contains a database, the results of the measurements of the various runs, code fragments identified as bottleneck, the log files, etc.

The full directory structure is removed after a proper run. In case you deliberate want to keep this directory structure, the user can add the ``-g'' (keep garbage) option.

\section{PerfExpert Automatic-mode}
\label{sec:PerfExpert_Automatic_Mode}

\strong{This component of the software is still under construction!}

PerfExpert can also automatically optimise your code. The user will still get the performance analysis report and the list of suggestion for bottleneck remediation when no automatic optimisation is possible.

In order to allow PerfExpert to automatically optimise, PerfExpert must be told where to find the single source file or where to find the Makefile to compile multiple files.

The automatic optimisation is achieved by using one the 2 extra options:

\begin{enumerate}
  \item  -s  $<$source-file$>$
  \item  -m  $<$Makefile$>$
\end{enumerate}

If you select the~-m~option and the application is composed of multiple files, your source code tree should have a Makefile file to enable PerfExpert compile your code. If your application is composed of a single source code file, the option~-s~is sufficient for you. If you do not select~-m~or~-s~options, PerfExpert requires only the binary code and will show you only the performance analysis report and the list of suggestion for bottleneck remediation.

\strong{Caution:~} PerfExpert may, if you choose to use the full capabilities for automated optimisation, change your source code during the process of optimisation. PerfExpert always saves the original file with a different name (e.g., mm\_omp.c.old\_27301) as well as adds annotations to your source code for each optimisation it makes. We cannot, however, fully guarantee that code modifications for optimisations will not break your code. We recommend having a full backup of your original source code before using PerfExpert.

\subsection{Automatic Optimisation of single source file}
\label{subsec:Automatic_Optimization}

PerfExpert will automatically try a number of code transformations, and it will recompile it for every attempt according to a default compilation command:

\begin{prompt}
gcc -O3 -g -o <binary-file> <source-file>
\end{prompt}

Now, try it on a sequential source:

\begin{prompt}
%\shellcmd{perfexpert -s mm.c -r50 0.1 ./mm}%
\end{prompt}

The new code will have been compiled with the command:
\begin{prompt}
gcc -O3 -g -o mm mm.c
\end{prompt}

You can use CC, CFLAGS and LDFLAGS to select different compiler and compilation/linkage flags. The command line would become:
\begin{prompt}
%\shellcmd{CC="icc" CFLAGS="-g -O2" perfexpert -s mm.c -r50 0.05 ./mm}%
\end{prompt}

When using PerfExpert's automatic optimisations on the OpenMP version of the simple matrix multiply code, and if we want that the OpenMP-enabled binary will run with 16 threads, we will use the following command line options:
\begin{prompt}
%\shellcmd{OMP\_NUM\_THREADS=16 CFLAGS="-fopenmp" perfexpert -s mm\_omp.c -r50 0.05 ./mm\_omp}%
\end{prompt}

In this case, PerfExpert will compile the mm\_omp.c code using the system's default compiler, which is GCC in the case of Stampede. PerfExpert will take into consideration only code fragments (loops and functions) that take more 5\% of the runtime.

To select a different compiler, you should specify the CC environment variable as below:
\begin{prompt}
%\shellcmd{CC="icc" OMP\_NUM\_THREADS=16 CFLAGS="-fopenmp" perfexpert -s mm\_omp.c -r50 0.05 ./mm\_omp}%
\end{prompt}

\subsection{Automatic Optimisation of multiple source files}
\label{subsec:Automatic_Optimization_multiple}

The same approach can be followed with multiple source files. In his case, the user need to create a Makefile, which can be used to recompile the executable.

Explore the contents of the Makefile and list the files in the directory:

\begin{prompt}
%\shellcmd{more Makefile}%
%\shellcmd{ls}%
compute1.c  compute1.h  compute2.c  compute2.h  compute3.c  compute3.h  main.c  main.h  Makefile
\end{prompt}

And run in automatic mode:

\begin{prompt}
%\shellcmd{perfexpert -m Makefile -r50 -c 0.1 ./main}%
\end{prompt}

\section{Running PerfExpert in Batch mode (with Job Script)}
\label{sec:Batch_Mode}

One can also run perfexpert in batch mode. This is often desired when the runtime of the program is a bit higher, and you do not want to wait for the results.  You can replace the normal command line to start up your executable (e.g., ``./mm\_omp'') in the job script with the perfexpert command line. (e.g., ``\textit{perfexpert -c 0.3 mm\_omp}''). When running your application with a job script, be sure to specify a running time that is about 6x the normal running time of the program.

Compile the parallel Matrix Multiplication program, explore the job script and submit the job:

\begin{prompt}
%\shellcmd{gcc -fopenmp mm\_omp.c -g -o mm\_omp}%
%\shellcmd{more mm\_omp.job}%
\end{prompt}

\begin{tabular}{|p{\dimexpr 0.3\textwidth-2\tabcolsep}|p{\dimexpr 0.7\textwidth-2\tabcolsep}|} \hline
\strong{Institute}  & \strong{Command} \\ \hline
TACC                & \$ \strong{sbatch mm\_omp.job} \\ \hline
VSC                 & \$ \strong{qsub mm\_omp.job} \\ \hline
\end{tabular}

The job will now be submitted to a node on the supercomputer. PerfExpert will run on the allocated node, start the users application and generate the performance analysis report. The results are printed on standard output (stdout), which was re-directed to the file ``MyJob.o\%j'' (``\%j'' is replaced by the jobID).

When the job is finished, the generated report can be viewed in the file in our current directory.

\begin{prompt}
%\shellcmd{ls}%
...
MyJob.o2225027
MyJob.e2225027
...
%\shellcmd{more MyJob.o2225027}%
[perfexpert] Collecting measurements [hpctoolkit]
[perfexpert]    [1] 1.992842066 seconds (includes measurement overhead)
[perfexpert]    [2] 1.564767383 seconds (includes measurement overhead)
[perfexpert]    [3] 1.373591388 seconds (includes measurement overhead)
[perfexpert] Analysing measurements
---------------------------------------------------------------------------
Total running time for mm_omp is 8.86 seconds between all 16 cores
The wall-clock time for mm_omp is approximately 0.55 seconds
...
\end{prompt}

