\chapter{Easybuild}
\label{ch:easybuild}

\section{What is Easybuild?}

You can use EasyBuild to build and install supported software in your own VSC account,
rather than requesting a central installation by the HPC support team.

EasyBuild (\url{https://easybuilders.github.io/easybuild}) is the software build and installation framework that was created by the HPC-UGent
team, and has recently been picked up by HPC sites around the world. It allows you to manage
(scientific) software on High Performance Computing (HPC) systems in an efficient way.

\section{When should I use Easybuild?}

For general software installation requests, please see \autoref{sec:software-installation}. However,
there might be reasons to install the software yourself:

\begin{itemize}
    \item applying custom patches to the software that only you or your group are using
    \item evaluating new software versions prior to requesting a central software installation
    \item installing (very) old software versions that are no longer eligible for central installation (on new clusters)
\end{itemize}

\section{Configuring EasyBuild}

Before you use EasyBuild, you need to configure it:

\subsection{Path to sources}

This is where EasyBuild can find software sources:

\begin{prompt}
%\shellcmd{export EASYBUILD\_SOURCEPATH=\$VSC\_DATA/easybuild/sources:/apps/gent/source}%
\end{prompt}

\begin{itemize}
    \item the first directory \lstinline|$VSC_DATA/easybuild/sources| is where EasyBuild will (try to) automatically
        download sources if they're not available yet
    \item \lstinline|/apps/gent/source| is the central ``cache'' for already downloaded sources,
        and will be considered by EasyBuild before downloading anything
\end{itemize}




\subsection{Build directory}

This directory is where EasyBuild will build software in. To have good performance,
this needs to be on a fast filesystem.

\begin{prompt}
%\shellcmd{export EASYBUILD\_BUILDPATH=\${TMPDIR:-/tmp/\$USER}}%
\end{prompt}

On cluster nodes, you can use the fast, in-memory \lstinline|/dev/shm/$USER| location
as a build directory.

\subsection{Software install location}

This is where EasyBuild will install the software (and accompanying modules) to.

For example, to let it use \lstinline|$VSC_DATA/easybuild|, use:


% Don't put the whole command inside \shellcmd{} because it would show ugly white boxes

\begin{prompt}
%\shellcmd{export}% EASYBUILD_INSTALLPATH=$VSC_DATA/easybuild/$VSC_OS_LOCAL/$VSC_ARCH_LOCAL$VSC_ARCH_SUFFIX
\end{prompt}

Using the \lstinline|$VSC_OS_LOCAL|, \lstinline|$VSC_ARCH| and \lstinline|$VSC_ARCH_SUFFIX| environment variables
ensures that your install software to a location that is specific to the cluster you are
building for.

Make sure you \strong{do not build software on the login nodes}, since the loaded \lstinline|cluster|
module determines the location of the installed software. Software built on the login
nodes may not work on the cluster you want to use the software on (see also \autoref{sec:running-software-incompatible-with-host}).

To share custom software installations with members of your VO, replace \lstinline|$VSC_DATA|
with \lstinline|$VSC_DATA_VO| in the example above.

\section{Using EasyBuild}

Before using EasyBuild, you first need to load the \lstinline|EasyBuild| module. We don't specify
a version here (this is an exception, for most other modules you should, see \autoref{subsec:explicit-version-numbers}) because newer versions
might include important bug fixes.

\begin{prompt}
module load EasyBuild
\end{prompt}

\subsection{Installing supported software}

EasyBuild provides a large collection of readily available software versions,
combined with a particular toolchain version. Use the \lstinline|--search|
(or \lstinline|-S|) functionality to see which different 'easyconfigs'
(build recipes, see \url{http://easybuild.readthedocs.org/en/latest/Concepts_and_Terminology.html#easyconfig-files}) are available:

\begin{prompt}
%\shellcmd{eb -S example-1.2}%
CFGS1=/apps/gent/CO7/sandybridge/software/EasyBuild/3.6.2/lib/python2.7/site-packages/easybuild_easyconfigs-3.6.2-py2.7.egg/easybuild/easyconfigs
 * $CFGS1/e/example/example-1.2.1-foss-%\the\year{}%a.eb
 * $CFGS1/e/example/example-1.2.3-foss-%\the\year{}%b.eb
 * $CFGS1/e/example/example-1.2.5-intel-%\the\year{}%a.eb
\end{prompt}

For readily available easyconfigs, just specify the name of the easyconfig file to build
and install the corresponding software package:

\begin{prompt}
%\shellcmd{eb example-1.2.1-foss-\the\year{}a.eb -{}-robot}%
\end{prompt}

\subsection{Installing variants on supported software}

To install small variants on supported software, e.g. a different software version,
or using a different compiler toolchain, use the corresponding \lstinline|--try-X| options:

To try to install \lstinline|example v1.2.6|, based on the easyconfig file for \lstinline|example v1.2.5|:

\begin{prompt}
%\shellcmd{eb example-1.2.5-intel-\the\year{}a.eb -{}-try-software-version=1.2.6}%
\end{prompt}

To try to install example v1.2.5 with a different compiler toolchain:

\begin{prompt}
%\shellcmd{eb example-1.2.5-intel-\the\year{}a.eb -{}-robot -{}-try-toolchain=intel,\the\year{}b}%
\end{prompt}

\subsection{Install other software}

To install other, not yet supported, software, you will need to provide the required
easyconfig files yourself. See \url{https://easybuild.readthedocs.org/en/latest/Writing_easyconfig_files.html}
for more information.

\section{Using the installed modules}

To use the modules you installed with EasyBuild, extend \lstinline|$MODULEPATH|
to make them accessible for loading:

\begin{prompt}
%\shellcmd{module use \$EASYBUILD\_INSTALLPATH/modules/all}%
\end{prompt}

It makes sense to put this \lstinline|module use| command and all \lstinline|export|
commands in your \lstinline|.bashrc| login script. That way you don't have to type
these commands every time you want to use EasyBuild or you want to load modules generated
with EasyBuild. See also
\href{\LinuxManualURL#sec:bashrc-login-script}{the section on \texttt{.bashrc} in the ``Beyond the basics'' chapter of the intro to Linux}.
