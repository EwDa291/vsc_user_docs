\chapter{Checkpointing}
\label{ch:checkpointing}
\section{Why checkpointing?}
If you want to run jobs that require wall time than the maximum wall time per job
and/or want to avoid you lose work because of power outages or system crashes,
you need to resort to checkpointing.

\section{What is checkpointing?}

Checkpointing allows for running jobs that run for weeks or months, by splitting
the job into smaller parts (called subjobs) which are executed consecutively.
Each time a subjob is running out of requested wall time, a snapshot of the
application memory (and much more) is taken and stored, after which a subsequent
subjob will pick up the checkpoint and continue.

\section{How to use checkpointing?}

Using checkpointing is very simple: just use \lstinline|csub| instead of \lstinline|qsub| to submit a job.

The \lstinline|csub| command creates a wrapper around your job script, to take care
of all the checkpointing stuff.

In practice, you (usually) don't need to adjust anything, except for the command used to submit your job.

Checkpointing does not require any changes to the application you are running, and should support most software.

\section{Usage and parameters}

An overview of the usage and various command line parameters is given here.

\subsection{Submitting a job}

Typically, a job script is submitted with checkpointing support enabled by running:

\begin{prompt}
csub -s job_script.sh
\end{prompt}

The  \lstinline|-s| flag specifies the job script to run.

\subsection{Caveat: don't create local directories}

One important caveat is that the job script (or the applications run in the script)
should \emph{not} create its own local temporary directories, because those will not (always)
be restored when the job is restarted from checkpoint.

\subsection{PBS directives}

Most PBS directives (\texttt{\#PBS \ldots} specified in the job script will be ignored.
There are a few exceptions however, i.e., \lstinline|# PBS -N <name>| (job name)
and all \lstinline|-l| directives (\lstinline|# PBS -l|), e.g., \lstinline|nodes|,
\lstinline|ppn|, \lstinline|vmem| (virtual memory limit), etc.
Controlling other job parameters (like requested walltime per sub-job) should be
specified on the \lstinline|csub| command line.

\subsection{Getting help}

Help on the various command line parameters supported by \lstinline|csub| can be
obtained using \lstinline|-h| or \lstinline|--help|.

\subsection{Local files (\texttt{--pre} / \texttt{--post})}

The \lstinline|--pre| and \lstinline|--post| parameters control whether local files
are copied or not. The job submitted using \lstinline|csub| is (by default) run on the local
storage provided by a particular workernode. Thus, no changes will be made to the
files on the shared storage.

If the job script needs (local) access to the files of the directory where \lstinline|csub|
is executed, \lstinline|--pre| flag should be used. This will copy all the files in the job
script directory to the location where the job script will execute.

If the output of the job (\lstinline|stdout|/\lstinline|stderr|) that was run,
or additional output files created by the job in its working directory are required,
the \lstinline|--post| flag should be used. This will copy the entire job working directory
to the location where \lstinline|csub| was executed, in a directory named \lstinline|result.<jobname>|.
An alternative is to copy the interesting files to the shared storage at the end of the job script.


\subsection{Running on shared storage (\texttt{--shared})}

If the job needs to be run on the shared storage, \lstinline|--shared| should be specified.
You should enable this option by default, because it makes the execution of the
underlying \lstinline|csub| script more robust: it doesn't have to copy from/to
the local storage on the workernode. When enabled, the job will be run in a
subdirectory of \lstinline|$VSC_SCRATCH/chkpt|. All files produced by the job will
be in \lstinline|$VSC_SCRATCH/chkpt/<jobid>/| while the job is running.

Note that if files in the directory where the job script is located are required to
run the job, you should also use \lstinline|--pre|.

\subsection{Job wall time (\texttt{--job\_time}, \texttt{--chkpt\_time})}

To specify the requested wall time per subjob, use the \lstinline|--job-time| parameter.
The default setting is 10 hours per (sub)job. Lowering this will result in more
frequent checkpointing, and thus more (sub)jobs.

To specify the time that is reserved for checkpointing the job, use \lstinline|--chkpt_time|.
By default, this is set to 15 minutes which should be enough for most applications/jobs.
\emph{Don't change this unless you really need to}.

The total requested wall time per subjob is the sum of both \lstinline|job_time| and \lstinline|chkpt_time|.

If you would like to time how long the job executes, just prepend the main command in your job script with
the time command \lstinline|time|, e.g.:

\begin{prompt}
time main_command
\end{prompt}

The \lstinline|real| time will not make sense, as it will also include the time
passed between two checkpointed subjobs. However, the \lstinline|user| time should give a good
indication of the actual time it took to run your command, even if multiple checkpoints were performed.

\subsection{Resuming from last checkpoint (\texttt{--resume})}

The \lstinline|--resume| option allows you to resume a job from the
last available checkpoint in case something went wrong
(e.g., accidentally deleting a (sub)job using \lstinline|qdel|, a power outage or other system failure, \ldots).

Specify the job name as returned after submission (and as listed in \texttt{\$VSC\_SCRATCH/chkpt}).
The full job name consists of the specified job name (or the script name if no job name was specified),
a timestamp and two random characters at the end, for example \texttt{my\_job\_name.\the\year0102\_133755.Hk}
or \texttt{script.sh.\the\year0102\_133755.Hk}.

Note: When resuming from checkpoint, you can change the wall time resources for your job using the \lstinline|--job_time| and
\lstinline|--chkpt_time| options. This should allow you to continue from the last checkpoint
in case your job crashed due to an excessively long checkpointing time.

In case resuming fails for you, please contact \hpcinfo, and include the output
of the \lstinline|csub --resume| command in your message.

\section{Additional options}

\subsection{Array jobs (\texttt{-t})}

\lstinline|csub| has support for checkpointing array jobs with the \lstinline|-t <spec>|
flag on the \lstinline|csub| command line. This behaves the same as \lstinline|qsub|, see \autoref{sec:worker-framework-job-arrays}.


\subsection{Pro/epilogue mimicking (\texttt{--no\_mimic\_pro\_epi})}

The option \lstinline|--no_mimic_pro_epi| disables the workaround currently required to resolve
a permissions problem when using actual Torque prologue/epilogue scripts. Don't use
this option unless you really know what you are doing.

\subsection{Cleanup checkpoints (\texttt{--cleanup\_after\_restart})}

Specifying this option will make the wrapper script remove the checkpoint files after
a successful job restart. This may be desirable in cause you are short on storage space.

Note that we don't recommend setting this option, because this way you won't be able
to resume from the last checkpoint when something goes wrong. It may also prevent
the wrapper script from reattempting to resubmit a new job in case an infrequent known
failure occurs. So, don't set this unless you really need to.

\subsection{No cleanup after job completion (\texttt{--no\_cleanup\_chkpt})}

Specifying this option will prevent the wrapper script from cleaning up the
checkpoints and related information once the job has finished. This may be useful
for debugging, since this also preserves the \lstinline|stdout|/\lstinline|stderr|
of the wrapper script.

\emph{Don't set this unless you know what you are doing}.
