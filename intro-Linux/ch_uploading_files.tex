\chapter{Uploading/downloading/editing files}

\subsection{Uploading/downloading files}

Detailed information about uploading/downloading files with ``scp'' (using an OS
X or Linux client) or WinSCP (Windows client), see
\usrl{http://hpc.ugent.be/userwiki/index.php/User:VscCopy}.

After copying files it is advised to run 

\begin{prompt}
%\shellcmd{dos2unix filename}
\end{prompt}

as this will fix any problems with windows / unix conversion.

\section{Symlinks for data/scratch}

As we end up in the home directory when connecting, it would be convenient if we
could access our data and vo storage. To facilitate this we shall create
symlinks to them in our home directory.

%% TODO: scratchdir doesnt exist.
\begin{prompt}
%\shellcmd{cd \$HOME}%
%\shellcmd{ln -s \$VSC\_SCRATCH scratch}%
%\shellcmd{ln -s \$VSC\_DATA data}%
%\shellcmd{ls -l scratch data}%
%\shellcmd{ls -l scratch data}%
lrwxrwxrwx 1 %\userid% %\userid% 31 Mar 27  2009 data -> %\homedir%
lrwxrwxrwx 1 %\userid% %\userid% 34 Jun  5  2012 scratch -> %\scratchdir%
\end{prompt}

this will create 4 directories pointing to the respective storages

\section{Editing: ``nano''}

Nano is the simplest editor available on Linux. To open Nano, just type
``nano''. To edit a file, you use ``nano the\_file\_to\_edit.txt''. You will
be presented with the contents of the file and a menu at the bottom with
commands like ``\^{}O Write Out''. The ``\^{}'' is the Control key. So ``\^{}O'' means
``Ctrl-O''. The main commands are:

\begin{enumerate}
\item Open ("Read"): ``\^{}R''
\item Save ("Write Out"): ``\^{}O''
\item Exit: ``\^{}X''
\end{enumerate}

More advanced editors (beyond the scope of this page) are ``vim'' and ``emacs''.


