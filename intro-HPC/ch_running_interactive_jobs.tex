\chapter{Running interactive jobs}
\label{ch:running-interactive-jobs}

\section{Introduction}

Interactive jobs are jobs which give you an interactive session on one of the
compute nodes. Importantly, accessing the compute nodes this way means that the
job control system guarantees the resources that you have asked for.

Interactive PBS jobs are similar to non-interactive PBS jobs in that they are
submitted to PBS via the command \strong{qsub}. Where an interactive job
differs is that it does not require a job script, the required PBS directives
can be specified on the command line.

The syntax for \emph{qsub} for submitting an interactive PBS job is:
\begin{prompt}
%\shellcmd{qsub -I \ldots pbs directives \ldots}%
\end{prompt}

\section{Interactive jobs, without X support}

\begin{tip}
Find the code in
``\tilde/\exampledir''
\end{tip}

First of all, in order to know on which computer you're working, enter:
\begin{prompt}
%\shellcmd{hostname -f}%
%\loginhost%
\end{prompt}

This means that you're now working on the login-node ``\emph{\loginnode}'' of
the \hpc cluster.

The most basic way to start an interactive job is the following:
\begin{prompt}
%\shellcmd{qsub -I}%
qsub: waiting for job %\jobid% to start
qsub: job %\jobid% ready
\end{prompt}

There are two things of note here.

\begin{enumerate}
  \item  The ``\emph{qsub''} command (with the interactive -I
      flag) waits until a node is assigned to your interactive session,
      connects to the compute node and shows you the terminal prompt on that
      node.
  \item  You'll see that your directory structure of your home directory has
      remained the same. Your home directory is actually located on a shared
      storage system.
\end{enumerate}

In order to know on which compute-node you're working, enter again:
\begin{prompt}
%\shellcmd{hostname -f}%
%\computenode%
\end{prompt}

Note that we are now working on the compute-node called ``\emph{\computenode}''.
This is the compute node, which was assigned to us by the scheduler after
issuing the ``\emph{qsub -I}'' command.

\ifantwerpen
This computer name looks strange, but bears some logic in it.  It provides the
system administrators with information where to find the computer in the
computer room.

The computer ``\emph{\computenodeshort}'' stands for:

\begin{enumerate}
\item  ``r5'' is rack \#5.
\item  ``c3'' is enclosure/chassis \#3.
\item  ``cn13'' is compute node \#13.
\end{enumerate}

With this naming convention, the system administrator can easily find the
physical computers when they need to execute some maintenance activities.
\fi

Now, go to the directory of our second interactive example and run the program
``primes.py''. This program will ask you for an upper limit ($>$ 1) and will
print all the primes between 1 and your upper limit:

\begin{prompt}
%\shellcmd{cd ~/\exampledir}%
%\shellcmd{./primes.py}%
This program calculates all primes between 1 and your upper limit.
Enter your upper limit (>1): %\strong{50}%
Start Time:  2013-09-11 15:49:06
[Prime#1] = 1
[Prime#2] = 2
[Prime#3] = 3
[Prime#4] = 5
[Prime#5] = 7
[Prime#6] = 11
[Prime#7] = 13
[Prime#8] = 17
[Prime#9] = 19
[Prime#10] = 23
[Prime#11] = 29
[Prime#12] = 31
[Prime#13] = 37
[Prime#14] = 41
[Prime#15] = 43
[Prime#16] = 47
End Time:  2013-09-11 15:49:06
Duration:  0 seconds.
\end{prompt}

You can exit the Interactive session with:
\begin{prompt}
%\shellcmd{exit}%
\end{prompt}

Note that you can now use this allocated node for 1 hour.  After this hour you
will be automatically disconnected. You can change this ``usage time'' by
explicitly specifying a ``walltime'', i.e., the time that you want to work on
this node.  (Think of walltime as the time elapsed when watching the clock on
the wall.)

You can work for 3 hours by:
\begin{prompt}
%\shellcmd{qsub -I -l walltime=03:00:00}%
\end{prompt}

If the walltime of the job is exceeded, the (interactive) job will be killed
and your connection to the compute node will be closed. So do make sure to
provide adequate walltime and that you save your data before your (wall)time is
up (exceeded)!  When you do not specify a walltime, you get a default walltime
of 1 hour.

\section{Interactive jobs, with X support}

\subsection{Software Installation}

To display graphical applications from a Linux computer (such as the VSC
clusters) on your machine, you need to install an X Window server on your
local computer.

The X Window system (commonly known as \strong{X11}, based on its current major
version being 11, or shortened to simply \strong{X}) is the system-level
software infrastructure for the windowing GUI on Linux, BSD and other UNIX-like
operating systems. It was designed to handle both local displays, as well as
displays sent across a network. More formally, it is a computer software system
and network protocol that provides a basis for graphical user interfaces (GUIs)
and rich input device capability for networked computers.

\ifmac
  Download the latest version of the XQuartz package on:
  http://xquartz.macosforge.org/landing/
  and install the XQuartz.pkg package.

  \includegraphics*[width=5.79in, height=4.12in, keepaspectratio=false]{img0512}

  The installer will take you through the installation procedure, just continue
  clicking $<$continue$>$ on the various screens that will pop-up until your
  installation was succesful.

  A reboot is required before XQuartz will correctly open X applications.
  \includegraphics*[width=5.81in, height=4.31in, keepaspectratio=false]{img0513}
\fi
\iflinux
%if X is not installed under linux the user will probably be expierienced enought
%to know what he/she's doing and be aware that he/she can't run X applications
\fi

\ifwindows
  \paragraph{Install XMing}

  The first task is to install the Xming software.

  \begin{enumerate}
    \item  Download the Xming installer from the following
      address: \url{http://www.straightrunning.com/XmingNotes/}. Either download Xming
      from the \strong{Public Domain Releases} (free) or from the \strong{Website
      Releases} (after a donation) on the website.
    \item  Run the Xming setup program on your Windows desktop.
    \item  Keep the proposed default folders for the Xming installation.
    \item  When selecting the components that need to be installed, make sure to
      select ``\emph{XLaunch wizard}'' and ``\emph{Normal PuTTY Link SSH client}''.

  \includegraphics*[width=4.27in, height=3.32in, keepaspectratio=false]{img0500}

  \item  We suggest to create a Desktop icon for Xming and XLaunch.
  \item  And \keys{Install}.
  \end{enumerate}

  And now we can run Xming:

  \begin{enumerate}
  \item Select XLaunch from the Start Menu or by double-clicking the Desktop icon.

  \item  Select \keys{Multiple Windows}. This will open each application in a separate window.

  \includegraphics*[width=3.11in, height=3.11in, keepaspectratio=false]{img0501}


  \item  Select \keys{Start no client} to make XLaunch wait for other programs (such as PuTTY).

  \includegraphics*[width=4.60in, height=3.57in, keepaspectratio=false]{img0502}


  \item  Select \keys{Clipboard} to share the clipboard.

  \includegraphics*[width=4.01in, height=3.11in, keepaspectratio=false]{img0503}


  \item  Finally \keys{Save configuration} into a file. You can keep the default filename and save it in your Xming installation directory.

  \includegraphics*[width=3.90in, height=3.03in, keepaspectratio=false]{img0504}

  \item  Now Xming is running in the background \ldots\\ and you can launch a graphical application in your PuTTY terminal.
  \item  Open a PuTTY terminal and connect  to the HPC.
  \item  In order to test the X-server, run ``\emph{xclock}''. ``\emph{xclock}'' is the standard GUI clock for the X Window System.
  \end{enumerate}

  \begin{prompt}
  %\shellcmd{xclock}%
  \end{prompt}

  You should see the XWindow clock application appearing on your Windows machine.
  The ``\emph{xclock}'' application runs on the login-node of the \hpc, but is
  displayed on your Windows machine.

  \includegraphics*[width=1.44in, height=1.62in, keepaspectratio=false]{img0505}

  You can close your clock and connect further to a compute node with again your
  X-forwarding enabled:

  \begin{prompt}
  %\shellcmd{qsub -I -X}%
  qsub: waiting for job %\jobid% to start
  qsub: job %\jobid% ready
  %\shellcmd{hostname -f}%
  %\computenode%
  %\shellcmd{xclock}%
  \end{prompt}

  and you should see your clock again.

  \paragraph{SSH Tunnel}

  In order to work in client/server mode, it is often required to establish an
  SSH tunnel between your Windows desktop machine and the compute node your job
  is running on.  PuTTY must have been installed on your computer, and you should
  be able to connect via SSH to the HPC cluster's login node.

  Because of one or more firewalls between your desktop and the HPC clusters, it
  is generally impossible to communicate directly with a process on the cluster
  from your desktop except when the network managers have given you explicit
  permission (which for security reasons is not often done). One way to work
  around this limitation is SSH tunneling.

  There are several cases where this is useful:

  \begin{enumerate}
  \item  Running X applications on the cluster: The X program cannot directly
    communicate with the X server on your local system. In this case, the
    tunneling is easy to set up as PuTTY will do it for you if you select the
    right options on the X11 settings page as explained on the page about
    text-mode access using PuTTY.
  \item  Running a server application on the cluster that a client on the desktop
    connects to. One example of this scenario is ParaView in remote visualization
    mode, with the interactive client on the desktop and the data processing and
    image rendering on the cluster. This scenario is explained on this page.
  \item  Running clients on the cluster and a server on your desktop. In this
    case, the source port is a port on the cluster and the destination port is on
    the desktop.
  \end{enumerate}

  Procedure: A tunnel from a local client to a specific computer node on the cluster

  \begin{enumerate}
    \item  Log in on the login node via PuTTY.
    \item  Start the server job, note the compute node's name the job is running
    on (e.g., \computenode), as well as the port the server is listening on
  (e.g., ``54321'').  \item  Set up the tunnel:

    \begin{enumerate}
      \item  In the ``\emph{Category}'' pane, expand \menu[,]{Connection,SSH}, and select \keys{Tunnels} as show below:
    \end{enumerate}
  \end{enumerate}

  \includegraphics*[width=4.85in, height=4.67in, keepaspectratio=false]{img0506}

  \begin{enumerate}
    \item  In the ``\emph{Source port}'' field, enter the local port to use (e.g., \emph{12345}).
    \item  In the ``\emph{Destination}'' field, enter \emph{$<$hostname$>$:$<$server-port$>$} (e.g., \computenode:54321 as in the example above).
    \item  Click the \keys{Add} button.
    \item  Click the \keys{Open} button
  \end{enumerate}

  The tunnel is now ready to use.
\fi % ENDIF WINDOWS

\ifmacORlinux
\subsection{Connect with X-forwarding}

In order to get the graphical output of your application (which is running on a
compute node on the \hpc) transferred to your personal screen, you will need
to re-connect to the \hpc with X-forwarding enabled, which is done with the
``-X'' option.

\includegraphics*[width=5.78in, height=1.78in, keepaspectratio=false]{ch5-interactive-mode}

First exit and re-connect to the \hpc with X-forwarding enabled:

\begin{prompt}
%\shellcmd{exit}%
%\shellcmd{ssh --X $<$vsc-account$>$@\loginnode}%
%\shellcmd{hostname -f}%
%\loginhost%
\end{prompt}

We first check whether our GUIs on the login node are decently forwarded to
your screen on your local machine. An easy way to test it is by running a small
X-application on the login node. Type:

\begin{prompt}
%\shellcmd{xclock}%
\end{prompt}

And you should see a clock appearing on your screen.

\includegraphics*[width=2.47in, height=2.78in, keepaspectratio=false]{img0507}

You can close your clock and connect further to a compute node with again your
X-forwarding enabled:

\begin{prompt}
%\shellcmd{qsub -I -X}%
qsub: waiting for job %\jobid% to start
qsub: job %\jobid% ready
%\shellcmd{hostname -f}%
%\computenode%
%\shellcmd{xclock}%
\end{prompt}

and you should see your clock again.
\fi

\subsection{Run simple example}

We have developed a little interactive program that shows the communication in
2 directions. It will send information to your local screen, but also asks you
to click a button.

Now run the message program:
\begin{prompt}
%\shellcmd{cd ~/examples/Running-jobs-with-input}%
%\shellcmd{./message.py}%
\end{prompt}

You should see the following message appearing.

\includegraphics*[width=3.51in, height=1.47in, keepaspectratio=false]{img0508}

Click any button and see what happens.

\begin{prompt}
-----------------------
< Enjoy the day! Mooh >
-----------------------
     ^__^
     (oo)\_______
     (__)\       )\/\
         ||----w |
         ||     ||

\end{prompt}

\subsection{Run your interactive application}

In this last example, we will show you that you can just work on this compute
node, just as if you were working locally on your desktop.  We will run the
Fibonacci example of the previous chapter again, but now in full interactive
mode. Remember that it was written in MATLAB, so we first load the MATLAB
module.

\begin{prompt}
%\shellcmd{module load MATLAB}%
\end{prompt}

And start the MATLAB interactive environment:

\begin{prompt}
%\shellcmd{matlab}%
\end{prompt}

And start the fibo2.m program in the command window:
\begin{prompt}
fx >> %\textbf{run fibo2.m}%
\end{prompt}

\includegraphics*[width=5.84in, height=4.11in, keepaspectratio=false]{img0509}

And see the displayed calculations, \dots

\includegraphics*[width=5.84in, height=4.11in, keepaspectratio=false]{img0510}

as well as the nice ``plot'' appearing:

\includegraphics*[width=5.76in, height=2.94in, keepaspectratio=false]{img0511}

You can work in this MATLAB GUI, and finally terminate the application by
entering ``\strong{exit}'' in the command window again.

\begin{prompt}
fx >> %\textbf{exit}%
\end{prompt}


Please not that you are not running MATLAB on the \hpc, but only on the
login nodes. So it is not recommended to run large programs this way.
You are allowed to run X programs on the login nodes for visualization
related tasks only. This is here to let you speed up your workflow,
so you can do a quick visualization without having to copy over all your data
from the \hpc to your local system to quickly view a plot.

On the login nodes memory and processor time is very limited, because these
nodes are shared with a large amount op people.

\ifgent
For more information about how you can run computationally expensive
MATLAB programs on the \hpc, see \url{http://hpc.ugent.be/userwiki/index.php/User:MATLAB}.
%TODO: add a matlab chapter/big section to ch_programming_examples
\fi
