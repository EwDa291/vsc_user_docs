\usepackage[english]{babel}
\usepackage{amssymb}
\usepackage{amsmath}
\usepackage{hyperref}
% allow conditional compilation
\usepackage{etoolbox}
\usepackage[toc,nonumberlist]{glossaries}

\usepackage{color}
\usepackage{listings}
\usepackage{tikz}
\usepackage[T1]{fontenc}
\usepackage{graphics}
\graphicspath{{img//}}

\lstset{basicstyle=\ttfamily,breaklines=true}
\lstdefinestyle{prompt}{language=bash,frame=tb,columns=fullflexible,
  escapechar=\%}
\lstdefinestyle{code}{numbers=left}

\lstnewenvironment{prompt} {\lstset{style=prompt}} {}
\lstnewenvironment{code}[1]{\lstset{style=code, language=#1}} {}
\lstnewenvironment{prog}{\lstset{style=code}} {}

% Taken from: http://tex.stackexchange.com/questions/9363/how-does-one-insert-a-backslash-or-a-tilde-into-latex
\renewcommand{\tilde}{\raise.17ex\hbox{$\scriptstyle\mathtt{\sim}$}}
\newcommand{\shellcmd}[1]{\textbf{\texttt{#1}\\}}
\newcommand{\iftoggleverb}[1]{%
  \ifcsdef{etb@tgl@#1}
    {\csname etb@tgl@#1\endcsname\iftrue\iffalse}
    {\etb@noglobal\etb@err@notoggle{#1}\iffalse}%
}
\parindent=0pt
\parskip=7pt

\newcommand{\ignore}[1]{}

\newif\ifremark
\long\def\remark#1{
  \ifremark%
  \begingroup%
  \dimen0=\columnwidth
  \advance\dimen0 by -1in%
  \setbox0=\hbox{\parbox[b]{\dimen0}{\protect\em #1}}
  \dimen1=\ht0\advance\dimen1 by 2pt%
  \dimen2=\dp0\advance\dimen2 by 2pt%
  \vskip 0.25pt%
  \hbox to \columnwidth{%
    \vrule height\dimen1 width 3pt depth\dimen2%
    \hss\copy0\hss%
    \vrule height\dimen1 width 3pt depth\dimen2%
  }%
  \endgroup%
\fi}

\remarktrue
\newtoggle{windows}
\newtoggle{mac}
\newtoggle{leuven}
\newtoggle{hasselt}
\newtoggle{brussel}
\newtoggle{gent}
\newtoggle{antwerpen}

\ifdefined\iswindows
  \toggletrue{windows}
\fi
\ifdefined\ismac
  \toggletrue{mac}
\fi
\ifdefined\isleuven
  \toggletrue{leuven}
\fi
\ifdefined\ishasselt
  \toggletrue{hasselt}
\fi
\ifdefined\isbrussel
  \toggletrue{brussel}
\fi
\ifdefined\isgent
  \toggletrue{gent}
\fi
\ifdefined\isantwerpen
  \toggletrue{antwerpen}
\fi
