\chapter{MATLAB}
\label{ch:matlab}

\section{Why is the MATLAB compiler required?}

The main reason behind this alternative way of using MATLAB is licensing: only
a limited number of MATLAB sessions can be active at the same time. However, once
the MATLAB program is compiled using the MATLAB compiler, the resulting stand-alone
executable can be run without needing to contact the license server.
A license is required for the MATLAB Compiler, see \url{https://nl.mathworks.com/help/compiler/index.html}.

\ifgent
Only a limited amount of MATLAB sessions can be active at the same time because
there are only a limited amount of MATLAB research licences available on
the \university MATLAB license server. Since there are a lot of compute nodes,
if each node would need a license, the licenses would quickly run out.
\fi

\section{How to compile MATLAB code}

Compiling MATLAB code can only be done from the login nodes, because only login
nodes can access the server, workernodes on clusters cannot.

To access the MATLAB compiler, the MATLAB module should be loaded first. Make sure
you are using the same MATLAB version to compile and to run the compiled MATLAB
program.

\begin{prompt}
%\shellcmd{module avail MATLAB}%
----------------------/apps/gent/CO7/sandybridge/modules/all ----------------------
   MATLAB/2016b    MATLAB/2017b    MATLAB/2018a (D)
%\shellcmd{module load MATLAB/2018a}
\end{prompt}

After loading the module, the \verb|mcc| command can be used. To get help on
\verb|mcc|, you can run \verb|mcc -?|.

To compile a single file, the \verb|-m|
flag is used (the \verb|-v| flag means verbose ouput).
To show how \verb|mcc| can be used, we use the \verb|magicsquare| example
that comes with MATLAB.

\begin{prompt}
%\shellcmd{cp \$EBROOTMATLAB/extern/examples/compiler/magicsquare.m .}%
%\shellcmd{mcc -mv magicsquare.m}%
Opening log file:  %\homedir/java.log.34090%
Compiler version: 6.6 (R2018a)
Dependency analysis by REQUIREMENTS.
Parsing file "%\homedir/magicsquare.m%"
	(Referenced from: "Compiler Command Line").
Deleting 0 temporary MEX authorization files.
Generating file "%\homedir/readme.txt%".
Generating file "run\_magicsquare.sh".
\end{prompt}

\subsection{Libraries}

To \emph{compile} a MATLAB library, the \verb|-l| flag should be used.

To compile a MATLAB program that \emph{needs a library}, you can use the
\texttt{-I \emph{library\_path}} flag. This will tell the compiler to also
look for files in \texttt{\emph{library\_path}}.

It's also possible to use the \texttt{-a \emph{path}} flag. That will result in
all files under the \texttt{\emph{path}} getting added to the final executable.

\subsection{Memory issues during compilation}

If you are seeing Java memory issues during the compilation of your MATLAB program
on the login nodes, consider tweaking the default maximum heap size (128M) of Java
using the \verb|_JAVA_OPTIONS| environment variable with:

\begin{prompt}
%\shellcmd{export \_JAVA\_OPTIONS="-Xmx64M"}%
\end{prompt}

The MATLAB compiler spawns multiple Java processes, and because of the default memory
limits that are in effect on the login nodes, this might lead to a crash of the compiler
if it's trying to create to many Java processes. If we lower the heap size, more
Java processes will be able to fit in memory.

\section{Multithreading}

MATLAB can only use the cores in a single
workernode (unless the Distributed Computing toolbox is used, see
\url{https://nl.mathworks.com/products/distriben.html}).

The amount of workers used by MATLAB for the parallel toolbox can be controlled
via the \verb|parpool| function: \verb|parpool('cluster1', 16)| will run the
\verb|cluster1| profile with 16 workers.

\section{Java output logs}

Each time MATLAB is executed, it generates a Java log file in the users home directory.
The output log directory can be changed using:

\begin{prompt}
%\shellcmd{MATLAB\_LOG\_DIR=\emph{<OUTPUT\_DIR>}}%
\end{prompt}

where \verb|<OUTPUT_DIR>| is the name of the desired output directory. To create
and use a temporary directory for these logs:

\begin{prompt}
# create unique temporary directory in $TMPDIR (or /tmp/$USER if $TMPDIR is not defined)
# instruct MATLAB to use this directory for log files by setting $MATLAB_LOG_DIR
%\shellcmd{export MATLAB\_LOG\_DIR=\$(mktemp -d -p  \${TMPDIR:-/tmp/\$USER})}%
\end{prompt}

You should remove the directory at the end of your job script:
\begin{prompt}
%\shellcmd{rm -rf \$MATLAB\_LOG\_DIR}%
\end{prompt}

\section{MATLAB job script}

% TODO: copy MATLAB job script
