\chapter{Troubleshooting}
\label{ch:troubleshooting}



\section{Walltime issues}
If you get from your job output an error message similar to this:

\begin{prompt}
 =>> PBS: job killed: walltime %\emph{<value in seconds>}% exceeded limit %\emph{<value in seconds>}%
\end{prompt}

This occurs when your job did not complete within the requested walltime.
See section~\ref{sec:specifying-walltime-requirements} for more information about how to request the walltime.
It is recommended to use \emph{checkpointing} if the job requires \strong{72 hours} of walltime or more to be executed.
% FIXME: Refer to Checkpointing section.



\section{Out of quota issues}

Sometimes a job hangs at some point or it stops writing in the disk. These errors are usually
related to the quota usage. You may have reached your quota limit at some storage endpoint.
You should move (or remove) the data to a different storage endpoint (or request more quota) to be able to write to the disk and then resubmit the jobs.
\ifgent
Another option is to request extra quota for your VO to the VO moderator/s.
See section~\ref{subsec:predefined-user-directories} and section~\ref{subsec:predfined-quotas} for more information about
quotas and how to use the storage endpoints in an efficient way.
\fi
% FIXME: Add how to request quota section

\section{DOS/Windows text format}

If you get errors like:

\begin{prompt}
%\shellcmd{qsub fibo.pbs}%
qsub:  script is written in DOS/Windows text format
\end{prompt}

It's probably because you transferred the files from a Windows computer.
Please go to the section about \verb|dos2unix| in \href{\LinuxManualURL#sec:dos2unix}{chapter 5 of the intro to Linux}
to fix this error.

\section{Warning message when first connecting to new host}
\label{sec:warning-message-new-host}

\ifmacORlinux
\begin{prompt}
%\shellcmd{ssh \userid{}@\loginnode{}}%
The authenticity of host %\loginnode% (<IP-adress>)
can't be established.
%\underline{<algorithm> key fingerprint is <hash>}%
Are you sure you want to continue connecting (yes/no)?
\end{prompt}

Now you can check the authenticity by checking if the line that is at the place
of the underlined piece of text matches one of the following lines:
\begin{prompt}
%\opensshFirstConnect%
\end{prompt}

If it does, type  \strong{\emph{yes}}. If it doesn't, please contact support: \hpcinfo.
\else
\firsttimeconnection
\fi
