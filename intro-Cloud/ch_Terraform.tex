\chapter{Orchestration Using Terraform}\label{cha:orch-using-terraform}

HashiCorp \gls{Terraform} \url{https://www.terraform.io/} is an infrastructure
as code tool (IaC), similar to OpenStack \gls{Heat} orchestrator
(See chapter \ref{cha:orch-using-heat}).
Users can deploy a data center infrastructure using a declarative
configuration language known as HashiCorp Configuration Language (HCL), or using JSON.
\gls{Terraform} has some advantages over OpenStack \gls{Heat} service.
It is has a simple syntax, it can provision virtual infrastructures across multiple cloud
providers (not only OpenStack) and it provides important features not supported
by \gls{Heat} at this moment, like network port forwarding rules (see \ref{sec:floating-ip}).
This means that with Terraform, scripts like \lstinline{neutron_port_forward}
(see \ref{port-forwarding}) are no longer needed.

\gls{Terraform} is currently one of the most popular infrastructure automation tools available.
VSC Cloud also provides some template examples that could be used to deploy virtual
infrastructures within VSC Tier-1 Cloud in an automated way
(\url{https://github.com/hpcugent/openstack-templates/tree/master/terraform}).

\gls{Terraform} client is available for different Operating Systems like Windows,
Linux or macOS (\url{https://www.terraform.io/downloads}) but it is also available from
UGent login node \lstinline{login.hpc.ugent.be}.

\section{Create application credentials for Terraform}\label{sec:app-cred-terraform}

\gls{Terraform} uses OpenStack application credentials to authenticate
to VSC Cloud Tier-1 public API. It is a good practice to generate a new application credential
just to be used with \gls{Terraform} framework. The process is the same described in section \ref{sec:appl-cred}.

\strong{Note:} Make sure you download the new application credential as yaml file instead of openRC.

At this point you should have a clouds.yaml text file with these lines:
\begin{code}{}
# This is a clouds.yaml file, which can be used by OpenStack tools as a source
# of configuration on how to connect to a cloud. If this is your only cloud,
# just put this file in ~/.config/openstack/clouds.yaml and tools like
# python-openstackclient will just work with no further config. (You will need
# to add your password to the auth section)
# If you have more than one cloud account, add the cloud entry to the clouds
# section of your existing file and you can refer to them by name with
# OS_CLOUD=openstack or --os-cloud=openstack
clouds:
  openstack:
    
    auth:
      
      auth_url: https://cloud.vscentrum.be:13000
      
      application_credential_id: "xxxxxxxxxxxxxxxxxxxxxxxxx"
      application_credential_secret: "xxxxxxxxxxxxxxxxxxxxx"
    
      
        
    region_name: "regionOne"
        
      
    interface: "public"
    identity_api_version: 3
    auth_type: "v3applicationcredential"
\end{code}

As the file comments state, you should copy the current clouds.yaml to your VSC login node \$HOME
\lstinline{login.hpc.ugent.be}: \texttt{\emph{\~\//.config/openstack/clouds.yaml}}, or locally
if you have installed \gls{Terraform} in your own laptop or computer.
\gls{Terraform} will use this file to authenticate to OpenStack API automatically.


\section{Getting Terraform examples}\label{sec:getting-terraform-templ}

You can connect to UGent login node \lstinline{login.hpc.ugent.be} to use terraform.
Login to the login node with your VSC account first:

\begin{prompt}
%\shellcmd{ssh -A vscxxxxx@login.hpc.ugent.be}
\end{prompt}

If this is the first time using Terraform, download the VSC Terraform examples from github from
\url{https://github.com/hpcugent/openstack-templates}:

\begin{prompt}
%\shellcmd{git clone https://github.com/hpcugent/openstack-templates}
\end{prompt}

Make sure you have \texttt{\emph{\~\//.config/openstack/clouds.yaml}} available from the login node
(see previous section \ref{sec:app-cred-terraform}).

\strong{Note:} Do not share your application credential file clouds.yaml or put this file in a public place.

\begin{prompt}
%\shellcmd{chmod 600 ~/.config/openstack/clouds.yaml}
\end{prompt}

\section{Generate Terraform template variables}\label{sec:generate-terraform-variables}
\gls{Terraform} requires some variables to know which resources are available from the cloud provider for the user or project.
You do not have to include these variables manually, we have included a script to gather these variable IDs automatically.
From the Terraform directory cloned from git in the previous step (see: \ref{sec:getting-terraform-templ}),
go to the scripts directory:

\begin{prompt}
%\shellcmd{cd ~/openstack-templates/terraform/scripts}
\end{prompt}

And now run the script (usually you only have to run this script once).

\begin{prompt}
%\shellcmd{./modify\_variable.sh}
\end{prompt}

This step will take some seconds. The script will contact the VSC OpenStack public API to gather
all the resources available for your project and fetch all the resource's IDs.
Usually you only have to run this script once, unless something was changed/updated for your project's
resources (like a new network or floating IPs) or if you want to deploy a new
Terraform template from scratch.

You will see some messages like this (IDs and IPs may change depending on your project's resources).

\begin{prompt}
Variable OS_CLOUD is not set. Using openstack as a value.
Image id: 749f4f24-7222-45fc-b571-996f5b68c28f. (Image name: CentOS-8-stream)
Flavor name: CPUv1.small.
Root FS volume size based on flavor disk size: 20.
VM network id: 4d72c0ec-c000-429e-89c6-8c3607a28b3d.
VM subnet id: f0bc8307-568f-457d-adff-219005a054e2.
NFS network id: 119d8617-4000-47c0-9c6e-589b3afce144.
NFS subnet id: e4e07edd-39cf-42ea-9fe4-5bf2891d2592.
VSC network id: f6eba915-06ad-4e50-bc4b-1538cdc39296.
VSC subnet id: b5ed8dc2-6d3f-42d4-87f8-3ffee19c1a9c.
Using first ssh access key "ssh-ed25519 AAAAC3Nz_A02TxLd9 lsimngar_varolaptop".
Using floating ip id: 64f2705c-43ec-4bdf-864e-d18fee013e3f. (floating ip: 193.190.80.3)
Using VSC floating ip: 172.24.49.7.
Using ssh forwarded ports: 56469 59112 54872 51280.
Using http forwarded port: 52247.
Modifying ../environment/main.tf file.
Modifying provider.tf files.
SSH commands for VMs access:
(myvm)           ssh -p 56469 <user>@193.190.80.3
(myvm-nginx)     ssh -p 59112 <user>@193.190.80.3
(myvm-vsc_net)   ssh -p 54872 <user>@193.190.80.3
(myvm-nfs_share) ssh -p 51280 <user>@193.190.80.3
\end{prompt}

After this step your Terraform templates will be ready to be deployed.

\strong{Note:} Please note that the script shows you the ssh command to connect to each VM
after instantiation (including the port which is generated automatically by the script).
You can copy this list or you can review it later. Also note that you should use a valid user
to connect to the VM, for instance for CentOS images is \emph{centos},
for Ubuntu images is \emph{ubuntu} and so on.
You can also try to connect as \emph{root} user, in that case the system will show you a message
with the user that you should use.

\section{Modify default Terraform modules}\label{sec:modify-terraform-modules}
In section \ref{sec:getting-terraform-templ} we have downloaded the Terraform module examples from
the VSC repository.
If you deploy these modules as it is it will deploy several VM examples by default such as:


\begin{enumerate}
\item \strong{myvm:} simple VM with 20Gb persistent volume and ssh access with port forwarding.
\item \strong{myvm-nginx:} Like previous example but with an ansible playbook to install
nginx and access to port 80 besides ssh.
\item \strong{myvm-vsc\_net:} Similar to the first example but also includes a VSC network
interface (only available for some projects).
\item \strong{myvm-nfs\_share:} Similar to the first example but it creates a NFS share
filesystem and it mounts it during instantiation (only available for some projects).
\end{enumerate}


But usually you do not want to deploy all these examples, you can just keep the required module
and comment out the rest. You can do this from environment directory:

\begin{prompt}
%\shellcmd{cd ~/openstack-templates/terraform/environment}
\end{prompt}

And edit \texttt{\emph{main.tf}} Terraform file with any text editor like vim or nano.
If you want to deploy just the simple VM (first example) only keep these lines (remenber variable
IDs may change depending on your project's resources):

\begin{code}{}
 module "vm_with_pf_rules_with_ssh_access" {
  source   = "../modules/vm_with_pf_rules_with_ssh_access"

  vm_name              = "MyVM"
  floating_ip_id       = "64f2705c-43ec-4bdf-864e-d18fee013e3f"
  vm_network_id        = "4d72c0ec-c000-429e-89c6-8c3607a28b3d"
  vm_subnet_id         = "f0bc8307-568f-457d-adff-219005a054e2"
  access_key           = "ssh-ed25519 AAAAC3Nz_A02TxLd9 lsimngar_varolaptop"
  image_id             = "749f4f24-7222-45fc-b571-996f5b68c28f"
  flavor_name          = "CPUv1.small"
  ssh_forwarded_port   = "56469"
  root_fs_volume_size  = "20"
}
\end{code}

And remove or comment out the rest of the lines.
In the previous example Terraform will deploy a simple VM and use 20Gb for a persistent volume
and port 56469 to connect via ssh (it also creates all required security groups).

\section{Deploy Terraform templates}\label{sec:deploy-terraform-templates}
If you have followed the previous steps now you can init and deploy your infrastucture to
Tier-1 VSC cloud.

You have to inititate Terraform first, if you didnt have deployed any template yet do this just once.

Move to environment directory first:

\begin{prompt}
%\shellcmd{cd ~/openstack-templates/terraform/environment}
\end{prompt}

This command performs several different initialization steps in order to prepare the
current working directory for use with Terraform:


\begin{prompt}
%\shellcmd{terraform init}
\end{prompt}

Now you can check and review your Terraform plan, from the same directory:

\begin{prompt}
%\shellcmd{terraform plan}
\end{prompt}

You will see a list of the resources required to deploy your infrastructure, Terraform also
checks if there is any systax error in your templates.
Your infrastructure is not deployed yet, review the plan and then just deploy it to VSC Tier-1 Cloud running:

\begin{prompt}
%\shellcmd{terraform apply}
\end{prompt}

Terraform will show your plan again and you will see this message:

\begin{prompt}
..
..
Do you want to perform these actions?
  Terraform will perform the actions described above.
  Only 'yes' will be accepted to approve.

  Enter a value: 
\end{prompt}

Type \texttt{\emph{yes}} and press enter and wait a few seconds or minutes.
If everything is correct and if you have enough quota Terraform will show you a message
after creating all the required resources.


\begin{prompt}
..
..
module.vm_with_pf_rules_with_ssh_access.openstack_compute_instance_v2.instance_01: Still creating... [1m30s elapsed]
module.vm_with_pf_rules_with_ssh_access.openstack_compute_instance_v2.instance_01: Creation complete after 1m35s [id=88c7d037-5c44-45b7-acce-f5e4e58b1c35]

Apply complete! Resources: 4 added, 0 changed, 0 destroyed.
\end{prompt}

Your cloud infrastrucuture is ready to be used.

It is important to keep a backup of your terraform directory, specially all the files
within the environment directory:

\begin{prompt}
%\shellcmd{~/openstack-templates/terraform/environment}
\end{prompt}

Terraform generates several files in this directory to keep track of any change in your infrastructure.
If for some reason you lost or remove these files you will not able to modify or change the
current Terraform plan (only directly from OpenStack).

You can also modify and add more resources for the current templates. This task is out of the scope
of this document, please refer to official Terraform documentation to add you own changes
\url{https://www.terraform.io/docs} or ask to VSC Cloud admins via email at \cloudinfo.

