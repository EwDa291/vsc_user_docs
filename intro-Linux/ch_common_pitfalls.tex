\chapter{Common Pitfalls}

\subsection{Files}

\subsubsection{Location}
If you receive an error message which contains something like the following:

\begin{prompt}
No such file or directory
\end{prompt}

It probably means that you haven't placed your files in the correct directory or
you have mistyped the file name or path.

Try and figure out the correct location using \lstinline|ls|, \lstinline|cd| and using the
different \lstinline|$VSC_*| variables.

\subsubsection{Spaces}

Filenames should \strong{not} contain any spaces! If you have a long filename
you should use underscores or dashes (e.g., \lstinline|very_long_filename|).

\begin{prompt}
%\shellcmd{cat some file}%
No such file or directory 'some'
\end{prompt}

Spaces are permitted, however they result in surprising behaviour. To cat the
file \lstinline|'some file'| as above, you can escape the space with a backslash (``\lstinline|\ |'')
or you can put the filename in quotes:

\begin{prompt}
%\shellcmd{cat some\textbackslash{} file}%
...
%\shellcmd{cat "some file"}%
...
\end{prompt}

This is especially error prone if you are piping results of \lstinline|find|:

\begin{prompt}
%\shellcmd{find . -type f | xargs cat}%
No such file or directory name 'some'
No such file or directory name 'file'
\end{prompt}

This can be worked around using the \lstinline|-print0| flag:

\begin{prompt}
%\shellcmd{find . -type f -print0 | xargs -0 cat}%
...
\end{prompt}

But, this is tedious and you can prevent errors by simply colouring within the
lines and not using spaces in filenames.

\subsubsection{Missing/mistyped environment variables}

If you use a command like \lstinline|rm -r| with environment variables you need to be
careful to make sure that the environment variable exists. If you mistype an
environment variable then it will resolve to a blank string. This means the
following resolves to \lstinline|rm -r ~/*| which will remove every file in your home
directory!

\begin{prompt}
%\shellcmd{rm -r ~/\$PROJETC/*}%
\end{prompt}

\subsubsection{Typing dangerous commands}

A good habit when typing dangerous commands is to precede the line with \lstinline|#|,
the comment character. This will let you type out the command without fear of
accidentally hitting enter and running something unintended.

\begin{prompt}
%\shellcmd{\#rm -r ~/\$POROJETC/*}%
\end{prompt}

Then you can go back to the beginning of the line (\lstinline|Ctrl-A|) and remove the
first character (\lstinline|Ctrl-D|) to run the command. You can also just press enter
to put the command in your history so you can come back to it later (e.g., while
you go check the spelling of your environment variables).

\ifwindows
\subsubsection{Copying files with WinSCP}

After copying files from a windows machine, a file might look funny when looking
at it on the cluster.

\begin{prompt}
%\shellcmd{cat script.sh}%
#!/bin/bash^M
#PBS -l nodes^M
...
\end{prompt}

Or you can get errors like:

\begin{prompt}
%\shellcmd{qsub fibo.pbs}%
qsub:  script is written in DOS/Windows text format
\end{prompt}

See section \ref{subsec:dos2unix} to fix these errors with \lstinline|dos2unix|.

\fi
\subsubsection{Permissions}

\begin{prompt}
%\shellcmd{ls -l script.sh \# File with correct permissions}%
-rwxr-xr-x 1 %\userid{}% %\userid{}% 2983 Jan 30 09:13 script.sh
%\shellcmd{ls -l script.sh \# File with incorrect permissions}%
-rw-r--r-- 1 %\userid{}% %\userid{}% 2983 Jan 30 09:13 script.sh
\end{prompt}

Before submitting the script, you'll need to add execute permissions
to make sure it can be executed:

\begin{prompt}
%\shellcmd{chmod +x script\_name.sh}%
\end{prompt}



\subsection{Help\!}

If you stumble upon an error, don't panic! Read the error output, it might
contain a clue as to what went wrong. You can copy the error message into Google
(selecting a small part of the error without filenames). It can help if you
surround your search terms in double quotes (for example \lstinline|"No such file or directory"|),
that way Google will consider the error as one thing, and won't show results just containing
these words in random order.

If you need help about a certain command, you should consult its so called
``man page'':

\begin{prompt}
%\shellcmd{man command}%
\end{prompt}

This will open the manual of this command. This manual contains detailed explanation of
all the options the command has. Exiting the manual is done by pressing 'q'.

\strong{Don't be afraid to contact \hpcinfo. They are here to help and will do so for even the smallest of problems!}

\chapter{More information}

\begin{enumerate}
 \item \href{http://www.docstore.mik.ua/orelly/unix/upt/index.htm}{Unix Power
 Tools - A \strong{fantastic} book about most of these tools} (see also
 \href{http://www.docstore.mik.ua/orelly/unix2.1/index.htm}{The Second Edition})
 \item \url{http://linuxcommand.org/}: A great place to start with many examples. There is an associated book which gets a lot of good reviews
 \item \href{http://www.tldp.org/guides.html}{The Linux Documentation Project} - More guides on various topics relating to the Linux command line
 \item \href{http://linuxcommand.org/lc3_learning_the_shell.php}{basic shell usage}
 \item \href{http://www.tldp.org/LDP/Bash-Beginners-Guide/html/Bash-Beginners-Guide.html}{Bash for beginners}
 \item \href{https://www.edx.org/course/introduction-linux-linuxfoundationx-lfs101x-0}{MOOC}
\end{enumerate}

\chapter{Q \& A}

Please don't hesitate to contact \hpcinfo in case of questions or problems.
