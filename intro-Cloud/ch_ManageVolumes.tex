\chapter{Create and manage volumes}
An \gls{OpenStack Volume} is a block storage device which you attach
to instances to enable persistent storage. You can attach a volume to
a running instance or detach a volume and attach it to another
instance at any time. You can also create a snapshot from or delete a
volume. Only administrative users can create volume types.

\subsubsection{Create a volume}\label{create-a-volume}
\begin{enumerate}
\item Open the Volumes tab and click Volumes category.
\item Click Create Volume.

  In the dialog box that opens, enter or select the following values.

  \begin{description}
  \item[Volume Name] Specify a name for the volume.
  \item[Description] Optionally, provide a brief description for the volume.
  \item[Volume Source] Select one of the following options:

    \begin{itemize}
    \item No source, empty volume: Creates an empty volume. An empty
      volume does not contain a file system or a partition table.
    \item Snapshot: If you choose this option, a new field for Use
      snapshot as a source displays. You can select the snapshot from
      the list.
    \item Image: If you choose this option, a new field for Use image
      as a source displays. You can select the image from the list.
    \item Volume: If you choose this option, a new field for Use
      volume as a source displays. You can select the volume from the
      list. Options to use a snapshot or a volume as the source for a
      volume are displayed only if there are existing snapshots or
      volumes.
    \end{itemize}
  \item[Type] Leave this field blank.
  \item[Size (GB)] The size of the volume in gibibytes (GiB).
  \item[Availability Zone] Select the Availability Zone from the
    list. By default, this value is set to the availability zone given
    by the cloud provider (for example, \strong{us-west} or
    \strong{apac-south}). For some cases, it could be \strong{nova}.
  \end{description}
\item Click Create Volume.
\end{enumerate}

The dashboard shows the volume on the Volumes tab.

\subsubsection{Attach a volume to an
  instance}\label{attach-a-volume-to-an-instance}
After you create one or more volumes, you can attach them to
instances.  You can attach a volume to one instance at a time.

\begin{enumerate}
\item Open the Volumes tab and click Volumes
  category.
\item Select the volume to add to an instance and click Manage
  Attachments.
\item In the Manage Volume Attachments dialog box, select an instance.
\item Enter the name of the device from which the volume is accessible
  by the instance.

  \strong{Note:} The actual device name might differ from the volume
  name because of hypervisor settings.
\item Click Attach Volume.

  The dashboard shows the instance to which the volume is now attached
  and the device name.
\end{enumerate}
You can view the status of a volume in the Volumes tab of the dashboard.
The volume is either Available or In-Use.

Now you can log in to the instance and mount, format, and use the disk.

\subsubsection{Detach a volume from an instance}\label{detach-a-volume-from-an-instance}
\begin{enumerate}
\item Open the Volumes tab and click the Volumes
  category.
  \item Select the volume and click Manage Attachments.
  \item Click Detach Volume and confirm your changes.
  \end{enumerate}

A message indicates whether the action was successful.

\subsubsection{Create a snapshot from a
  volume}\label{create-a-snapshot-from-a-volume}
\begin{enumerate}
\item Open the Volumes tab and click Volumes category.
\item Select a volume from which to create a snapshot.
\item In the Actions column, click Create Snapshot.
\item In the dialog box that opens, enter a snapshot name and a brief
  description.
\item Confirm your changes.
\end{enumerate}

The dashboard shows the new volume snapshot in Volume Snapshots tab.

\subsubsection{Edit a volume}\label{edit-a-volume}
\begin{enumerate}
\def\labelenumi{\arabic{enumi}.}
\item Open the Volumes tab and click Volumes category.
\item Select the volume that you want to edit.
\item In the Actions column, click Edit Volume.
\item In the Edit Volume dialog box, update the name and description
  of the volume.
\item Click Edit Volume.
\end{enumerate}

\strong{Note:} You can extend a volume by using the Extend Volume
option available in the More dropdown list and entering the new value
for volume size.

\subsubsection{Delete a volume}\label{delete-a-volume}
When you delete an instance, the data in its attached volumes is not
deleted.

\begin{enumerate}
\item Open the Volumes tab and click Volumes
  category.
\item Select the check boxes for the volumes that you want to delete.
\item Click Delete Volumes and confirm your choice.
\end{enumerate}

A message indicates whether the action was successful.

%%% Local Variables:
%%% mode: latex
%%% TeX-master: "intro-Cloud"
%%% End:
