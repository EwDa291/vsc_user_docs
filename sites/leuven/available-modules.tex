\begin{prompt}
%\shellcmd{module av}%
------------------- /apps/leuven/thinking/2014a/modules/all --------------------
Abaqus/6.12-2
Abaqus/6.13-3
Abaqus/6.14-1
accounting
allinea-ddt/4.2
annovar/2013Jun21
ANSYS/13.0
ANSYS/14.5
ANSYS/15.0
ANSYS/17.0
%\dots{}%
\end{prompt}

``module av'' is an abbreviation for ``module available''.

When you want to switch to the more recent versions of the compilers you can check the software:

\begin{prompt}
%\shellcmd{source switch\_to\_2015a; module av 2>\&1 | more}%
------------------- /apps/leuven/thinking/2015a/modules/all --------------------
ABySS/1.3.7-intel-2015a-Python-2.7.9
ABySS/1.9.0-intel-2015a-Python-2.7.9
accounting
allinea-forge/5.0.1
allinea-forge/5.1
ANSYS/16.2
ant/1.9.4-Java-1.8.0_31
APBS/1.4.1-foss-2015a
APBS/1.4.1-intel-2015a
APBS/1.4-linux-static-x86_64
--More--
\end{prompt}

Additionally, different software is installed on the shared memory cluster, GPU cluster and Haswell nodes:

\begin{prompt}
%\shellcmd{module av}%
%\dots{}%
-------------------------- /apps/leuven/etc/modules/ ---------------------------
cerebro/2014a   K20Xm/2014a     M2070/2014a     thinking/2014a
ictstest/2014a  K40c/2014a      phi/2014a       thinking2/2014a
%\end{prompt}

When you want to check whether some specific software, some compiler or some
application (e.g., worker) is installed on the \hpc.

\begin{prompt}
%module av 2>\&1 | grep -i -e "worker"%
worker/1.4.2-foss-2014a
worker/1.5.0-intel-2014a
worker/1.5.1-intel-2014a
worker/1.5.2-intel-2014a
worker/1.5.3-intel-2014a
\end{prompt}

As you are not aware of the capitals letters in the module name, we looked for a case-insensitive name with the ``-i'' option.
