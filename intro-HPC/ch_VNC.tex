\chapter{Graphical applications with VNC}
\label{ch:vnc}

\ifgent
\strong{VNC is still available at UGent site but we encorage our users to replace VNC by X2Go client}.
Please see \autoref{ch:x2go} for more information.
\fi

Virtual Network Computing is a graphical desktop sharing system that enables you
to interact with graphical software running on the HPC infrastructure from your own
computer.

\strong{Please carefully follow the instructions below, since the procedure to
connect to a VNC server running on the HPC infrastructure is not trivial, due
to security constraints.}

\section{Starting a VNC server}
\label{sec:start-vnc}

First login on the login node (see \autoref{sec:first-time-connection-to-the-hpc}),
then start \lstinline|vncserver| with:

\begin{prompt}
%\shellcmd{vncserver -geometry 1920x1080 -localhost}%
You will require a password to access your desktops.

Password:%\emph{<{}enter a secure password>{}}%
Verify:%\emph{<{}enter the same password>{}}%
Would you like to enter a view-only password (y/n)? %\emph{n}%
A view-only password is not used

New '%\strong{\loginhost{}}%:%\strong{6}% (%\userid{}%)' desktop is %\loginhost{}%:6

Creating default startup script %\homedir{}%/.vnc/xstartup
Creating default config %\homedir{}%/.vnc/config
Starting applications specified in %\homedir{}%/.vnc/xstartup
Log file is %\homedir{}%/.vnc/%\loginhost{}%:6.log

\end{prompt}

\strong{When prompted for a password, make sure to enter a secure password: if someone
can guess your password, they will be able to do anything with your account you can!}

Note down the details in bold: the hostname (in the example: \texttt{\loginhost{}})
and the (partial) port number (in the example: \lstinline|6|).

It's important to remember that VNC sessions are permanent. They survive network
problems and (unintended) connection loss. This means you can logout and go home
without a problem (like the terminal equivalent \lstinline|screen| or \lstinline|tmux|).
This also means you don't have to start \lstinline|vncserver| each time you want to connect.

\section{List running VNC servers}
\label{sec:list-vnc}

You can get a list of running VNC servers on a node with

\begin{prompt}
%\shellcmd{vncserver -list}%
TigerVNC server sessions:

X DISPLAY #	PROCESS ID
:6		    30713
\end{prompt}

This only displays the running VNC servers on \strong{the login node you run the command on}.

To see what login nodes you are running a VNC server on, you can run the \lstinline|ls .vnc/*.pid|
command in your home directory: the files shown have the hostname of the login node in the filename:

\begin{prompt}
%\shellcmd{cd \$HOME}%
%\shellcmd{ls .vnc/*.pid}%
.vnc/%\loginhost{}%:6.pid
.vnc/%\altloginhost{}%:8.pid
\end{prompt}

This shows that there is a VNC server running on \texttt{\loginhost{}} on port 5906
and another one running \texttt{\altloginhost{}} on port 5908 (see also \autoref{sec:source-port-vnc}).

\section{Connecting to a VNC server}

The VNC server runs on a \strong{specific login node} (in the example above, on \texttt{\loginhost{}}).

In order to access your VNC server, you will need to set up an SSH tunnel from your workstation
to this login node (see \autoref{sec:ssh-tunnel-vnc}).

Login nodes are rebooted from time to time. You can check that the VNC server is still
running in the same node by executing \lstinline|vncserver -list| (see also \autoref{sec:list-vnc}).
If you get an empty list, it means that there is no VNC server running on the login node.

\strong{To set up the SSH tunnel required to connect to your VNC server, you will need to port forward the VNC port
to your workstation.}

The \emph{host} is \lstinline|localhost|, which means ``your own computer'': we set up an SSH tunnel that connects
the VNC port on the login node to the same port on your local computer.

\subsection{Determining the source/destination port}
\label{sec:source-port-vnc}

The \emph{destination port} is the port on which the VNC server is running (on the login node),
which is \strong{the sum of \lstinline|5900| and the partial port number} we noted down earlier (\lstinline|6|);
in the running example, that is \lstinline|5906|.

The \emph{source port} is the port you will be connecting to with your VNC client on your workstation.
Although you can use any (free) port for this, we strongly recommend to use the \strong{same value as the destination port}.

So, in our running example, both the source and destination ports are \lstinline|5906|.

\subsection{Picking an intermediate port to connect to the right login node}
\label{sec:intermediate-port-vnc}

In general, you have no control over which login node you will be on when setting up the SSH tunnel from
your workstation to \texttt{\loginnode{}} (see \autoref{sec:ssh-tunnel-vnc}).

If the login node you end up on is a different one than the one where your VNC server is running
(i.e., \texttt{\altloginhost} rather than \texttt{\loginhost} in our running example),
you need to create a \strong{\emph{second} SSH tunnel} on the login node you are connected to,
in order to "patch through" to the correct port on the login node where your VNC server is running.

In the remainder of these instructions, we will assume that we are indeed connected to a different login node.
Following these instructions should always work, even if you happen to be connected to the correct login node.

To set up the second SSH tunnel, you need to \strong{pick an (unused) port on the login node you are connected to},
which will be used as an \emph{intermediate} port.

Now we have a chicken-egg situation: you need to pick a port before setting up the SSH tunnel from your workstation
to \texttt{\loginhost}, but only after starting the SSH tunnel will you be able to determine whether the port you
picked is actually free or not\ldots

In practice, if you \strong{pick a \strong{random} number between $10000$ and $30000$}, you have a good chance that the
port will not be used yet.

We will proceed with $12345$ as intermediate port, but \strong{you should pick another value that
other people are not likely to pick}. If you need some inspiration, run the following command on a Linux server
(for example on a login node): \texttt{echo \$RANDOM} (but do not use a value lower than $1025$).

\subsection{Setting up the SSH tunnel(s)}
\label{sec:ssh-tunnel-vnc}

\subsubsection{Setting up the first SSH tunnel from your workstation to \loginnode}

First, we will set up the SSH tunnel from our workstation to \loginnode.

Use the settings specified in the sections above:

\begin{itemize}
\item \emph{source port}: the port on which the VNC server is running (see \autoref{sec:source-port-vnc});
\item \emph{destination host}: \lstinline|localhost|;
\item \emph{destination port}: use the intermediate port you picked (see \autoref{sec:intermediate-port-vnc})
\end{itemize}

\ifwindows

See \autoref{par:ssh-tunnel-windows} for detailed information on how to configure PuTTY to set up the SSH tunnel,
by entering the settings in the \menu{Source port} and \menu{Destination} fields in \menu[,]{Connection,SSH,Tunnels}.

\else

Execute the following command to set up the SSH tunnel.\\

\begin{prompt}
%\shellcmd{ssh -L 5906:localhost:12345 \userid{}@\loginnode{}}%
\end{prompt}

\strong{Replace the source port (\lstinline|5906|, destination port (\lstinline|12345|)
and the user ID (\texttt{\userid{}}) with your own!}

\fi

With this, we have forwarded port \lstinline|5906| on our workstation to port \lstinline|12345|
on the login node we are connected to.

\strong{Again, do \emph{not} use \lstinline|12345| as destination port,
as this port will most likely be used by somebody else already;
replace it with a port number you picked yourself, which is unlikely to be used already
(see \autoref{sec:intermediate-port-vnc}).}

\subsubsection{Checking whether the intermediate port is available}

Before continuing, it's good to check whether the intermediate port that you have picked
is actually still available (see \autoref{sec:intermediate-port-vnc}).

You can check using the following command (\strong{do not forget to replace \lstinline|12345| the
value you picked for your intermediate port}):

\begin{prompt}
%\shellcmd{netstat -an | grep -i listen | grep tcp | grep 12345}%
%\shellcmd{}%
\end{prompt}

If you see no matching lines, then the port you picked is still available, and you can continue.

If you see one or more matching lines as shown below, \strong{you must disconnect the first SSH tunnel,
pick a different intermedate port, and set up the first SSH tunnel again using the new value}.

\begin{prompt}
%\shellcmd{netstat -an | grep -i listen | grep tcp | grep 12345}%
%tcp        0      0 0.0.0.0:12345           0.0.0.0:*               LISTEN%
%tcp6       0      0 :::12345                :::*                    LISTEN%
%\shellcmd{}%
\end{prompt}

\subsubsection{Setting up the second SSH tunnel to the correct login node}

In the session on the login node you created by setting up an SSH tunnel from your workstation
to \texttt{\loginnode}, you now need to set up the second SSH tunnel to "patch through" to the login
node where your VNC server is running (\texttt{\loginhost} in our running example, see \autoref{sec:start-vnc}).

To do this, run the following command:

\begin{prompt}
%\shellcmd{ssh -L 12345:localhost:5906 \loginhost{}}%
%\shellcmd{hostname}%
%\loginhost{}%
\end{prompt}

With this, we are forwarding port \lstinline|12345| on the login node we are connected to (which is referred
to as \lstinline|localhost|) through to port \lstinline|5906| on our target login node (\texttt{\loginhost}).

Combined with the first SSH tunnel, port \lstinline|5906| on our workstation is now connected to
port \lstinline|5906| on the login node where oru VNC server is running (via the intermediate port \lstinline|12345|
on the login node we ended up one with the first SSH tunnel).

\strong{Do not forget to change the intermediate port (\lstinline|12345|), destination port (\lstinline|5906|),
and hostname of the login node (\texttt{\loginhost}) in the command shown above!}

As shown above, you can check again using the \lstinline|hostname| command whether you are indeed connected
to the right login node. If so, you can go ahead and connect to your VNC server (see \autoref{sec:vnc-client}).

\subsection{Connecting using a VNC client}
\label{sec:vnc-client}

\ifwindows

You can download a free VNC client from \url{https://sourceforge.net/projects/turbovnc/files/}.
You can download the latest version by clicking the top-most folder that has a version number
in it that doesn't also have \lstinline|beta| in the version. Then download a file that looks like
\lstinline|TurboVNC64-2.1.2.exe| (the version number can be different, but the \lstinline|64|
should be in the filename) and execute it.
\fi
\ifmac
You can download a free VNC client from \url{https://sourceforge.net/projects/turbovnc/files/}.
You can download the latest version by clicking the top-most folder that has a version number
in it that doesn't also have \lstinline|beta| in the version. Then download a file ending in
\lstinline|TurboVNC64-2.1.2.dmg| (the version number can be different) and execute it.
\fi
\iflinux
Download and setup a VNC client. A good choice is \lstinline|tigervnc|. You can start
it with the \lstinline|vncviewer| command.
\fi

Now start your VNC client and connect to \lstinline|localhost:5906|.
\strong{Make sure you replace the port number \lstinline|5906| with your own destination port}
(see \autoref{sec:source-port-vnc}).

When prompted for a password, use the password you used to setup the VNC server.

When prompted for default or empty panel, choose default.

If you have an empty panel, you can reset your settings with the following commands:

\begin{prompt}
%\shellcmd{xfce4-panel --quit ; pkill xfconfd}%
%\shellcmd{mkdir ~/.oldxfcesettings}%
%\shellcmd{mv ~/.config/xfce4 ~/.oldxfcesettings}%
%\shellcmd{xfce4-panel}%
\end{prompt}

\section{Stopping the VNC server}
\label{sec:stop-vnc}

The VNC server can be killed by running

\begin{prompt}
vncserver -kill :6
\end{prompt}

where \lstinline|6| is the port number we noted down earlier. If you forgot,
you can get it with \lstinline|vncserver -list| (see \autoref{sec:list-vnc}).

\section{I forgot the password, what now?}

You can reset the password by first stopping the VNC server (see \autoref{sec:stop-vnc}),
then removing the \lstinline|.vnc/passwd| file (with \lstinline|rm .vnc/passwd|) and then
starting the VNC server again (see \autoref{sec:start-vnc}).
