\chapter{Graphical applications with VNC}
\label{ch:vnc}

Virtual Network Computing is a graphical desktop sharing system that enables you
to interact with graphical software running on HPC infrastructure from your own
computer.

\section{Start the VNC server}
\label{sec:start-vnc}

First login on the login node (see \autoref{sec:first-time-connection-to-the-hpc}),
then start \lstinline|vncserver| with:

\begin{prompt}
%\shellcmd{vncserver -geometry 1920x1080 -localhost}%
You will require a password to access your desktops.

Password:%\emph{<{}enter a secure password>{}}%
Verify:%\emph{<{}enter the same password>{}}%
Would you like to enter a view-only password (y/n)? %\emph{n}%
A view-only password is not used

New '%\strong{login02}%.login.os:%\strong{6}% (%\userid{}%)' desktop is login02.login.os:6

Creating default startup script %\homedir{}%/.vnc/xstartup
Creating default config %\homedir{}%/.vnc/config
Starting applications specified in %\homedir{}%/.vnc/xstartup
Log file is %\homedir{}%/.vnc/login02.login.os:6.log

\end{prompt}

When prompted for a password, make sure to enter a secure password: if someone
can guess your password, they will be able to do anything with your account you can.

Note down the details in bold: the hostname (in the example \lstinline|login02|)
and the port number (in the example \lstinline|6|).

It's important to remember that VNC sessions are permanent. They survive network
problems and (unintended) connection loss. This means you can logout and go home
without a problem (like the terminal equivalent \lstinline|screen| or \lstinline|tmux|).
This also means you don't have to start \lstinline|vncserver| each time you want to connect.

\section{Connecting to the VNC server}

\subsection{VNC on the login nodes}

The VNC server runs on a \strong{specific login node} (in this example \lstinline|login02|).
Make sure you connect to this login node: the domain should be like \lstinline|login02.institute.be|,
but the number can be different. If you're not sure how to do this, please follow the steps
in \autoref{sec:first-time-connection-to-the-hpc}, but replace \loginnode with your specific
domain (here \lstinline|login02.institute.be|).

Login nodes are rebooted from time to time. You can check that the VNC server is still
running in the same note by executing `vncserver -list`. If you get an empty list,
it means that \lstinline|vncserver| is not running. You will need to start it again,
see \autoref{sec:start-vnc}.

You will now need to portforward the VNC port. The source port is the sum of \lstinline|5900|
and the number we noted down earlier. In this case, it would be \lstinline|5906|.
The destination port is the same as the source port. The host is \lstinline|localhost|:
\lstinline|localhost| here means ``your own computer'': we set up an SSH tunnel that would
connect the VNC port on the login node to the same port on your local computer.

\ifwindows
See \autoref{par:ssh-tunnel-windows}. Use the details specified here (host, destination port,
source port).
\else

Execute the following command. Make sure to replace the port numbers, userid and login node
with your own.

\begin{prompt}
%shellcmd{ssh -L 5906:localhost:5906 \userid{}@login02.institute.be}
\end{prompt}
\fi

\ifwindows

You can download a free VNC client from \url{https://sourceforge.net/projects/turbovnc/files/}.
You can download the latest version by clicking the top-most folder that has a version number
in it that doesn't also have \lstinline|beta| in the version. Then download a file that looks like
\lstinline|TurboVNC64-2.1.2.exe| (the version number can be different, but the \lstinline|64|
should be in the filename) and execute it.
\fi
\ifmac
You can download a free VNC client from \url{https://sourceforge.net/projects/turbovnc/files/}.
You can download the latest version by clicking the top-most folder that has a version number
in it that doesn't also have \lstinline|beta| in the version. Then download a file ending in
\lstinline|TurboVNC64-2.1.2.dmg| (the version number can be different) and execute it.
\fi
\iflinux
Download and setup a VNC client. A good choice is \lstinline|tigervnc|. You can start
it with the \lstinline|vncviewer| command.
\fi

Now start your VNC client and connect to \lstinline|localhost:5906|, again replacing
the port with your own.

When promted for a password, use the password you used to setup the VNC server.
When prompted for default or empty panel, choose default.

If you have an empty panel, you can reset your settings with the following commands:

\begin{prompt}
%\shellcmd{xfce4-panel --quit ; pkill xfconfd}%
%\shellcmd{mkdir ~/.oldxfcesettings}%
%\shellcmd{mv ~/.config/xfce4 ~/.oldxfcesettings}%
%\shellcmd{xfce4-panel}%
\end{prompt}

\section{Stopping the VNC server}

The VNC server can be killed by running

\begin{prompt}
vncserver -kill :6
\end{prompt}

where \lstinline|6| is the port number we noted down earlier. If you forgot,
you can get it with \lstinline|vncserver -list|.

\section{List running VNC servers}

You can get a list of running VNC servers on a node with

\begin{prompt}
\shellcmd{vncserver -list}
TigerVNC server sessions:

X DISPLAY #	PROCESS ID
:6		    30713
\end{prompt}

This only displays the running VNC servers on \strong{the node you run the command on}.

To see what login nodes you are running a VNC server on, you can use the \lstinline|ls ~/.vnc/*.pid|
command: the files shown have the hostname of the login node in the filename:

\begin{prompt}
%\shellcmd{find ~/.vnc -iname '*.pid'  -printf "\%{}f\textbackslash{}n"}%
login02.login.os:6.pid
login01.login.os:8.pid
\end{prompt}

This shows that there is a VNC server running on \lstinline|login02| on port 5906
and another one running on \lstinline|login01| on port 5908.

\section{I forgot the password, what now?}

You can reset the password by first stopping the VNC server, then removing
the \lstinline|.vnc/passwd| file (with \lstinline|rm .vnc/passwd|) and then
starting the VNC server again.
