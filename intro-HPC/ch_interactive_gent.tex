\chapter{HPC-UGent interactive and debug cluster}
\label{ch:interactive_ugent}


\section{Purpose}
\label{sec:interactive_ugent_pupose}
The purpose of this cluster is to give the user an environment where
there should be no waiting in the queue to get access to a limited
number of resources. This environment allows a user to immediatelty
start working, and is the ideal place for interactive work such as
development, debugging and light production workloads (typically sufficient
for training and/or courses). \\ This enviroment should be seen as an
extension of the login nodes, instead of a dedicated compute resource.
The interactive cluster is \emph{overcommitted}, which means that more CPU cores can be
requested for jobs than physically exist in the cluster. Obviously, the performance of this cluster
heavily depends on the workloads and the actual overcommit usage. Be aware that jobs can slow
down or speed up during their execution.
 \\
Due to the restrictions and sharing of the CPU resources (see section~\ref{subsec:interactive_ugent_restrictions}) 
jobs on slaking should normally start more or less immediately.
The tradeoff is that performance must not be an issue for the submitted jobs.
This means that typical workloads for this cluster should be limited to:
\begin{itemize}
  \item  Interactive jobs (see chapter~\ref{ch:running-interactive-jobs})
  \item  Cluster desktop sessions (see chapter~\ref{ch:web_portal})
  \item  Jobs requiring few resources
  \item  Debugging programs
  \item  Testing and debuging submit scripts
\end{itemize} 

\section{Submitting jobs}
\label{sec:interactive_ugent_jobs}

To submit jobs to the HPC-UGent interactive and debug cluster nicknamed \lstinline|slaking|, first use:

\begin{prompt}
%\shellcmd{module swap cluster/slaking}
\end{prompt}

Then use the familiar \lstinline|qsub|, \lstinline|qstat|, etc. commands. (see chapter~\ref{ch:running-batch-jobs})

\subsection{Restrictions and overcommit factor}
\label{subsec:interactive_ugent_restrictions}

Some limits are in place for this cluster:
\begin{itemize}
  \item each user may have at most 5 jobs in the queue (both running and waiting to run);
  \item at most 3 jobs per user can be running at the same time;
  \item running jobs may allocate no more than 8 CPU cores and no more than 27200 MiB in total per user;
\end{itemize}

In addition, the cluster has an overcommit factor of 6. This means that 6 times more cores
can be allocated than physically exist. Simultaneously, the default memory per core is 6
times less than what would physically be available on a non-overcommitted cluster.
 \\
Please note that based on the (historical) workload of the interactive and debug cluster, the above
restrictions and the overcommitment ratio might change without prior notice.
