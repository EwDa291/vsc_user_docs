\pagestyle{empty}

\begin{center}

\includegraphics*[width=8truecm]{logo_vsc}

\vspace*{1.5\baselineskip}

\Huge \strong{PerfExpert Tutorial} \\
\LARGE version 1.0

\vspace*{1.5\baselineskip}

\includegraphics*[width=7truecm]{logo_tacc}

\vspace*{0.75\baselineskip}
\ifantwerpen
\includegraphics*[width=7truecm]{logo_ua}
\fi
\vspace*{0.75\baselineskip}


\normalsize\strong{Author:}

Geert Borstlap

\vspace*{.5\baselineskip}

\strong{Written/contributions by:}

James C.\ Browne, Leonardo Fialho

\vspace*{.5\baselineskip}
Acknowledgement: TACC-staff, Susan H.\ Mehringer

\vspace*{\baselineskip}

\ifvsc
\begin{tabular}{ >{\centering\arraybackslash}m{0.25\textwidth}  >{\centering\arraybackslash}m{0.05\textwidth}  >{\centering\arraybackslash}m{0.20\textwidth}  >{\centering\arraybackslash}m{0.2\textwidth}} \\
\includegraphics*[width=0.2\textwidth, height=0.7in, keepaspectratio=true]{logo_auha} & \multicolumn{2}{ >{\centering\arraybackslash}m{.2\textwidth} }{\includegraphics*[width=0.2\textwidth, height=0.7in,, keepaspectratio=true]{logo_akuleuven}} & \includegraphics*[width=0.2\textwidth, height=0.7in,, keepaspectratio=true]{logo_auhl} \\ 
\multicolumn{2}{ >{\centering\arraybackslash}m{.32\textwidth} }{\includegraphics*[width=0.3\textwidth, height=0.7in, keepaspectratio=true]{logo_augent}} & \multicolumn{2}{ >{\centering\arraybackslash}m{.38\textwidth} }{\includegraphics*[width=0.3\textwidth, height=0.7in, keepaspectratio=false]{logo_uab}} \\ 
\end{tabular}
\fi

\end{center}

\strong{Audience:}

This PerfExpert Tutorial is designed for \strong{HPC-Users} at the \strong{University of Texas}, the members of the \strong{Flemish Supercomputing Centre (VSC)}, being the \strong{Antwerp University Association}, the \strong{Brussels University Association}, the \strong{Ghent University Association}, the \strong{Hasselt University Association}, and the \strong{KU~Leuven Association} who want to profile and optimise their applications.

The audience may be completely unaware of the profiling and optimisation concepts but must have some basic understanding of HPC programming.

\strong{Contents:}

This tutorial gives answers to the typical questions that a new PerfExpert user may have. The aim is to learn how to use PerfExpert.

\begin{tabular}{|c|p{0.3\textwidth}|c|p{0.4\textwidth}|} \hline
\strong{Part} & \strong{Questions}                          & \strong{Chapter}  & \strong{Title}    \\ \hline
Tutorial      & What's PerfExpert?                          & \strong{\ref{ch:ch01_introduction}}        & \nameref{ch:ch01_introduction}      \\ \hline
              & How to profile and interpret the report?    & \strong{\ref{ch:ch02_profiling}}        & \nameref{ch:ch02_profiling} \\ \hline
              & How shall I optimise my code?               & \strong{\ref{ch:ch03_optimizing}}        & \nameref{ch:ch03_optimizing} \\ \hline
              & But what about my complex parallel program? & \strong{\ref{ch:ch04_multi_core_multi_node_profiling}}        & \nameref{ch:ch04_multi_core_multi_node_profiling} \\ \hline
              & What can I do more?                         & \strong{\ref{ch:ch05_more_perfexpert_functionality}}        & \nameref{ch:ch05_more_perfexpert_functionality} \\ \hline
\end{tabular}

The \strong{Annexes} contains some useful reference guides.

\begin{tabular}{|c|l|c|} \hline
\strong{Part}   & \strong{Title}        & \strong{Chapter}  \\ \hline
Annex           & \nameref{ch:ch06_perfexpert_options}  & \strong{\ref{ch:ch06_perfexpert_options}}        \\ \hline
                & \nameref{ch:ch07_optimization_patterns}  & \strong{\ref{ch:ch07_optimization_patterns}}        \\ \hline
\end{tabular}

\strong{Notification:}

In this tutorial specific actions dealing with the software are separated from the accompanying text:

\begin{prompt}
%\shellcmd{Actions}% (i.e., to be entered at a command line in your Terminal) in an exercise are preceded by a $ and printed in \strong{bold}. You'll find those actions in a grey frame.
\end{prompt}

\strong{``Directory''} is the notation for directories or specific files. (e.g., ``~/examples/'')
\strong{``Text''} Is the notation for text to be entered.

\begin{tip}
A ``Tip'' paragraph is used for remarks or tips.
\end{tip}

\strong{More support:} \\

Before starting the course, the example programs and configuration files used in this PerfExpert Tutorial must be copied to your own work directory.

\iftacc
The examples can also be downloaded from \url{https://portal.tacc.utexas.edu/training/} or copied from the training directory  ``\emph{\tutorialdir/perfexpert/examples}'' on Stampede to your home directory.
\fi
\ifvsc
If you have received a new VSC account, all examples are present in the tutorials directory. In order to have your own local copy, copy the full directory to your home directory.
\fi

\begin{prompt}
%\shellcmd{cp --r \tutorialdir/perfexpert/examples/ \tilde/}%
\end{prompt}

Apart from this Tutorial, the PerfExpert documentation on Texas Advanced Computing Center (TACC) website (\url{https://www.tacc.utexas.edu/perfexpert}) will serve as a reference for all PerfExpert operations.

\begin{tip}
The users are advised to get self-organised. There are only limited resources available at the TACC, which are best effort based. The user applications and own developed software remain solely the responsibility of the end-user.
\end{tip}


\strong{Contact Information:}

For all technical questions, please join the PerfExpert mailing list or contact the TACC staff: perfexpert-subscribe@lists.tacc.utexas.edu

We also welcome your feedback, comments and suggestions for improving the PerfExpert Tutorial via this mailing list.
