\chapter{Introduction to HPC}
\label{ch:introduction-to-hpc}

\section{What is HPC?}
\label{sec:what-is-hpc}

``\strong{High Performance Computing}'' (HPC) is computing on a
``\emph{Supercomputer}'', a computer with at the frontline of contemporary
processing capacity -- particularly speed of calculation and available memory.

While the supercomputers in the early days (around 1970) used only a few
processors, in the 1990s machines with thousands of processors began to appear
and, by the end of the 20th century, massively parallel supercomputers with
tens of thousands of ``off-the-shelf'' processors were the norm. A large number
of dedicated processors are placed in close proximity to each other in a
computer cluster.

A \strong{computer cluster} consists of a set of loosely or tightly connected
computers that work together so that in many respects they can be viewed as a
single system.

The components of a cluster are usually connected to each other through fast
local area networks (``LAN'') with each \emph{node} (computer used as a
server) running its own instance of an operating system. Computer clusters
emerged as a result of convergence of a number of computing trends including
the availability of low cost microprocessors, high-speed networks, and software
for high performance distributed computing.

Compute clusters are usually deployed to improve performance and availability
over that of a single computer, while typically being much more cost-effective
than single computers of comparable speed or availability.

Supercomputers play an important role in the field of computational science,
and are used for a wide range of computationally intensive tasks in various
fields, including quantum mechanics, weather forecasting, climate research, oil
and gas exploration, molecular modeling (computing the structures and
properties of chemical compounds, biological macromolecules, polymers, and
crystals), and physical simulations (such as simulations of the early moments
of the universe, airplane and spacecraft aerodynamics, the detonation of
nuclear weapons, and nuclear fusion).\footnote{Wikipedia}

\section{What is the \hpc?}
\label{sec:what-is-the-hpc}

The \hpc is a collection computers with Intel processors, running a Linux
operating system, shaped in the form of lunch boxes and stored above and next
to each other in racks, interconnected with copper and fiber lines. Their number crunching
power is (presently) measured in hundreds of billions of floating point
operations (gigaflops) and even in Teraflops.

\includegraphics*[width=1.88in, height=1.26in, keepaspectratio=false]{ch1-cables1}
\includegraphics*[width=1.88in, height=1.26in, keepaspectratio=false]{ch1-cables2}
\includegraphics*[width=1.88in, height=1.26in, keepaspectratio=false]{ch1-cables3}

The \hpc relies on parallel-processing technology to offer \university researchers an
extremely fast solution for all their data processing needs.

The \hpc consists of:

\ifantwerpen
\begin{center}
\begin{tabular}{|p{1.8in}|p{2.1in}|} \hline
\strong{In technical terms}         & \strong{\dots\  in human terms}                    \\ \hline
336 nodes and 4992 cores            & \dots\  or the equivalent of 1248 quad-core PCs    \\ \hline
100 Terabyte of online storage     & \dots\  or the equivalent of over 20000 DVDs            \\ \hline
up to 40 Gbit infiniband fiber connections & \dots\  or allowing to transfer 8 DVDs per second \\ \hline
\end{tabular}
\end{center}
\fi
%TODO insert specs for other sites

The \hpc currently consists of:

\ifantwerpen
Turing:
  \begin{enumerate}
    \item  64 compute nodes with 2 quad core Harpertown processors, 16 GB memory,
           120 GB local disk, GbE network
    \item  32 nodes with 2 quad core Harpertown processors, 16 GB memory,
           120 GB local disk, DDR-IB network
    \item  64 nodes with 2 six core Westmere processors, 24 GB memory,
           120 GB local disk, QDR-IB network
    \item  8 nodes with 2 six core Westmere processors, 24 GB memory,
           120 GB local disk, GbE network
  \end{enumerate}
and Hopper:
  \begin{enumerate}
    \item  96 compute nodes with 2 ten core Ivy Bridge processors, 64 GB memory,
           500 GB local disk, FDR10 network
    \item  24 nodes with 2 ten core Harpertown processors, 64 GB memory,
           500 GB local disk, FDR10 network
  \end{enumerate}
\fi
\ifleuven
  \begin{enumerate}
    \item  112 compute nodes with 2 10-core Ivy Bridge processors, 64 GB memory,
           300 GB local disk, QDR-IB network
    \item  32 compute nodes with 2 10-core Ivy Bridge processors, 128 GB memory,
           300 GB local disk, QDR-IB network
    \item  8 nodes with 2 8-core Sandy Bridge processors, 64 GB memory,
           300 GB local disk, DDR-IB network, 2 K20 NVIDIA GPGPUs
    \item  8 nodes with 2 8-core Sandy Bridge processors, 64 GB memory,
           300 GB local disk, DDR-IB network, 2 Intel Xeon Phis
  \end{enumerate}
\fi
%TODO insert specs for other sites

\ifantwerpen
All the nodes in the \hpc run under the ``\operatingsystemSL'' operating
system, which is a clone of ``\operatingsystemRHEL'', 
with \emph{cpuset} support and \emph{BLCR} modules.
\fi
\ifleuven
All the nodes in the \hpc run under the ``\operatingsystem'' operating system, version 6.
\fi
%TODO insert specs for other sites

Two tools perform the \strong{job management} and \strong{job scheduling}:
\begin{enumerate}
  \item  TORQUE: a resource manager (based on PBS);
  \item  Moab: job scheduler and management tools.
\end{enumerate}
\ifantwerpen
%\strong{Accounting} is handled by a third tool, i.e., GOLD.
\fi
\ifleuven
\strong{Accounting} is handled by a third tool, i.e., MAM (Moab Accounting Manager).
\fi
%TODO check whether others use accounting

For maintenance and monitoring, we use:
\ifantwerpen
    \begin{enumerate}
      \item  Ganglia: monitoring software;
      \item  Icinga: alert manager.
    \end{enumerate}
\fi
\ifleuven
    \begin{enumerate}
      \item  Ganglia: monitoring software;
      \item  Icinga: alert manager.
    \end{enumerate}
\fi
%TODO check tools for other sites


\section{What is the \hpc not!}
\label{sec:what-is-the-hpc-not}

A computer that automatically:
\begin{enumerate}
  \item  runs your PC-applications code much faster for bigger problems;
  \item  develops your applications;
  \item  solves your bugs;
  \item  does your thinking;
  \item  \dots
  \item  allows you to play games even faster.
\end{enumerate}
The \hpc does not replace your desktop computer.

\section{Is the \hpc a solution for my computational needs?}
\label{sec:is-the-hpc-a-solution-for-my-computational-needs}

\subsection{Batch or interactive mode?}
\label{sec:batch-or-interactive-mode}

Typically, the strength of a supercomputer comes from its ability to run a huge
number of programs (i.e., executables) in parallel without any user interaction
in real time. This is what is called ``running in batch mode''.

It is also possible to run programs at the \hpc, which require user
interaction. (pushing buttons, entering input data, etc.).  Although
technically possible, the use of the \hpc might not always be the best and
smartest option to run those interactive programs.  Each time some user
interaction is needed, the computer will wait for user input. The available
computer resources (CPU, storage, network, etc.) might not be optimally used in
those cases. A more in-depth analysis with the \hpc staff can unveil whether
the \hpc is the desired solution to run interactive programs.

\subsection{Parallel or sequential programs?}
\label{sec:parallel-or-sequential-programs}

\strong{Parallel computing} is a form of computation in which many calculations
are carried out simultaneously. They are based on the principle that large
problems can often be divided into smaller ones, which are then solved
concurrently (``in parallel'').

Parallel computers can be roughly classified according to the level at which
the hardware supports parallelism, with multicore computers having multiple
processing elements within a single machine, while clusters use multiple
computers to work on the same task. Parallel computing has become the dominant
computer architecture, mainly in the form of multicore processors.

\strong{Parallel programs} are more difficult to write than sequential ones,
because concurrency introduces several new classes of potential software bugs,
of which race conditions are the most common. Communication and synchronization
between the different subtasks are typically some of the greatest obstacles to
getting good parallel program performance.

It is perfectly possible to also run purely \strong{sequential programs} on the
\hpc.

Running your sequential programs on the most modern and fastest computers in
the \hpc can save you a lot of time.  But it also might be possible to run
multiple instances of your program (e.g., with different input parameters) on
the \hpc, in order to solve one overall problem (e.g., to perform a parameter
sweep). This is another form of running your sequential programs in parallel.

\subsection{What programming languages can I use?}
\label{sec:what-programming-languages-can-i-use}

You can use \emph{any} programming language, \emph{any} software package and
\emph{any} library provided it has a version that runs on Linux, specifically,
on the version of Linux that is installed on the compute nodes, \operatingsystembase.

For the most common \strong{programming languages}, a compiler is available on
\operatingsystem. Supported and common programming languages on the \hpc are
C/C++, FORTRAN, Java, Perl, Python, MATLAB, R, etc.

Supported and commonly used compilers are GCC, clang, J2EE and Intel Cluster
Studio.

Commonly used software packages are ABINIT, CP2K, Gaussian, Gromacs, Molpro,
NWChem, Quantum Espresso, R, Siesta, TURBOMOLE, WIEN2k,\ldots

Commonly used Libraries are Intel MKL, FFTW, HDF5, PETSc and Intel MPI, M(VA)PICH, OpenMPI.

Additional software can be installed ``\emph{on demand}''. Please check the
\hpc staff to see whether the \hpc can handle your specific requirements.

\subsection{What operating systems can I use?}
\label{sec:what-operating-systems-can-i-use}

All nodes in the \hpc cluster run under \operatingsystemSL, which is a specific version of \operatingsystemRHEL. This means that
all programs (executable) must be compiled for \operatingsystemSL.

But a user can connect from any desk computer in the \university network to the \hpc,
regardless of the Operating System that he is using on his personal computer.
The user can use any of the common Operating Systems (such as Windows, OS X or
any version of Linux/Unix/BSD) and run and control his programs on the \hpc.

A user does not need to have prior knowledge about Linux; all of the required
knowledge is explained in this tutorial.

\subsection{What is the next step?}
\label{sec:what-is-the-next-step}

When you think that the \hpc is a useful tool to support your computational
needs, we encourage you to acquire a VSC-account (as explained in
\autoref{ch:getting-a-hpc-account}), read \autoref{ch:preparing-the-environment}, ``Setting up the
environment'', and explore chapters~\ref{ch:running-interactive-jobs} to~\ref{ch:fine-tuning-job-specifications} which will help you to transfer and
run your programs on the \hpc cluster.

Do not hesitate to contact the \hpc team members for any help.
