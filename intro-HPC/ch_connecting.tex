% Can't use \hpc in a chapter title
\chapter{Connecting to the HPC infrastructure}
\label{ch:connecting}

Before you can really start using the \hpc clusters, there are several things you need to do or know:

\begin{enumerate}
\item  You need to \textbf{log on to the cluster} using an SSH client to one of the login nodes. This will give you command-line access. The software you'll need to use on your client system depends on its operating system.
\item  Before you can do some work, you'll have to \textbf{transfer the files} that you need from your desktop computer to the cluster. At the end of a job, you might want to transfer some files back.
\item  Optionally, if you wish to use programs with a \textbf{graphical user interface}, you will need an X-server on your client system and log in to the login nodes with X-forwarding enabled.
\item  Often several versions of \textbf{software packages and libraries} are installed, so you need to select the ones you need. To manage different versions efficiently, the VSC clusters use so-called \textbf{modules}, so you will need to select and load the modules that you need.
\end{enumerate}

\section{Connection restrictions}
\label{sec:connection-restrictions}

Since March 20th 2020, restrictions are in place that limit from where you can connect
to the VSC HPC infrastructure, in response to security incidents involving several European HPC centres.

VSC login nodes are only directly accessible from within university networks,
and from (most) Belgian commercial internet providers.

All other IP domains are blocked by default.
If you are connecting from an IP address that is not allowed direct access,
you have the following options to get access to VSC login nodes:

\begin{itemize}
  \item Use an VPN connection to connect to the \university network (recommended).
      \ifgent
      See \url{https://helpdesk.ugent.be/vpn/en/} for more information.
      \fi
  \item Register your IP address by accessing \url{https://firewall.hpc.kuleuven.be} (same URL regardless of the university your affiliated with) and log in with your \university account.
      \begin{itemize}
      \item While this web connection is active new SSH sessions can be started.
      \item Active SSH sessions will remain active even when this web page is closed.
      \end{itemize}
  \item Contact your HPC support team (via \hpcinfo) and ask them to whitelist your IP range (e.g., for industry access, automated processes).
\end{itemize}

Trying to establish an SSH connection from an IP address that does not adhere to these restrictions
will result in an immediate failure to connect, with an error message like:

\begin{prompt}
ssh_exchange_identification: read: Connection reset by peer
\end{prompt}


\section{First Time connection to the HPC infrastructure}
\label{sec:first-time-connection-to-the-hpc}

If you have any issues connecting to the \hpc after you've followed these steps,
see \autoref{sec:connecting-issues} to troubleshoot.

\ifantwerpen
 When connecting from outside Belgium, you need a VPN client to connect to the \university network first.
\fi

\ifwindows

  \subsection{Open a Terminal}
  \label{sec:windows-open-a-terminal}

  You've generated a public/private key pair with PuTTYgen and have an approved
  account on the VSC clusters.  The next step is to setup the connection to (one
  of) the \hpc.

   In the screenshots, we show the setup for user
\ifantwerpen
``\strong{\emph{vsc20167}}''
\fi
\ifbrussel
``\strong{\emph{vsc20167}}''
\fi
\ifgent
``\strong{\emph{vsc20167}}''
\fi
\ifleuven
``\strong{\emph{vsc30468}}''
\fi
  to the \hpc cluster via the login node
  ``\strong{\emph{\loginnode}}''.

  \begin{enumerate}
    \item  Start the PuTTY executable \lstinline|putty.exe| in your
      directory \lstinline|C:\Program Files (x86)\PuTTY| and the
      configuration screen will pop up. As you will often use the PuTTY tool,
      we recommend adding a shortcut on your desktop.
    \item  Within the category <{}\strong{\emph{Session}}>{}, in the field
      <{}\strong{\emph{Host Name}}>{}, enter the name of the login node of the
      \hpc cluster (i.e., ``\strong{\emph{\loginnode}}'')
      you want to connect to.

\ifantwerpen
      \begin{center}
      \includegraphics*[width=3.59in, height=3.45in, keepaspectratio=false]{ch2-putty-configuration}
      \end{center}
\fi
\ifleuven
      \begin{center}
      \includegraphics*[width=3.59in, height=3.45in, keepaspectratio=false]{ch2-putty-configuration-leuven}
      \end{center}
\fi
\ifbrussel
      \begin{center}
      \includegraphics*[width=3.59in, height=3.45in, keepaspectratio=false]{ch2-putty-configuration}
      \end{center}
\fi
\ifgent
      \begin{center}
      \includegraphics*[width=3.59in, height=3.45in, keepaspectratio=false]{ch2-putty-configuration}
      \end{center}
\fi

    \item  In the category \menu[,]{Connection,Data}, in
      the field \keys{Auto-login username}, put in
      <{}\emph{\userid}>{}, which is your VSC username that you have
      received by e-mail after your request was approved.

\ifantwerpen
      \begin{center}
      \includegraphics*[width=3.11in, height=2.99in, keepaspectratio=false]{ch2-putty-configuration2}
      \end{center}
\fi
\ifleuven
      \begin{center}
      \includegraphics*[width=3.11in, height=2.99in, keepaspectratio=false]{ch2-putty-configuration2-leuven}
      \end{center}
\fi
\ifbrussel
      \begin{center}
      \includegraphics*[width=3.11in, height=2.99in, keepaspectratio=false]{ch2-putty-configuration2}
      \end{center}
\fi
\ifgent
      \begin{center}
      \includegraphics*[width=3.11in, height=2.99in, keepaspectratio=false]{ch2-putty-configuration2}
      \end{center}
\fi

    \item  In the category \menu[,]{Connection,SSH,Auth}, in the
      field \keys{Private key file for authentication} click on
      \keys{Browse} and select the private key (i.e., ``id\_rsa.ppk'')
      that you generated and saved above.

      \begin{center}
      \includegraphics*[width=3.25in, height=3.13in, keepaspectratio=false]{ch2-putty-key-setup}
      \end{center}

    \item  In the category \menu[,]{Connection,SSH,X11}, click
      the \keys{Enable X11 Forwarding} checkbox.

      \begin{center}
      \includegraphics*[width=3.54in, height=3.41in, keepaspectratio=false]{ch2-putty-x-forwarding}
      \end{center}

    \item  Now go back to <{}Session>{}, and fill in ``\emph{\hpcname}'' in the
      \keys{Saved Sessions} field and press \keys{Save} to
      store the session information.

\ifantwerpen
      \begin{center}
      \includegraphics*[width=3.30in, height=3.17in, keepaspectratio=false]{ch2-putty-saved-session}
      \end{center}
\fi
\ifleuven
      \begin{center}
      \includegraphics*[width=3.30in, height=3.17in, keepaspectratio=false]{ch2-putty-saved-session-leuven}
      \end{center}
\fi
\ifbrussel
      \begin{center}
      \includegraphics*[width=3.30in, height=3.17in, keepaspectratio=false]{ch2-putty-saved-session}
      \end{center}
\fi
\ifgent
      \begin{center}
      \includegraphics*[width=3.30in, height=3.17in, keepaspectratio=false]{ch2-putty-saved-session}
      \end{center}
\fi


    \item  Now pressing \keys{Open}, will open a terminal window and
      asks for you passphrase.

\ifantwerpen
      \begin{center}
      \includegraphics*[width=3.94in, height=2.47in, keepaspectratio=false]{ch2-putty-enter-password}
      \end{center}
\fi
\ifleuven
      \begin{center}
      \includegraphics*[width=3.94in, height=2.47in, keepaspectratio=false]{ch2-putty-enter-password-leuven}
      \end{center}
\fi
\ifbrussel
      \begin{center}
      \includegraphics*[width=3.94in, height=2.47in, keepaspectratio=false]{ch2-putty-enter-password}
      \end{center}
\fi
\ifgent
      \begin{center}
      \includegraphics*[width=3.94in, height=2.47in, keepaspectratio=false]{ch2-putty-enter-password}
      \end{center}
\fi

    \item If this is your first time connecting, you will be asked to verify the authenticity of
        the login node. Please see section~\ref{sec:warning-message-new-host} on how to do this.

    \item  After entering your correct passphrase, you will be connected to the
      login-node of the \hpc.
    \item  To check you can now ``Print the Working Directory'' (pwd) and check
      the name of the computer, where you have logged in (hostname):

\begin{prompt}
%\shellcmd{pwd}%
%\homedir{}%
%\shellcmd{hostname --f}%
%\loginhost{}%
\end{prompt}

    \item  For future PuTTY sessions, just select your saved session (i.e.
      ``\emph{\hpcname}'') from the list, \keys{Load} it and press
      \keys{Open}.
  \end{enumerate}

%ENDIF windows
\fi

\ifmacORlinux

  \subsection{Connect}
  \label{sec:connect}


  Open up a terminal and enter the following command to connect to the \hpc.
  \ifmac
  You can open a terminal by navigation to Applications and then Utilities in the finder and open Terminal.app, or enter Terminal in Spotlight Search.
  \fi

\begin{prompt}
%\shellcmd{ssh \userid{}@\loginnode{}}%
\end{prompt}

  Here, user \userid wants to make a connection to the ``\hpcname'' cluster at
  \university via the login node ``\loginnode'', so replace \userid with your own
  VSC id in the above command.

  The first time you make a connection to the login node, you will be asked to
  verify the authenticity of the login node. Please check section~\ref{sec:warning-message-new-host}
  on how to do this.

A possible error message you can get if you previously saved your private key
somewhere else than the default location (\lstinline|$HOME/.ssh/id_rsa|):

\begin{prompt}
%\shellcmd{Permission denied (publickey,gssapi-keyex,gssapi-with-mic).}%
\end{prompt}

In this case, use the \lstinline|-i| option for the \lstinline|ssh| command to specify the location
of your private key. For example:

\begin{prompt}
%\shellcmd{ssh -i /home/example/my\_keys/id\_rsa \userid{}@\loginnode{}}%
\end{prompt}

% ENDIF \ifmacORlinux
\fi

\strong{Congratulations, you're on the \hpc infrastructure now!}
To find out where you have landed you can print the current working directory:

\begin{prompt}
%\shellcmd{pwd}%
%\homedir{}%
\end{prompt}


% TODO The preferred location for storing files, certainly the tutorial-files, should be VSC_DATA at Gent.
Your new private home directory is ``\homedir''.
Here you can create your own subdirectory structure, copy and prepare your
applications, compile and test them and submit your jobs on the \hpc.

\begin{prompt}
%\shellcmd{cd \tutorialdir{}}%
%\shellcmd{ls }%
Intro-HPC/
\end{prompt}

This directory currently contains all training material for the \strong{\emph{Introduction to the \hpc}}.
More relevant training material to work with the \hpc can always be added
later in this directory.

You can now explore the content of this directory with the ``ls --l''
(\strong{l}ist\strong{s} \strong{l}ong) and the ``cd'' (\strong{c}hange
\strong{d}irectory) commands:

As we are interested in the use of the \strong{\emph{HPC}}, move further to
\strong{\emph{Intro-HPC}} and explore the contents up to 2 levels deep:

\begin{prompt}
%\shellcmd{cd Intro-HPC}%
%\shellcmd{tree -L 2}%
.
`-- examples
    |-- Compiling-and-testing-your-software-on-the-HPC
    |-- Fine-tuning-Job-Specifications
    |-- Multi-core-jobs-Parallel-Computing
    |-- Multi-job-submission
    |-- Program-examples
    |-- Running-batch-jobs
    |-- Running-jobs-with-input
    |-- Running-jobs-with-input-output-data
    |-- example.pbs
    `-- example.sh
9 directories, 5 files
\end{prompt}

This directory contains:

\begin{enumerate}
  \item This \strong{\emph{HPC Tutorial}} (in either a Mac, Linux or Windows version).
  %TODO: these are not here yet? \item The according \strong{\emph{HPC slides}} (used during training sessions).
  %TODO: this is not here yet? \item A \strong{\emph{docs}} subdirectory, containing interesting \hpc related documents.
  \item An \strong{\emph{examples}} subdirectory, containing all the
    examples that you need in this Tutorial, as well as examples that might be
    useful for your specific applications.
\end{enumerate}

\begin{prompt}
%\shellcmd{cd examples}%
\end{prompt}

\begin{tip}
Typing \lstinline|cd ex| followed by \keys{\tab} (the Tab-key) will generate the
\lstinline|cd examples| command. \strong{Command-line completion
}(also tab completion) is a common feature of the bash command line
interpreter, in which the program automatically fills in partially typed
commands.
\end{tip}

\begin{tip}
For more exhaustive tutorials about Linux usage, see Appendix \ref{ch:useful-linux-commands}
\end{tip}

The first action is to copy the contents of the \hpc examples directory
to your home directory, so that you have your own personal copy and that
you can start using the examples. The ``-r'' option of the copy command
will also copy the contents of the sub-directories ``\emph{recursively}''.

\begin{prompt}
%\shellcmd{cp --r \examplesdir{} \tilde{}/}%
\end{prompt}

\ifgent
Go to your home directory, check your own private examples directory, \dots\
and start working.

\begin{prompt}
%\shellcmd{cd}%
%\shellcmd{ls -l}%
\end{prompt}

Upon connecting you will see a login message containing your last login
time stamp and a basic overview of the current cluster utilisation.

\begin{prompt}
Last login: Mon Sep  3 14:00:00 2018 from helios.ugent.be
STEVIN HPC-UGent infrastructure status on Mon, 03 Sep 2018 14:30:00

   cluster - full - free -  part - total - running - queued
             nodes  nodes   free   nodes    jobs      jobs
-------------------------------------------------------------------
    golett    168      2     16     200      N/A       N/A
    phanpy     15      0      1      16      N/A       N/A
    swalot      0      0      0     128      N/A       N/A
    skitty     69      2      0      73      N/A       N/A
   victini     83      1      3      96      N/A       N/A

For a full view of the current loads and queues see:
 http://hpc.ugent.be/clusterstate/
Updates on maintenance and unscheduled downtime can be found on
 https://www.ugent.be/hpc/en/infrastructure/status

\end{prompt}
\fi %ENDIF \ifgent
%
\ifbrussel
Upon connecting you will see a login message containing your last login
time stamp, some useful environment variable definitions and the message of the
day (MOTD).

\begin{prompt}
Last login: Thu Nov  6 16:05:21 2014 from example.vub.ac.be

Initialising your working environment...
System variables to use in your scripts/programs:
  Temporary directory:   \$TMPDIR as /tmp/vsc40485
  Temporary work directory:    \$WORKDIR as /work/vsc40485
  Home directory:              \$HOME as /user/home/gent/vsc404/vsc40485
Message of the day:

  --------------------------------------------------------------------
  Welcome to the Hydra cluster.

  The old work directory access has been closed.


  The Hydra Team
  --------------------------------------------------------------------
       \   ,__,
        \  (oo)____
           (__)    )\
              ||--||

\end{prompt}
\fi %ENDIF \ifbrussel

\ifleuven
Upon connecting you will see a login message containing your last login
time stamp and some useful links.

\begin{prompt}
Last login: Mon Jan 12 18:52:20 2015 from example.kuleuven.be
**********************************************
*                                            *
* Please check the following site for        *
* status messages concerning KU Leuven       *
* services (incl. HPC):                      *
*                                            *
*   http://status.kuleuven.be/               *
*                                            *
* For VSC user documentation:                *
*                                            *
*  https://www.vscentrum.be/en/user-portal   *
*                                            *
*                                            *
**********************************************
\end{prompt}
\fi  % ENDIF \ifleuven

\ifantwerpen
Upon connection, you will get a welcome message containing your last login
timestamp and some pointers to information about the system. On Leibniz, the system
will also show your disk quota.

\begin{prompt}
Last login: Mon Feb  2 17:58:13 2015 from mylaptop.uantwerpen.be

---------------------------------------------------------------

Welcome to LEIBNIZ !

Useful links:
  https://vscdocumentation.readthedocs.io
  https://vscdocumentation.readthedocs.io/en/latest/antwerp/tier2_hardware.html
  https://www.uantwerpen.be/hpc

Questions or problems? Do not hesitate and contact us:
  hpc@uantwerpen.be

Happy computing!

---------------------------------------------------------------

Your quota is:

                   Block Limits
   Filesystem       used      quota      limit    grace
   user             740M         3G       3.3G     none
   data           3.153G        25G      27.5G     none
   scratch        12.38M        25G      27.5G     none
   small          20.09M        25G      27.5G     none

                   File Limits
   Filesystem      files      quota      limit    grace
   user            14471      20000      25000     none
   data             5183     100000     150000     none
   scratch            59     100000     150000     none
   small            1389     100000     110000     none

---------------------------------------------------------------

\end{prompt}
\fi  %ENDIF \ifantwerpen

You can exit the connection at anytime by entering:

\begin{prompt}
%\shellcmd{exit}%
logout
Connection to %\loginnode{}% closed.
\end{prompt}

\begin{tip}[Setting your Language right]

You may encounter a warning message similar to the following one during connecting:

\begin{prompt}
perl: warning: Setting locale failed.
perl: warning: Please check that your locale settings:
LANGUAGE = (unset),
LC_ALL = (unset),
LC_CTYPE = "UTF-8",
LANG = (unset)
    are supported and installed on your system.
perl: warning: Falling back to the standard locale ("C").
\end{prompt}

or any other error message complaining about the locale.

This means that the correct ``locale'' has not yet been properly specified on
your local machine. Try:

\begin{prompt}
%\shellcmd{locale}%
LANG=
LC_COLLATE="C"
LC_CTYPE="UTF-8"
LC_MESSAGES="C"
LC_MONETARY="C"
LC_NUMERIC="C"
LC_TIME="C"
LC_ALL=
\end{prompt}
\end{tip}

A \strong{locale} is a set of parameters that defines the user's language,
country and any special variant preferences that the user wants to see in their
user interface. Usually a locale identifier consists of at least a language
identifier and a region identifier.

\ifgent
\strong{Note:} If you try to set a non-supported locale, then it will be
automatically set to the default. Currently the default is
\lstinline|en_US.UFT-8| or \lstinline|en_US|, depending on whether your
originally (non-supported) locale was \lstinline|UTF-8| or not.
\fi

%% TODO: how to fix this on windows/putty? generally this is not an issue I believe
\ifmacORlinux
Open the  \lstinline|.bashrc| on your local machine with your
favourite editor and add the following lines:

\begin{prompt}
%\shellcmd{nano ~/.bashrc}%
%\dots{}%
export LANGUAGE="en_US.UTF-8"
export LC_ALL="en_US.UTF-8"
export LC_CTYPE="en_US.UTF-8"
export LANG="en_US.UTF-8"
%\dots{}%
\end{prompt}

\begin{tip}[vi]
To start entering text in vi: move to the place you want to start entering text
with the arrow keys and type ``i'' to switch to insert mode.  You can easily
exit vi by entering: ``\keys{ESC}:wq''
To exit vi without saving your changes, enter ``\keys{ESC}:q!''
\end{tip}

or alternatively (if you are not comfortable with the Linux editors), again on your local machine:

\begin{prompt}
%\shellcmd{echo "export LANGUAGE=\textbackslash{"}en\_US.UTF-8\textbackslash{"}" >{}>{} ~/.profile}%
%\shellcmd{echo "export LC\_ALL=\textbackslash{"}en\_US.UTF-8\textbackslash{"}" >{}>{} ~/.profile}%
%\shellcmd{echo "export LC\_CTYPE=\textbackslash{"}en\_US.UTF-8\textbackslash{"}" >{}>{} ~/.profile}%
%\shellcmd{echo "export LANG=\textbackslash{"}en\_US.UTF-8\textbackslash{"}" >{}>{} ~/.profile}%
\end{prompt}

You can now log out, open a new terminal/shell on your local machine and reconnect to the \hpc, and you should not get these
warnings anymore.

\fi
\section{Transfer Files to/from the \hpc}
\label{sec:filetransfer}
\hypertarget{sec:filetransfer}{}


Before you can do some work, you'll have to \strong{transfer the files}
you need from your desktop or department to the cluster. At the end of a job,
you might want to transfer some files back.

\ifmacORlinux
The preferred way to transfer files is by using an scp or sftp via the secure
OpenSSH protocol.  \OS  ships with an implementation of
OpenSSH, so you don't need to install any third-party software to use it. Just
open a terminal window and jump in!
\fi

\ifwindows
  \subsection{WinSCP}

  To transfer files to and from the cluster, we recommend the use of
  WinSCP, a graphical file management tool which can transfer files
  using secure protocols such as SFTP and SCP.  WinSCP is freely
  available from \url{http://www.winscp.net}.

  To transfer your files using WinSCP,

  \begin{enumerate}
  \item  Open the program
  \item The \keys{Login} menu is shown automatically (if it is closed, click \keys{New
    Session} to open it again).  Fill in the necessary fields under
  \keys{Session}
  \begin{enumerate}
  \item  Click \keys{New Site}.
  \item  Enter ``\emph{\loginnode}'' in the \keys{Host name} field.
  \item  Enter your ``\emph{vsc-account}'' in the \keys{User name} field.
  \item  Select \keys{SCP} as the \keys{file} protocol.
  \item  Note that the password field remains empty.
  \begin{center}
\ifantwerpen
  \includegraphics*[height=3.04in]{img0214}
\fi
\ifbrussel
  \includegraphics*[width=3.42in, height=3.04in, keepaspectratio=false]{img0214}
\fi
\ifgent
  \includegraphics*[width=3.42in]{img0214-gent}
\fi
\ifleuven
  \includegraphics*[width=3.42in, height=3.04in, keepaspectratio=false]{img0214-leuven}
\fi
  \includegraphics*[height=3.04in]{img0214b}
  \end{center}

  \item  Click \keys{Advanced...}.
  \item  Click \menu[,]{SSH,Authentication}.
  \item  Select your private key in the field \keys{Private key file}.
  \end{enumerate}

  \item Press the \keys{Save} button, to save the session under
  \menu[,]{Session,Sites} for future access.
  \item  Finally, when clicking on \keys{Login}, you will be asked for your key passphrase.

  \begin{center}
\ifantwerpen
  \includegraphics*[height=2.47in]{img0215}
\fi
\ifbrussel
  \includegraphics*[width=3.14in, height=2.47in, keepaspectratio=false]{img0215}
\fi
\ifgent
  \includegraphics*[width=3.14in, height=2.47in, keepaspectratio=false]{img0215}
\fi
\ifleuven
  \includegraphics*[width=3.14in, height=2.47in, keepaspectratio=false]{img0215-leuven}
\fi
  \end{center}

  \end{enumerate}

  \firsttimeconnection

  \begin{center}
\ifantwerpen
  \includegraphics*[height=2.81in]{img0216}
\fi
\ifbrussel
  \includegraphics*[width=5.74in, height=2.81in, keepaspectratio=false]{img0216}
\fi
\ifgent
  \includegraphics*[width=5.74in, height=2.81in, keepaspectratio=false]{img0216}
\fi
\ifleuven
  \includegraphics*[width=5.74in, height=2.81in, keepaspectratio=false]{img0216-leuven}
\fi
  \end{center}

  Now, try out whether you can transfer an arbitrary file from your local
  machine to the HPC and back.

\fi

\ifmacORlinux
  \subsection{Using scp}

  \strong{Secure copy} or \strong{SCP} is a tool (command) for securely
  transferring files between a local host (= your computer) and a remote host
  (the \hpc). It is based on the Secure Shell (SSH) protocol.  The \strong{scp}
  command is the equivalent of the \strong{cp}  (i.e., \strong{c}o\strong{p}y)
  command, but can copy files to or from remote machines.

  It's easier to copy files directly to \lstinline|$VSC_DATA| and \lstinline|$VSC_SCRATCH|
  if you have symlinks to them in your home directory. See
  \href{\LinuxManualURL#sec:symlink-for-data}{the chapter titled ``Uploading/downloading/editing files'', section ``Symlinks for data/scratch'' in the intro to Linux}
  for how to do this.


  Open an additional terminal window and check that you're working on your local
  machine.

\begin{prompt}
%\shellcmd{hostname}%
<local-machine-name>
\end{prompt}

  If you're still using the terminal that is connected to the \hpc, close the
  connection by typing ``exit'' in the terminal window.
  %Alternatively, open a new
  %terminal window using the shortcut \keys{command}\keys{N}.

  For example, we will copy the (local) file ``\emph{localfile.txt}'' to your
  home directory on the \hpc cluster. We first generate a small dummy
  ``\emph{localfile.txt}'', which contains the word ``Hello''.  Use your own
  VSC account, which is something like ``\emph{\userid}''. Don't
  forget the colon (\lstinline|:|) at the end: if you forget it, it will just create a file
  named \texttt{\userid{}@\loginnode{}} on your local filesystem. You can even
  specify where to save the file on the remote filesystem by putting a path after the colon.

\begin{prompt}
%\shellcmd{echo "Hello" >{} localfile.txt}%
%\shellcmd{ls -l}%
%\dots{}%
-rw-r--r-- 1 user  staff   6 Sep 18 09:37 localfile.txt
%\shellcmd{scp localfile.txt \userid{}@\loginnode{}:}%
localfile.txt    100%\%%   6     0.0KB/s   00:00
\end{prompt}

  Connect to the \hpc via another terminal, print the working directory (to make
  sure you're in the home directory) and check whether the file has arrived:

\begin{prompt}
%\shellcmd{pwd}%
%\homedir{}%
%\shellcmd{ls -l}%
total 1536
drwxrwxr-x  2 %\userid{}% 131072 Sep 11 16:24 bin/
drwxrwxr-x  2 %\userid{}% 131072 Sep 17 11:47 docs/
drwxrwxr-x 10 %\userid{}% 131072 Sep 17 11:48 examples/
-rw-r--r--  1 %\userid{}%      6 Sep 18 09:44 localfile.txt
%\shellcmd{cat localfile.txt}%
Hello
\end{prompt}

  The \strong{scp} command can also be used to copy files from the cluster to your local machine.
  Let us copy the remote file ``\jobname.pdf'' from your ``docs''
  subdirectory on the cluster to your local computer.

  First, we will confirm that the file is indeed in the ``docs'' subdirectory.
  On the terminal on the \hpc, enter:

\begin{prompt}
%\shellcmd{cd ~/docs}%
%\shellcmd{ls -l}%
total 1536
-rw-r--r-- 1 %\userid{}% Sep 11 09:53 %\jobname{}%.pdf
\end{prompt}

  Now we will copy the file to the local machine. On the terminal on your own local computer, enter:

\begin{prompt}
%\shellcmd{scp \userid{}@\loginnode{}:./docs/\jobname{}.pdf .}%
%\jobname{}.pdf% 100%\%%  725KB 724.6KB/s   00:01
%\shellcmd{ls -l}%
total  899
-rw-r--r--   1 user  staff  741995 Sep 18 09:53 %\jobname{}%.pdf
-rw-r--r--   1 user  staff       6 Sep 18 09:37 localfile.txt
\end{prompt}

The file has been copied from the HPC to your local computer.

It's also possible to copy entire directories (and their contents) with the \lstinline|-r| flag.
For example, if we want to copy the local directory \lstinline|dataset| to \lstinline|$VSC_SCRATCH|,
we can use the following command (assuming you've created the \lstinline|scratch| symlink):

\begin{prompt}
%\shellcmd{scp -r dataset \userid{}@\loginnode{}:scratch}%
\end{prompt}

If you don't use the \lstinline|-r| option to copy a directory, you will run into the following error:

\begin{prompt}
%\shellcmd{scp dataset \userid{}@\loginnode{}:scratch}%
dataset: not a regular file
\end{prompt}



  \subsection{Using sftp}

  The \strong{SSH File Transfer Protocol} (also \strong{Secure File Transfer
  Protocol}, or \strong{SFTP}) is a network protocol that provides file access,
  file transfer and file management functionalities over any reliable data
  stream. It was designed as an extension of the Secure Shell protocol (SSH)
  version 2.0. This protocol assumes that it is run over a secure channel, such
  as SSH, that the server has already authenticated the client, and that the
  identity of the client user is available to the protocol.

  The sftp is an equivalent of the ftp command, with the difference that it uses
  the secure ssh protocol to connect to the clusters.

  One easy way of starting a sftp session is

\begin{prompt}
%\shellcmd{sftp \userid{}@\loginnode{}}%
\end{prompt}

  Typical and popular commands inside an sftp session are:

  \begin{tabular}{|p{\dimexpr 0.3\textwidth-2\tabcolsep}|p{\dimexpr 0.7\textwidth-2\tabcolsep}|} \hline
  \strong{cd \tilde/examples/fibo} & Move to the examples/fibo subdirectory on the \hpc (i.e., the remote machine)\\  \hline
  \strong{ls}                      & Get a list of the files in the current directory on the \hpc. \\ \hline
  \strong{get fibo.py}             & Copy the file ``fibo.py'' from the \hpc \\ \hline
  \strong{get tutorial/HPC.pdf}    & Copy the file ``HPC.pdf'' from the \hpc, which is in the ``tutorial'' subdirectory. \\ \hline
  \strong{lcd test}                & Move to the ``test'' subdirectory on your local machine. \\ \hline
  \strong{lcd ..}                  & Move up one level in the local directory. \\ \hline
  \strong{lls}                     & Get local directory listing \\ \hline
  \strong{put test.py}             & Copy the local file test.py to the \hpc. \\ \hline
  \strong{put test1.py test2.py}  & Copy the local file test1.py to the \hpc and rename it to test2.py. \\ \hline
  \strong{bye}                     & Quit the sftp session \\ \hline
  \strong{mget *.cc}               & Copy all the remote files with extension ``.cc'' to the local directory.  \\ \hline
  \strong{mput *.h}                & Copy all the local files with extension ``.h'' to the \hpc. \\ \hline
  \end{tabular}

\fi
\iflinux
  \subsection{Using a GUI}
  % don't suggest filezilla, it installs adware by default.

  If you prefer a GUI to transfer files back and forth to the \hpc, you can use your file browser.
  Open your file browser  and press

  \iflinux
    \keys{Ctrl} + \keys{l}
  \fi

  \ifmac
    \keys{cmd} + \keys{k}
  \fi

  This should open up a address bar where you can enter a URL. Alternatively, look for the ``connect to server'' option in your file browsers menu.

  Enter: \emph{sftp://\userid{}@\loginnode/} and press enter.

  You should now be able to browse files on the \hpc in your file browser.
\fi

\ifmac
  \subsection{Using a GUI (Cyberduck)}

  Cyberduck is a graphical alternative to the \lstinline|scp| command. It can be installed
  from \url{https://cyberduck.io}.

  This is the one-time setup you will need to do before connecting:

  \begin{enumerate}
      \item After starting Cyberduck, the Bookmark tab will show up. To add a new
        bookmark, click on the ``+'' sign on the bottom left of the window. A new window will open.
      \item In the ``Server'' field, type in \texttt{\loginnode}. In the ``Username'' field,
        type in your VSC account id (this looks like \texttt{\userid}).
      \item Click on ``ore Options'', select ``Use Public Key Authentication'' and point
        it to your private key (the filename will be shown underneath).
      \item Finally, type in a name for the bookmark in the ``Nickname'' field and
        close the window by pressing on the red circle in the top left corner of the window.
  \end{enumerate}

  To open the scp connection, click on the ``Bookmarks'' icon (which resembles an
  open book) and double-click on the bookmark you just created.

\fi

\subsection{Fast file transfer for large datasets}

See \href{\LinuxManualURL#sec:rsync}{the section on \texttt{rsync} in chapter 5 of the Linux intro manual}.
