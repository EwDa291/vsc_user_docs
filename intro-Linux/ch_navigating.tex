\chapter{Navigating}

\section{Current directory: ``pwd'' and ``$PWD''}

To print the current directory, use ``pwd'' or ``$PWD'':
\begin{prompt}
%\shellcmd{cd \$HOME}%
%\shellcmd{pwd}%
  /user/home/gent/vsc400/vsc40000
%\shellcmd{echo "The current directory is: \$PWD"}%
  The current directory is: /user/home/gent/vsc400/vsc40000
\end{prompt}

\section{Listing files and directories: ``ls''}

A very basic and commonly used command is ``ls'', which can be used to list files and directories.

In it's basic usage, it just prints the names of files and directories in the current directory. For example:

\begin{prompt}
  %\shellcmd{ls}%
  afile.txt   some_directory
\end{prompt}

When provided an argument, it can be used to list the contents of a directory:

\begin{prompt}
  %\shellcmd{ls some_directory}%
  one.txt  two.txt
\end{prompt}

A couple of commonly used options include:

\item detailed listing using ``ls -l''

\begin{prompt}
  %\shellcmd{ls -l}%
  total 4224
  -rw-rw-r-- 1 vsc40023 vsc40023 2157404 Apr 12 13:17 afile.txt
  drwxrwxr-x 2 vsc40023 vsc40023     512 Apr 12 12:51 some_directory
\end{prompt}

To print the size information in human-readable form, use ``-h'':

\begin{prompt}
  %\shellcmd{ls -lh}%
  total 4.1M
  -rw-rw-r-- 1 vsc40023 vsc40023 2.1M Apr 12 13:16 afile.txt
  drwxrwxr-x 2 vsc40023 vsc40023  512 Apr 12 12:51 some_directory
\end{prompt}

\item also listing hidden files using ``ls -''a
\begin{prompt}
  %\shellcmd{ls -lah}%
  total 3.9M
  drwxrwxr-x   3 vsc40000 vsc40000  512 Apr 12 13:11 .
  drwx------ 188 vsc40000 vsc40000 128K Apr 12 12:41 ..
  -rw-rw-r--   1 vsc40000 vsc40000 1.8M Apr 12 13:12 afile.txt
  -rw-rw-r--   1 vsc40000 vsc40000    0 Apr 12 13:11 .hidden_file.txt
  drwxrwxr-x   2 vsc40000 vsc40000  512 Apr 12 12:51 some_directory
\end{prompt}

\item ordering files by the most recent change using ``-rt''

\begin{prompt}
  %\shellcmd{ls -lrth}%
  total 4.0M
  drwxrwxr-x 2 vsc40000 vsc40000  512 Apr 12 12:51 some_directory
  -rw-rw-r-- 1 vsc40000 vsc40000 2.0M Apr 12 13:15 afile.txt
\end{prompt}

If you try to use ``ls'' on a file that doesn't exist, you will get a clear error message:

\begin{prompt}
  %\shellcmd{ls nosuchfile}%
  ls: cannot access nosuchfile: No such file or directory
\end{prompt}

\section{Changing directory: ``cd''}

To change to a different directory, you can use the ``cd'' command:

\begin{prompt}
  %\shellcmd{cd some_directory}%
\end{prompt}

To change back to the previous directory you were in, there's a shortcut: ``cd -''

Using ``cd'' without an argument results in returning back to your home directory:

\begin{prompt}
  %\shellcmd{cd}%
  %\shellcmd{pwd}%
  /user/home/gent/vsc400/vsc40000  
\end{prompt}

\section{Inspecting file type: ``file''}

The file command can be used to inspect what type of file you're dealing with:

\begin{prompt}
  %\shellcmd{file afile.txt}%
  afile.txt: ASCII text

  %\shellcmd{file some_directory}%
  some_directory: directory
\end{prompt}

\section{Absolute vs relative file paths}

An \emph{absolute} filepath starts with ``/'' (or a variable which value starts
with ``/''), which is also called the \emph{root} of the filesystem.

Example: absolute path to your home directory: ``/user/home/gent/vsc400/vsc40000''.

A \emph{relative} path starts from the current directory, and points to another
location up or down the filesystem hierarchy.

Example: ``some_directory/one.txt'' points to the file ``one.txt'' that is
located in the subdirectory named ``some_directory'' of the current directory.

There are two special relative paths worth mentioning:

\item \'``.''\' is a shorthand for the current directory
\item \'``..''\' is a shorthand for the parent of the current directory

You can use ``..'' also when constructing relative paths, for example:

\begin{prompt}
  %\shellcmd{cd \$HOME/some_directory}%
  %\shellcmd{ls ../afile.txt}%
  ../afile.txt
\end{prompt}

\section{Permissions}

Each file and directory has particular \emph{permissions} set on it, which can be queried using ``ls -l''.

For example:

\begin{prompt}
  %\shellcmd{ls -l afile.txt}%
  -rw-rw-r-- 1 vsc40000 agroup 2929176 Apr 12 13:29 afile.txt
\end{prompt}

The ``-rwxrw-r--'' specifies both the type of file (``-'' for files, ``d'' for directories), and the permissions for user/group/others:

\begin{enumerate}
\item each triple of characters indicates whether the read (``r''), write (``w''), execute (``x'') permission bits are set or not 
\item the 1st part ``rwx'' indicates that the \emph{owner} ``vsc40000'' of the file has all the rights
\item the 2nd part ``rw-'' indicates the members of the \emph{group} ``agroup'' only have read/write permissions (not execute)
\item the 3rd part ``r--'' indicates that \emph{other} users only have read permissions
\end{enumerate}

The default permission settings for new files/directories are determined by the so-called \emph{umask} setting, and are by default:

\begin{enumerate}
\item read-write permission on files for user/group (no execute), read-only for others (no write/execute)
\tem read-write-execute permission for directories on user/group, read/execute-only for others (no write)
\end{enumerate}

See also \url{http://hpc.ugent.be/userwiki/index.php/Tips:Introduction_to_Linux#Changing_permissions:_chmod}.

\section{Finding files/directories: ``find''}

``find'' will crawl a series of directories and lists files matching given criteria.

For example, to look for the file named ``one.txt'':

\begin{prompt}
  %\shellcmd{cd \$HOME}%
  %\shellcmd{find . -name one.txt}%
  ./some_directory/one.txt
\end{prompt}

To look for files using incomplete names, you can use a wildcard ``*''; note
that you need to escape the ``*'' to avoid that Bash \emph{expands} it into
``afile.txt'':

\begin{prompt}
  %\shellcmd{find . -name '*.txt'}%
  ./.hidden_file.txt
  ./afile.txt
  ./some_directory/one.txt
  ./some_directory/two.txt 
\end{prompt}

More advanced use of ``find'' is to use ``-exec'' to perform actions on the
found file, rather than just printing their paths (see ``man find'').

\chapter{Manipulating files and directories}

Being able to manage your data is an important part of using the HPC infrastructure. 

The bread and butter commands for doing this are mentioned here.

It might seem annoyingly terse at first, but with practice you will realise that it's very practical to have such common commands short to type.

\section{File contents: ``cat'', ``head'', ``tail'', ``less'', ``more''}

To print the contents of an entire file, you can use ``cat''; to only see the first or last N lines, you can use ``head'' or ``tail'':

\begin{prompt}
  %\shellcmd{cat one.txt}%
  1
  2
  3
  4
  5

  %\shellcmd{head -n 2 one.txt}%
  1
  2

  %\shellcmd{tail -2 one.txt}%
  4 
  5
\end{prompt}

To check the contents of long text files, you can use the ``less'' or ``more''
commands which support scrolling with ``<up>'', ``<down>'', ``<space>'', etc.

\section{Copying files: ``cp''}

\begin{prompt}
   %\shellcmd{cp source target}%
\end{prompt}

This is the ``cp'' command, which copies a file from source to target. To copy a directory, we use the ``-r'' option:

\begin{prompt}
   %\shellcmd{cp -r sourceDirectory target}%
\end{prompt}

A last more complicated example:
\begin{prompt}
   %\shellcmd{cp -a sourceDirectory target}%
\end{prompt}

Here we used the same cp command, but instead we gave it the ``-a'' option
which tells cp to copy all the files and keep timestamps and permissions.

\section{Creating directories: ``mkdir''}

\begin{prompt}
  %\shellcmd{mkdir directory}%
\end{prompt}

which will create a directory with the given name inside the current directory.

\section{Renaming/moving files: ``mv''}

\begin{prompt}
  %\shellcmd{mv source target}%
\end{prompt}

``mv'' will move the source path to the destination path. Works for both directories as files.

\section{Removing files: ``rm''}

\strong{Note: there are **NO** backups, there is no 'trash bin'. If you remove files/directories, they are *gone*.}

\begin{prompt}
  %\shellcmd{ rm filename}%
\end{prompt}

``rm'' will remove a file or directory. (rm -rf directory will remove every
file inside a given directory). WARNING: files removed will be lost forever,
there are no backups, so beware when using this command!

\subsubsection{Removing a directory: ``rmdir''}

You can remove directories using ``rm -r directory'', however, this is error
prone and can ruin your day if you make a mistake in typing. To prevent this
type of error, you can remove the contents of a directory using ``rm''
and then finally removing the directory with:

\begin{prompt}
    %\shellcmd{rmdir directory}%
\end{prompt}

\section{Changing permissions: ``chmod''}

Every file, directory, and link has a set of permissions. These permissions
consist of permission groups and permission types. The permission groups are: 

\begin{enumerate}
\item User - a particular user (account)
\item Group - a particular group of users (may be user-specific group with only one member)
\item Other - other users in the system
\end{enumerate}

The permission types are:

\begin{enumerate}
\item Read - For files, this gives permission to read the contents of a file
\item Write - For files, this gives permission to write data to the file. For directories it allows users to add or remove files to a directory.
\item Execute - For files this gives permission to execute a file as through it were a script. For directories, it allows users to open the directory and look at the contents.
\end{enumerate}

Any time you run ``ls -l'' you'll see a familiar line of ``rwx------'' or similar combination of the letters ``r'', ``w'', ``x'' and ``-'' (dashes). These are the permissions for the file or directory.

\begin{prompt}
    %\shellcmd{ls -l}%
    total 1
    -rw-r--r--. 1 myname mygroup 4283648 Apr 12 15:13 articleTable.csv
    drwxr-x---. 2 myname mygroup      40 Apr 12 15:00 Project_GoldenDragon
\end{prompt}

Here, we see that ``articleTable.csv'' is a file (beginning the line with ``-'') has read and write permission for the user ``myname''(``rw-''), and read permission for the group ``mygroup'' as well as all other users (``r--'' and ``r--'').

The next entry is ``Project_GoldenDragon''. We see it is a directory because the line begins with a ``d''. It also has read, write, and execute permission for the ``myname'' user (``rwx''). So that user can look into the directory and add or remove files. Users in the ``mygroup'' can also look into the directory and read the files. But they can't add or remove files (``r-x''). Finally, other users can read files in the directory, but other users have no permissions to look in the directory at all (``---'').

Maybe we have a colleague who wants to be able to add files to the directory. We use ``chmod'' to change the modifiers to the directory to let people in the group write to the directory:

\begin{prompt}
    %\shellcmd{chmod g+w Project_GoldenDragon}%
    %\shellcmd{ls -l}%
    total 1
    -rw-r--r--. 1 myname mygroup 4283648 Apr 12 15:13 articleTable.csv
    drwxrwx---. 2 myname mygroup      40 Apr 12 15:00 Project_GoldenDragon
\end{prompt}

The syntax used here is ``g+x'' which means ''g''roup was given ''w''rite permission. To revoke it again, we use ``g-w''. The other roles are ``u'' for user and ``o'' for other.

You can put multiple changes on the same line: ``chmod o-rwx,g-rxw,u+rx,u-w somefile'' will take everyone's permission away except the user's ability to read or execute the file.

You can also use the ``-R'' to affect all the files within a directory but this is dangerous. It's best to refine your search using ``find'' and then pass the resulting list to ``chmod'' since it's not usual for all files in a directory structure to have the same permissions.

\subsection{Access control lists (ACLs)}

However, this means that all users in ``mygroup'' can add or remove files. This could be problematic if you only wanted one person to be allowed to help you administer the files in the project. We need a new group. To do this in the HPC environment, we need to use access control lists (ACLs):

\begin{prompt}
    %\shellcmd{setfacl -m u:otheruser:w Project_GoldenDragon}%
    %\shellcmd{ls -l Project_GoldenDragon}%
    drwxr-x---+ 2 myname mygroup      40 Apr 12 15:00 Project_GoldenDragon
\end{prompt}

Now there is a ``+'' at the end of the line. This means there is an ACL attached to the directory. ``getfacl Project_GoldenDragon'' will print the ACLs for the directory.

Note: most people don't use ACLs, but it's sometimes the right thing and you should be aware it exists.

See http://linuxcommand.org/lc3_lts0090.php for more information.

\section{Zipping: ``gzip''/``gunzip'', ``zip''/``unzip''}

Files should usually be stored in a compressed file if they're not being used
frequently. This means they will use less space and thus you get more out of
your quota. Some types of files (e.g. CSV files with a lot of numbers) compress
as much as 9:1. The most commonly used compression format on Linux is gzip. To
compress a file using gzip, we use:

\begin{prompt}
    %\shellcmd{ls -lh myfile}%
    -rw-r--r--. 1 myname myname 4.1M Dec  2 11:14 myfile
    %\shellcmd{gzip myfile}%
    %\shellcmd{ls -lh myfile.gz}%
    -rw-r--r--. 1 myname myname 1.1M Dec  2 11:14 myfile.gz
\end{prompt}


To compress the file again, use gzip:

\begin{prompt}
    %\shellcmd{gzip myfile}%
\end{prompt|

Note: if you zip a file, the original file will be removed. If you unzip a file, the compressed file will be removed. To keep both, we send the data to ``stdout'' and redirect it to the target file:

\begin{prompt}
    %\shellcmd{gzip -c myfile > myfile.gz}%
    %\shellcmd{gunzip -c myfile.gz > myfile}%
\end{prompt}

\subsection{``zip'' and ``unzip''} 

Windows and OS X seem to favour the zip file format, so it's also important to know how to unpack those. We do this using unzip:

\begin{prompt}
    %\shellcmd{unzip myfile.zip}%
\end{prompt}

If we would like to make our own zip archive, we use zip:

\begin{prompt}
    %\shellcmd{zip myfiles.zip myfile1 myfile2 myfile3}%
\end{prompt}

\section{Working with tarballs: ``tar''}

Tar stands for "tape archive" and is a way to bundle files together in a bigger file.

You will normally want to unpack these files more often than you make them. To unpack a ``tar'' file you use:

\begin{prompt}
    %\shellcmd{tar -xf tarfile.tar}%
\end{prompt}

Often, you will find ``gzip'' compressed tar files on the web. These are called tarballs. You can uncompress these using gunzip and then unpacking them using tar. But tar knows how to open them using the ``-z'' option:

\begin{prompt}
    %\shellcmd{tar -zxf tarfile.tar.gz}%
    %\shellcmd{tar -zxf tarfile.tgz}%
\end{prompt}

\subsection{Order of arguments}

Note: Archive programs like ``zip'', ``tar'', and ``jar'' use arguments in the ''opposite direction'' of copy commands. 

\begin{prompt}
    # cp, ln: <source(s)> <target>
    %\shellcmd{cp source1 source2 source3 target}%
    %\shellcmd{ln -s source target}%
    
    # zip, tar: <target> <source(s)>
    %\shellcmd{zip zipfile.zip source1 source2 source3}%
    %\shellcmd{tar -cf tarfile.tar source1 source2 source3}%
\end{prompt}

If you use tar with the source files first then the first file will be overwritten. You can control the order of arguments of ``tar'' if it helps you remember:

\begin{prompt}
    %\shellcmd{tar -c source1 source2 source3 -f tarfile.tar}%
\end{prompt}
