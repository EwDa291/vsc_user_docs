\chapter{Graphical applications with VNC}
\label{ch:vnc}

Virtual Network Computing is a graphical desktop sharing system that enables you
to interact with graphical software running on HPC infrastructure from your own
computer.

\section{Starting a VNC server}
\label{sec:start-vnc}

First login on the login node (see \autoref{sec:first-time-connection-to-the-hpc}),
then start \lstinline|vncserver| with:

\begin{prompt}
%\shellcmd{vncserver -geometry 1920x1080 -localhost}%
You will require a password to access your desktops.

Password:%\emph{<{}enter a secure password>{}}%
Verify:%\emph{<{}enter the same password>{}}%
Would you like to enter a view-only password (y/n)? %\emph{n}%
A view-only password is not used

New '%\strong{\loginhost{}}%:%\strong{6}% (%\userid{}%)' desktop is %\loginhost{}%:6

Creating default startup script %\homedir{}%/.vnc/xstartup
Creating default config %\homedir{}%/.vnc/config
Starting applications specified in %\homedir{}%/.vnc/xstartup
Log file is %\homedir{}%/.vnc/%\loginhost{}%:6.log

\end{prompt}

When prompted for a password, make sure to enter a secure password: if someone
can guess your password, they will be able to do anything with your account you can.

Note down the details in bold: the hostname (in the example: \texttt{\loginhost{}})
and the port number (in the example: \lstinline|6|).

It's important to remember that VNC sessions are permanent. They survive network
problems and (unintended) connection loss. This means you can logout and go home
without a problem (like the terminal equivalent \lstinline|screen| or \lstinline|tmux|).
This also means you don't have to start \lstinline|vncserver| each time you want to connect.

\section{List running VNC servers}
\label{sec:list-vnc}

You can get a list of running VNC servers on a node with

\begin{prompt}
%\shellcmd{vncserver -list}%
TigerVNC server sessions:

X DISPLAY #	PROCESS ID
:6		    30713
\end{prompt}

This only displays the running VNC servers on \strong{the login node you run the command on}.

To see what login nodes you are running a VNC server on, you can run the \lstinline|ls .vnc/*.pid|
command in your home directory: the files shown have the hostname of the login node in the filename:

\begin{prompt}
%\shellcmd{cd \$HOME}%
%\shellcmd{ls .vnc/*.pid}%
.vnc/%\loginhost{}%:6.pid
.vnc/%\altloginhost{}%:8.pid
\end{prompt}

This shows that there is a VNC server running on \texttt{\loginhost{}} on port 5906
and another one running \texttt{\altloginhost{}} on port 5908 (see also \autoref{sec:source-port-vnc}).

\section{Connecting to a VNC server}

The VNC server runs on a \strong{specific login node} (in the example above, on \texttt{\loginhost{}}).

In order to access your VNC server, you will need to set up an SSH tunnel from your workstation
to this login node (see \autoref{sec:ssh-tunnel-vnc}).

Login nodes are rebooted from time to time. You can check that the VNC server is still
running in the same node by executing \lstinline|vncserver -list| (see also \autoref{sec:list-vnc}).
If you get an empty list, it means that there is no VNC server running on the login node.

\subsection{Determining the source port}
\label{sec:source-port-vnc}

To set up the SSH tunnel required to connect to your VNC server, you will need to port forward the VNC port
to your workstation. The \emph{source port} is the sum of \lstinline|5900|
and the port number we noted down earlier (\lstinline|6|); in this case, that is \lstinline|5906|.
The \emph{destination port} should be the same as the source port. The \emph{host} is \lstinline|localhost|,
which means ``your own computer'': we set up an SSH tunnel that connects
the VNC port on the login node to the same port on your local computer.

\subsection{Setting up the SSH tunnel}
\label{sec:ssh-tunnel-vnc}

\ifwindows
See \autoref{par:ssh-tunnel-windows}. Use the details specified here (host, destination port,
source port).
\else

Execute the following command to set up the SSH tunnel.\\
\strong{Replace the port number (\lstinline|5906|) and the user ID (\texttt{\userid{}}) with your own!}

\begin{prompt}
%\shellcmd{ssh -L 5906:localhost:5906 \userid{}@\loginnode{}}%
\end{prompt}
\fi

\subsubsection{Connecting to the right login node}

After setting up the SSH tunnel, you need to make sure you are connected to the correct login node,
i.e.\ the one on which the VNC server was started (see \autoref{sec:start-vnc}). In the example,
the VNC server was started on \texttt{\loginhost}.

In the session you created to set up the SSH tunnel, check which login node you are connected to
using the \lstinline|hostname| command:

\begin{prompt}
%\shellcmd{hostname}%
%\altloginhost{}%
\end{prompt}

Note that in the example above, we are not (yet) connected to the correct login node (i.e., \texttt{\loginhost}).

There are two possible scenarios:

\begin{itemize}

\item If you are connected to the right login node, your SSH tunnel is set up correctly to connect to your VNC server
      (see \autoref{sec:vnc-client}).

\item If you are connected to a different login node (e.g., \texttt{\altloginhost} rather than \texttt{\loginhost},
      as shown above), you need to set up another SSH tunnel on the login node you are connected to,
      to forward the port you will be connecting to from your workstation to the correct login node.

      To set up an SSH tunnel to forward port \lstinline|5906| to \texttt{\loginhost}, run the following command
      on the login node you are connected to:

\begin{prompt}
%\shellcmd{ssh -L 5906:localhost:5906 \loginhost{}}%
%\shellcmd{hostname}%
%\loginhost{}%
\end{prompt}

      \strong{Do not forget to change the port (\lstinline|5906|) and hostname of the login node (\texttt{\loginhost})
              in the command shown above, if needed.}

      As shown above, you can check again using the \lstinline|hostname| command whether you are indeed connected
      to the right login node.
\end{itemize}

Once you are connected to the right login node through an SSH tunnel, you can go ahead and connect to yout VNC server
(see \autoref{sec:vnc-client}).

\subsection{Connecting using a VNC client}
\label{sec:vnc-client}

\ifwindows

You can download a free VNC client from \url{https://sourceforge.net/projects/turbovnc/files/}.
You can download the latest version by clicking the top-most folder that has a version number
in it that doesn't also have \lstinline|beta| in the version. Then download a file that looks like
\lstinline|TurboVNC64-2.1.2.exe| (the version number can be different, but the \lstinline|64|
should be in the filename) and execute it.
\fi
\ifmac
You can download a free VNC client from \url{https://sourceforge.net/projects/turbovnc/files/}.
You can download the latest version by clicking the top-most folder that has a version number
in it that doesn't also have \lstinline|beta| in the version. Then download a file ending in
\lstinline|TurboVNC64-2.1.2.dmg| (the version number can be different) and execute it.
\fi
\iflinux
Download and setup a VNC client. A good choice is \lstinline|tigervnc|. You can start
it with the \lstinline|vncviewer| command.
\fi

Now start your VNC client and connect to \lstinline|localhost:5906|.
\strong{Make sure you replace the port number \lstinline|5906| with your own.}

When promted for a password, use the password you used to setup the VNC server.
When prompted for default or empty panel, choose default.

If you have an empty panel, you can reset your settings with the following commands:

\begin{prompt}
%\shellcmd{xfce4-panel --quit ; pkill xfconfd}%
%\shellcmd{mkdir ~/.oldxfcesettings}%
%\shellcmd{mv ~/.config/xfce4 ~/.oldxfcesettings}%
%\shellcmd{xfce4-panel}%
\end{prompt}

\section{Stopping the VNC server}
\label{sec:stop-vnc}

The VNC server can be killed by running

\begin{prompt}
vncserver -kill :6
\end{prompt}

where \lstinline|6| is the port number we noted down earlier. If you forgot,
you can get it with \lstinline|vncserver -list|.

\section{I forgot the password, what now?}

You can reset the password by first stopping the VNC server (see \autoref{sec:stop-vnc}),
then removing the \lstinline|.vnc/passwd| file (with \lstinline|rm .vnc/passwd|) and then
starting the VNC server again (see \autoref{sec:start-vnc}).
