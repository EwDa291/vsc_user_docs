\chapter{TORQUE options}
\label{ch:torque-options}

\section{TORQUE Submission Flags: common and useful directives}

Below is a list of the most common and useful directives.

\begin{tabular}{|p{\dimexpr 0.15\textwidth-2\tabcolsep}|p{\dimexpr 0.15\textwidth-2\tabcolsep}|p{\dimexpr 0.7\textwidth-2\tabcolsep}|} \hline
\strong{Option} & \strong{System type}  & \strong{Description} \\ \hline
-k              & All                   & Send ``stdout'' and/or ``stderr'' to your home directory when the job runs\newline \strong{\#PBS -k o} or \strong{\#PBS -k e} or \strong{\#PBS -koe}  \\ \hline
-l              & All                   & Precedes a resource request, e.g., processors, wallclock \\ \hline
-M              & All                   & Send an e-mail messages to an alternative e-mail address\newline \strong{\#PBS -M me@mymail.be} \\ \hline
-m              & All                   & Send an e-mail address when a job \strong{b}egins execution and/or \strong{e}nds or \strong{a}borts\newline \strong{\#PBS -m b} or \strong{\#PBS -m be} or \strong{\#PBS -m ba} \\ \hline
mem             & Shared\newline Memory & Specifies the amount of memory you need for a job.\newline \strong{\#PBS -l mem=80gb} \\ \hline
mpiprocs        & Clusters              & Number of processes per node on a cluster. This should equal number of processors on a node in most cases.\newline \strong{\#PBS -l mpiprocs=4} \\ \hline
-N              & All                   & Give your job a unique name\newline \strong{\#PBS -N galaxies1234} \\ \hline
-ncpus          & Shared\newline Memory & The number of processors to use for a shared memory job. \newline \strong{\#PBS ncpus=4} \\ \hline
-r              & All                   & Control whether or not jobs should automatically re-run from the start if the system crashes or is rebooted. Users with check points might not wish this to happen.\newline \strong{\#PBS -r n\newline \#PBS -r y} \\ \hline
select          & Clusters              & Number of compute nodes to use. Usually combined with the mpiprocs directive\newline \strong{\#PBS -l select=2} \\ \hline
-V              & All                   & Make sure that the environment in which the job \strong{runs} is the same as the environment in which it was \strong{submitted.\newline \#PBS -V} \\ \hline
Walltime        & All                   & The maximum time a job can run before being stopped. If not used a default of a few minutes is used. Use this flag to prevent jobs that go bad running for hundreds of hours. Format is HH:MM:SS\newline \strong{\#PBS -l walltime=12:00:00} \\ \hline
\end{tabular}

\section{Environment Variables in Batch Job Scripts}

TORQUE-related environment variables in batch job scripts.
\begin{code}{bash}
# Using PBS - Environment Variables:
# When a batch job starts execution, a number of environment variables are
# predefined, which include:
#
#      Variables defined on the execution host.
#      Variables exported from the submission host with
#                -v (selected variables) and -V (all variables).
#      Variables defined by PBS.
#
# The following reflect the environment where the user ran qsub:
# PBS_O_HOST    The host where you ran the qsub command.
# PBS_O_LOGNAME Your user ID where you ran qsub.
# PBS_O_HOME    Your home directory where you ran qsub.
# PBS_O_WORKDIR The working directory where you ran qsub.
#
# These reflect the environment where the job is executing:
# PBS_ENVIRONMENT       Set to PBS_BATCH to indicate the job is a batch job,
#         or to PBS_INTERACTIVE to indicate the job is a PBS interactive job.
# PBS_O_QUEUE   The original queue you submitted to.
# PBS_QUEUE     The queue the job is executing from.
# PBS_JOBID     The job's PBS identifier.
# PBS_JOBNAME   The job's name.
\end{code}

\strong{\emph{IMPORTANT!!}} All PBS directives MUST come before the first line
of executable code in your script, otherwise they will be ignored.

When a batch job is started, a number of environment variables are created that
can be used in the batch job script. A few of the most commonly used variables
are described here.

\begin{tabular}{|p{\dimexpr 0.3\textwidth-2\tabcolsep}|p{\dimexpr 0.7\textwidth-2\tabcolsep}|} \hline
\strong{Variable} & \strong{Description} \\ \hline
PBS\_ENVIRONMENT & set to PBS\_BATCH to indicate that the job is a batch job; otherwise, set to PBS\_INTERACTIVE to indicate that the job is a PBS interactive job. \\ \hline
PBS\_JOBID       & the job identifier assigned to the job by the batch system. This is the same number you see when you do \emph{qstat}. \\ \hline
PBS\_JOBNAME     & the job name supplied by the user \\ \hline
PBS\_NODEFILE    & the name of the file that contains the list of the nodes assigned to the job . Useful for Parallel jobs if you want to refer the node, count the node etc. \\ \hline
PBS\_QUEUE       & the name of the queue from which the job is executed \\ \hline
PBS\_O\_HOME     & value of the HOME variable in the environment in which \emph{qsub} was executed \\ \hline
PBS\_O\_LANG     & value of the LANG variable in the environment in which \emph{qsub} was executed \\ \hline
PBS\_O\_LOGNAME  & value of the LOGNAME variable in the environment in which \emph{qsub} was executed \\ \hline
PBS\_O\_PATH     & value of the PATH variable in the environment in which \emph{qsub} was executed \\ \hline
PBS\_O\_MAIL     & value of the MAIL variable in the environment in which \emph{qsub} was executed \\ \hline
PBS\_O\_SHELL    & value of the SHELL variable in the environment in which \emph{qsub} was executed \\ \hline
PBS\_O\_TZ       & value of the TZ variable in the environment in which \emph{qsub} was executed \\ \hline
PBS\_O\_HOST     & the name of the host upon which the \emph{qsub} command is running \\ \hline
PBS\_O\_QUEUE    & the name of the original queue to which the job was submitted \\ \hline
PBS\_O\_WORKDIR  & the absolute path of the current working directory of the \emph{qsub} command. This is the most useful. Use it in every job-script. The first thing you do is, cd \$PBS\_O\_WORKDIR after defining the resource list. This is because, pbs throw you to your \$HOME directory. \\ \hline
PBS\_O\_NODENUM  & node offset number \\ \hline
PBS\_O\_VNODENUM & vnode offset number \\ \hline
PBS\_VERSION     & Version Number of TORQUE, e.g., TORQUE-2.5.1 \\ \hline
PBS\_MOMPORT     & active port for mom daemon \\ \hline
PBS\_TASKNUM     & number of tasks requested \\ \hline
PBS\_JOBCOOKIE   & job cookie \\ \hline
PBS\_SERVER      & Server Running TORQUE \\ \hline
\end{tabular}

