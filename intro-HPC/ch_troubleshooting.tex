\chapter{Troubleshooting}
\label{ch:troubleshooting}



\section{Walltime issues}
If you get from your job output error a message similar to this:

\begin{prompt}
 =>> PBS: job killed: walltime 146 exceeded limit 120
\end{prompt}

You job has exceeded the requested walltime. Probably the job requires more time to be executed.
See section~\ref{sec:specifying-walltime-requirements} for more information about how to request the walltime in an accurate way.
If you need more than \strong{72h} to execute your job you \strong{must} use \emph{checkpointing} in that case.
% FIXME: Refer to Checkpointing section.



\section{Out of quota issues}

Sometimes the jobs hang at some point or they stop writing in the disk. These errors are usually
related to the quota usage. Probably you are reached your quota limit at some storage endpoint. 
You should move (or remove) the data to a different storage endpoint to be able to write to the disk and then resubmit the jobs. 
See section~\ref{subsec:predfined-user-directories} and section~\ref{subsec:predfined-quotas} for more information about 
quotas and how to use the storage endpoints in an efficient way.


\section{SSH keys do not match}
Each SSH user and host key has its own and unique fingerprint.
If you run into a warning similar to this:

\begin{prompt}
@@@@@@@@@@@@@@@@@@@@@@@@@@@@@@@@@@@
@ WARNING: REMOTE HOST IDENTIFICATION HAS CHANGED! @
@@@@@@@@@@@@@@@@@@@@@@@@@@@@@@@@@@@
IT IS POSSIBLE THAT SOMEONE IS DOING SOMETHING NASTY!
Someone could be eavesdropping on you right now (man-in-the-middle attack)!
It is also possible that the RSA host key has just been changed.
The fingerprint for the RSA key sent by the remote host is
xxxxxxxxxxxxxxxxxxxxxxxxxxxxxxxxxxxxxxxxxxxxxxxxx.
Please contact your system administrator.
Add correct host key in /home/user/.ssh/known_hosts to get rid of this message.
Offending key in /home/user/.ssh/known_hosts:1
RSA host key for ras.mydomain.com has changed and you have requested strict checking.
Host key verification failed.
\end{prompt}

Probably the login node SSH host key was updated or changed,  if this is the case it means you have an old 
version of our key cached somewhere, you can remove it by:

\begin{prompt}
%\shellcmd{ssh-keygen -R \emph{\loginnode}}%
\end{prompt}

and try again, you should be able to add the key again and ssh to the login node ``\strong{\emph{\loginnode}}''.
If you still have issues to connect to the login node you should contact to your HPC administrators.

