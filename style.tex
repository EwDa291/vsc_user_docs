\usepackage[english]{babel}
\usepackage{amssymb}
\usepackage{amsmath}
\usepackage{hyperref}
% allow conditional compilation
\usepackage{etoolbox}
\usepackage[toc,nonumberlist]{glossaries}
\usepackage{makeidx}

\usepackage{color}
\usepackage{listings}
\usepackage{tikz}
\usepackage[T1]{fontenc}
\usepackage{graphics}
\usepackage{multirow}
\usepackage{substr}
\usepackage{xstring}

\lstset{basicstyle=\ttfamily,breaklines=true, keepspaces=true}
\lstdefinestyle{prompt}{
  language=bash,
  frame=tblr,
  columns=fullflexible,
  escapechar=\%,
  fillcolor=\color{lightgray},
  backgroundcolor=\color{lightgray},
  }
\lstdefinestyle{code}{
  numbers=left,
  escapechar=\%,
}

\lstnewenvironment{prompt} {\lstset{style=prompt}} {}
\lstnewenvironment{code}[1]{\lstset{style=code, language=#1}} {}
\lstnewenvironment{prog}{\lstset{style=code}} {}

% Taken from: http://tex.stackexchange.com/questions/9363/how-does-one-insert-a-backslash-or-a-tilde-into-latex
\renewcommand{\tilde}{\raise.17ex\hbox{$\scriptstyle\mathtt{\sim}$}}
\newcommand{\shellcmd}[1]{\$ \textbf{\texttt{#1}\\}}

% LaTeX hacks to allow underscores in filenames. Adapted from
% http://tex.stackexchange.com/questions/69142/include-figure-from-macro-with-underscore-in-filename
\newcommand{\examplecodeStub}[2]{
  \lstinputlisting[
    language=#1,
    numbers=left,
    title=\IfSubStringInString{/}{#2}{\BehindSubString{/}{#2}}{#2},
    frame=tlbr]
    {\exampledir/#2}
    \endgroup
}
\def\examplecode{\begingroup
  \catcode`\_=12
\examplecodeStub}

\newcommand{\exampledir}{examples/\Chaptername}

% http://tex.stackexchange.com/questions/62241/how-to-get-the-current-chapter-name-section-name-subsection-name-etc
% getting the chapter name automatically
\let\Chaptermark\chaptermark
\def\chaptermark#1{
  \StrSubstitute{#1}{ }{-}[\x]
  \StrSubstitute{\x}{/}{-}[\Chaptername]
  \Chaptermark{#1}}


\parindent=0pt
\parskip=7pt
\newcommand{\ignore}[1]{}

\newif\ifremark
\long\def\remark#1{
  \ifremark%
  \begingroup%
  \dimen0=\columnwidth
  \advance\dimen0 by -1in%
  \setbox0=\hbox{\parbox[b]{\dimen0}{\protect\em #1}}
  \dimen1=\ht0\advance\dimen1 by 2pt%
  \dimen2=\dp0\advance\dimen2 by 2pt%
  \vskip 0.25pt%
  \hbox to \columnwidth{%
    \vrule height\dimen1 width 3pt depth\dimen2%
    \hss\copy0\hss%
    \vrule height\dimen1 width 3pt depth\dimen2%
  }%
  \endgroup%
\fi}

% custom \strong declaration
% http://tex.stackexchange.com/questions/14667/does-latex-define-a-semantic-equivalent-of-textbf
\makeatletter
\newcommand{\strong}[1]{\@strong{#1}}
\newcommand{\@@strong}[1]{\textbf{\let\@strong\@@@strong#1}}
\newcommand{\@@@strong}[1]{\textnormal{\let\@strong\@@strong#1}}
\let\@strong\@@strong
\makeatother

\remarktrue

\newcommand{\includesite}[1]{
  \ifleuven
    \include{leuven/#1}
  \fi
  \ifantwerpen
    \include{antwerpen/#1}
  \fi
  \ifhasselt
    \include{hasselt/#1}
  \fi
  \ifbrussel
    \include{brussel/#1}
  \fi
  \ifgent
    \include{gent/#1}
  \fi
}
\newcommand{\inputsite}[1]{
  \ifleuven
    \input{leuven/#1}
  \fi
  \ifantwerpen
    \input{antwerpen/#1}
  \fi
  \ifhasselt
    \input{hasselt/#1}
  \fi
  \ifbrussel
    \input{brussel/#1}
  \fi
  \ifgent
    \input{gent/#1}
  \fi
}


\newif\ifwindows
\newif\iflinux
\newif\ifmac
\newif\ifleuven
\newif\ifhasselt
\newif\ifbrussel
\newif\ifgent
\newif\ifantwerpen

\windowsfalse
\linuxfalse
\macfalse
\leuvenfalse
\hasseltfalse
\brusselfalse
\gentfalse
\antwerpenfalse

\ifdefined\iswindows
  \windowstrue
\fi
\ifdefined\islinux
  \linuxtrue
\fi
\ifdefined\ismac
  \mactrue
\fi
\ifdefined\isleuven
  \leuventrue
\fi
\ifdefined\ishasselt
  \hasselttrue
\fi
\ifdefined\isbrussel
  \brusseltrue
\fi
\ifdefined\isgent
  \genttrue
\fi
\ifdefined\isantwerpen
  \antwerpentrue
\fi
