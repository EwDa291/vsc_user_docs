\chapter{MATLAB}
\label{ch:matlab}

\section{Why is the MATLAB compiler required?}

The main reason behind this alternative way of using MATLAB is licensing: only
a limited number of MATLAB sessions can be active at the same time. However, once
the MATLAB program is compiled using the MATLAB compiler, the resulting stand-alone
executable can be run without needing to contact the license server.

Note that a license is required for the MATLAB Compiler,
see \url{https://nl.mathworks.com/help/compiler/index.html}. If the \lstinline|mcc|
command is provided by the MATLAB installation you are using, the MATLAB compiler
can be used as explained below.

\ifgent
Only a limited amount of MATLAB sessions can be active at the same time because
there are only a limited amount of MATLAB research licenses available on
the \university MATLAB license server. If each job would need a license,
licenses would quickly run out.
\fi

\section{How to compile MATLAB code}

Compiling MATLAB code can only be done from the login nodes, because only login
nodes can access the MATLAB license server, workernodes on clusters can not.

To access the MATLAB compiler, the \lstinline|MATLAB| module should be loaded first. Make sure
you are using the same \lstinline|MATLAB| version to compile and to run the compiled MATLAB
program.

\begin{prompt}
%\shellcmd{module avail MATLAB}%
----------------------%\modulelocation{}%----------------------
   MATLAB/2016b    MATLAB/2017b    MATLAB/2018a (D)
%\shellcmd{module load MATLAB/2018a}
\end{prompt}

After loading the \lstinline|MATLAB| module, the \lstinline|mcc| command can be used. To get help on
\lstinline|mcc|, you can run \lstinline|mcc -?|.

To compile a standalone application, the \lstinline|-m|
flag is used (the \lstinline|-v| flag means verbose output).
To show how \lstinline|mcc| can be used, we use the \lstinline|magicsquare| example
that comes with MATLAB.

First, we copy the \lstinline|magicsquare.m| example that comes with MATLAB to \lstinline|example.m|:

\begin{prompt}
%\shellcmd{cp \$EBROOTMATLAB/extern/examples/compiler/magicsquare.m example.m}%
\end{prompt}

To compile a MATLAB program, use \lstinline|mcc -mv|:

\begin{prompt}
%\shellcmd{mcc -mv example.m}%
Opening log file:  %\homedir{}/java.log.34090%
Compiler version: 6.6 (R2018a)
Dependency analysis by REQUIREMENTS.
Parsing file "%\homedir{}/example.m%"
	(Referenced from: "Compiler Command Line").
Deleting 0 temporary MEX authorization files.
Generating file "%\homedir{}/readme.txt%".
Generating file "run\_example.sh".
\end{prompt}

\subsection{Libraries}

To compile a MATLAB program that \emph{needs a library}, you can use the
\texttt{-I \emph{library\_path}} flag. This will tell the compiler to also
look for files in \texttt{\emph{library\_path}}.

It's also possible to use the \texttt{-a \emph{path}} flag. That will result in
all files under the \texttt{\emph{path}} getting added to the final executable.

For example, the command \lstinline|mcc -mv example.m -I examplelib -a datafiles| will
compile \lstinline|example.m| with the MATLAB files in \lstinline|examplelib|, and will
include all files in the \lstinline|datafiles| directory in the binary it produces.

\subsection{Memory issues during compilation}

If you are seeing Java memory issues during the compilation of your MATLAB program
on the login nodes, consider tweaking the default maximum heap size (128M) of Java
using the \lstinline|_JAVA_OPTIONS| environment variable with:

\begin{prompt}
%\shellcmd{export \_JAVA\_OPTIONS="-Xmx64M"}%
\end{prompt}

The MATLAB compiler spawns multiple Java processes, and because of the default memory
limits that are in effect on the login nodes, this might lead to a crash of the compiler
if it's trying to create to many Java processes. If we lower the heap size, more
Java processes will be able to fit in memory.

Another possible issue is that the heap size is too small. This could result
in errors like:

\begin{prompt}
Error: Out of memory
\end{prompt}

A possible solution to this is by setting the maximum heap size to be bigger:

\begin{prompt}
%\shellcmd{export \_JAVA\_OPTIONS="-Xmx512M"}%
\end{prompt}

\section{Multithreading}

MATLAB can only use the cores in a single
workernode (unless the Distributed Computing toolbox is used, see
\url{https://nl.mathworks.com/products/distriben.html}).


The amount of workers used by MATLAB for the parallel toolbox can be controlled
via the \lstinline|parpool| function: \lstinline|parpool(16)| will use 16 workers.
It's best to specify the amount of workers,
because otherwise you might not harness the full compute power available (if you have
too few workers), or you might negatively impact performance (if you have too much workers).
By default, MATLAB uses a fixed number of workers (12).

You should use a number of workers that is equal to the number of cores you requested
when submitting your job script (the \lstinline|ppn| value, see \autoref{subsec:generic-resource-requirements}).
You can determine the right number of workers to use via the following code snippet in your MATLAB program:

\examplecode{MATLAB}{parpool.m}


See also \href{https://nl.mathworks.com/help/distcomp/parpool.html}{the parpool documentation}.


\section{Java output logs}

Each time MATLAB is executed, it generates a Java log file in the users home directory.
The output log directory can be changed using:

\begin{prompt}
%\shellcmd{MATLAB\_LOG\_DIR=\emph{<OUTPUT\_DIR>}}%
\end{prompt}

where \lstinline|<OUTPUT_DIR>| is the name of the desired output directory. To create
and use a temporary directory for these logs:

\begin{prompt}
# create unique temporary directory in $TMPDIR (or /tmp/$USER if $TMPDIR is not defined)
# instruct MATLAB to use this directory for log files by setting $MATLAB_LOG_DIR
%\shellcmd{export MATLAB\_LOG\_DIR=\$(mktemp -d -p  \${TMPDIR:-/tmp/\$USER})}%
\end{prompt}

You should remove the directory at the end of your job script:
\begin{prompt}
%\shellcmd{rm -rf \$MATLAB\_LOG\_DIR}%
\end{prompt}

\section{Cache location}

When running, MATLAB will use a cache for performance reasons. This location
and size of this cache can be changed trough the \lstinline|MCR_CACHE_ROOT| and
\lstinline|MCR_CACHE_SIZE| environment variables.

The snippet below would set the maximum cache size to 1024MB and the
location to \lstinline|/tmp/testdirectory|.

\begin{prompt}
%\shellcmd{export MATLAB\_CACHE\_ROOT=/tmp/testdirectory}%
%\shellcmd{export MATLAB\_CACHE\_SIZE=1024M}%
\end{prompt}

So when MATLAB is running, it can fill up to 1024MB of cache in
\lstinline|/tmp/testdirectory|.

\section{MATLAB job script}

All of the tweaks needed to get MATLAB working have been implemented in an example
job script. This job script is also available on the HPC. %TODO: where?

\examplecode{bash}{jobscript.sh}
