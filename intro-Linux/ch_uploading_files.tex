\chapter{Uploading/downloading/editing files}

\section{Uploading/downloading files}
\label{sec:uploading-files}

To transfer files from and to the HPC, see
\href{\HPCManualURL#sec:filetranfer}{the section about tranferring files in chapter 3 of the HPC manual}.

\ifwindows

\subsection{\texttt{dos2unix}}
\label{subsec:dos2unix}
\hypertarget{sec:dos2unix}{}
After copying files it is advised to run

\begin{prompt}
%\shellcmd{dos2unix filename}
\end{prompt}

as this will fix any problems with Windows/Unix conversion.
\fi

\section{Symlinks for data/scratch}
\label{sec:symlink-for-data}

As we end up in the home directory when connecting, it would be convenient if we
could access our data and VO storage. To facilitate this we will create
symlinks to them in our home directory. This will create 4 symbolic links
% You're probably wondering about the duplication? I tried to fix it by putting
% the ifwindows inline. This resulted in either an ugly lone ) or the newly-invented word
% "desktopadd". total_wasted_minutes=10
\ifwindows
(they're like ``shortcuts'' on your desktop and they look like directories in WinSCP)
\else
(they're like ``shortcuts'' on your desktop)
\fi
pointing to the respective storages:

\begin{prompt}
%\shellcmd{cd \$HOME}%
%\shellcmd{ln -s \$VSC\_SCRATCH scratch}%
%\shellcmd{ln -s \$VSC\_DATA data}%
%\shellcmd{ls -l scratch data}%
lrwxrwxrwx 1 %\userid{}% %\userid{}% 31 Mar 27  2009 data -> %\datadir{}%
lrwxrwxrwx 1 %\userid{}% %\userid{}% 34 Jun  5  2012 scratch -> %\scratchdir{}%
\end{prompt}

\section{Editing with \texttt{nano}}

Nano is the simplest editor available on Linux. To open Nano, just type
\lstinline|nano|. To edit a file, you use \lstinline|nano the_file_to_edit.txt|. You will
be presented with the contents of the file and a menu at the bottom with
commands like \lstinline|^O Write Out| The \lstinline|^| is the Control key. So \lstinline|^O| means
\lstinline|Ctrl-O|. The main commands are:

\begin{enumerate}
\item Open ("Read"): \lstinline|^R|
\item Save ("Write Out"): \lstinline|^O|
\item Exit: \lstinline|^X|
\end{enumerate}

More advanced editors (beyond the scope of this page) are \lstinline|vim| and \lstinline|emacs|.
A simple tutorial on how to get started with \lstinline|vim| can be found at \url{https://www.openvim.com/}.

\section{Copying faster with \texttt{rsync}}
\hypertarget{sec:rsync}{}

\lstinline|rsync| is a fast and versatile copying tool. It can be much faster than \lstinline|scp|
when copying large datasets. It's famous for its ``delta-transfer algorithm'', which
reduces the amount of data sent over the network by only sending the differences between
files.

You will need to run \lstinline|rsync| from a computer where it is installed. Installing
\lstinline|rsync| is the easiest on Linux: it comes pre-installed with a lot of distributions.

For example, to copy a folder with lots of CSV files:

\begin{prompt}
%\shellcmd{rsync -rzv testfolder \userid{}@\loginnode{}:data/}
\end{prompt}

will copy the folder \lstinline|testfolder| and its contents to \lstinline|$VSC_DATA| on
the \hpc, assuming the \lstinline|data| symlink is present in your home directory,
see \autoref{sec:symlink-for-data}.

The \lstinline|-r| flag means ``recursively'', the \lstinline|-z| flag means that compression
is enabled (this is especially handy when dealing with CSV files because they compress well) and the \lstinline|-v|
enables more verbosity (more details about what's going on).

To copy large files using \lstinline|rsync|, you can use the \lstinline|-P| flag: it enables
both showing of progress and resuming partially downloaded files.

To copy files from the \hpc to your local computer, you can also use \lstinline|rsync|:

\begin{prompt}
%\shellcmd{rsync -rzv \userid{}@\loginnode{}:data/bioset local\_folder}
\end{prompt}

This will copy the folder \lstinline|bioset| and its contents that on \lstinline|$VSC_DATA| of
the \hpc to a local folder named \lstinline|local_folder|.

See \lstinline|man rsync| or \url{https://linux.die.net/man/1/rsync} for more information about rsync.

\section{Exercises}

\begin{enumerate}
    \item Download the file \lstinline|/etc/hostname| to your local computer.
    \item Upload a file to a subdirectory of your personal \lstinline|$VSC_DATA| space.
    \item Create a file named \lstinline|hello.txt| and edit it using \lstinline|nano|.
\end{enumerate}
