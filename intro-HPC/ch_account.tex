\chapter{Getting an HPC Account}
\label{ch:getting-a-hpc-account}

\section{Getting ready to request an account}
\label{sec:getting-ready-to-request-an-account}

All users of \association can request an account on
the \hpc, which is part of the Flemish Supercomputing Center (VSC).

The VSC, abbreviation of Flemish Supercomputer Center, is a virtual
supercomputer center. It is a partnership between the five Flemish
associations: the Association KULeuven,  Ghent University Association, Brussels
University Association, Antwerp University Association and the University
Colleges-Limburg. The VSC is funded by the Flemish Government.

The \hpc clusters use public/private key pairs for user authentication
(rather than passwords). Technically, the private key is stored on your local
computer and always stays there; the public key is stored on the \hpc.
Access to the \hpc is granted to anyone who can prove to have access to the
corresponding private key on his local computer.

\includegraphics*[width=5.75in, height=3.01in, keepaspectratio=false]{ch2-authentication-general}

% IFDEF WINDOWS
\ifwindows

  Since all \hpc clusters use Linux as their main operating system, you will
  need to get acquainted with using the command-line interface and using the
  Terminal. A typical Windows environment does not come with pre-installed
  software to connect and run command-line executables on a \hpc. Some tools
  needs to be installed on your Windows machine first, before we can start the
  actual work.

  \subsection{Install Putty: A free telnet/SSH client}
  \label{sec:install-putty}

  We recommend to use the PuTTY tools package, which is freely available. You
  do not even need to install; just download the executable and run it!
  Alternatively, an installation package is also available.

  Download PuTTY and install it in the C:/Program Files (x86)/PuTTY directory.

  Google ``\emph{PuTTY download}'' or go directly to the official download
  address: \url{http://www.chiark.greenend.org.uk/\~sgtatham/putty/}

  The PuTTY package consists of several components:

  \begin{enumerate}
    \item  \strong{PuTTY}: the Telnet and SSH client itself
    \item  \strong{PSCP}: an SCP client, i.e. command-line secure file copy
    \item  \strong{PSFTP}: an SFTP client, i.e. general file transfer sessions much like FTP
    \item  \strong{Plink}: a command-line interface to the PuTTY back ends
    \item  \strong{Pageant}: an SSH authentication agent for PuTTY, PSCP and Plink
    \item  \strong{PuTTYgen}: an RSA and DSA key generation utility
    \item  \strong{pterm}: a standalone terminal emulator
  \end{enumerate}

  \subsection{Generating a public/private key pair}
  \label{sec:generate-key-pair}

  Before requesting a VSC account, you need to generate a pair of \emph{ssh}
  keys. By default, there is no \strong{\emph{ssh}} client software available
  on Windows, so you will typically have to install one yourself. One popular and
  recommended way to do this on Windows is using the freely available
  \strong{\emph{PuTTY}} client, which you can then also use to log on to the
  clusters.

  To generate a public/private key pair, you can use the PuTTYgen key generator.

  Start \strong{\emph{PuTTYgen.exe}} it and follow these steps:

  \begin{enumerate}
    \item  In $<$\emph{Parameters}$>$ (at the bottom of the window), choose
      'SSH-2 RSA' and set the number of bits in the key to 2048.

    \includegraphics*[width=3.49in, height=3.38in, keepaspectratio=false]{ch2-puttygen-bits}

    \item  Click on $<$\emph{Generate}$>$. To generate the key, you must move
      the mouse cursor over the PuTTYgen window (this generates some random
      data that PuTTYgen uses to generate the key pair). Once the key pair is
      generated, your public key is shown in the field $<$\emph{Public key for pasting into OpenSSH authorized\_keys file}$>$.
    \item  Next, it is advised to fill in the $<$\emph{Key comment}$>$ field
      to make it easier identifiable afterwards.
    \item  Next, you should specify a passphrase in the $<$\emph{Key passphrase}$>$
      field and retype it in the $<$\emph{Confirm passphrase}$>$ field.
      Remember, the passphrase protects the private key against unauthorized
      use, so it is best to choose one that is not too easy to guess but that
      you can still remember.

    \includegraphics*[width=3.39in, height=3.28in, keepaspectratio=false]{ch2-puttygen-password}

    \item  Finally, save both the public and private keys in a folder on your
      personal computer (We recommend to create and put them in the folder
      ``C:\textbackslash Users\textbackslash \%USERNAME\%\textbackslash
      AppData\textbackslash Local\textbackslash PuTTY\textbackslash .ssh'')
      with the buttons $<$\emph{Save public key}$>$ and $<$\emph{Save private key}$>$.
      We recommend using the name "\strong{id\_rsa.pub}" for the public key,
      and "\strong{id\_rsa.ppk}" for the private key.
  \end{enumerate}

  If you use another program to generate a key pair, please remember that they
  need to be in the OpenSSH format to access the \hpc clusters.

%iftoggle WINDOWS
\fi

%IFDEF MAC

\ifmac

  \subsection{Open a terminal}
  \label{sec:open-a-terminal}

  Since all VSC clusters use Linux as their main operating system, you will
  need to get acquainted with using the command-line interface and using the
  Terminal.

  To open a Terminal window in OS X, open the Finder and choose

  \emph{$>$$>$ Applications $>$ Utilities $>$ Terminal}

  Before requesting an account, you need to generate a pair of ssh keys. One
  popular way to do this on OS X is using the OpenSSH client included with OS
  X, which you can then also use to log on to the clusters.

  \subsection{Test OpenSSH}
  \label{sec:test-openssh}

  Secure Shell (ssh) is a cryptographic network protocol for secure data
  communication, remote command-line login, remote command execution, and other
  secure network services between two networked computers. In short, ssh
  provides a secure connection between 2 computers via insecure channels
  (Network, Internet, telephone lines, \ldots).

  ``Secure'' means that:

  \begin{enumerate}
    \item  the User is authenticated to the System; and
    \item  the System is authenticated to the User; and
    \item  all data is encrypted during transfer.
  \end{enumerate}

  OpenSSH is a FREE implementation of the SSH connectivity protocol. OS X comes
  with its own implementation of OpenSSH, so you don't need to install any
  third-party software to use it. Just open a Terminal window and jump in!

  On all popular Linux distributions, the OpenSSH software is readily
  available, and most often installed by default. You can check whether the
  OpenSSH software is installed by opening a terminal and typing:

\begin{prompt}
%\shellcmd{ssh -V}%
OpenSSH_4.3p2, OpenSSL 0.9.8e-fips-rhel5 01 Jul 2008
\end{prompt}

  To access the clusters and transfer your files, you will use the following commands:

  \begin{enumerate}
    \item  ssh-keygen: to generate the ssh keys
    \item  ssh: to open a shell on a remote machine;
    \item  sftp: a secure equivalent of ftp;
    \item  scp: a secure equivalent of the remote copy command rcp.
  \end{enumerate}

  \subsection{Generate a public/private key pair with OpenSSH}
  \label{sec:generate-key-pair-with-openssh}

  A key pair might already be present in the default location inside your home
  directory. Therefore, we first check if a key is available with the ``list
  short'' (``ls'')  command:

\begin{prompt}
%\shellcmd{ls  \tilde/.ssh}%
\end{prompt}


  If a key-pair is already available, you would normally get:
\begin{prompt}
authorized_keys2    id_rsa            id_rsa.pub         known_hosts
\end{prompt}

  Otherwise, the command will show:

\begin{prompt}
ls: .ssh: No such file or directory
\end{prompt}

  You can recognize a public/private key pair when a pair of files has the same
  name except for the extension ".pub" added to one of them. In this particular
  case, the private key is "id\_rsa" and public key is "id\_rsa.pub". You may
  have multiple keys (not necessarily in the directory "\tilde/.ssh") if you or
  your operating system requires this.

  You will need to generate a new key pair, when:
  \begin{enumerate}
    \item  you don't have a key pair yet,
    \item  you forgot the passphrase protecting your private key,
    \item  or your private key was compromised.
  \end{enumerate}

  For extra security, the private key itself is encrypted using a 'passphrase',
  to prevent anyone from using your private key even when they manage to copy
  it. You have to 'unlock' the private key by typing the passphrase.

\begin{prompt}
%\shellcmd{ssh-keygen}%
Generating public/private rsa key pair.
Enter file in which to save the key (/home/user/.ssh/id_rsa):
Enter passphrase (empty for no passphrase):
Enter same passphrase again:
Your identification has been saved in /home/user/.ssh/id_rsa.
Your public key has been saved in /home/user/.ssh/id_rsa.pub.
\end{prompt}

  This will ask you for a file name to store the private and public key, and a
  passphrase to protect your private key. It needs to be emphasized that you
  really should choose the passphrase wisely! The system will ask you for it
  every time you want to use the private key that is every time you want to
  access the cluster or transfer your files.

  \strong{Without your key pair, you won't be able to apply for a personal VSC account.}

%iftoggle mac
\fi

\section{Applying for the account}
\label{sec:applying-for-the-account}

Users of the \university need to follow the instructions in \autoref{sec:users-of-university}.

\ifantwerpen
Other users of the Antwerp University Association (AUHA), need to follow the
instructions in \autoref{sec:other-users-of-antwerp-university}.
\fi

\subsection{Users of \university}
\label{sec:users-of-university}

Open the following webpage from within the \university campus network (or using a VPN
connection) \url{https://account.vscentrum.be/req} and choose `\sitename' as your VSC
site.

Browse to \url{https://account.vscentrum.be/req}

Choose `\sitename' as your VSC site.

Click $<$Submit$>$

\includegraphics*[width=5.77in, height=3.51in, keepaspectratio=false]{ch2-browser-select-site}

% IF ANTWERPEN
\ifantwerpen
  You will now be taken to a website of BELNET, where you must authenticate yourself.

  \includegraphics*[width=5.73in, height=4.00in, keepaspectratio=false]{ch2-browser-authenticate}

  Select `Universiteit Antwerpen' in the dropdown box.

  Click $<$Select$>$
\fi
% END IF ANTWERPEN

You will now be taken to the authentification page of your institute.

\ifantwerpen
\includegraphics*[width=5.73in, height=4.00in, keepaspectratio=false]{ch2-browser-authenticate-institute}
\fi

After you log in using your \university login and password (the one you also
use for Webmail, \ldots ), you will be asked to upload the file that contains
your public key, i.e. the file `id\_rsa.pub' which you should already have
generated.

\ifwindows

  This file should have been stored in the directory
  `\emph{/Users/$<$Username$>$/PuTTY/.ssh/}'.

\fi



\ifmac

  This file has been stored in the directory `\tilde/.ssh/'.

  \strong{\underbar{Tip:}} As ".ssh" is an invisible directory, the Finder will
  not show it by default. The easiest way to access the folder, is by pressing
  Apple-Shift--G, which will allow you to enter the name of a directory, which
  you would like to open in Finder. Here, type "\tilde/.ssh" and press enter.

\fi

You will get an e-mail to confirm your account request. After the VSC staff has
approved the account, your account will be created and you will get
confirmation e-mail.

% IF ANTWERPEN
\ifantwerpen
\strong{The site is only accessible from within the University of Antwerp
domain}, so the page won't load from (e.g.) home. However, you can also get
external access to the UA domain using VPN. We refer to the UA ICT infocenter
(\url{http://www.ua.ac.be/main.aspx?c=.INFOCENTER}) for more information.

\subsection{Other users of the Antwerp University Association (AUHA)}
\label{sec:other-users-of-antwerp-university}

All users (researchers, academic staff, etc.) from the higher education
institutions associated with UA can get a VSC account via the UA. There is not
yet an automated form to request for your personal VSC account.

Please e-mail the \hpc staff to get an account (see Contacts information).
You will have to provide a public \emph{ssh} key generated as described
above. Please attach your public key (i.e., the file named `id\_rsa.pub'),
which you will normally find in your .ssh subdirectory within your HOME
Directory. (i.e. /Users/$<$username$>$/.ssh/id\_rsa.pub).


% IF MAC
\ifmac

  \strong{\underbar{Tip:}} As ".ssh" is an invisible directory, the Finder will
  not show it by default. The easiest way to access the folder, is by pressing
  Apple-Shift-G, which will allow you to enter the name of a directory, which
  you would like to open in Finder. Here, type "\tilde/.ssh" and press enter.

\fi
% END IF MAC
\fi
% END IF ANTWERPEN

\subsection{Welcome e-mail}
\label{sec:welcome-email}

Within one day, you should receive a Welcome e-mail with your VSC account details.

\begin{verbatim}
Welcome xxx,
* Your vsc-username is \userid
* You can view your profile details at https://account.vscentrum.be/req
* For further info please visit https://vscentrum.be/neutral/documentation
Your VSC-admin
\end{verbatim}

Now, you can start using the \hpc.

\section{First Time connection to the \hpc}
\label{sec:first-time-connection-to-the-hpc}

\ifwindows

  \subsection{Open a Terminal}
  \label{sec:windows-open-a-terminal}

  You've generated a public/private key pair with PuTTYgen and have an approved
  account on the VSC clusters.  The next step is to setup the connection to (one
  of) the \hpc.

  In the screenshots, we show the setup for user ``\strong{\emph{vsc20167}}''
  to the \hpc cluster via the loginnode
  ``\strong{\emph{\loginnode}}''.

  \begin{enumerate}
    \item  Start the PuTTY executable $<$\strong{\emph{putty.exe}}$>$ in your
      directory `\strong{\emph{C:/Program Files (x86)/PuTTY}}' and the
      configuration screen will pop up. As you will often use the PuTTY tool,
      we recommend adding a shortcut on your desktop.
    \item  Within the category $<$\strong{\emph{Session}}$>$, in the field
      $<$\strong{\emph{Host Name}}$>$, enter the name of the loginnode of the
      \hpc cluster (i.e., ``\strong{\emph{\loginnode}}'')
      you want to connect to.

    \includegraphics*[width=3.59in, height=3.45in, keepaspectratio=false]{ch2-putty-configuration}

    \item  In the category $<$\emph{Connection}$>$ $<$\emph{Data}$>$, in
      the field $<$\emph{Auto-login username}$>$, put in
      $<$\emph{vsc-account}$>$, which is your VSC username that you have
      received by mail after your request was approved.

  \includegraphics*[width=3.11in, height=2.99in, keepaspectratio=false]{ch2-putty-configuration2}

    \item  In the category $<$\emph{Connection$>$$<$SSH$>$$<$Auth$>$}, in the
      field $<$\emph{Private key file for authentication}$>$ click on
      $<$\emph{Browse}$>$ and select the private key (i.e., `id\_rsa.ppk')
      that you generated and saved above.

  \includegraphics*[width=3.25in, height=3.13in, keepaspectratio=false]{ch2-putty-key-setup}

    \item  In the category $<$\emph{Connection$>$$<$SSH $>$$<$X11$>$}, click
      the $<$\emph{Enable X11 Forwarding}$>$ checkbox.

  \includegraphics*[width=3.54in, height=3.41in, keepaspectratio=false]{ch2-putty-x-forwarding}

    \item  Now go back to $<$Session$>$, and fill in ``\emph{Turing}'' in the
      $<$\emph{Saved Sessions}$>$ field and press $<$\emph{Save}$>$ to
      store the session information.

  \includegraphics*[width=3.30in, height=3.17in, keepaspectratio=false]{ch2-putty-saved-session}

    \item  Now pressing $<$\emph{Open}$>$, will open a terminal window and
      asks for you passphrase.

  \includegraphics*[width=3.94in, height=2.47in, keepaspectratio=false]{ch2-putty-enter-password}

    \item  The first time you make a connection to the loginnode, a Security
      Alert will appear and you will be asked to verify the authenticity of the
      loginnode.

  \includegraphics*[width=3.91in, height=2.42in, keepaspectratio=false]{ch2-putty-verify-authenticity}

    \item  After entering your correct passphrase, you will be connected to the
      login-node of the \hpc.
    \item  To check you can now ``Print the Working Directory'' (pwd) and check
      the name of the computer, where you have logged in (hostname):

  \begin{prompt}
  %\shellcmd{pwd}%
  /user/antwerpen/201/vsc20167
  %\shellcmd{hostname --f}%
  ln01.turing.antwerpen.vsc
  \end{prompt}

    \item  For future PuTTY sessions, just select your saved session (i.e.
      ``\emph{Turing}'') from the list, $<$\emph{Load}$>$ it and press
      $<$\emph{Open}$>$.
  \end{enumerate}

%ENDIF windos
\fi

\subsection{Connect}
\label{sec:connect}

The \hpc is only accessible from within the UA campus network, but you can
get external access (e.g., from home) by using a UA-VPN connection.

To make a connection to the \hpc cluster, the ssh command is used:

\begin{prompt}
%\shellcmd{ssh $<$vsc-account$>$@\loginnode}%
\end{prompt}

Here,

\begin{enumerate}
  \item  \emph{$<$vsc-account$>$} is your VSC username that you have received
    by mail after your request was approved,
  \item  \emph{``\loginnode''} is the name of the login
    node of the \hpc cluster you want to connect to.
\end{enumerate}

The first time you make a connection to the login node, you will be asked to
verify the authenticity of the login node, e.g.,

\begin{prompt}
%\shellcmd{ssh \userid{}@\loginnode{}}%
The authenticity of host '%\loginnode (<IP-adress>)%' can't be established.
RSA key fingerprint is b7:66:42:23:5c:d9:43:e8:b8:48:6f:2c:70:de:02:eb.
Are you sure you want to continue connecting (yes/no)? %\strong{\emph{yes}}%
\end{prompt}

Here, user \userid wants to make a connection to the `Turing' cluster at
\university via the login node '\loginnode'.

\ifantwerpen
\strong{Congratulations, you're on the UA Super-Computer now !!!}
\fi
Okay, I am on the supercomputer, but where am I?

So, print the current working directory:
\begin{prompt}
%\shellcmd{pwd}%
%\homedir%
\end{prompt}

Your new private home-directory is ``\homedir''.
Here you can create your own sub-directory structure, copy and prepare your
applications, compile and test them and submit your jobs on the \hpc.

%TODO: maybe do this as well on other sites?
\ifantwerpen
For now, we already have created a symbolic link to the tutorials (including
all the exercises) for you, which you can explore with the ``ls --l''
(\strong{l}ist\strong{s} \strong{l}ong) and the ``cd'' (\strong{c}hange
\strong{d}irectory) commands:
\fi

\begin{prompt}
%\shellcmd{ls -l}%
total 256
lrwxrwxrwx 1 %\userid \userid %25 Nov 13 09:53 tutorials ->
  /apps/antwerpen/tutorials/
%\shellcmd{cd tutorials}%
\end{prompt}

\strong{Tip: }Typing ``\strong{\emph{cd tu$<$TAB$>$}}'' will generate the
``\strong{\emph{cd tutorials}}'' command. \strong{Command-line completion
}(also tab completion) is a common feature of the bash command line
interpreter, in which the program automatically fills in partially typed
commands.

\begin{prompt}
%\shellcmd{ls  -1}%
calcua/
eclipse/
python/
vi/
\end{prompt}

This directory currently contains all training material for the use of:

\begin{enumerate}
\item  The \strong{\emph{\hpc}}.
\item  The \strong{\emph{Eclipse}} environment.
\item  The \strong{\emph{Python}} programming language.
\item  The \strong{\emph{VI}} editor.
\end{enumerate}

More relevant training material to work with the \hpc can always be added
later in this directory.

As we are interested in the use of the \strong{\emph{HPC}}, move further to
\strong{\emph{calcua}} and explore the contents up to 2 levels deep:

\begin{prompt}
%\shellcmd{cd calcua}%
%\shellcmd{tree -L 2}%
.
%\textbar%-- HPC_Tutorial_Mac.pdf
%\textbar%-- HPC_Tutorial_Win.pdf
%\textbar%-- HPC_slides.pdf
%\textbar%-- docs
`-- examples
    %\textbar%-- Chapter04_Batch
    %\textbar%-- Chapter05_Interactive
    %\textbar%-- Chapter06_Files
    %\textbar%-- Chapter07_Parallel
    %\textbar%-- Chapter08_MultiJob
    %\textbar%-- Chapter10_Compile
    %\textbar%-- Chapter11_TuneJobs
    %\textbar%-- Chapter13_Examples
    %\textbar%-- example.pbs
    `-- example.sh
10 directories, 5 files
\end{prompt}

This directory contains:

\begin{enumerate}
  \item This \strong{\emph{HPC Tutorial}} (in either a Mac or Windows
    version).
  \item The according \strong{\emph{HPC slides}} (used during training
    sessions).
  \item A \strong{\emph{docs}} sub-directory, containing interesting \hpc
    related documents.
  \item An \strong{\emph{examples}} sub-directory, containing all the
    examples that you need in this Tutorial, as well as examples that might be
    useful for your specific applications.
\end{enumerate}

\strong{\underbar{Tip:}} For more exhaustive tutorials about Linux usage,
please refer to the following sites:

\begin{enumerate}
  \item \url{http://www.linux.org/lessons/}
  \item \url{http://linux.about.com/od/nwb\_guide/a/gdenwb06.htm}
\end{enumerate}

Of course, all documents are ``write-protected''. The first action is to copy
the contents of the \hpc examples directory to your home directory, so that you
have your own personal copy and that you can start using the examples. The
``-r'' option of the copy command will also copy the contents of the
sub-directories ``\emph{recursively}''.  We first go back to your HOME
directory.

\begin{prompt}
%\shellcmd{cd}%
%\shellcmd{cp --r \tilde/tutorials/calcua/examples \tilde/}%
\end{prompt}

Go to your home directory, check your own private examples directory, \dots  and start working.

\begin{prompt}
%\shellcmd{cd}%
%\shellcmd{ls -l}%
total 792
drwxrwxr-x 10 vsc20167 vsc20167 131072 Nov  4 13:15 examples/
lrwxrwxrwx 1  root     root         25 Nov 12 14:51 tutorials -%$>$% /apps/antwerpen/tutorials/
\end{prompt}

\section{Computation Workflow on the \hpc}
\label{sec:compuation-workflow-on-the-hpc}

A typical Computation workflow will be:

\begin{enumerate}
  \item  Connect to the \hpc
  \item  Transfer your files to the \hpc
  \item  Compile your code and test it
  \item  Create a job script
  \item  Submit your job
  \item  Wait while
  \begin{enumerate}
    \item  your job gets into the queue
    \item  your job gets executed
    \item  your job finishes
  \end{enumerate}
  \item  Move your results
\end{enumerate}

We'll take you through the different tasks one by one in the following
chapters.
