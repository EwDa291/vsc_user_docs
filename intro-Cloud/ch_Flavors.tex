\chapter{Instance types and flavors}

VSC Tier1 Cloud provides several virtual machine instance types
and flavors to fit different use cases.
Each instance type provides several flavor sizes to give different
combinations of CPU, memory, GPU and network resources.

\section{Instance Types}\label{sec:instance-types}
The following table provides the current main instance types available
from the VSC Tier1 Cloud infrastructure:

\begin{table}[h!]
\centering
\begin{tabular}{ |p{6cm}|p{4cm}|p{5cm}| }
  \hline
  \rowcolor{lightgray} \textbf{UPSv1} & \textbf{CPUv1} & \textbf{GPUv1} \\
  \hline
  \begin{itemize}
    \item AMD Epyc 7542 2.9GHz
    \item vCPU oversubscription 2:1
    \item 25Gbit Ethernet
    \item Uninterruptible Power Supply (UPS)
  \end{itemize}
  &
  \begin{itemize}
    \item Intel Xeon CPU E5-2670 2.60GHz
    \item 10Gbit Ethernet
  \end{itemize}
  &
  \begin{itemize}
    \item AMD Epyc 7542 2.9GHz
    \item 25Gbit Ethernet
    \item 1 vGPU NVIDIA Tesla 4
  \end{itemize}
  \\
  \hline
\end{tabular}
\caption{Instance types hardware profiles}
\label{table:instance-type}
\end{table}

Each instance type is appropriate for different workloads:
(\strong{CPUv1}) for regular CPU usage,
(\strong{GPUv1}) for GPU computations,
or (\strong{UPSv1}) for VMs that need to be connected to an uninterruptible power supply.
VMs using UPS will keep up and running even if the datacenter suffers
an unexpected power cut.
(\strong{CPUv1}) and (\strong{GPUv1}) virtual machines are not supported by an UPS and will go offline when an unexpected power cut occurs.

VSC Tier-1 Cloud instance types also provide different kind of network
performance specifications. All the instance types are able to connect
to the available networks: public network, VSC network and shared filesystem
network (NFS). Note that VSC and shared file system network access is only made available
if explicitly requested in the project application.

VSC network gives an optimal path towards other VSC sites. This is ideal for high
performance connections between different clusters and services within VSC.
E.g. when you intend to do high data volume reshuffling between VMs and other Tier-1 components.

\strong{Note:} Cloud projects should request VSC network if they want to
connect to VSC Data component (\url{https://www.vscentrum.be/data}) with iRODS
and Globus from their Tier1 Cloud VMs.

On the other hand, the shared filesystem network is required by the OpenStack
shared filesystem service (Manila) (see chapter \ref{cha:shared-file-systems}
for more information).


\section{Flavor Sizes}\label{sec:flavor-sizes}
A flavor size is a set of virtualized hardware resources to a virtual
machine (VM) instance like system memory size (RAM), virtual cores (vCPUs)
or the root filesystem size. 

The flavor's root disk size is the amount of disk space
used by the root (\emph{/}) partition, an ephemeral disk that the
base image is copied into (see section \ref{launch-an-instance} for
more information about VM persistent/non-persistent instances).

The flavor's root ephemeral storage is only used when booting from
a non-persistent VM, but is not used when booting from a persistent storage volume or
persistent VM.
The flavor's root ephemeral size is not taken into account to
calculate the project's local storage quota either. You can also
create a persistent volume and choose the desired filesystem size for
your persistent VM during the instantiation.
VM persistent volumes could be resized later if that is necessary
(see chapter \ref{cha:launch-manage-inst} for more information).

VSC Tier-1 Cloud VM flavors are grouped by instance types
(see table \ref{table:instance-type}). The current flavor sizes are
available for each instance type and can be used in different
combinations to fit different workload hardware requirements.

\begin{table}[h!]
\centering
\begin{tabular}{ |p{3cm}|p{3cm}|p{3cm}|p{3cm}|p{3cm}| }
  \hline
  \rowcolor{lightgray} \textbf{Flavor name} & \textbf{RAM} & \textbf{Root Disk} & \textbf{# vCPUs} & \textbf{# vGPUs} \\
  \hline
  CPUv1.nano & 64Kb & 1Gb & 1 & 0 \\
  \hline
  CPUv1.tiny & 512Kb & 10Gb & 1 & 0 \\
  \hline
  CPUv1.small & 2Gb & 20Gb & 1 & 0 \\
  \hline
  CPUv1.medium & 4Gb & 30Gb & 2 & 0 \\
  \hline
  CPUv1.large & 8Gb & 40Gb & 4 & 0 \\
  \hline
  CPUv1.xlarge & 16Gb & 40Gb & 8 & 0 \\
  \hline
  CPUv1.1_2xlarge & 60Gb & 40Gb & 8 & 0 \\
  \hline
  CPUv1.2xlarge & 60Gb & 40Gb & 16 & 0 \\  
  \hline
  CPUv1.1_3xlarge & 180Gb & 80Gb & 14 & 0 \\
  \hline
  CPUv1.3xlarge & 120Gb & 80Gb & 16 & 0 \\
  \hline
  CPUv1.4xlarge & 360Gb & 80Gb & 20 & 0 \\
  \hline
  UPSv1.small & 2Gb & 20Gb & 1 & 0 \\
  \hline
  UPSv1.medium & 4Gb & 30Gb & 2 & 0 \\
  \hline
  UPSv1.large & 8Gb & 40Gb & 4 & 0 \\
  \hline
  UPSv1.2xlarge & 60Gb & 40Gb & 16 & 0 \\
  \hline
  UPSv1.3xlarge & 120Gb & 80Gb & 16 & 0 \\
  \hline
  GPUv1.small & 2Gb & 20Gb & 1 & 1 \\
  \hline
  GPUv1.medium & 4Gb & 30Gb & 2 & 1 \\
  \hline
  GPUv1.large & 8Gb & 40Gb & 4 & 1 \\
  \hline
  GPUv1.2xlarge & 60Gb & 40Gb & 16 & 1 \\
  \hline
\end{tabular}
\caption{Flavor sizes}
\label{table:flavor-size}
\end{table}


E.g. The \strong{GPUv1}.\emph{large} OpenStack flavor will instantiate a
VM with 4 AMD Epyc 7542 2.9GHz vCPUs, with 1 NVIDIA Tesla 4 vGPUs, 8GB
of RAM, and a 40GB root disk.


%%% Local Variables:
%%% mode: latex
%%% TeX-master: "intro-Cloud"
%%% End:
