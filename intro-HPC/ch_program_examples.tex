\chapter{Program examples}

Go to our examples in Chapter 11:
\begin{prompt}
$ %\textbf{cd \~/examples/Chapter13\_Examples}%
\end{prompt}

Here, we just have put together a number of examples for your convenience.
We did an effort to put comments inside the source files, so the source code files are (should be) self-explanatory.

\begin{enumerate}
\item  01\_Python
\item  02\_C\_C++
\item  03\_Matlab
\item  04\_MPI\_C
\item  05a\_OMP\_C
\item  05b\_OMP\_FORTRAN
\end{enumerate}

The above 2 OMP directories contain the following examples:

\begin{tabular}{|p{1.0in}|p{1.1in}|p{1.7in}|} \hline
\textbf{C Files} & \textbf{Fortran Files} & \textbf{Description} \\ \hline
omp\_hello.c & omp\_hello.f & Hello world \\ \hline
omp\_workshare1.c & omp\_workshare1.f & Loop work-sharing \\ \hline
omp\_workshare2.c & omp\_workshare2.f & Sections work-sharing \\ \hline
omp\_reduction.c & omp\_reduction.f & Combined parallel loop reduction \\ \hline
omp\_orphan.c\newline  & omp\_orphan.f & Orphaned parallel loop reduction \\ \hline
omp\_mm.c & omp\_mm.f & Matrix multiply \\ \hline
omp\_getEnvInfo.c & omp\_getEnvInfo.f & Get and print environment information \\ \hline
omp\_bug1.c \newline omp\_bug1fix.c \newline omp\_bug2.c \newline omp\_bug3.c \newline omp\_bug4.c \newline omp\_bug4fix \newline omp\_bug5.c \newline omp\_bug5fix.c \newline omp\_bug6.c & omp\_bug1.f \newline omp\_bug1fix.f \newline omp\_bug2.f \newline omp\_bug3.f \newline omp\_bug4.f \newline omp\_bug4fix \newline omp\_bug5.f \newline omp\_bug5fix.f \newline omp\_bug6.f & Programs with bugs and their solution \\ \hline
\end{tabular}

Compile by any of the following commands:

\begin{tabular}{|p{0.5in}|p{3.3in}|} \hline
\textbf{C:} & icc -openmp omp\_hello.c -o hello\newline pgcc -mp omp\_hello.c -o hello\newline gcc -fopenmp omp\_hello.c -o hello \\ \hline
\textbf{Fortran:} & ifort -openmp omp\_hello.f -o hello\newline pgf90 -mp omp\_hello.f -o hello\newline gfortran -fopenmp omp\_hello.f -o hello \\ \hline
\end{tabular}

\begin{enumerate}
\item  06\_NWChem
\item  07\_Wien2k
\item  08\_Gaussian
\item  09\_Fortran
\item  10\_PQS
\end{enumerate}

Be invited to explore the examples.
