\chapter{HPC Policies}
\label{ch:hpc-policies}

The cluster has been setup to support the ongoing research in all the domains
at the Free University of Brussels.  As such, it should be considered as valuable
research equipment.  User shall not abuse the system for other purposes. By
registering, users accept this implicit user agreement.

In order to shared resources in a fair way, a number of site policies have been
implemented:

\begin{enumerate}
\item  Priority based scheduling
\item  Fairshare mechanism
\end{enumerate}

\section{Priority based scheduling?}

The scheduler associates a priority number to each job:

\begin{enumerate}
\item  the highest priority job will (usually) be the next one to run;
\item  jobs with a negative priority will (temporarily) be blocked.
\end{enumerate}

Currently, the priority calculation is based on:

\begin{enumerate}
\item  the time a job is queued: the priority of a job is increased in accordance to the time (in minutes) it is waiting in the queue to get started;
\item  a ``\emph{Fairshare}'' window of 24*8 hours with a target of 10\%: the priority of jobs of a user who has used too many resources over the current ``\emph{Fairshare}'' window is lowered (with a maximum of 10\% per 8 hours window).
\end{enumerate}

The Priority system can of course be adapted in the future, but this will be will be communicated.

\section{Fairshare mechanism}

A ``\emph{Fairshare}'' system has been setup to allow all users to get their
fair share of the \hpcname utilisation.  The mechanism incorporates historical
resource utilisation over the last week.  A users ``Fairshare'' index is used
to adapt its job priority in the queue.  It only affects its job's priority
relative to other available jobs. When there are no other competing jobs, the
job will immediately run.

As such, the ``\emph{Fairshare}'' mechanism has nothing to do with the
allocation of computing time.

\includegraphics*[width=5.77in, height=2.37in, keepaspectratio=false]{img0900}

\section{Crediting system}

There is no crediting system in use yet at the \hpc.
