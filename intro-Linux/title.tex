\pagestyle{empty}

\begin{center}

\includegraphics*[width=8truecm]{logo_vsc}

\vspace*{1.5\baselineskip}

\Huge \strong{Linux Tutorial} \\
\LARGE Version 20180828.03

\ifwindows
\LARGE For Windows Users
\fi
\ifmac
\LARGE For macOS Users
\fi
\iflinux
\LARGE For Linux Users
\fi

\vspace*{.75\baselineskip}
\ifantwerpen
\includegraphics*[width=7truecm]{logo_ua}
\fi

\vspace*{0.75\baselineskip}


\normalsize\strong{Authors:}

Kenneth Hoste (UGent), Ewan Higgs (UGent)

\vspace*{.5\baselineskip}

\vspace*{.5\baselineskip}

Acknowledgement: VSCentrum.be

\vspace*{\baselineskip}

\begin{tabular}{ >{\centering\arraybackslash}m{0.25\textwidth}  >{\centering\arraybackslash}m{0.05\textwidth}  >{\centering\arraybackslash}m{0.20\textwidth}  >{\centering\arraybackslash}m{0.2\textwidth}} \\
\includegraphics*[width=0.2\textwidth, height=0.7in, keepaspectratio=true]{logo_auha} & \multicolumn{2}{ >{\centering\arraybackslash}m{.2\textwidth} }{\includegraphics*[width=0.2\textwidth, height=0.7in,, keepaspectratio=true]{logo_akuleuven}} & \includegraphics*[width=0.2\textwidth, height=0.7in,, keepaspectratio=true]{logo_auhl} \\
\multicolumn{2}{ >{\centering\arraybackslash}m{.32\textwidth} }{\includegraphics*[width=0.3\textwidth, height=0.7in, keepaspectratio=true]{logo_augent}} & \multicolumn{2}{ >{\centering\arraybackslash}m{.38\textwidth} }{\includegraphics*[width=0.3\textwidth, height=0.7in, keepaspectratio=false]{logo_uab}} \\
\end{tabular}
\end{center}


\strong{Audience:}
This document is a hands-on guide for using the Linux command line in the
context of the \strong{\university} \hpc infrastructure. The
command line (sometimes called 'shell') can seems daunting at first, but with a
little understanding can be very easy to use. Everything you do starts at the
prompt. Here you have the liberty to type in any commands you want. Soon, you
will be able to move past the limited point and click interface and express
interesting ideas to the computer using the shell.

Gaining an understanding of the fundamentals of Linux will help accelerate your
research using the HPC infrastructure.  You will learn about commands, managing
files, and some scripting basics.

\strong{\underbar{Notification:}}

In this tutorial specific commands are separated from the accompanying text:
\begin{prompt}
%\shellcmd{commands}%
\end{prompt}

These should be entered by the reader at a command line in a Terminal on the \hpcInfra. They appear in all exercises preceded by a \$ and printed in \textbf{bold}. You'll find those actions in a grey frame.

\keys{Button} are menus, buttons or drop down boxes to be pressed or selected.

``Directory'' is the notation for directories (called ``folders'' in
Windows terminology) or specific files. (e.g., ``\homedir'')

``Text'' Is the notation for text to be entered.

\begin{tip}
A ``Tip'' paragraph is used for remarks or tips.
\end{tip}

They can also be downloaded from the VSC website at
\url{https://www.vscentrum.be}.
Apart from this \hpc Tutorial, the documentation on the VSC website
will serve as a reference for all the
operations.


\begin{tip}
The users are advised to get self-organised. There are
only limited resources available at the \hpc, which are best effort based.
The \hpc cannot give support for code fixing, the user applications and own
developed software remain solely the responsibility of the end-user.
\end{tip}

More documentation can be found at:

\begin{enumerate}
  \item  VSC documentation: \url{https://www.vscentrum.be/en/user-portal}
  \ifantwerpen
    \item CalcUA Core Facility web pages: \url{https://www.uantwerpen.be/hpc}
  \fi
  \ifbrussel
    \item \hpcname documentation: \url{http://cc.ulb.ac.be/hpc}
  \fi
  \ifgent
    \item \hpcname documentation: \url{http://hpc.ugent.be/userwiki}
  \fi
  \item  External documentation (TORQUE, Moab): \url{http://docs.adaptivecomputing.com}
\end{enumerate}

This tutorial is intended for users working on \strong{\OS} who want to connect to the HPC of the \strong{\university}.

This tutorial is available in a Windows, Mac or Linux version.

This tutorial is available for UAntwerpen, UGent, KU~Leuven, UHasselt and VUB users.

Request your appropriate version at \hpcinfo.

\strong{\underbar{Contact Information:}}

We welcome your feedback, comments and suggestions for improving the Linux
Tutorial  (contact: \hpcinfo).

For all technical questions, please contact the \hpc staff:

\inputsite{contact-information}


