\begin{prompt}
%\shellcmd{module av}%
------------------- /apps/leuven/thinking/2014a/modules/all --------------------
Abaqus/6.12-2
Abaqus/6.13-3
Abaqus/6.14-1
accounting
allinea-ddt/4.2
annovar/2013Jun21
ANSYS/13.0
ANSYS/14.5
ANSYS/15.0
ant/1.9.3-Java-1.7.0_51
Armadillo/4.320.0-foss-2014a-Python-2.7.6
Armadillo/6.400.3-foss-2014a-Python-2.7.6
Autoconf/2.69-foss-2014a
Autoconf/2.69-GCC-4.7.2
Autoconf/2.69-GCC-4.8.2
Autoconf/2.69-iccifort-11.1.080
Autoconf/2.69-intel-2014a
Automake/1.14-GCC-4.8.2
Automake/1.14-iccifort-11.1.080
Automake/1.14-intel-2014a
Bash/4.3
Bash/4.3-GCC-4.8.2
beagle-lib/20140304-intel-2014a
BEAST/1.8.0
BEAST/1.8.2
BEAST/2.1.2
bedtools/2.19.1-intel-2014a
Biopython/1.66-foss-2014a-Python-2.7.6
Bison/2.5-foss-2014a
Bison/2.7.1-GCC-4.8.2
Boost/1.54.0-foss-2014a
Boost/1.55.0-foss-2014a
Boost/1.55.0-foss-2014a-Python-2.7.6
Boost/1.55.0-intel-2014a
Boost/1.55.0-intel-2014a-Python-2.7.6
Bowtie/1.0.1-intel-2014a
Bowtie2/2.1.0-intel-2014a
Bowtie2/2.2.1-intel-2014a
BWA/0.7.5a-foss-2014a
BWA/0.7.5a-intel-2014a
bzip2/1.0.6-foss-2014a
bzip2/1.0.6-intel-2014a
%\dots% 
\end{prompt}

``module av'' is an abbreviation for ``module available''.

When you want to switch to the more recent versions of the compilers you can check the software:

\begin{prompt}
%\shellcmd{source switch\_to\_2015a; module av 2>\&1 | more}%
------------------- /apps/leuven/thinking/2015a/modules/all --------------------
ABySS/1.3.7-intel-2015a-Python-2.7.9
ABySS/1.9.0-intel-2015a-Python-2.7.9
accounting
allinea-forge/5.0.1
allinea-forge/5.1
ANSYS/16.2
ant/1.9.4-Java-1.8.0_31
APBS/1.4.1-foss-2015a
APBS/1.4.1-intel-2015a
APBS/1.4-linux-static-x86_64
Armadillo/4.600.4-intel-2015a-Python-2.7.9
arpack-ng/3.1.5-intel-2015a
Autoconf/2.69-GCC-4.9.2
Autoconf/2.69-intel-2015a
Automake/1.15-GCC-4.9.2
Autotools/20150119-GCC-4.9.2
Bash/4.3-GCC-4.9.2
bazel/0.1.2
beagle-lib/20120124-intel-2015a
beagle-lib/2.1.2-intel-2015a
Beast/2.3.1
--More--
\end{prompt}

Additionally, different software is installed on the shared memory cluster, GPU cluster and Haswell nodes:

\begin{prompt}
%\shellcmd{module av}%
%\dots% 
-------------------------- /apps/leuven/etc/modules/ ---------------------------
cerebro/2014a   K20Xm/2014a     M2070/2014a     thinking/2014a
ictstest/2014a  K40c/2014a      phi/2014a       thinking2/2014a
%\end{prompt}

When you want to check whether some specific software, some compiler or some
application (e.g., worker) is installed on the \hpc.

\begin{prompt}
%module av 2>&1 | grep -i -e "worker"%
worker/1.4.2-foss-2014a
worker/1.5.0-intel-2014a
worker/1.5.1-intel-2014a
worker/1.5.2-intel-2014a
worker/1.5.3-intel-2014a
\end{prompt}

As you are not aware of the capitals letters in the module name, we looked for a case-insensitive name with the ``-i'' option.
