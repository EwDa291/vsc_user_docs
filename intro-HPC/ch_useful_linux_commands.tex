\chapter{Useful Linux Commands}
\label{ch:useful-linux-commands}

\section{Basic Linux Usage}

All the \hpc clusters run the ``\operatingsystembase'' operating system.  This
means that, when you connect to one of them, you get a command line interface,
which looks something like this:

\begin{prompt}
%\userid{}%@ln01[203] $
\end{prompt}

When you see this, we also say you are inside a ``shell''. The shell will accept
your commands, and execute them.

\begin{tabular}{|p{\dimexpr 0.15\textwidth-2\tabcolsep}|p{\dimexpr 0.85\textwidth-2\tabcolsep}|} \hline
ls   & Shows you a list of files in the current directory \\ \hline
cd   & Change current working directory \\ \hline
rm   & Remove file or directory \\ \hline
joe  & Text editor \\ \hline
echo & Prints its parameters to the screen \\ \hline
\end{tabular}

Most commands will accept or even need parameters, which are placed after the
command, separated by spaces. A simple example with the ``echo'' command:

\begin{prompt}
%\shellcmd{echo This is a test}%
This is a test
\end{prompt}

Important here is the ``\$'' sign in front of the first line. This should not be
typed, but is a convention meaning ``the rest of this line should be typed at
your shell prompt''. The lines not starting with the ``\$'' sign are usually the
feedback or output from the command.

More commands will be used in the rest of this text, and will be explained then
if necessary. If not, you can usually get more information about a command, say
the item or command ``ls'', by trying either of the following:

\begin{prompt}
%\shellcmd{ls --help}%
%\shellcmd{man ls}%
%\shellcmd{info ls}%
\end{prompt}

(You can exit the last two ``manuals'' by using the ``q'' key.)
For more exhaustive tutorials about Linux usage, please refer to the following sites:
\url{http://www.linux.org/lessons/}
\url{http://linux.about.com/od/nwb\_guide/a/gdenwb06.htm}

\section{How to get started with shell scripts}

In a shell script, you will put the commands you would normally type at your
shell prompt in the same order. This will enable you to execute all those
commands at any time by only issuing one command: starting the script.

Scripts are basically non-compiled pieces of code: they are just text files.
Since they don't contain machine code, they are executed by what is called a
``parser'' or an ``interpreter''. This is another program that understands the
command in the script, and converts them to machine code. There are many kinds
of scripting languages, including Perl and Python.

Another very common scripting language is shell scripting. In a shell script,
you will put the commands you would normally type at your shell prompt in the
same order. This will enable you to execute all those commands at any time by
only issuing one command: starting the script.

Typically in the following examples they'll have on each line the next command
to be executed although it is possible to put multiple commands on one line. A
very simple example of a script may be:

\begin{code}{bash}
echo "Hello! This is my hostname:"
hostname
\end{code}

You can type both lines at your shell prompt, and the result will be the following:
\begin{prompt}
%\shellcmd{echo "Hello! This is my hostname:"}%
Hello! This is my hostname:
%\shellcmd{hostname}%
%\loginhost%
\end{prompt}

Suppose we want to call this script ``foo''. You open a new file for editing, and
name it ``foo'', and edit it with your favourite editor

\begin{prompt}
%\shellcmd{vi foo}%
\end{prompt}

or use the following commands:
\begin{prompt}
%\shellcmd{echo "echo Hello! This is my hostname:" $>$ foo}%
%\shellcmd{echo hostname $>$$>$ foo}%
\end{prompt}

The easiest ways to run a script is by starting the interpreter and pass the
script as parameter. In case of our script, the interpreter may either be ``sh''
or ``bash'' (which are the same on the cluster). So start the script:

\begin{prompt}
%\shellcmd{bash foo}%
Hello! This is my hostname:
%\loginhost%
\end{prompt}

Congratulations, you just created and started your first shell script!

A more advanced way of executing your shell scripts is by making them
executable by their own, so without invoking the interpreter manually. The
system can not automatically detect which interpreter you want, so you need to
tell this in some way. The easiest way is by using the so called
``shebang''-notation, explicitly created for this function: you put the
following line on top of your shell script ``\#!/path/to/your/interpreter''.

You can find this path with the ``which'' command. In our case, since we use bash
as an interpreter, we get the following path:

\begin{prompt}
%\shellcmd{which bash}%
/bin/bash
\end{prompt}

We edit our script and change it with this information:

\begin{code}{bash}
#!/bin/bash
echo "Hello! This is my hostname:"
hostname
\end{code}

Note that the ``shebang'' must be the first line of your script! Now the
operating system knows which program should be started to run the script.

Finally, we tell the operating system that this script is now executable. For
this we change its file attributes:

\begin{prompt}
%\shellcmd{chmod +x foo}%
\end{prompt}

Now you can start your script by simply executing it:

\begin{prompt}
%\shellcmd{./foo}%
Hello! This is my hostname:
%\loginhost%
\end{prompt}

The same technique can be used for all other scripting languages, like Perl and Python.

Most scripting languages understand that lines beginning with ``\#'' are
comments, and should be ignored. If the language you want to use does not
ignore these lines, you may get strange results \ldots

\section{Linux Quick reference Guide}

\subsection{Archive Commands}

\begin{tabular}{|p{\dimexpr 0.25\textwidth-2\tabcolsep}|p{\dimexpr 0.75\textwidth-2\tabcolsep}|} \hline
tar                   & An archiving program designed to store and extract files from an archive known as a tar file.  \\ \hline
tar -cvf foo.tar foo/ & compress the contents of foo folder to foo.tar \\ \hline
tar -xvf foo.tar      & extract foo.tar \\ \hline
tar -xvzf foo.tar.gz  & extract gzipped foo.tar.gz \\ \hline
\end{tabular}


\subsection{Basic Commands}

\begin{tabular}{|p{\dimexpr 0.15\textwidth-2\tabcolsep}|p{\dimexpr 0.85\textwidth-2\tabcolsep}|} \hline
ls     & Shows you a list of files in the current directory \\ \hline
cd     & Change the current directory \\ \hline
rm     & Remove file or directory \\ \hline
mv     & Move file or directory \\ \hline
echo   & Display a line or text \\ \hline
pwd    & Print working directory \\ \hline
mkdir  & Create directories \\ \hline
rmdir  & Remove directories \\ \hline
\end{tabular}


\subsection{Editor}

\begin{tabular}{|p{\dimexpr 0.15\textwidth-2\tabcolsep}|p{\dimexpr 0.85\textwidth-2\tabcolsep}|} \hline
emacs  & \\ \hline
nano   & Nano's ANOther editor, an enhanced free Pico clone \\ \hline
vi     & A programmers text editor \\ \hline
\end{tabular}

\subsection{File Commands}

\begin{tabular}{|p{\dimexpr 0.15\textwidth-2\tabcolsep}|p{\dimexpr 0.85\textwidth-2\tabcolsep}|} \hline
cat   & Read one or more files and print them to standard output \\ \hline
cmp   & Compare two files byte by byte \\ \hline
cp    & Copy files from a source to the same or different target(s) \\ \hline
du    & Estimate disk usage of each file and recursively for directories \\ \hline
find  & Search for files in directory hierarchy \\ \hline
grep  & Print lines matching a pattern \\ \hline
ls    & List directory contents \\ \hline
mv    & Move file to different targets \\ \hline
rm    & Remove files \\ \hline
sort  & Sort lines of text files \\ \hline
wc    & Print the number of new lines, words, and bytes in files \\ \hline
\end{tabular}


\subsection{Help Commands}

\begin{tabular}{|p{\dimexpr 0.15\textwidth-2\tabcolsep}|p{\dimexpr 0.85\textwidth-2\tabcolsep}|} \hline
man   & Displays the manual page of a command with its name, synopsis, description, author, copyright etc. \\ \hline
\end{tabular}

\subsection{Network Commands}

\begin{tabular}{|p{\dimexpr 0.15\textwidth-2\tabcolsep}|p{\dimexpr 0.85\textwidth-2\tabcolsep}|} \hline
hostname  & show or set the system's host name \\ \hline
ifconfig  & Display the current configuration of the network interface. It is also useful to get the information about IP address, subnet mask, set remote IP address, netmask etc. \\ \hline
ping      & send ICMP ECHO\_REQUEST to network hosts, you will get back ICMP packet if the host responds.  This command is useful when you are in a doubt whether your computer is connected or not. \\ \hline
\end{tabular}


\subsection{Other Commands}
\begin{tabular}{|p{\dimexpr 0.15\textwidth-2\tabcolsep}|p{\dimexpr 0.85\textwidth-2\tabcolsep}|} \hline
logname & Print user's login name \\ \hline
quota   & Display disk usage and limits \\ \hline
su      & Switch to super user or change user ID \\ \hline
which   & Returns the pathnames of the files that would be executed in the current environment \\ \hline
whoami  & Displays the login name of the current effective user \\ \hline
\end{tabular}

\subsection{Process Commands}

\begin{tabular}{|p{\dimexpr 0.15\textwidth-2\tabcolsep}|p{\dimexpr 0.85\textwidth-2\tabcolsep}|} \hline
\&      & In order to execute a command in the background, place an ampersand (\&) on the command line at the end of the command. A user job number (placed in brackets) and a system process number are displayed. A system process number is the number by which the system identifies the job whereas a user job number is the number by which the user identifies the job \\ \hline
at      & executes commands at a specified time \\ \hline
bg      & Places a suspended job in the background \\ \hline
crontab & crontab is a file which contains the schedule of entries to run at specified times \\ \hline
fg      & A process running in the background will be processed in the foreground \\ \hline
jobs    & Lists the jobs being run in the background \\ \hline
kill    & Cancels a job running in the background, it takes argument either the user job number or the system process number \\ \hline
ps      & Reports a snapshot of the current processes \\ \hline
sudo    & Execute a command as a superuser \\ \hline
top     & Display Linux tasks \\ \hline
\end{tabular}


\subsection{User Account Commands}

\begin{tabular}{|p{\dimexpr 0.15\textwidth-2\tabcolsep}|p{\dimexpr 0.85\textwidth-2\tabcolsep}|} \hline
chmod    & Modify properties for users \\ \hline
chown    & Change file owner and group \\ \hline
\end{tabular}

