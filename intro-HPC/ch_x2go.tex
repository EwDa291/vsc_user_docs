\chapter{Graphical applications with X2Go}
\label{ch:x2go}

X2Go is a graphical desktop software for Linux similar to VNC but with extra advantages.
It does not require to execute a server in the login node and it is possible to setup a SSH proxy to
connect to an specific login node.
It can also be used to access Windows, Linux and macOS desktops.
X2Go provides several advantages such:
\begin{enumerate}
 \item A graphical remote desktop that works well over low bandwidth connections.
 \item Copy/paste support from client to server and vice-versa.
 \item File sharing from client to server.
 \item Support for sound.
 \item Printer sharing from client to server.
 \item The ability to access single applications by specifying the name of the desired executable like a terminal
 or an internet browser.
\end{enumerate}



\section{Install X2Go client}
\label{sec:x2go-client}
X2Go is available for several operating systems. You can download the latest client from 
\url{https://wiki.x2go.org/doku.php/doc:installation:x2goclient}.

X2Go requires a valid private SSH key to connect to the login node,
this is described in \autoref{subsec:how-do-ssh-keys-work}.
This section also describes how to use X2Go client with a SSH agent. The SSH agent setup is optional but it 
is the easiest way to connect to the login nodes using several SSH keys and applications.
Please see \autoref{sec:using-ssh-agent-with-openssh} if you want to know how to setup an SSH agent in your system.


\section{Create a new X2Go session}
\label{sec:sessions-x2go}

After the X2Go client installation just start the client.
When you launch the client for the first time, it will start the new session dialogue automatically.

There are two ways to connect to the login node:
\begin{itemize}
\item \emph{Option A}: A direct connection to ``\strong{\emph{\loginnode}}''. This is the simpler option, the system will decide which
login node to use based on a load-balancing algorithm.
\item \emph{Option B}: You can use the node ``\strong{\emph{\loginnode}}'' as SSH proxy to connect to a specific login node. Use this option if you want to resume an old X2Go session.
\end{itemize}

\subsection{Option A: direct connection}
\label{subsec:sessions-x2go-optiona}
This is the easier way to setup X2Go, a direct connection to the login node.

\begin{center}
\includegraphics*[width=3.30in, height=3.17in, keepaspectratio=false]{ch19-x2go-configuration-gent}
\end{center}

\begin{enumerate}
    \item  Include a session name. This will help you to identify 
    the session if you have more than one, you can choose any name (in our example ``HPC login node``).
    \item  Set the login hostname (In our case: ``\strong{\emph{\loginnode}}'')
    \item  Set the Login name. In the example is ``\strong{\userid}'' but you must change it by your
    current VSC account.
    \item  Set the SSH port (22 by default).
    \item  Skip this step if you are using an SSH agent (see \autoref{sec:x2go-client}).
    If not add your SSH private key into ``Use RSA/DSA key..'' field. In this case:
    \begin{enumerate}
     \item Click on the ``Use RSA/DSA..'' folder icon. This will open a file browser. 
     \begin{center}
       \includegraphics*[width=3.2in, keepaspectratio=true]{ch19-x2go-ssh-key}
     \end{center}
     \item
     \ifmacORlinux
     You should look for your \strong{private} SSH key generated in \autoref{sec:generate-key-pair-with-openssh}.
     This file has been stored in the directory ``\emph{\tilde/.ssh/}'' (by default ``\strong{id\_rsa}''.
     `.ssh'' is an invisible directory, the Finder will
      not show it by default. The easiest way to access the folder, is by pressing
      \keys{cmd} + \keys{shift} + \keys{g}, which will allow you to enter the name of a directory, which
     you would like to open in Finder. Here, type ``\tilde/.ssh'' and press enter.
     Choose that file and click on \keys{open}.
     \fi
     \ifwindows
     You should look for your \strong{private} SSH key generated by \keys{puttygen} in \autoref{sec:generate-key-pair}
     (by default ``\strong{id\_rsa.ppk}'').
     Choose that file and click on \keys{open}.
     \fi
    \end{enumerate}
    \item  Check ``Try autologin'' option.
    \item  Choose Session type to \keys{XFCE}. Only the \keys{XFCE} desktop is available for the moment.
    It is also possible to choose single applications instead of a full desktop, like the \keys{Terminal}
    or \keys{Internet browser} (you can change this option later directly from the X2Go session tab if you want).
    \begin{enumerate}
      \item \strong{[optional]:} Set a single application like \keys{Terminal} instead of \keys{XFCE} desktop.
      \ifwindows This option is much better than PuTTY because the X2Go client includes copy-pasting support.\fi
      \begin{center}
        \includegraphics*[width=3.2in, keepaspectratio=true]{ch19-x2go-configuration-xterm}
      \end{center}
    \end{enumerate}
    \item  \strong{[optional]:} Change the session icon.
    \item  Click the \keys{OK} button after these changes.
\end{enumerate}

\subsection{Option B: use the login node as SSH proxy}
\label{subsec:sessions-x2go-optionb}
This option is useful if you want to resume a previous session or if you want to set explicitly the login node to use.
In this case you should include a few more options. Use the same \strong{Option A} setup but with these changes:

\begin{center}
\includegraphics*[width=3.30in, height=3.17in, keepaspectratio=false]{ch19-x2go-configuration-gent-proxy}
\end{center}

\begin{enumerate}
    \item  Include a session name. This will help you to identify 
    the session if you have more than one (in our example ``HPC UGent proxy login``).
    \item  Set the login hostname. This is the login node that you want to use at the end
    (In our case: ``\strong{\emph{\loginhost{}}}'')
    \item  Set ``Use Proxy server..'' to enable the proxy.
    Within ``Proxy section'' set also these options:
    \begin{enumerate}
      \item  Set Type ``SSH'', ``Same login'', ``Same Password'' and ``SSH agent'' options.
      \item  Set Host to ``\strong{\emph{\loginnode}}'' within ``Proxy Server'' section as well.
      \item  Skip this step if you are using an SSH agent (see \autoref{sec:x2go-client}).
      Add your \strong{private} SSH key within ``RSA/DSA key'' field within ``Proxy Server'' as you
      did for the server configuration (The ``RSA/DSA key'' field must be set in both sections)
      \begin{center}
       \includegraphics*[width=3.2in, keepaspectratio=true]{ch19-x2go-proxy-key}
      \end{center}
      \item  Click the \keys{OK} button after these changes.
    \end{enumerate}
\end{enumerate}

\section{Connect to your X2Go session}
\label{sec:connect-x2go}
Just click on any session that you already have to start/resume any session. It will take a few seconds to open the session the first time.
It is possible to terminate a session if you logout from the current open session or if you click on the ``shutdown'' button from X2Go.
If you want to suspend your session to continue working with it later just click on the ``pause'' icon.

\begin{center}
  \includegraphics*[width=3.2in, keepaspectratio=true]{ch19-x2go-pause}
\end{center}

X2Go will keep the session open for you (but only if the login node is not rebooted).

\section{Resume a previous session}
\label{sec:re-connect-x2go}

If you want to re-connect to the same login node, or resume a previous session, you should know which login node were used at first place.
You can get this information before logging out from your X2Go session. Just open a terminal and execute: 

\begin{prompt}
%\shellcmd{hostname}%
\end{prompt}

\includegraphics*[width=3.30in, keepaspectratio=true]{ch19-x2go-xterm}

This will give you the full login name (like ``\strong{\emph{\loginhost{}}}'' but the hostname in your situation may be slightly different).
You should set the same name to resume the  session the next time.
Just add this full hostname into ``login hostname'' section in your X2Go session (see \autoref{subsec:sessions-x2go-optionb}).
