\chapter{Preparing the Environment}
\label{ch:setting-up-the-environment}

Before you can really start using the \hpc clusters, there are several things you need to do or know:

\begin{enumerate}
\item  You need to \textbf{log on to the cluster} using an ssh-client to one of the login nodes. This will give you command-line access. The software you'll need to use on your client system depends on its operating system.
\item  Before you can do some work, you'll have to \textbf{transfer the files} that you need from your desktop computer to the cluster. At the end of a job, you might want to transfer some files back.
\item  Optionally, if you wish to use programs with a \textbf{graphical user interface}, you will need an X-server on your client system.
\item  Often several versions of \textbf{software packages and libraries} are installed, so you need to select the ones you need. To manage different versions efficiently, the VSC clusters use so-called \textbf{modules}, so you will need to select and load the modules that you need.
\end{enumerate}

\section{Connecting to the \hpc}

To make a connection to the \hpc clusters, the ssh command is used:

\begin{prompt}
%\shellcmd{ssh $<$vsc-account$>$@\loginnode}%
\end{prompt}

Here,

\begin{enumerate}
\item  \textit{$<$vsc-account$>$} is your VSC username that you have received by mail after your request was approved,
\item  \textit{``\loginnode''} is the name of the login node of the \hpc cluster you want to connect to.
\end{enumerate}

Upon connection, you will get a welcome message and your current disk and file quotas are shown.

\begin{prompt}
%\dots%
Your quota is:
Block Limits
   Filesystem         KB      quota      limit    grace
   data            82944   26214400   28835840     none
   home            35328    3145728    3461120     none
   scratch         47808   26214400   28835840     none

File Limits
   Filesystem      files      quota      limit    grace
   data               27     100000     150000     none
   home              125      20000      25000     none
   scratch            11     100000     150000     none
----------------------------------------------------------
\end{prompt}

To check on which login node you are connected:

\begin{prompt}
%\shellcmd{hostname -f}%
%\loginnode%
\end{prompt}

You can exit the connection at anytime by entering:

\begin{prompt}
%\shellcmd{exit}%
logout
Connection to %\loginnode% closed.
\end{prompt}

\strong{Tip: Setting your Language right}

You may encounter the following warning message during connecting:
\begin{prompt}
perl: warning: Setting locale failed.
perl: warning: Please check that your locale settings:
LANGUAGE = (unset),
LC_ALL = (unset),
LC_CTYPE = "UTF-8",
LANG = (unset)
    are supported and installed on your system.
perl: warning: Falling back to the standard locale ("C").
\end{prompt}

This means that the correct `locale' has not yet been properly specified on your local machine. Try:

\begin{prompt}
%\shellcmd{locale}%
LANG=
LC_COLLATE="C"
LC_CTYPE="UTF-8"
LC_MESSAGES="C"
LC_MONETARY="C"
LC_NUMERIC="C"
LC_TIME="C"
LC_ALL=
\end{prompt}

A \strong{locale} is a set of parameters that defines the user's language,
country and any special variant preferences that the user wants to see in their
user interface. Usually a locale identifier consists of at least a language
identifier and a region identifier.

Open the .bash\_profile or the .profile on your local machine with your
favorite editor and add the following lines:

\begin{prompt}
%\shellcmd{vi ~/.bash\_profile}%
%\dots%
export LANGUAGE="en_US.UTF-8"
export LC\_ALL="en_US.UTF-8"
export LC\_CTYPE="en_US.UTF-8"
export LANG="en_US.UTF-8"
%\dots%
\end{prompt}

or alternatively (if you are not comfortable with the Linux editor's):

\begin{prompt}
%\shellcmd{echo "export LANGUAGE=\textbackslash "en\_US.UTF-8\textbackslash " $>$$>$ ~/.profile"}%
%\shellcmd{echo ``export LC\_ALL=\textbackslash "en\_US.UTF-8\textbackslash " $>$$>$ ~/.profile''}%
%\shellcmd{echo ``export LC\_CTYPE=\textbackslash "en\_US.UTF-8\textbackslash " $>$$>$ ~/.profile''}%
%\shellcmd{echo ``export LANG="en\_US.UTF-8\textbackslash " $>$$>$ ~/.profile''}%
\end{prompt}

You can now log-out and re-connect to the \hpc, and you should not get these warnings anymore.

\section{Transfer Files to/from the \hpc}

Before you can do some work, you'll have to \strong{transfer the files} that
you need from your desktop or department to the cluster. At the end of a job,
you might want to transfer some files back.

The preferred way to transfer files is by using an scp or sftp via the secure
OpenSSH protocol.  OS X comes with its own implementation of OpenSSH, so you
don't need to install any third-party software to use it. Just open a Terminal
window and jump in!

\ifwindows

\subsection{WinSCP}

To transfer files to and from the cluster, we recommend the use of WinSCP,
which is a graphical ftp-style program (but than one that uses the ssh way of
communicating with the cluster rather then the less secure ftp) that is also
freely available. WinSCP can be downloaded both as an installation package and
as a standalone portable executable.

Google ``WinSCP Download'' and download it from \url{http://www.winscp.net}

To transfer your files using WinSCP,

\begin{enumerate}
\item  Open the program
\item  Fill in the necessary fields under $<$\emph{Session}$>$
\begin{enumerate}
\item  Press $<$\emph{New}$>$
\item  Enter ``\emph{\loginnode}'' in the $<$Host name$>$ field
\item  Put your ``\emph{vsc-account}'' in $<$\emph{User name}$>$ field
\item  Select your private key in the field $<$\emph{Private key file}$>$.
\item  Select $<$\emph{SFTP}$>$ as the $<$\emph{file}$>$ protocol.
\item  Note that the password field remains empty.
\end{enumerate}
\end{enumerate}

\includegraphics*[width=3.42in, height=3.04in, keepaspectratio=false]{img0214}

\begin{enumerate}
\item  By pressing on the $<$\emph{Save}$>$ button, you can save the session under $<$\emph{Stored sessions}$>$ for future access.
\item  Finally, when clicking on 'Login', you will be asked for your key passphrase.
\end{enumerate}

\includegraphics*[width=3.14in, height=2.47in, keepaspectratio=false]{img0215}

The first time you make the connection, you will be asked to 'Continue
connecting and add host key to the cache'; select 'Yes'.

\includegraphics*[width=5.74in, height=2.81in, keepaspectratio=false]{img0216}

Now, try out whether you can transfer an arbitrary file from your local machine to the HPC and back.

\fi
\ifmac

\subsection{Using scp}

\strong{Secure copy} or \strong{SCP} is a tool (commando) for securely
transferring files between a local host (= your computer) and a remote host
(the UA-HPC). It is based on the Secure Shell (SSH) protocol.  The \strong{scp}
command works equivalent `as the \strong{cp}  (i.e. \strong{c}o\strong{p}y)
command, but can copy files to or from remote machines.

Open a additional Terminal window in OS X, open the Finder and choose

\strong{\emph{$>$$>$ Applications $>$ Utilities $>$ Terminal}}

Check that you're working on your local machine.

\begin{prompt}
%\shellcmd{hostname}%
<local-machine-name>
\end{prompt}

If you're still using the terminal that is connected to the \hpc, close the
connection by typing "exit" in the terminal window. Alternatively, open a new
terminal window using the shortcut $<$command$>$$<$N$>$.

For example, we will copy the (local) file ``\emph{localfile.txt}'' to your
home directory on the \hpc cluster. We first generate a small dummy
``\emph{localfile.txt}'', which contains the word ``Hello''.  Use your own
$<$vsc-account$>$, which is something like ``\emph{\userid}''.

\begin{prompt}
%\shellcmd{echo "Hello" $>$ localfile.txt}%
%\shellcmd{ls -l}%
%\dots%
-rw-r--r-- 1 gborstlap  staff   6 Sep 18 09:37 localfile.txt
%\shellcmd{scp localfile.txt $<$vsc-account$>$@\loginnode:}%
localfile.txt    100%\%%   6     0.0KB/s   00:00
\end{prompt}

Connect to the \hpc via another terminal, print the working directory (to make
sure you're in the home-directory) and check whether the file has arrived:

\begin{prompt}
%\shellcmd{pwd}%
%\homedir%
%\shellcmd{ls -l}%
total 1536
drwxrwxr-x  2 %\userid% 131072 Sep 11 16:24 bin/
drwxrwxr-x  2 %\userid% 131072 Sep 17 11:47 docs/
drwxrwxr-x 10 %\userid% 131072 Sep 17 11:48 examples/
-rw-r--r--  1 %\userid%      6 Sep 18 09:44 localfile.txt
%\shellcmd{cat localfile.txt}%
Hello
\end{prompt}

% TODO: rename files to intro-HPC
Likewise, to copy the remote file ``intro-HPC.pdf'' from your ``docs''
sub-directory on the cluster to your local computer, try:

On the Terminal on the \hpc, enter:

\begin{prompt}
%\shellcmd{cd ~/docs}%
%\homedir%/docs
%\shellcmd{ls -l}%
total 1536
-rw-r--r-- 1 %\userid% Sep 11 09:53 intro-HPC.pdf
\end{prompt}

On the Terminal on your own local computer, enter:

\begin{prompt}
%\shellcmd{scp \userid\@\loginnode:./docs/intro-HPC.pdf .}%
intro-HPC.pdf 100%\%%  725KB 724.6KB/s   00:01
%\shellcmd{ls -l}%
total  899
-rw-r--r--   1 gborstlap  staff     413 Sep 10 10:29 id_rsa.pub
-rw-r--r--   1 gborstlap  staff  741995 Sep 18 09:53 intro-HPC.pdf
-rw-r--r--   1 gborstlap  staff       6 Sep 18 09:37 localfile.txt
\end{prompt}

\subsection{Using sftp}

The \strong{SSH File Transfer Protocol} (also \strong{Secure File Transfer
Protocol}, or \strong{SFTP}) is a network protocol that provides file access,
file transfer and file management functionalities over any reliable data
stream. It was designed as an extension of the Secure Shell protocol (SSH)
version 2.0. This protocol assumes that it is run over a secure channel, such
as SSH, that the server has already authenticated the client, and that the
identity of the client user is available to the protocol.

The sftp is an equivalent of the ftp command, with the difference that it uses
the secure ssh protocol to connect to the clusters.

One easy way of starting a sftp session is
\begin{prompt}
%\shellcmd{sftp $<$vsc-account$>$@$<$vsc-login node$>$}%
\end{prompt}

Typical and popular commando's inside an sftp session are:

\begin{tabular}{|p{0.3\textwidth}|p{0.6\textwidth}|} \hline
\strong{cd \tilde/examples/fibo} & Move to the examples/fibo subdirectory on the \hpc (i.e. the remote machine)\\  \hline
\strong{ls}                      & Get a list of the files in the current directory on the \hpc. \\ \hline
\strong{get fibo.py}             & Copy the file `fibo.py' from the \hpc \\ \hline
\strong{get tutorial/HPC.pdf}    & Copy the file `HPC.pdf' from the \hpc, which is in the `tutorial' subdirectory. \\ \hline
\strong{lcd test}                & Move to the `test' subdirectory on your local machine. \\ \hline
\strong{lcd ..}                  & Move up one level in the local directory. \\ \hline
\strong{lls}                     & Get local directory listing \\ \hline
\strong{put test.py}             & Copy the local file test.py to the \hpc. \\ \hline
\strong{put test1.py test2.py }  & Copy the local file test1.py to the \hpc and rename it to test2.py. \\ \hline
\strong{bye}                     & Quit the sftp session \\ \hline
\strong{mget *.cc}               & Copy all the remote files with extension ``.cc'' to the local directory.  \\ \hline
\strong{mput *.h}                & Copy all he local files with extension ``.h'' to the \hpc. \\ \hline
\end{tabular}

\subsection{Using FileZilla}

If you orefer a GUI to transfer files back and forth to the \hpc, we can
suggest to use FileZilla. FileZilla is a fast and free FTP and SFTP client,
which is widely used.

Download and install FileZilla client at:

\url{https://filezilla-project.org/download.php}

Tip: Some users might get the notification that Filezilla.app cannot be opened
because it is from an unidentified developer. Check out the Mac Gatekeeper at
\url{http://support.apple.com/kb/HT5290}

Start FileZilla and login to the \hpc with the following details:

\begin{enumerate}
\item  Host: sftp://\loginnode
\item  Username:  $<$your vsc-username$>$
\end{enumerate}

\includegraphics*[width=4.75in, height=4.19in, keepaspectratio=false]{img0300}
\fi

\section{Modules}

Software installation and maintenance on a \hpc cluster such as the VSC
clusters poses a number of challenges not encountered on a workstation or a
departmental cluster. We therefore need a system on the \hpc, which is able
to easily activate or de-activate the software packages that you require for
your program execution.

\subsection{Environment Variables}

The program environment on the \hpc is controlled by pre-defined settings,
which are stored in environment (or shell) variables.

You can use shell variables to store data, set configuration options and
customize the environment on the \hpc. The default shell under Scientific
Linux on the \hpc is Bash (Bourne Again Shell) and can be used for the
following purposes:

\begin{enumerate}
\item  Configure look and feel of shell.
\item  Setup terminal settings depending on which terminal you're using.
\item  Set the search path for running executable's.
\item  Set environment variables as needed by programs.
\item  Set convenient abbreviations for heavily used values
\item  Run commands that you want to run whenever you log in or log out.
\item  Setup aliases and/or shell function to automate tasks to save typing and time.
\item  Changing bash prompt.
\item  Setting shell options.
\end{enumerate}

The environment variables are typically set at login by a script, whenever you
connect to the \hpc. These pre-defined variables usually impact the run time
behavior of the programs that we want to run.

All the software packages that are installed on the \hpc cluster require
different settings. These packages include compilers, interpreters,
mathematical software such as MATLAB and SAS, as well as other applications and
libraries.

In order to administer the active software and their environment variables, a
\strong{\emph{``module''}} package has been developed, which:

\begin{enumerate}
\item  Activates or de-activates \underbar{software packages} and their dependencies.
\item  Allows setting and unsetting of \underbar{environment variables}, including adding and deleting entries from database-type environment variables.
\item  Does this in a \underbar{shell-independent} fashion (necessary information is stored in the accompanying module configuration file).
\item  Takes care of\underbar{versioning aspects:} For many libraries, multiple versions are installed and maintained. The module system also takes care of the versioning of software packages in case multiple versions are installed. For instance, it?does not allow multiple versions to be loaded at same time.
\item  Takes care of\underbar{dependencies:} Another issue arises when one considers library versions and the dependencies they create. Some software requires an older version of a particular library to run correctly (or at all). Hence a variety of version numbers is available for important libraries.
\end{enumerate}

This is all managed with the ``\emph{module}'' command, which is explained in
the next sections.

\subsection{Available modules}

A large number of software packages are installed on the \hpc clusters. A
list of all currently available software can be obtained by typing:

\begin{prompt}
%\shellcmd{module av}%
\end{prompt}
or
\begin{prompt}
%\shellcmd{module available}%
\end{prompt}

This will give some output such as:

\inputsite{available-modules}

This gives a full list of software packages that can be loaded. Note that
modules starting with a capital letter are listed first.

\subsection{Activating and de-activating modules}

A module is loaded using the following command:

\begin{prompt}
%\shellcmd{module load MATLAB}%
\end{prompt}

This will load the most recent version of MATLAB.

For some packages, e.g., OpenMPI, multiple versions are installed; the load
command will automatically choose the most recent version (i.e., the
lexicographically last after the "/") or the default version (as set by the
system administrators). However, the user can (and probably should, to avoid
surprises when never versions are installed) specify a particular version,
e.g.:

\begin{prompt}
%\shellcmd{module load Python/2.7.3-ictce-4.0.1}%
\end{prompt}

Obviously, the user needs to keep track of the modules that are currently
loaded. If he executed the two load commands stated above, he will get the
following:

\begin{prompt}
%\shellcmd{module list}%
Currently Loaded Modulefiles:
  1) MATLAB/R2013a        5) imkl/10.3.1.107
  2) icc/2011.1.107       6) ictce/4.0.1
  3) ifort/2011.1.107     7) Python/2.7.3-ictce-4.0.1
  4) impi/4.0.1.007
\end{prompt}

It is important to note at this point that other modules (e.g., ictce/4.0.1)
are also listed, although the user did not explicitly load them. This is
because ``Python/2.7.3-ictce-4.0.1'' depends on it (as indicated in it's name),
and the system administrator specified that the ``ictce/4.0.1'' module should
be loaded whenever the Python module is loaded. There are advantages and
disadvantages to this, so be aware of automatically loaded modules whenever
things go wrong: they may have something to do with it!

To unload a module, one can use the ``module unload'' command. It works
consistently with the load command, and reverses the latter's effect. However,
the dependencies of the package are NOT automatically unloaded; the user shall
unload the packages one by one. One can however unload automatically loaded
modules manually, to debug some problem. When the 'Python' module is unloaded,
only the following module remains:

\begin{prompt}
%\shellcmd{module unload Python}%
%\shellcmd{module list}%
Currently Loaded Modulefiles:
  1) MATLAB/R2013a        4) impi/4.0.1.007
  2) icc/2011.1.107       5) imkl/10.3.1.107
  3) ifort/2011.1.107     6) ictce/4.0.1
\end{prompt}

Notice that the version was not specified: the module system is sufficiently
clever to figure out what the user intends. However, checking the list of
currently loaded modules is always a good idea, just to make sure\ldots

In order to unload all modules at once, and hence be sure to start in a clean
state, you can use:

\begin{prompt}
%\shellcmd{module purge}%
\end{prompt}

It is a good habit to use this command in job-scripts, prior to loading the
modules specifically needed by applications in that job description. This
ensures that no version conflicts occur if the user loads module using his
'.bashrc' file.

Finally, modules need not be loaded one by one; the two 'load' commands can be
combined as follows:

\begin{prompt}
%\shellcmd{module load MATLAB Python/2.7.3-ictce-4.0.1}%
\end{prompt}

This will load the two modules as well as their dependencies.

\subsection{Explicit version numbers}

As a rule, once a module has been installed on the cluster, the executables or
libraries it comprises are never modified. This policy ensures that the user's
programs will run consistently, at least if the user specifies a specific
version. Failing to specify a version may result in unexpected behavior.

Consider the following example: the user decides to use the GSL library for
numerical computations, and at that point in time, just a single version 1.12,
compiled with Intel is installed on the cluster. The user loads the library
using:

\begin{prompt}
%\shellcmd{module load GSL}%
\end{prompt}
rather than
\begin{prompt}
%\shellcmd{module load GSL/1.12}%
\end{prompt}

Everything works fine, up to the point where a new version of GSL is installed,
1.13 compiled for gcc. From then on, the user's load command will load the
latter version, rather than the one he intended, which may lead to unexpected
problems.

Lets now generate a version conflict with the ``scripts'' module, and see what
is happening.

\begin{prompt}
%\shellcmd{module av 2$>$\&1 \textbar grep scripts}%
scripts/1.1.0
scripts/1.2.0
scripts/2.2.2
scripts/2.3.3
scripts/2.3.6
scripts/2.5.1
scripts/2.6.0
%\shellcmd{module load scripts/1.2.0}%
%\shellcmd{module load scripts /2.6.0 }%
scripts/2.6.0(13):ERROR:150: Module 'scripts/2.6.0' conflicts with the currently loaded module(s) 'scripts/1.2.0'
scripts/2.6.0(13):ERROR:102: Tcl command execution failed: conflict scripts
%\shellcmd{module switch scripts/2.6.0}%
\end{prompt}

Note: A ``module switch'' command combines the appropriate ``module unload''
and ``module load'' commands.

\subsection{Get detailed info}

In order to know more about a certain package, and to know what environment
variables will be changed by a certain module, try:

% TODO: this probably needs to be site-specific, or not shown as it seems a bit advanced and out of place
\begin{prompt}
%\shellcmd{module show Molpro}%
-----------------------
/apps/local/turing/harpertown/modules/Molpro/2010.1-ga-4.3.2-TCGMSG-MPI-ictce-3.2.2.u2:
module-whatis Molpro is a complete system for molecular electronic structure calculations - Homepage: http://www.molpro.net/
conflict  Molpro
module  load ictce/3.2.2.u2
prepend-path  PATH  /apps/local/turing/harpertown/software/Molpro/2010.1-ga-4.3.2-TCGMSG-MPI-ictce-3.2.2.u2/bin
setenv     SOFTROOTMOLPRO /apps/local/turing/harpertown/
software/Molpro/2010.1-ga-4.3.2-TCGMSG-MPI-ictce-3.2.2.u2
setenv   SOFTVERSIONMOLPRO 2010.1
-----------------------
\end{prompt}

To get a list of all possible commands, type:

\begin{prompt}
%\shellcmd{module help}%
\end{prompt}
Or to get more information about one specific module package:

\begin{prompt}
%\shellcmd{module help zlib}%
- Module Specific Help for 'zlib/1.2.8-ictce-5.5.0' -
zlib is designed to be a free, general-purpose, legally unencumbered -- that is, not covered by any patents -- lossless data-compression library for use on virtually any computer hardware and operating system. -
Homepage: http://www.zlib.net/
\end{prompt}
