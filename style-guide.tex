\documentclass[11pt,a4paper]{article}
\usepackage[english]{babel}
\usepackage{amssymb}
\usepackage{amsmath}
\usepackage{hyperref}
% allow conditional compilation
\usepackage{etoolbox}
\usepackage[toc,nonumberlist]{glossaries}

\usepackage{color}
\usepackage{listings}
\usepackage{tikz}
\usepackage[T1]{fontenc}
\usepackage{graphics}
\graphicspath{{img//}}

\lstset{basicstyle=\ttfamily,breaklines=true}
\lstdefinestyle{prompt}{language=bash,frame=tb,columns=fullflexible,
  escapechar=\%}
\lstdefinestyle{code}{numbers=left}

\lstnewenvironment{prompt} {\lstset{style=prompt}} {}
\lstnewenvironment{code}[1]{\lstset{style=code, language=#1}} {}
\lstnewenvironment{prog}{\lstset{style=code}} {}

\newcommand{\shellcmd}[1]{\textbf{\texttt{#1}\\}}
\parindent=0pt
\parskip=7pt
\newcommand{\ignore}[1]{}

\newif\ifremark
\long\def\remark#1{
  \ifremark%
  \begingroup%
  \dimen0=\columnwidth
  \advance\dimen0 by -1in%
  \setbox0=\hbox{\parbox[b]{\dimen0}{\protect\em #1}}
  \dimen1=\ht0\advance\dimen1 by 2pt%
  \dimen2=\dp0\advance\dimen2 by 2pt%
  \vskip 0.25pt%
  \hbox to \columnwidth{%
    \vrule height\dimen1 width 3pt depth\dimen2%
    \hss\copy0\hss%
    \vrule height\dimen1 width 3pt depth\dimen2%
  }%
  \endgroup%
\fi}

\remarktrue
\newif\ifwindows
\newif\ifmac
\newif\ifleuven
\newif\ifhasselt
\newif\ifbrussel
\newif\ifgent
\newif\ifantwerpen

\windowsfalse
\macfalse
\leuvenfalse
\hasseltfalse
\brusselfalse
\gentfalse
\antwerpenfalse

\ifdefined\iswindows
  \windowstrue
\fi
\ifdefined\ismac
  \mactrue
\fi
\ifdefined\isleuven
  \leuventrue
\fi
\ifdefined\ishasselt
  \hasselttrue
\fi
\ifdefined\isbrussel
  \brusseltrue
\fi
\ifdefined\isgent
  \genttrue
\fi
\ifdefined\isantwerpen
  \antwerpentrue
\fi

\title{Style Guide}

\begin{document}
\maketitle

\section{File organisation}
\label{sec:file-organisation}

The main \texttt{HPC.tex} file contains all the necessary setup for the rest of
the chapters. Each chapter should be written in a separate file. Give the
chapter a filename by prepending \texttt{ch}, then the number of the chapter
followed by a meaningful name. For example: \texttt{ch2\_introduction.tex}.

After following these conventions you only need to add the include in the main
\texttt{HPC.tex} file.

\section{Sections and chapters}
\label{sec:sections-and-chapters}

Chapter titles should contain capitalized words. Section titles should only
have the first word capitalized. Every \texttt{\textbackslash{}chapter},
\texttt{\textbackslash{}section} and \texttt{\textbackslash{}subsection} should
be accompanied by a \texttt{\textbackslash{}label}.

The label name should start with \texttt{ch} for a chapter, \texttt{sec} for a
section and \texttt{subsec} for a subsection. The label should have a name
closely resembling the title, spaces should be replaced by dashes. Example:

\begin{verbatim}
\chapter{This Is A Chapter Title}
\label{ch:chapter-title}

\section{This is a section title}
\label{sec:section-title}
\end{verbatim}

\section{Tables}
\label{sec:tables}

When defining tables spend some time to align them column-wise, this looks much more attractive.

\begin{verbatim}
\begin{tabular}{|l|l|} \hline
  short        & \textbf{\dots in human terms} \\ \hline
  short        & \dots                         \\ \hline
  short        & \dots                         \\ \hline
  a bit longer & \dots                         \\ \hline
\end{tabular}
\end{verbatim}

\section{Lists}
\label{sec:lists}

When defining lists make sure to indent the items.

\begin{verbatim}
\begin{itemize}
  \item indented item
  \item multi-line item which spans
    multiple lines
  \begin{enumerate}
    \item you can
    \item have multiple
    \item levels of indentation
  \end{enumerate}
\end{itemize}
\end{verbatim}

\section{Displaying prompts}
\label{sec:displaying-prompts}

To display prompt output you can use the prompt environment. There is the
special command \texttt{\textbackslash{}shellcmd{}} which provides consistent
styling. Inside the prompt environment you can use \% as escape character to
type commands.

\begin{verbatim}
\begin{prompt}
$ %\shellcmd{bash --version}%
GNU bash, versie 4.1.2(1)-release (x86_64-redhat-linux-gnu)
\end{prompt}
\end{verbatim}

which will result in following output:
\begin{prompt}
$ %\shellcmd{bash --version}%
GNU bash, versie 4.1.2(1)-release (x86_64-redhat-linux-gnu)
\end{prompt}

\section{Job scripts}
\label{sec:job-scripts}

When including code of any kind you can use the code environment, there is a
required argument which sets the language. The following LaTeX code:

\begin{verbatim}
\begin{code}{bash}
#!/usr/bin/bash
#PBS -m e
echo test
\end{code}
\end{verbatim}

will result in following output:

\begin{code}{bash}
#!/usr/bin/bash
#PBS -m e
echo "test"
\end{code}

\section{Windows and Mac-specific}
TODO

\section{Site-specific sections}
TODO

\end{document}


