\chapter{Filesystems}

In this section, we briefly explain which different filesystems are available on
the HPC infrastructure.

See \href{\HPCManualURL#predefined-user-directories}{Chapter 6, section titled "Where to store your data on the \hpc{}" of the HPC manual}
for a list of available locations.

\subsubsection{VO storage}

If you're member of a (non-default) virtual organisation (VO), see
\url{http://hpc.ugent.be/userwiki/index.php/User:VSCVos}, you have access to
additional directories (with more quota) on the data and scratch filesystems,
which you can share with other members in the VO.

See \url{http://hpc.ugent.be/userwiki/index.php/User:StorageDetails} for more
information.

\subsection{Quota}

Space is limited on the cluster's storage. To check your quota, see
\href{\HPCManualURL#predefined-quotas}{Chapter 6, section titled "Pre-defined quota" of the HPC manual}.

To figure out where you're quota is being spent, the \verb|du| (\strong{d}isk \strong{u}sage)
command can come in useful:

\begin{prompt}
%\shellcmd{du -sh test}%
59M   test
\end{prompt}

Do \emph{not} (frequently) run \verb|du| on directories where large amounts
of data are stored, since that will:
\begin{enumerate}
    \item take a long time
    \item result in increased
        load on the shared storage since (the metadata of) every file in those directories
        will have to be inspected.
\end{enumerate}
