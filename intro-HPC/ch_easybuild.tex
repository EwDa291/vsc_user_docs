\chapter{Easybuild}
\label{ch:easybuild}

\section{What is Easybuild?}

You can use EasyBuild to build and install supported software in your own VSC account,
rather than requesting a central installation by the HPC support team.

EasyBuild is the software build and installation framework that was created by the HPC-UGent
team, and has recently been picked up by HPC sites around the world. It allows you to manage
(scientific) software on High Performance Computing (HPC) systems in an efficient way.

\section{When should I use Easybuild?}

For general software installation requests, please see \autoref{sec:software-installation}. However,
there might be reasons to install the software yourself:

\begin{enumerate}
    \item applying custom patches to the software that only you or your group are using
    \item evaluating new software versions prior to requesting a central software installation
    \item installing (very) old software versions that are no longer eligible for central installation (on new systems)
\end{enumerate}

\section{Configuring EasyBuild}

Before you use EasyBuild, you need to configure it:

\subsection{Path to sources}

This is where EasyBuild can find software sources:

\begin{prompt}
%\shellcmd{export EASYBUILD\_SOURCEPATH=/apps/gent/source}%
\end{prompt}

\subsection{Build directory}

This directory is where EasyBuild will build software in. To have good performance,
this needs to be on a fast filesystem.

\begin{prompt}
%\shellcmd{export EASYBUILD\_BUILDPATH=\${TMPDIR:-/tmp/\$USER}}%
\end{prompt}

On cluster nodes, you can use the fast, in-memory \lstinline|/dev/shm/$USER| location
as a build directory.

\subsection{Software install location}

This is where EasyBuild will install the software (and accompanying modules) to.

For example, to let it use \lstinline|$VSC_DATA/easybuild|, use:


% Don't put the whole command inside \shellcmd{} because it would show ugly white boxes

\begin{prompt}
%\shellcmd{export}% EASYBUILD_INSTALLPATH=$VSC_DATA/easybuild/$VSC_OS_LOCAL/$VSC_ARCH_LOCAL$VSC_ARCH_SUFFIX
\end{prompt}

Using the \lstinline|$VSC_ARCH| and \lstinline|$VSC_ARCH_SUFFIX| environment variables
ensures that your install software to a location that is specific to the cluster you're
building for.

Make sure you \strong{do not build software on the login nodes}, since the loaded \lstinline|cluster|
module determines the location of the installed software. Software built on the login
nodes may not work on the cluster you want to use the software on.

To share custom software installations with members of your VO, replace \lstinline|$VSC_DATA|
with \lstinline|$VSC_DATA_VO| in the example above.

\section{Using EasyBuild}

Before using EasyBuild, you first need to load the \lstinline|EasyBuild| module. We don't specify
a version here (this is an exception, for other modules we would) because newer versions
might include important bugfixes.

\begin{prompt}
module load EasyBuild
\end{prompt}

\subsection{Installing supported software}

EasyBuild provides a large collection of readily available software versions,
combined with a particular toolchain version. Use the \lstinline|--search|
(or \lstinline|-S|) functionality to see which different 'easyconfigs'
(build recipes, see \url{http://easybuild.readthedocs.org/en/latest/Concepts_and_Terminology.html#easyconfig-files}) are available:

\begin{prompt}
%\shellcmd{eb -S OpenFOAM-2.2}%
CFGS1=/apps/gent/CO7/sandybridge/software/EasyBuild/3.6.2/lib/python2.7/site-packages/easybuild_easyconfigs-3.6.2-py2.7.egg/easybuild/easyconfigs
 * $CFGS1/o/OpenFOAM/OpenFOAM-2.2.0-goolf-1.4.10.eb
 * $CFGS1/o/OpenFOAM/OpenFOAM-2.2.0-ictce-5.3.0.eb
 * $CFGS1/o/OpenFOAM/OpenFOAM-2.2.0_libreadline.patch
 * $CFGS1/o/OpenFOAM/OpenFOAM-2.2.2-intel-2015a.eb
 * $CFGS1/o/OpenFOAM/OpenFOAM-2.2.2-intel-2016a.eb
 * $CFGS1/o/OpenFOAM/OpenFOAM-2.2.2-intel-2017a.eb
 * $CFGS1/o/OpenFOAM/cleanup-OpenFOAM-2.2.0.patch
 * $CFGS1/o/OpenFOAM/cleanup-OpenFOAM-2.2.2.patch
\end{prompt}

For readily available easyconfigs, just specify the name of the easyconfig file to build
and install the corresponding software package:

\begin{prompt}
%\shellcmd{eb OpenFOAM-2.2.2-intel-2017a.eb --robot}%
\end{prompt}

\subsection{Installing variants on supported software}

To install small variants on supported software, e.g. a different software version,
or using a different compiler toolchain, use the corresponding \lstinline|--try-X| options:

To try to install \lstinliline|GCC v4.9.2|, based on the easyconfig file for \lstinline|GCC v4.9.1|:

\begin{prompt}
%\shellcmd{eb GCC-4.9.1.eb --try-software-version=4.9.2}%
\end{prompt}

To try to install OpenFOAM v2.2.2 with a different compiler toolchain:

\begin{prompt}
%\shellcmd{eb OpenFOAM-2.2.2-intel-2015a.eb --robot --try-toolchain=foss,2015a}%
\end{prompt}

\subsection{Install other software}

To install other, not yet supported, software, you will need to provide the required
easyconfig files yourself. See \url{https://easybuild.readthedocs.org/en/latest/Writing_easyconfig_files.html}
for more information

\section{Using the installed modules}

To use the modules you installed with EasyBuild, extend \lstinline|$MODULEPATH|
to make them accessible for loading:

\begin{prompt}
module use $EASYBUILD_INSTALLPATH/modules/all
\end{prompt}
