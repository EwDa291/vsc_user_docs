% TODO: this needs to be replaced with something pretty
\includegraphics*[width=5.76in, height=2.85in, keepaspectratio=false]{img_VSC_logo}

University of Antwerp

HPC Tutorial

For Mac OS X and Linux Users

version 1.1


\textbf{CALCUA}

Department of Mathematics and Computer Science

Prof. A. Cuyt



\textbf{Written by:}

Geert Borstlap, Stefan Becuwe, Franky Backeljauw, Bert Tijskens

Acknowledgement: VSCentrum.be


University of Antwerp
HPC Tutorial
For Windows Users
Version 1.1

\textbf{CALCUA}

Department of Mathematics and Computer Science

Prof. A. Cuyt



\textbf{Written by:}

Geert Borstlap, Stefan Becuwe, Franky Backeljauw,Bert Tijskens



Acknowledgement: VSCentrum.be

\textbf{\underbar{Audience:}}

This \hpc Tutorial is designed for \textbf{researchers} at the \textbf{University of Antwerp} and affiliated institutes who are in need of computational power (computer resources) and wish to explore and use the High Performance Computing (HPC) core facility to execute their computational intensive tasks.


The audience may be completely unaware of the \hpc concepts but must have some basic understanding of computers and computer programming.


\strong{\underbar{Contents:}}

This \strong{Beginners Part} of this tutorial gives answers to the typical
questions that a new \hpc user has. The aim is to learn how to use of the
HPC.

\begin{tabular}{|p{0.4\textwidth}|c|p{0.4\textwidth}|} \hline
\multicolumn{3}{|c|}{\strong{Beginners Part}} \\ \hline
\strong{Questions}                                                      & \strong{Chapter} & \strong{Title} \\ \hline
What is a \hpc exactly?\newline Can it solve my computational needs?    & \strong{1} & Introduction to \hpc \\ \hline
How to get an account?                                                  & \strong{2} & Getting an \hpc account \\ \hline
How do I connect to the \hpc and transfer my files and programs?        & \strong{3} & Preparing the environment \\ \hline
How to start background jobs?                                           & \strong{4} & Running batch jobs \\ \hline
How to start jobs with user interaction?                                & \strong{5} & Running interactive jobs \\ \hline
Where do the input and output go? Where to collect my results?          & \strong{6} & Running jobs with input/output data \\ \hline
Can I speed up my program by exploring parallel programming techniques? & \strong{7} & Multi-core jobs / Parallel programming \\ \hline
Can I start many jobs at once?                                          & \strong{8} & Multi job submission \\ \hline
What are the rules and priorities of jobs?                              & \strong{9} & HPC policies \\ \hline
\end{tabular}

The \strong{Advanced Part} focuses on in-depth issues. The aim is to assist the
end-user in running his own software on the \hpc.

\begin{tabular}{|p{0.4\textwidth}|c|p{0.4\textwidth}|} \hline
\multicolumn{3}{|c|}{\strong{Advanced Part}} \\ \hline
\strong{Questions}                           & \strong{Chapter} & \strong{Title} \\ \hline
Can I compile my software on the \hpc?       & \strong{10}      & Compiling on the \hpc \\ \hline
What are the optimal Job Specifications?     & \strong{11}      & Fine-tuning Job Specifications \\ \hline
How do I check my programs at runtime?       & \strong{12}      & Monitoring \hpc Utilization with Ganglia \\ \hline
Can I stop my program and continue later on? & \strong{13}      & Checkpointing \\ \hline
Do you have more examples for me?            & \strong{14}      & Examples \\ \hline
Any more advice?                             & \strong{15}      & Good practices \\ \hline
\end{tabular}

The \strong{Annexes} contains some useful reference guides.

\begin{tabular}{|l|c|} \hline
\multicolumn{2}{|c|}{\strong{Annex}} \\ \hline
\strong{Title}             & \strong{Chapter} \\ \hline
\hpc Quick Reference Guide & \strong{16} \\ \hline
Torque Options             & \strong{17} \\ \hline
Useful Linux commands      & \strong{18} \\ \hline
\end{tabular}

\strong{\underbar{Notification:}}

In this tutorial specific actions dealing with the software are separated from
the accompanying text:

\begin{prompt}
%\shellcmd{Actions}%
\end{prompt}
(e.g., to be entered at a command line in your Terminal) in an exercise are preceded by a \$ and printed in \textbf{bold}. You'll find those actions in a grey frame.

\strong{<Button>} are menus, buttons or drop down boxes to be presses or selected.

\strong{`Directory'} is the notation for directories (called ``folders'' in
Windows terminology) or specific files. (e.g., `\homedir')

\strong{``Text''} Is the notation for text to be entered.

\strong{Tip:}A ``Tip'' paragraph is used for remarks or tips.

\strong{\underbar{More support:}}

Before starting the course, the example programs and configuration files used
in this \hpc-Tutorial must be copied to your home directory.

\ifantwerpen
%TODO: consider this for other sites
If you have
received a new VSC-account, all examples are present in an
`\tilde/tutorials/calcua/examples' sub-directory within your home-directory.

\begin{prompt}
%\shellcmd{ls \tilde/tutorials/calcua/examples}%
docs/  example.pbs  examples/
\end{prompt}

They can also be downloaded from \url{http://calcua.ua.ac.be/} or copied from
the directory `\emph{/apps/antwerpen/examples/}' on the \hpc.

\begin{prompt}
%\shellcmd{cd}%
%\shellcmd{cp -r /apps/antwerpen/examples/ \tilde/}%
\end{prompt}
\fi

Apart from this \hpc Tutorial, the \hpc documentation on the web
(https://www.vscentrum.be) will serve as a reference for all the \hpc
operations.

\strong{\underbar{Tip:}} The users are advised to get self-organized. There are
only limited resources available at the UA-HPC, which are best effort based.
The \hpc cannot give support for code fixing, the user applications and own
developed software remain solely the responsibility of the end-user.

More documentation can be found at:

\begin{enumerate}
  \item  VSC documentation: \url{https://www.vscentrum.be/vsc-help-center}
  \item  External documentation: \url{http://www.adaptivecomputing.com}
\end{enumerate}

Manuals (Torque, MOAB):

\begin{enumerate}
  \item  Torque: \url{http://docs.adaptivecomputing.com/torque/help.htm}
  \item  MOAB: \url{http://docs.adaptivecomputing.com/mwm/help.htm}
\end{enumerate}


\inputsite{contact-information}
