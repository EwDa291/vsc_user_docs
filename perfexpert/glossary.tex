\newglossaryentry{Cluster}
{
  name={Cluster},
  description={A group of compute nodes},
}
\newglossaryentry{Compute-Node}
{
  name={Compute Node},
  description={The computational units on which batch or interactive jobs are processed. A compute node is pretty much comparable to a single personal computer. It contains one or more sockets, each holding a single processor or CPU. The compute node is equipped with memory (RAM) that is accessible by all its CPUs},
}
\newglossaryentry{Core}
{
  name={Core},
  description={An individual compute unit inside a CPU},
}
\newglossaryentry{CPU}
{
  name={CPU},
  description={A single processing unit. A CPU is a consumable resource. Compute nodes typically contain of one or more CPUs},
}
\newglossaryentry{Distributed-memory-system}
{
  name={Distributed memory system},
  description={Computing system consisting of many compute nodes connected by a network, each with their own memory. Accessing memory on a neighboring node is possible but require explicit communication},
}
\newglossaryentry{Flop}
{
  name={Flop},
  description={Floating-point Operations Per second},
}
\newglossaryentry{HPC}
{
  name={HPC},
  description={High Performance Computing, high performance computing and multiple-task computing on a supercomputer},
}
\newglossaryentry{L1d}
{
  name={L1d},
  description={Level 1 data cache, often called \textbf{primary cache}, is a static memory integrated with processor core that is used to store data recently accessed by a processor and also data which may be required by the next operations.},
}
\newglossaryentry{L2d}
{
  name={L2d},
  description={Level 2 data cache, also called \textbf{secondary cache}, is a memory that is used to store recently accessed data and also data, which may be required for the next operations. The goal of having the level 2 cache is to reduce data access time in cases when the same data was already accessed before.},
}
\newglossaryentry{L3d}
{
  name={L3d},
  description={Level 3 data cache. Extra cache level built into motherboards between the microprocessor and the main memory.},
}
\newglossaryentry{LCC}
{
  name={LCC},
  description={The Last Level Cache is the last level in the memory hierarchy before main memory. Any memory requests missing here must be serviced by local or remote DRAM, with significant increase in latency when compared with data serviced by the cache memory.},
}
\newglossaryentry{Login-Node}
{
  name={Login Node},
  description={On UA-HPC clusters, login nodes serve multiple functions. From a login node you can submit and monitor batch jobs, analyze computational results, run editors, plots, debuggers, compilers, do housekeeping chores as adjust shell settings, copy files and in general manage your account},
}
\newglossaryentry{Memory}
{
  name={Memory},
  description={A quantity of physical memory (RAM). Memory is provided by compute nodes. It is required as a constraint or consumed as a consumable resource by jobs. Within Moab, memory is tracked and reported in megabytes (MB)},
}
\newglossaryentry{Metrics}
{
  name={Metrics},
  description={A measure of some property, activity or performance of a computer sub-system. These metrics are visualized by graphs in, e.g., Ganglia},
}
\newglossaryentry{Modules}
{
  name={Modules},
  description={UA-HPC uses an open source software package called ``Environment Modules'', (Modules for short) which allows you to add various path definitions to your shell environment},
}
\newglossaryentry{MPI}
{
  name={MPI},
  description={MPI stands for Message-Passing Interface. It supports a parallel programming method designed for distributed memory systems, but can be used very well on shared memory systems also},
}
\newglossaryentry{Node}
{
  name={Node},
  description={Typically, a machine, one computer. A node is the fundamental object associated with compute resources. Each node contains the a list of},
}
\newglossaryentry{processor}
{
  name={Processor},
  description={A processing unit. A processor is a consumable resource. Nodes typically consist of one or more processors. (same as CPU)},
}
\newglossaryentry{Scratch}
{
  name={Scratch},
  description={Supercomputers generally have what is called scratch space: disk space available for temporary use. Use /scratch when, for example you are downloading and uncompressing applications, reading and writing input/output data during a batch job, or when you work with large datasets},
}
\newglossaryentry{Shared-memory-system}
{
  name={Shared memory system},
  description={Computing system in which all of the processors share one global memory space. However, access times from a processor to different regions of memory are not necessarily uniform},
}
\newglossaryentry{SSH}
{
  name={SSH},
  description={Secure Shell (SSH), sometimes known as Secure Socket Shell, is a Unix-based command interface and protocol for securely getting access to a remote computer. It is widely used by network administrators to control Web and other kinds of servers remotely. SSH is actually a suite of three utilities - slogin, ssh, and scp - that are secure versions of the earlier UNIX utilities, rlogin, rsh, and rcp. SSH commands are encrypted and secure in several ways. Both ends of the client/server connection are authenticated using a digital certificate, and passwords are protected by encryption},
}
\newglossaryentry{ssh-keys}
{
  name={ssh-keys},
  description={OpenSSH is a network connectivity tool, which encrypts all traffic including passwords to effectively eliminate eavesdropping, connection hijacking, and other network-level attacks. SSH-keys are part of the OpenSSH bundle. On NYU UA-HPC clusters, ssh-keys allow password-less access between compute nodes while running batch or interactive parallel jobs},
}
\newglossaryentry{super-computer}
{
  name={super-computer},
  description={A computer with an extremely high processing capacity or processing power},
}
\newglossaryentry{swap-space}
{
  name={swap space},
  description={A quantity of virtual memory available for use by batch jobs. Swap is a consumable resource provided by nodes and consumed by jobs},
}
\newglossaryentry{TLB}
{
  name={TLB},
  description={Translation Look-aside Buffer, a table in the processor's memory that contains information about the virtual memory pages the processor has accessed recently. The table cross-references a program's virtual addresses with the corresponding absolute addresses in physical memory that the program has most recently used. The TLB enables faster computing because it allows the address processing to take place independent of the normal address-translation pipeline.},
}
\newglossaryentry{TACC}
{
  name={TACC},
  description={Texas Advanced Computing Center (creators of the PerfExpert tool)},
}
\newglossaryentry{UA}
{
  name={UA},
  description={University of Antwerp (creators of this tutorial)},
}
\newglossaryentry{Walltime}
{
  name={Walltime},
  description={Walltime is the length of time specified in the job-script for which the job will run on a batch system},
}
