\chapter{Graphical applications with X2Go [recommended]}
\label{ch:x2go}

X2Go is a graphical desktop software for Linux similar to VNC but with extra advantages.
It does not require to execute a server in the login node and it is possible to share a local folder with
the HPC account.
It can also be used to access Windows desktops. 


\section{Install X2Go client}
\label{sec:x2go-client}

X2Go requires a private SSH key to connect to the login node.
First login on the login node (see \autoref{sec:first-time-connection-to-the-hpc}),
then start \lstinline|vncserver| with:



\section{Connecting to the login node}

The VNC server runs on a \strong{specific login node} (in the example above, on \texttt{\loginhost{}}).

In order to access your VNC server, you will need to set up an SSH tunnel from your workstation
to this login node.

Login nodes are rebooted from time to time. You can check that the VNC server is still
running in the same node by executing \lstinline|vncserver -list| .
If you get an empty list, it means that there is no VNC server running on the login node.

\strong{To set up the SSH tunnel required to connect to your VNC server, you will need to port forward the VNC port
to your workstation.}

The \emph{host} is \lstinline|localhost|, which means ``your own computer'': we set up an SSH tunnel that connects
the VNC port on the login node to the same port on your local computer.



\ifwindows

You can download a free VNC client from \url{https://sourceforge.net/projects/turbovnc/files/}.
You can download the latest version by clicking the top-most folder that has a version number
in it that doesn't also have \lstinline|beta| in the version. Then download a file that looks like
\lstinline|TurboVNC64-2.1.2.exe| (the version number can be different, but the \lstinline|64|
should be in the filename) and execute it.
\fi
\ifmac
You can download a free VNC client from \url{https://sourceforge.net/projects/turbovnc/files/}.
You can download the latest version by clicking the top-most folder that has a version number
in it that doesn't also have \lstinline|beta| in the version. Then download a file ending in
\lstinline|TurboVNC64-2.1.2.dmg| (the version number can be different) and execute it.
\fi
\iflinux
Download and setup a VNC client. A good choice is \lstinline|tigervnc|. You can start
it with the \lstinline|vncviewer| command.
\fi

Now start your VNC client and connect to \lstinline|localhost:5906|.
\strong{Make sure you replace the port number \lstinline|5906| with your own destination port}

When prompted for a password, use the password you used to setup the VNC server.

When prompted for default or empty panel, choose default.

If you have an empty panel, you can reset your settings with the following commands:


\section{Stopping the X2Go session}
\label{sec:stop-x2go}

The VNC server can be killed by running


