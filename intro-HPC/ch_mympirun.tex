\chapter{Mympirun}
\label{ch:mympirun}

\lstinline|mympirun| is a tool to make it easier for users of HPC clusters to run
MPI programs with good performance. We strongly recommend to use \lstinline|mympirun|
instead of \lstinline|impirun|.

In this chapter, we give a high-level overview. For a more detailed description of all
options, see the
\href{https://github.com/hpcugent/vsc-mympirun/blob/master/README.md}{vsc-mympirun README}.

\section{Basic usage}
\label{sec:myrun-basic-usage}

Before using \lstinline|mympirun|, we first need to load its module:

\begin{prompt}
%\shellcmd{module load vsc-mympirun}%
\end{prompt}

As an exception, we don't specify a version here. The reason is that we want to ensure
that the latest version of the \lstinline|mympirun| script is always used, since it may
include important bug fixes or improvements.

The most basic form of using \lstinline|mympirun| is \lstinline|mympirun [mympirun options] your_program [your_program options]|.

For example, to run a program named \lstinline|example| and give it a single argument (\lstinline|5|),
we can run it with \lstinline|mympirun example 5|.


\section{Controlling number of processes}
\label{sec:mympirun-controlling-number-of-processes}

There are four options you can choose from to control the number of processes
\lstinline|mympirun| will start. In the following example, the program \lstinline|mpi_hello|
prints a single line: \lstinline|Hello world from processor <node> ...| (the sourcecode of
\lstinline|mpi_hello| is \href{https://github.com/hpcugent/vsc-mympirun/blob/master/testscripts/mpi_helloworld.c}{available in the vsc-mympirun repository}).

By default, \lstinline|mympirun| starts one process per \emph{core} on every node you assigned.
So if you assigned 2 nodes with 16 cores each, \lstinline|mympirun| will
start $2\cdot16=32$ test processes in total.

\subsection{\texttt{-{}-hybrid}/\texttt{-h}}

This is the most commonly used option for controlling the number of processing.

The \lstinline|--hybrid| option requires a positive number. This number specifies
the number of processes started on each available physical \emph{node}. It will ignore
the number of available \emph{cores} per node.

\begin{prompt}
%\shellcmd{echo \$PBS\_NUM\_NODES}%
2
%\shellcmd{mympirun --hybrid 2 ./mpi\_hello}%
Hello world from processor node2157.delcatty.os, rank 1 out of 4 processors
Hello world from processor node2158.delcatty.os, rank 3 out of 4 processors
Hello world from processor node2158.delcatty.os, rank 2 out of 4 processors
Hello world from processor node2157.delcatty.os, rank 0 out of 4 processors
\end{prompt}

\subsection{Other options}

There's also \lstinline|--universe|, which sets the exact amount of processes started by \lstinline|mympirun|;
\lstinline|--double|, which uses double the amount of processes it normally would;
and \lstinline|--multi| that does the same as \lstinline|--double|, but takes a multiplier
(instead of the implied factor 2 with \lstinline|--double|).

See \href{https://github.com/hpcugent/vsc-mympirun/blob/master/README.md}{vsc-mympirun README}
for a detailed explanation of these options.

\section{Dry run}

You can do a so-called ``dry run'', which doesn't have any side-effects, but just
prints the command that \lstinline|mympirun| would execute. You enable this with the \lstinline|--dry-run| flag:

\begin{prompt}
%\shellcmd{mympirun --dry-run ./mpi\_hello}%
mpirun ... -genv I_MPI_FABRICS shm:dapl ... -np 16 ... ./mpi_hello
\end{prompt}

\section{FAQ}

\subsection{\texttt{mympirun} seems to ignore its arguments}

For example, we have a simple script (\lstinline|./hello.sh|):

\begin{code}{bash}
#!/bin/bash
echo "hello world"
\end{code}

And we run it like \lstinline|mympirun ./hello.sh --output output.txt|.

To our surprise, this doesn't output to the file \lstinline|output.txt|, but to standard out!
This is because \lstinline|mympirun| expects the program name and the arguments of the program to
be its last arguments. Here, the \lstinline|--output output.txt| arguments are passed to
\lstinline|./hello.sh| instead of to \lstinline|mympirun|. The correct way to run it is:

\begin{prompt}
%\shellcmd{mympirun --output output.txt ./hello.sh}%
\end{prompt}



\subsection{I have other problems/questions}

Please don't hesitate to contact \hpcinfo.
