\chapter{Instance types and flavors}

VSC Tier1 Cloud provides several virtual machine instance types and flavors to fit different use cases.
Each instance type provides several flavor sizes to give different combinations of CPU, memory, GPU and network resource.

\section{Instance Types}\label{sec:instance-types}
The following table provides the current main instance types available from VSC Tier1 Cloud infrastructure:

\begin{table}[h!]
\centering
\begin{tabular}{ |p{6cm}|p{4cm}|p{5cm}| }
  \hline
  \rowcolor{lightgray} \textbf{UPSv1} & \textbf{CPUv1} & \textbf{GPUv1} \\
  \hline
  \begin{itemize}
    \item AMD Epyc 7542 2.9GHz
    \item vCPU oversubscription 2:1
    \item 25Gbit Ethernet
    \item Uninterruptible Power Supply (UPS)
  \end{itemize}
  &
  \begin{itemize}
    \item Intel Xeon CPU E5-2670 2.60GHz
    \item 10Gbit Ethernet
  \end{itemize}
  &
  \begin{itemize}
    \item AMD Epyc 7542 2.9GHz
    \item 25Gbit Ethernet
    \item NVIDIA Tesla 4 GPU
  \end{itemize}
  \\
  \hline
\end{tabular}
\caption{Instance types hardware profiles}
\label{table:instance-type}
\end{table}

Each instance type is appropriate for different workloads. From regular CPU usage (\strong{CPUv}\emph{<version>} instance types), to high performance GPU computations (\strong{GPUv}\emph{<version>}) or VMs connected to uninterruptible power supplies (\strong{UPSv}\emph{<version>}). VMs using UPS flavor types will be keep up and running even if the data centre suffers an unexpected power cut. 

\section{Flavor Sizes}\label{sec:flavor-sizes}
A flavor size is a set of virtualized hardware resouces to a virtual machine (VM) instance like system memory size, virtual CPU (vCPU) or the root filesystem size.
VSC Tier-1 Cloud VM flavors are grouped by instance types (see table \ref{table:instance-type}). The current flavor sizes are available for each instance type and can be used in different combinations to fit different workload hardware requirements.

\begin{table}[h!]
\centering
\begin{tabular}{ |p{3cm}|p{3cm}|p{3cm}|p{3cm}| }
  \hline
  \rowcolor{lightgray} \textbf{Flavor Size} & \textbf{RAM} & \textbf{Root Disk} & \textbf{vCPUs} \\
  \hline
  small & 2Gb & 20Gb & 1 \\
  \hline
  medium & 4Gb & 30Gb & 2 \\
  \hline
  large & 8Gb & 40Gb & 4 \\
  \hline
  xlarge & 16Gb & 40Gb & 8 \\
  \hline
  1\_2xlarge & 62Gb & 40Gb & 8 \\
  \hline
  2xlarge & 62Gb & 40Gb & 16 \\
  \hline
  3xlarge & 120Gb & 80Gb & 16 \\
  \hline
\end{tabular}
\caption{Flavor sizes}
\label{table:flavor-size}
\end{table}

E.g. \strong{GPUv1}.\emph{large} OpenStack flavor will instantiate a VM with 4 AMD Epyc 7542 2.9GHz CPUs, with a NVIDIA Tesla 4 GPU, 8GB of ram, 40Gb of root disk and connected to the UPS.

%%% Local Variables:
%%% mode: latex
%%% TeX-master: "intro-Cloud"
%%% End:
