\chapter{Multi core jobs/Parallel Computing}

\section{Why Parallel Programming?}

There are two important motivations to engage in parallel programming.

\begin{enumerate}

\item  Firstly, the need to decrease the time to solution: distributing your
  code over \emph{C} cores holds the promise of speeding up execution times
  by a factor \emph{C}. All modern computers (and probably even your
  smartphone) are equipped with multi-core processors capable of parallel
  processing.
\item  The second reason is problem size: distributing your code over
  \emph{N} nodes increases the available memory by a factor \emph{N}, and
  thus holds the promise of being able to tackle problems which are \emph{N}
  times bigger.
\end{enumerate}

On a desktop computer, this enables a user to run multiple programs and the
operating system simultaneously. For scientific computing, this means you have
th ability in principle of splitting up your computations into groups and
running each group on its own core.

There are multiple different ways to achieve parallel programming. The table
below gives a (non-exhaustive) overview of problem independent approaches to
parallel programming. In addition there are many problem specific libraries
that incorporate parallel capabilities. The next three sections explore some
common approaches: (raw) threads, OpenMP and MPI.

\begin{tabular}{|p{0.15\textwidth}|p{0.3\textwidth}|p{0.55\textwidth}|} \hline
\multicolumn{3}{|c|}{\strong{Parallel programming approaches}} \\ \hline
\strong{Tool}                                                          & \strong{Available language bindings}                                   & \strong{Limitations} \\ \hline
Raw threads\newline pthreads, boost::\newline threading, \dots         & Threading libraries are available for all common programming languages & Threads are limited to a shared memory systems. They are more often used on single node systems rather than for \hpc. Thread management is hard. \\ \hline
OpenMP                                                                 & Fortran/C/C++                                                      & Limited to shared memory systems, but large shared memory systems for \hpc are not uncommon (e.g. SGI UV). Loops and task can be parallelized by simple insertion of compiler directives. Under the hood threads are used. Hybrid approaches exist which use OpenMP to parallelize the work load on each node and MPI (see below) for communication between nodes. \\ \hline
Lightweight threads with clever scheduling, Intel TBB, Intel Cilk Plus & C/C++                                                                & Limited to shared memory systems, but may be combined with MPI. Thread management is taken care of by a very clever scheduler enabling the programmer to focus on parallelization itself. Hybrid approaches exist which use TBB and/or Cilk Plus to parallelize the work load on each node and MPI (see below) for communication between nodes. \\ \hline
MPI                                                                    & Fortran/C/C++ Python                                             & Applies to both distributed and shared memory systems. Cooperation between different nodes or cores is managed by explicit calls to library routines handling communication routines. \\ \hline
Global Arrays library                                                  & C/C++ Python                                                       & Mimics a global address space on distributed memory systems, by distributing arrays over many nodes and one sided communication. This library is used a lot for chemical structure calculation codes and was used in one of the first applications that broke the PetaFlop barrier. \\ \hline
Scoop                                                                  & Python                                                                 & Applies to both shared and distributed memory system. Not extremely advanced, but may present a quick road to parallelization of python code.  \\ \hline
\end{tabular}

\section{Parallel Computing with threads}

Multi-threading is a widespread programming and execution model that allows
multiple threads to exist within the context of a single process. These threads
share the process' resources, but are able to execute independently. The
threaded programming model provides developers with a useful abstraction of
concurrent execution. Multi-threading can also be applied to a single process
to enable parallel execution on a multiprocessing system.

\includegraphics*[width=5.75in, height=2.39in, keepaspectratio=false]{img0700}

This advantage of a multithreaded program allows it to operate faster on
computer systems that have multiple CPUs or across a cluster of machines ---
because the threads of the program naturally lend themselves to truly
concurrent execution. In such a case, the programmer needs to be careful to
avoid race conditions, and other non-intuitive behaviors. In order for data to
be correctly manipulated, threads will often need to synchronize in time in
order to process the data in the correct order. Threads may also require
mutually exclusive operations (often implemented using semaphores) in order to
prevent common data from being simultaneously modified, or read while in the
process of being modified. Careless use of such primitives can lead to
deadlocks.

Threads are a way that a program can spawn concurrent units of processing that
can then be delegated by the operating system to multiple processing cores.
Clearly the advantage of a multithreaded program (one that uses multiple
threads that are assigned to multiple processing cores) is that you can achieve
big speedups, as all cores of your CPU (and all CPUs if you have more than one)
are used at the same time.

Here is a simple example program that spawns 7 threads, where each one runs a
simple function that only prints ``Hello from thread''.

Go to the example directory:

\begin{prompt}
%\shellcmd{cd ~/\exampledir}%
\end{prompt}

Study the example first:

\examplecode{C}{T_hello.c}

And compile it (whilst including the thread library) and run and test it on the login-node:

\begin{prompt}
%\shellcmd{module load GCC}%
%\shellcmd{gcc -o T\_hello T\_hello.c -lpthread}%
%\shellcmd{./T\_hello}%
spawning thread 0
spawning thread 1
spawning thread 2
Hello from thread 0!
Hello from thread 1!
Hello from thread 2!
spawning thread 3
spawning thread 4
Hello from thread 3!
Hello from thread 4!
\end{prompt}

Now, run it on the cluster and check the output:

\begin{prompt}
%\shellcmd{qsub T\_hello.pbs}%
%\jobid%
%\shellcmd{more T\_hello.pbs.o\jobnumber}%
spawning thread 0
spawning thread 1
spawning thread 2
Hello from thread 0!
Hello from thread 1!
Hello from thread 2!
spawning thread 3
spawning thread 4
Hello from thread 3!
Hello from thread 4!
\end{prompt}

% FIXME: Refer to Hager & co insted of this.
\underbar{Tip:} If you plan engaging in parallel programming using threads,
this book may prove useful: \emph{Professional Multicore Programming: Design
and Implementation for C++ Developers. Cameron Hughes and Tracey Hughes. Wrox
2008.}

\section{Parallel Computing with OpenMP}

\strong{\emph{OpenMP}} is an API that implements a multi-threaded, shared
memory form of parallelism. It uses a set of compiler directives (statements
that you add to your code and that are recognized by your Fortran/C/C++
compiler if OpenMP is enabled or otherwise ignored) that are incorporated at
compile-time to generate a multi-threaded version of your code. You can think
of Pthreads (above) as doing multi-threaded programming "by hand", and OpenMP
as a slightly more automated, higher-level API to make your program
multithreaded. OpenMP takes care of many of the low-level details that you
would normally have to implement yourself, if you were using Pthreads from the
ground up.

An important advantage of OpenMP is that, because it uses compiler directives,
the original serial version stays intact, and minimal changes (in the form of
compiler directives) are necessary to turn a working serial code into a working
parallel code.

Here is the general code structure of an OpenMP program:

\begin{code}{C}
#include <omp.h>
main ()  {
int var1, var2, var3;
// Serial code
// Beginning of parallel section. Fork a team of threads.
// Specify variable scoping

#pragma omp parallel private(var1, var2) shared(var3)
  {
  // Parallel section executed by all threads
  // All threads join master thread and disband
  }
// Resume serial code
}
\end{code}

\subsection{Private versus Shared variables}

By using the private() and shared() clauses, you can specify variables within
the parallel region as being \strong{shared}, i.e. visible and accessible by
all threads simultaneously, or \strong{private}, i.e. private to each thread,
meaning each thread will have its own local copy. In the code example below for
parallelizing a for loop, you can see that we specify the thread\_id and nloops
variables as private.

\subsection{Parallelizing for loops with OpenMP}

Parallelizing for loops is really simple (see code below). By default, loop
iteration counters in OpenMP loop constructs (in this case the i variable) in
the for loop are set to private variables.

\examplecode{C}{omp1.c}

And compile it (whilst including the ``\emph{openmp}'' library) and run and test it on the login-node:

\begin{prompt}
%\shellcmd{module load GCC}%
%\shellcmd{gcc -fopenmp -o omp1 omp1.c}%
%\shellcmd{./omp1}%
Thread 6 performed 125 iterations of the loop.
Thread 7 performed 125 iterations of the loop.
Thread 5 performed 125 iterations of the loop.
Thread 4 performed 125 iterations of the loop.
Thread 0 performed 125 iterations of the loop.
Thread 2 performed 125 iterations of the loop.
Thread 3 performed 125 iterations of the loop.
Thread 1 performed 125 iterations of the loop.
\end{prompt}

Now run it in the cluster and check the result again.

\begin{prompt}
%\shellcmd{qsub omp1.pbs}%
%\shellcmd{cat omp1.pbs.o*}%
Thread 1 performed 125 iterations of the loop.
Thread 4 performed 125 iterations of the loop.
Thread 3 performed 125 iterations of the loop.
Thread 0 performed 125 iterations of the loop.
Thread 5 performed 125 iterations of the loop.
Thread 7 performed 125 iterations of the loop.
Thread 2 performed 125 iterations of the loop.
Thread 6 performed 125 iterations of the loop.
\end{prompt}

\subsection{Critical Code}

Using OpenMP you can specify something called a "critical" section of code.
This is code that is performed by all threads, but is only performed
\strong{one thread at a time} (i.e. in serial). This provides a convenient way
of letting you do things like updating a global variable with local results
from each thread, and you don't have to worry about things like other threads
writing to that global variable at the same time (a collision).

\examplecode{C}{omp2.c}

And compile it (whilst including the ``\emph{openmp}'' library) and run and test it on the login-node:

\begin{prompt}
%\shellcmd{module load GCC}%
%\shellcmd{gcc -fopenmp -o  omp2   omp2.c}%
%\shellcmd{./omp2}%
Thread 3 is adding its iterations (12500) to sum (0), total is now 12500.
Thread 7 is adding its iterations (12500) to sum (12500), total is now 25000.
Thread 5 is adding its iterations (12500) to sum (25000), total is now 37500.
Thread 6 is adding its iterations (12500) to sum (37500), total is now 50000.
Thread 2 is adding its iterations (12500) to sum (50000), total is now 62500.
Thread 4 is adding its iterations (12500) to sum (62500), total is now 75000.
Thread 1 is adding its iterations (12500) to sum (75000), total is now 87500.
Thread 0 is adding its iterations (12500) to sum (87500), total is now 100000.
Total # loop iterations is 100000
\end{prompt}

Now run it in the cluster and check the result again.

\begin{prompt}
%\shellcmd{qsub omp2.pbs}%
%\shellcmd{cat omp2.pbs.o*}%
Thread 2 is adding its iterations (12500) to sum (0), total is now 12500.
Thread 0 is adding its iterations (12500) to sum (12500), total is now 25000.
Thread 1 is adding its iterations (12500) to sum (25000), total is now 37500.
Thread 4 is adding its iterations (12500) to sum (37500), total is now 50000.
Thread 7 is adding its iterations (12500) to sum (50000), total is now 62500.
Thread 3 is adding its iterations (12500) to sum (62500), total is now 75000.
Thread 5 is adding its iterations (12500) to sum (75000), total is now 87500.
Thread 6 is adding its iterations (12500) to sum (87500), total is now 100000.
Total # loop iterations is 100000
\end{prompt}

\subsection{Reduction}

Reduction refers to the process of combining the results of several
sub-calculations into a final result. This is a very common paradigm (and
indeed the so-called "map-reduce" framework used by Google and others is very
popular). Indeed we used this paradigm in the code example above, where we used
the "critical code" directive to accomplish this. The map-reduce paradigm is so
common that OpenMP has a specific directive that allows you to more easily
implement this.

\examplecode{C}{omp3.c}

And compile it (whilst including the ``\emph{openmp}'' library) and run and
test it on the login-node:

\begin{prompt}
%\shellcmd{gcc -fopenmp -o omp3 omp3.c}%
%\shellcmd{./omp3}%
Total # loop iterations is 100000
\end{prompt}

Now run it in the cluster and check the result again.

\begin{prompt}
%\shellcmd{qsub omp3.pbs}%
%\shellcmd{cat omp3.pbs.o*}%
Total # loop iterations is 100000
\end{prompt}

\subsection{Other OpenMP directives}

There are a host of other directives you can issue using OpenMP.

Some other clauses of interest are:

\begin{enumerate}
\item  barrier: each thread will wait until all threads have reached this point in the code, before proceeding
\item  nowait: threads will not wait until everybody is finished
\item  schedule(type, chunk) allows you to specify how tasks are spawned out to threads in a for loop. There are three types of scheduling you can specify
\item  if: allows you to parallelize only if a certain condition is met
\item  \dots  and a host of others
\end{enumerate}

\underbar{Tip:} If you plan engaging in parallel programming using OpenMP, this
book may prove useful: \textit{Using OpenMP - Portable Shared Memory Parallel
Programming}. By Barbara Chapman Gabriele Jost and Ruud van der Pas Scientific
and Engineering Computation. 2005.

\section{Parallel Computing with MPI }

The Message Passing Interface (MPI) is a standard defining core syntax and
semantics of library routines that can be used to implement parallel
programming in C (and in other languages as well). There are several
implementations of MPI such as Open MPI, Intel MPI, M(VA)PICH(2) and LAM/MPI.

In the context of this tutorial, you can think of MPI, in terms of its
complexity, scope and control, as sitting in between programming with Pthreads,
and using a high-level API such as OpenMP. For a Message Passing Interface
(MPI) application, a parallel task usually consists of a single executable
running concurrently on multiple processors, with communication between the
processes.  This is shown in the following diagram:

\includegraphics*[width=1.50in, height=1.59in, keepaspectratio=false]{img0701}

The process numbers 0, 1 and 2 represent the process rank and have greater or
less significance depending on the processing paradigm. At the minimum, Process
0 handles the input/output and determines what other processes are running.

The MPI interface allows you to manage allocation, communication, and
synchronization of a set of processes that are mapped onto multiple nodes,
where each node can be a core within a single CPU, or CPUs within a single
machine, or even across multiple machines (as long as they are networked
together).

One context where MPI shines in particular is the ability to easily take
advantage not just of multiple cores on a single machine, but to run programs
on clusters of several machines. Even if you don't have a dedicated cluster,
you could still write a program using MPI that could run your program in
parallel, across any collection of computers, as long as they are networked
together. Just make sure to ask permission before you load up your lab-mate's
computer's CPU(s) with your computational tasks!

Here is a "Hello World" program in MPI written in C. In this example, we send a
"Hello" message to each processor, manipulate it trivially, return the results
to the main process, and print the messages.

Study the MPI-programme and the PBS-file:

\examplecode{C}{mpi_hello.c}
\examplecode{bash}{mpi_hello.pbs}

mpiicc is a wrapper of the Intel C++ compiler icc to compile MPI programs (see
the chapter on compilation for details).

Run the parallel program:

\begin{prompt}
%\shellcmd{qsub mpi\_hello.pbs}%
%\shellcmd{ls -l}%
total 1024
-rwxrwxr-x 1 %\userid% 8746 Sep 16 14:19 mpi_hello*
-rw-r--r-- 1 %\userid% 1626 Sep 16 14:18 mpi_hello.c
-rw------- 1 %\userid%    0 Sep 16 14:22 mpi_hello.o%\jobnumber%
-rw------- 1 %\userid%  697 Sep 16 14:22 mpi_hello.o%\jobnumber%
-rw-r--r-- 1 %\userid%  304 Sep 16 14:22 mpi_hello.pbs
%\shellcmd{cat mpi\_hello.o\jobnumber}%
0: We have 16 processors
0: Hello 1! Processor 1 reporting for duty
0: Hello 2! Processor 2 reporting for duty
0: Hello 3! Processor 3 reporting for duty
0: Hello 4! Processor 4 reporting for duty
0: Hello 5! Processor 5 reporting for duty
0: Hello 6! Processor 6 reporting for duty
0: Hello 7! Processor 7 reporting for duty
0: Hello 8! Processor 8 reporting for duty
0: Hello 9! Processor 9 reporting for duty
0: Hello 10! Processor 10 reporting for duty
0: Hello 11! Processor 11 reporting for duty
0: Hello 12! Processor 12 reporting for duty
0: Hello 13! Processor 13 reporting for duty
0: Hello 14! Processor 14 reporting for duty
0: Hello 15! Processor 15 reporting for duty
\end{prompt}

The runtime environment for the MPI implementation used (often called mpirun or
mpiexec) spawns multiple copies of the program, with the total number of copies
determining the number of process \emph{ranks} in MPI\_COMM\_WORLD, which is
an opaque descriptor for communication between the set of processes. A single
process, multiple data (SPMD = Single Program, Multiple Data) programming model
is thereby facilitated, but not required; many MPI implementations allow
multiple, different, executable's to be started in the same MPI job. Each
process has its own rank, the total number of processes in the world, and the
ability to communicate between them either with point-to-point (send/receive)
communication, or by collective communication among the group. It is enough for
MPI to provide an SPMD-style program with MPI\_COMM\_WORLD, its own rank, and
the size of the world to allow algorithms to decide what to do. In more
realistic situations, I/O is more carefully managed than in this example. MPI
does not guarantee how POSIX I/O would actually work on a given system, but it
commonly does work, at least from rank 0.

MPI uses the notion of process rather than processor. Program copies are
\emph{mapped} to processors by the MPI runtime. In that sense, the parallel
machine can map to 1 physical processor, or N where N is the total number of
processors available, or something in between. For maximum parallel speedup,
more physical processors are used. This example adjusts its behavior to the
size of the world N, so it also seeks to scale to the runtime configuration
without compilation for each size variation, although runtime decisions might
vary depending on that absolute amount of concurrency available.


\underbar{Tip:} If you plan engaging in parallel programming using MPI, this
book may prove useful: \emph{Parallel Programming with MPI. Peter Pacheo.
Morgan Kaufmann. 1996.}
